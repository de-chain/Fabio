\chapter{Analisi } % Capitolo 3
\label{Analisi} % Etichetta per riferimenti incrociati

\section{Энциклопедия русской жизни}

Русское  дворянство было сословием душе- и землевладельцев. 
Владение поместьями и крепостными крестьянами составляло одновременно сословную привилегию дворян  и было мерилом богатства, общественного положения 
и престижа. 
Это, в частности, приводило к тому, что  стремление увеличивать число душ доминировало над  попытками повысить доходность поместья путем рационального землепользования.
По имущественному положению различались мелко поместные (до 80— 100 душ), среднепоместные (число душ которых исчислялось сотнями) и крупнопоместные 
(около тысячи душ) дворяне. 
Кроме того, имелась количественно небольшая, но стоящая на вершинах 
власти и жизни группа помещиков, имущество которых насчитывало десятки или даже сотни тысяч душ. Иерархия душевладения, в значительной мере, определяла общественное положение. 
Герои ЕО довольно четко охарактеризованы в отношении их имущественного положения. Характеристика Ленского начинается с указания, что он «богат», Ларины же не были богаты. 
Старшая Ларина, вдова екатерининского бригадира, скорее всего., была помещицей среднего достатка. 
Но богатства оказывается связанной с мотивом разорения. Слова «долги», «залог», «заимодавцы» встречаются уже в первых строках романа.
 Долги, проценты по залогам, перезакладывание уже заложенных имений было уделом отнюдь не только бедных или стоящих на грани краха помещиков. Более  того, именно мелкие и средние провинциальные помещики, менее нуждающиеся в деньгах на покупку предметов роскоши и дорогостоящих импортных товаров довольствующиеся «домашним припасом», реже входили в долги и прибегали к разорительным финансовым операциям. Между тем столичное дворянство, начиная с екатерининских времен, поголовно было в долгах. 
Фонвизин во «Всеобщей придворной грамматике» писал: «Как у двора, так и в столице никто без долгу не живет, для того чаще всех спрягается глагол: быть должным...»
Он же спрашивал Екатерину II: «От чего все в долгах?» — и получил ответ: «Оттого в долгах, что проживают более, нежели дохода имеют».

 Повышение доходности хозяйства путем увеличения его производительности противоречило как природе крепостного труда, так и психологии дворянина-помещика, который предпочитал идти по более легкому пути роста крестьянских повинностей и оброков. Давая единовременный эффект повышения дохода, эта мера в конечном итоге разоряла крестьян и самого помещика, хотя умение выжимать из крестьян деньги считалось среди средних и мелких помещиков основой хозяйственного искусства. 
 
 \emph{Гвоздин, хозяин превосходный,\\
 Владелец нищих мужиков (V, XXVI, 3 — 4).}
 
Ещё  Более верными способами «подымать доходы по расходам» были различные формы пожалований от правительства. 

\subsection{Долги}
Долги могли образоваться от частных займов и заклада поместий в банк.
Жить на средства, полученные при закладе имения, называлось «жить долгами».Такой способ был прямым путем к разорению. Предполагалось, что дворянин на полученные при закладе деньги приобретет новые поместья или улучшит состояние старых и, повысив таким образом свой доход, получит средства на уплату процентов и выкуп поместья из заклада. Однако в большинстве случаев 
дворяне проживали полученные в банке суммы, тратя их на покупку или строительство домов в столице, туалеты, балы. Это приводило к перезакладыванию уже заложенных имений. Приходилось делать долги, вырубать леса, продавать еще не заложенйые деревни и т. д.
Когда отец Онегина, скончался, выяснилось, что наследство обременено большими долгами:
 \emph{Перед Онегиным собрался\\
 Заимодавцев жадный полк (Iу LI, 6-7).}
 
 В этом случае наследник мог принять наследство и вместе с ним взять на себя долги отца или отказаться от него, предоставив кредиторам самим улаживать 
счеты между собой. Первое решение диктовалось чувством чести, желанием не запятнать доброе имя отца или сохранить родовое имение.
Легкомысленный же Онегин пошел по второму пути.
Молодым людям охотно верили в долг рестораторы, портные, владельцы магазинов в расчете на их «грядущие доходы». Поэтому молодой человек из богатой семьи мог без больших денег вести в Петербурге безбедное существование при наличии надежд на наследство и известной беззастенчивости. Так, Лев Сергеевич, брат поэта, жил в Петербурге без копейки денег.
\testbf{Молодость — время надежд на наследство — была как бы узаконенным периодом долгов, от которых во вторую половину жизни следовало освобождаться, став «наследником ... своих родных» или выгодно женившись.}\\
\emph{ Блажен <...)\\
Кто в двадцать лет был франт иль хват,\\ 
А в тридцать выгодно женат;\\
Кто в пятьдесят освободился\\
 От частных и других долгов\\
 (VIII,»Х, 1 — 10).}
 \section{Образование и служба дворян}
 В России домашнее воспитание есть самое недостаточное, самое безнравственное. Оно ограничивается изучением двух или трех иностранных языков и начальным основанием всех наук, преподаваемых каким-нибудь нанятым учителем французем. Но Француз-гувернер и француз-учитель редко серьезно относились к своим педагогическим обязанностям. Если в XVIII в. (до французской революции 1789 г.) претендентами на учительские места в России были, главным образом, мелкие жулики и авантюристы, актеры, парикмахеры, беглые солдаты и просто люди неопределенных занятий, то после революции за границами Франции оказались тысячи аристократов-эмигрантов и в России возник новый тип учителя-француза. 
 Русский язык, словесность и историю, а также танцы, верховую езду и фехтование преподавали специальные учителя, которых приглашали «по билетам».
Альтернативой домашнему воспитанию, дорогому и малоудовлетворительному, были частные пансионы и государственные училища. Частные пансионы, как и уроки домашних учителей, не имели ни общей программы, ни каких-либо единых требований.%аббат Николь Пансион «Немца»
В значительно большем порядке находились государственные учебные заведения. Большинство русских дворян по традиции готовили своих детей к военному поприщу. По указу 21 марта 1805 г. в обеих столицах и ряде провинциальных городов (Смоленске, Киеве, Воронеже и др.) были открыты начальные военные училища в количестве 
«15 рот». В них зачислялись дети «от 7 до 9-летнего возраста, которые, пробыв в училище 7 лет.  Военное поприще представлялось настолько естественным для дворянина, что отсутствие этой черты в биографии должно было иметь какое-либо специальное объяснение. 
Штатская служба стояла значительно ниже военной. Однако в ее пределах имелись существенные отличия с точки зрения ценности в глазах современников. К наиболее «благородным» относили дипломатическую службу. Гоголь в «Невском проспекте» иронически писал, что чиновники иностранной коллегии «отличаются благородством 
своих занятий и привычек», и восклицал: «Боже, какие есть прекрасные должности и службы! как они возвышают и услаждают душу!».
Штатскими высшими учебными заведениями были университеты. В начале XIX в. их было в России пять: Московский, Дерптский, Виленский, Казанский и Харьковский. При университетах имелись подготовительные училища — пансионы..
Герой пушкинского романа получил только домашнее образование. Следует подчеркнуть, что отношение П. к такому воспитанию было резко отрицательным. В записке для Николая I в 1826 г. поэт писал категорически: \emph{«Нечего колебаться: во что бы то ни стало должно подавить воспитание частное»} 
 Онегин никогда не носил военного мундира, что выделяло его из числа сверстников, встретивших 1812 г. в возрасте 16—17 лет. 
Но то, что он вообще никогда нигде служил, не имел никакого, даже самого низшего чина, решительно делало Онегина белой вороной в кругу современников.
Манифестом Петра III от 18 февраля 1762 г. дворянство было освобождено от обязательной службы. 
Правительство Екатерины II пыталось указом от 11 февраля 1763 г. приостановить действие манифеста и сделать службу вновь обязательной. Указ 1774 года подтвердил обязательность военной службы для дворянских недорослей. Однако Жалованная дворянству грамота 1785 г. вновь вернула дворянам «вольность» служить и оставлять службу — как гражданскую, так и военную — по своему произволу.
Таким образом, неслужащий дворянин формально не нарушал законов империи. Однако его положение 
в обществе было совершенно особым. Правительство также весьма отрицательно смотрело на уклоняющегося от службы и не имеющего никакого 
чина дворянина. Служба входила в дворянское понятие чести и патриотизма. 
Однако была и противоположное мнение. Еще Новиков противопоставил пафосу государственной службы идею организованных усилий «приватных» людей. В поэзии Державина сталкивались пафос государственного служения и поэзия частного существования. Однако, пожалуй, именно Карамзин сделал впервые отказ от 
государственной службы предметом поэтизации в стихах, звучавших для своего времени достаточно дерзко:
 \\emph{...в войне добра не видя,\\
 В чиновных гордецах чины возненавидя,\\
 Вложил свой меч в ножны\\
 («Россия, торжествуй, — Сказал я, — без меня!»)...\\
 (Карамзин, с. 170).}\\
 То, что было предметом эгоизма и отсутствия любви к обществу, неожиданно приобретало контуры борьбы за личную независимость, отстаивания права человека самому определять род своих занятий, строить свою жизнь независимо от государственного надзора.  
 В свете сказанного видно, во-первых, что то, что Онегин никогда не служил, не имел чина, не было 
неважным и случайным признаком — это важная и заметная современникам черта. Во-вторых, черта эта по-разному просматривалась в свете различных культурных перспектив, бросая на героя то сатирический, 
то глубоко интимный для автора отсвет.анных путей. 
Право не служить, быть «сам большой» и оставаться верным «науке первой» — чтить самого себя,стало заповедью зрелого П. 
\section{Глава 3}
XI —XII.
Строфы посвящены сопоставлению моралистических романов XVIII в., о которых Карамзин писал: «Напрасно думают, что романы могут быть вредны
для сердца: все они представляют обыкновенно славу добродетели или нравоучительное следствие» (Карамзин),— с романами эпохи романтизма.
XII
\emph{Британской музы небылицы...},романтизм, в значительной мере, воспринимался как «английское» направление в европейской литературе.
\emph{...задумчивый Вампир...} — П снабдил упоминание Вампира примечанием: «Повесть, неправильно приписанная лорду Байрону». Помета указывает на
следующий эпизод: в 1816 г. в Швейцарии, спасаясь от дурной погоды, Байрон, Шелли, его восемнадцатилетняя жена Мэри и врач Полидори договорились развлекать друг друга страшными новеллами. Условие выполнила только Мэри Шелли, сочинившая роман «Франкенштейн», сделавшийся классическим произведением «черной литературы» и доживший до экранизаций в XX в. Байрон сочинил фрагмент романа «Вампир». Позже (1819) появился в печати роман «Вампир» на тот же сюжет, написанный Полидори, использовавшим, видимо, устные импровизации Байрона. Роман был приписан Байрону и под его именем переведен в том же,
1819 г., на французский язык «Le Vampire, nouvelle traduite de l’anglais de Lord Byron». (77, видимо, пользовался этим изданием). Байрон нервно реагировал на
публикацию, потребовал, чтобы Полидори раскрыл в печати свое авторство, а сам опубликовал сохранившийся у него отрывок действительно им написанного «Вампира», чтобы читатели могли убедиться в отличии байроновского текста от опубликованного Полидори.
\section{Глава 4}
VII
\emph{Чем меньше женщину мы любим...} Переложение в стихах отрывка из письма П к брату: «То, что я могу сказать тебе о женщинах, было бы совершенно бесполезно. Замечу только, что чем меньше любим мы женщину, тем вернее можем овладеть ею. Однако забава эта достойна старой обезьяны.
\emph{Со славой красных каблуков....} — Высокие красные каблуки были в моде при дворе Людовика XV. «Красные каблуки» сделалось прозванием предреволюционной аристократии. Большие парики были модны в первую половину XVIII века.
X
\emph{На вист вечерний приезжает... } Вист, карточная игра для четырех партнеров. Считалась игрой «степенных» солидных людей. Вист, — коммерческая,
а не азартная игра, носила спокойный характер.
XII — XVI\\
«Проповедь» Онегина\\
Отсутствие литературныq клише и реминисценций.
Смысл речи Онегина именно в том, что он неожиданно для Татьяны повел себя не как литературный герой («спаситель» или «соблазнитель»), а просто
как хорошо воспитанный светский и к тому же вполне порядочный человек, который «очень мило поступил с печальной Таней». Онегин повел себя не по законам литературы, а по нормам и правилам, которыми руководствовался достойный человек пушкинского круга.
XVIII —XXII\\
 Строфы представляют собой полемическое сопоставление литературного культа любви и дружбы и бытовой реальности светской жизни.
Противопоставляемая культу пламенных чувств проповедь эгоизма «любите самого себя» также имеет характер не философского обобщения, а практического рецепта относительно того, как себя следует вести в свете (ср. совет брату: «Будь холоден со всеми; фамильярность всегда вредит»), чтобы
сохранить независимость и личное достоинство.
XIX\\
\emph{А что? Да так. Я усыпляю.....}\\
Непосредственный и доверительный разговор с читателем.
XX\\
\emph{Что значит имянно родные}. Здесь оно является элементом бытовой разговорной дворянской речи, отражая
исключительное значение родственных связей в быту пушкинской эпохи. Всякое знакомство начиналось с того, чтобы «счесться родными», выяснить, если возможно, степень родства. Это же влияло и на все виды карьерных и должностных продвижений. \\
\emph{0 рожестве их навещать...} — Сочельник Рождества (обращает внимание подчеркнуто разговорная формула «о рожестве»!) был временем обязательных официальных визитов
XXIII\\
\emph{Так одевает бури тень\\
Едва рождающийся день.}
 Реминисценция из поэмы Баратынского «Эда»:
\emph{Что ж изменить ее могло?\\
Что ж это утро облекло\\
И так внезапно в сумрак ночи?} (Баратынский, II, с. 150)\\
Текстуальная близость стихов 13-14 к «Эде» посуществу полемична: у Баратынского они характеризуют состояние «падшей» героини, соблазненной «злодеем». Такой ситуации, повторенной в бесчисленном ряду литературных текстов.\\
П противопоставляет каждодневное бытовое течение вещей («не-событие», по литературным нормам), являющееся одновременно совершенно уникальным в литературе той поры.
Эда «увядает» в результате победы «лукавого соблазнителя», Татьяна «увядает», хотя ни ее романтическое письмо, ни «роковое» свидание ни в чем не изменило ее судьбы, а Онегин решительно отказался от роли литературного соблазнителя.\\
XXVI\\
\emph{...автор знает боле\\
Природу, чем Шатобриан... }\\
Шатобриан Рене (1768 —1848) — французский писатель и политический деятель. 
Литературные вкусы Ленского тяготеют к предромантизму, а не к романтизму : он окружен воспоминаниями о Шиллере, Гете (конечно, как авторе «Вертера»), Стерне, о нравоучительном романе XVIII в.\\
 К Шатобриану он относится отрицательно, романтические бунтари и пессимисты XIX в., в первую очередь Байрон, из его мира исключены. Энтузиазм и чувствительность, оптимизм и вера в свободу предромантической литературы противопоставлялись эгоизму, разочарованности и скепсису романтизма. Это следует подчеркнуть, поскольку в исследовательской литературе имеется тенденция трактовать Ленского как воплощение романтизма как такового.\\
 XXVII —XXX\\
 Альбом был важным фактом «массовой культуры» второй половины XVIII — первой половины XIX вв., являясь своеобразным рукописным альманахом. 
  Что был альбом 20 лет назад? Книжка в алом сафьяне в '32-ю долю листа. Что находили в таких книжках?
Песни Хованского, Николева, конфектные билетцы (бумажка, в которую завернута конфета, с напечатанным на ней стихотворением.) и любовные
объяснения. \\
Теперь, о! теперь не то! Переплетчики истощили все свое искусство на украшение этих книжек. Теперь редко найдете в них выписки из печатного, или дурные рисунки цветков й домиков. В нынешних альбомах хотят иметь рисунки лучших артистов, почерк известных литераторов. Есть альбомы,которые, через 50 лет, будут дороже целой русской библиотеки». Характеризуя разные типы альбомов, Яковлев так описывает альбом девиц : «В 8-ку. Переплет обернут веленевою бумажкою. На первом листке советы от матери, — стихи французские, английские, итальянские; выписки из Жуковского, много рисунков карандашом. Травки и сушеные цветы между листами».
XXVII\\
\emph{Прилежно украшает ей..}\\
 Альбомы начала XIX в. включали не толькЬ стихи, но и рисунки. Часто в них вклеивались вырезанные из книг офорты
и гравюры. Обучение живописи было весьма распространено в домашнем дворянском воспитании и входило в обязательную программу военных корпусов и ряда
гражданских училищ. 
\emph{Пониже подписи других...}
Хотя альбомы заполнялись в хронологической последовательности, место, на котором делалась запись, имело значение: первые
страницы отводились родителям и старшим, затем шли подруги и друзья. Для выражения более нежных
чувств предназначался конец альбома — особенно значимыми считались подписи на последнем листе. Самый первый лист часто оставался
незаполненным, поскольку существовало поверье, что с открывшим первую страницу альбома случится несчастье.
pg 243



\section{Глава 5}
\emph{О, не знай сих страшных снов\\ 
Ты, моя Светлана!\\
 Жуковский}\\
 Эпиграф из заключительных стихов баллады Жуковского «Светлана» (1812). Она считалась образцом романтического фольклоризма. Это Имя звучала для уха читателей тех лет шокирующе, поскольку «Светлана» не бытовое имя (оно 
отсутствует в святцах). Светланы Жуковского и Татьяны Лариной раскрывало не только параллелизм их народности, но и глубокое отличие в трактовке образов: одного, ориентированного на романтическую фантастику и игру, другого — на бытовую и психологическую реальность.
\emph{II, 1 —Зима!.. Крестьянин торжествуя... }\\
Столкновение церковнославянского «торжествовать» и «крестьянин» побудило критиков сделать автору замечание: «В первый раз, я думаю, дровни в завидном соседстве с торжеством. Крестьянин торжествуя выражение неверное»
\emph{IV —XXIV — Строфы} погружая героиню романа в атмосферу фольклорности, решительно изменили характеристику ее духовного облика. 77 не сгладил этого противоречия, противопоставив демонстративному заявлению в третьей главе «она по-русски плохо знала».
\emph{ V, 5— 14 — Ее тревожили приметы... }\\
«Заяц пересечет дорогу — несчастье»\\
 «поп попадет навстречу — путь несчастлив»\\
  «кот умывается — к гостям»\\
Пушкин сам был суеверен. Рационалисты XVIII столетия относились к народным поверьям, приметам и обычаям отрицательно, видя в них результат непросвещенности и предрассудков, которые использует деспотизм, а в суеверии народа — условие похищения у него с помощью обмана и мнимых чудес исконного суверенитета.\\
Эпоха романтизма, видела в традиции вековой опыт и отражение национального склада мысли, реабилитировала народные «суеверия», увидев в них поэзию и выражение народной души.
\emph{Предрассудок! он обломок\\
Древней правды. Храм упал;\\
А руин его, потомок\\ 
Языка не разгадал\\
 (Баратынский, I, 204).\\
 Вера в приметы становится знаком близости к народному сознанию.
 П. писал: «Есть образ мыслей и чувствований, есть тьма обычаев, поверий и привычек, принадлежащих исключительно какому-нибудь народу» 
(XI, 40). Отсюда напряженный интерес к приметам, обрядам, гаданиям, которые для П., наряду с народной поэзией, характеризуют склад народной души. Поэтическая вера в приметы Татьяны отличается от суеверия Германна из «Пиковой Дамы», который, «имея мало истинной веры (...), имел множество предрассудков».  Приметы воспринимались как результат вековых наблюдений над протеканием случайных процессов.\\
Возможность повторения случайных сцеплений событий, «странных сближений» весьма занимала П. верившего в народные приметы и пытавшегося одновременно с помощью математической теории вероятности разгадать секреты случайного выпадения карты в штоссе. Вера П. в приметы соприкасалась, с одной стороны, 
с убеждением в том, что случайные события повторяются, а с другой — с сознательным стремлением усвоить черты народной психологии.
\emph{VII, 5 — Настали святки. То-то радость!}\\
Святки зимние — 25 декабря — 6 января. Зимние святки представляют собой праздник, в ходе которого совершается ряд обрядов магического свойства, имеющих целью повлиять на будущий урожай и плодородие. Последнее 
связывается с обилием детей и семейным счастьем. 
Поэтому святки — время выяснения суженых и первых шагов к заключению будущих браков. «Никогда русская жизнь не является в таком раздолье, как на 
Святках: в эти дни все русские веселятся. Всматриваясь в святочные обычаи, мы всюду видим, что наши Святки созданы для русских дев. В посиделках, гаданиях, играх, песнях все направлено к одной цели — к сближению суженых <...) Только в Святочные дни юноши и девы сидят запросто рука об руку; суженые явно гадают при своих суженых, старики весело рассказывают про старину и с молодыми сами молодеют; старушки грустно вспоминают о житье девичьем и с радостью подсказывают девушкам песни и загадки.\\
\textbf{Наша старая Русь воскресает только на Святках.}
Традиционной центральной фигурой святочного маскарада является 
медведь, что, возможно, оказало воздействие на характер сна Татьяны. 
\emph{VIII, 1 — 14 — Татьяна любопытным взором... }\\
После всех увеселений вносили стол и ставили посреди комнаты (...) Являлась почетная сваха со скатертью и накрывала стол. Старшая нянюшка приносила блюдо с водою и ставила на стол. Красные девицы, молодушки, старушки, суженые снимали с себя кольца, перстни, серьги и клали на стол, загадывая над ними «свою судьбу». Хозяйка приносила скатерть-столечник, а сваха накрывала ею блюдо. Гости усаживались. В середине садилась сваха прямо против блюда. Нянюшки клали на столечник маленькие кусочки хлеба, соль и три уголька. 
Сваха запевала первую песню: «хлеба да соли». Все сидящие гости пели под ее голос. С окончанием первой песни сваха поднимала столечник и опускала в блюдо 
хлеб, соль, угольки, а гости клали туда же вещи. Блюдо снова закрывалось. За этим начинали петь святочные подблюдные песни. Во время пения сваха разводила в блюде, а с окончанием песни трясла блюдом.
pg263
