
\documentclass{book}

\usepackage[T2A]{fontenc}    % Codifica T2A per i caratteri cirillici
\usepackage[utf8]{inputenc}  % Codifica UTF-8
\usepackage[russian, english]{babel} % Supporto per il russo e l'inglese
\DeclareUnicodeCharacter{202F}{.}%dichiara il carattere 202F spazio stretto non divisibile, i puntini in O rus capitolo 2
\DeclareUnicodeCharacter{0301}{.}%dichiara il carattere 0301
\usepackage[hidelinks]{hyperref}%rimuove i bordi intorno ai link
\usepackage{marginnote}
\title{Эвгенин Онегин}
\author{А.С. Пушкин}
\date{}

\begin{document}
\maketitle

\href{https://www.culture.ru/poems/4481/evgenii-onegin}
\textbf{Роман в стихах}\\
Не мысля гордый свет забавить,\\
Вниманье дружбы возлюбя,\\
Хотел бы я тебе представить\\
Залог достойнее тебя,\\
Достойнее души прекрасной,\\
Святой исполненной мечты,\\
Поэзии живой и ясной,\\
Высоких дум и простоты;\\
Но так и быть — рукой пристрастной\\
Прими собранье пестрых глав,\\
Полусмешных, полупечальных,\\
Простонародных, идеальных,\\
Небрежный плод моих забав,\\
Бессонниц, легких вдохновений,\\
Незрелых и увядших лет,\\
Ума холодных наблюдений\\
И сердца горестных замет.\\

% Includi i capitoli
\chapter{} % Capitolo di Introduzione
\label{cap1} % Etichetta per riferimenti incrociati

\textit{И жить торопится, и чувствовать спешит.\\
Кн. Вяземский}\\

I\\
«Мой дядя самых честных правил,\\
Когда не в шутку\emph{занемог,}\marginpar{si ammalò}\\
Он уважать себя заставил\\
И лучше выдумать не мог.\\
Его пример другим наука;\\
Но, боже мой, какая скука\\
С больным сидеть и день и ночь,\\
Не отходя ни шагу прочь!\\
Какое низкое коварство\\
Полуживого забавлять,\\
Ему подушки поправлять,\\
Печально подносить лекарство,\\
Вздыхать и думать про себя:\\
Когда же\emph{черт} возьмет тебя!»\footnote{чертикаться мог толбко данди или простой человек}\\
II\\
Так думал молодой повеса,
Летя в пыли на почтовых,
Всевышней волею Зевеса
Наследник всех своих родных.
Друзья Людмилы и Руслана!
С героем моего романа
Без предисловий, сей же час
Позвольте познакомить вас:
Онегин, добрый мой приятель,
Родился на брегах Невы,
Где, может быть, родились вы
Или блистали, мой читатель;
Там некогда гулял и я:
Но вреден север для меня.
III
Служив отлично благородно,
Долгами жил его отец,
Давал три бала ежегодно
И промотался наконец.
Судьба Евгения хранила:
Сперва Madame за ним ходила,
Потом Monsieur ее сменил.
Ребенок был резов, но мил.
Monsieur l’Abbe, француз убогой,
Чтоб не измучилось дитя,
Учил его всему шутя,
Не докучал моралью строгой,
Слегка за шалости бранил
И в Летний сад гулять водил.
IV
Когда же юности мятежной
Пришла Евгению пора,
Пора надежд и грусти нежной,
Monsieur прогнали со двора.
Вот мой Онегин на свободе;
Острижен по последней моде,
Как dandy лондонский одет —
И наконец увидел свет.
Он по-французски совершенно
Мог изъясняться и писал;
Легко мазурку танцевал
И кланялся непринужденно;
Чего ж вам больше? Свет решил,
Что он умен и очень мил.
V
Мы все учились понемногу
Чему-нибудь и как-нибудь,
Так воспитаньем, слава богу,
У нас немудрено блеснуть.
Онегин был по мненью многих
(Судей решительных и строгих)
Ученый малый, но педант:
Имел он счастливый талант
Без принужденья в разговоре
Коснуться до всего слегка,
С ученым видом знатока
Хранить молчанье в важном споре
И возбуждать улыбку дам
Огнем нежданных эпиграмм.
VI
Латынь из моды вышла ныне:
Так, если правду вам сказать,
Он знал довольно по-латыне,
Чтоб эпиграфы разбирать,
Потолковать об Ювенале,
В конце письма поставить vale,
Да помнил, хоть не без греха,
Из Энеиды два стиха.
Он рыться не имел охоты
В хронологической пыли
Бытописания земли:
Но дней минувших анекдоты
От Ромула до наших дней
Хранил он в памяти своей.
VII
Высокой страсти не имея
Для звуков жизни не щадить,
Не мог он ямба от хорея,
Как мы ни бились, отличить.
Бранил Гомера, Феокрита;
Зато читал Адама Смита
И был глубокой эконом,
То есть умел судить о том,
Как государство богатеет,
И чем живет, и почему
Не нужно золота ему,
Когда простой продукт имеет.
Отец понять его не мог
И земли отдавал в залог.
VIII
Всего, что знал еще Евгений,
Пересказать мне недосуг;
Но в чем он истинный был гений,
Что знал он тверже всех наук,
Что было для него измлада
И труд, и мука, и отрада,
Что занимало целый день
Его тоскующую лень, —
Была наука страсти нежной,
Которую воспел Назон,
За что страдальцем кончил он
Свой век блестящий и мятежный
В Молдавии, в глуши степей,
Вдали Италии своей.
IX
. . . . . . . . . . . . . . . . . .
. . . . . . . . . . . . . . . . . .
. . . . . . . . . . . . . . . . . .
X
Как рано мог он лицемерить,
Таить надежду, ревновать,
Разуверять, заставить верить,
Казаться мрачным, изнывать,
Являться гордым и послушным,
Внимательным иль равнодушным!
Как томно был он молчалив,
Как пламенно красноречив,
В сердечных письмах как небрежен!
Одним дыша, одно любя,
Как он умел забыть себя!
Как взор его был быстр и нежен,
Стыдлив и дерзок, а порой
Блистал послушною слезой!
XI
Как он умел казаться новым,
Шутя невинность изумлять,
Пугать отчаяньем готовым,
Приятной лестью забавлять,
Ловить минуту умиленья,
Невинных лет предубежденья
Умом и страстью побеждать,
Невольной ласки ожидать,
Молить и требовать признанья,
Подслушать сердца первый звук,
Преследовать любовь, и вдруг
Добиться тайного свиданья…
И после ей наедине
Давать уроки в тишине!
XII
Как рано мог уж он тревожить
Сердца кокеток записных!
Когда ж хотелось уничтожить
Ему соперников своих,
Как он язвительно злословил!
Какие сети им готовил!
Но вы, блаженные мужья,
С ним оставались вы друзья:
Его ласкал супруг лукавый,
Фобласа давний ученик,
И недоверчивый старик,
И рогоносец величавый,
Всегда довольный сам собой,
Своим обедом и женой.
XIII, XIV
. . . . . . . . . . . . . . . . . .
. . . . . . . . . . . . . . . . . .
. . . . . . . . . . . . . . . . . .
XV
Бывало, он еще в постеле:
К нему записочки несут.
Что? Приглашенья? В самом деле,
Три дома на вечер зовут:
Там будет бал, там детский праздник.
Куда ж поскачет мой проказник?
С кого начнет он? Все равно:
Везде поспеть немудрено.
Покамест в утреннем уборе,
Надев широкий боливар,
Онегин едет на бульвар
И там гуляет на просторе,
Пока недремлющий брегет
Не прозвонит ему обед.
XVI
Уж темно: в санки он садится.
«Пади, пади!» — раздался крик;
Морозной пылью серебрится
Его бобровый воротник.
К Talon помчался: он уверен,
Что там уж ждет его Каверин.
Вошел: и пробка в потолок,
Вина кометы брызнул ток;
Пред ним roast-beef окровавленный,
И трюфли, роскошь юных лет,
Французской кухни лучший цвет,
И Страсбурга пирог нетленный
Меж сыром лимбургским живым
И ананасом золотым.
XVII
Еще бокалов жажда просит
Залить горячий жир котлет,
Но звон брегета им доносит,
Что новый начался балет.
Театра злой законодатель,
Непостоянный обожатель
Очаровательных актрис,
Почетный гражданин кулис,
Онегин полетел к театру,
Где каждый, вольностью дыша,
Готов охлопать entrechat,
Обшикать Федру, Клеопатру,
Моину вызвать (для того,
Чтоб только слышали его).
XVIII
Волшебный край! там в стары годы,
Сатиры смелый властелин,
Блистал Фонвизин, друг свободы,
И переимчивый Княжнин;
Там Озеров невольны дани
Народных слез, рукоплесканий
С младой Семеновой делил;
Там наш Катенин воскресил
Корнеля гений величавый;
Там вывел колкий Шаховской
Своих комедий шумный рой,
Там и Дидло венчался славой,
Там, там под сению кулис
Младые дни мои неслись.
XIX
Мои богини! что вы? где вы?
Внемлите мой печальный глас:
Все те же ль вы? другие ль девы,
Сменив, не заменили вас?
Услышу ль вновь я ваши хоры?
Узрю ли русской Терпсихоры
Душой исполненный полет?
Иль взор унылый не найдет
Знакомых лиц на сцене скучной,
И, устремив на чуждый свет
Разочарованный лорнет,
Веселья зритель равнодушный,
Безмолвно буду я зевать
И о былом воспоминать?
XX
Театр уж полон; ложи блещут;
Партер и кресла — все кипит;
В райке нетерпеливо плещут,
И, взвившись, занавес шумит.
Блистательна, полувоздушна,
Смычку волшебному послушна,
Толпою нимф окружена,
Стоит Истомина; она,
Одной ногой касаясь пола,
Другою медленно кружит,
И вдруг прыжок, и вдруг летит,
Летит, как пух от уст Эола;
То стан совьет, то разовьет
И быстрой ножкой ножку бьет.
XXI
Все хлопает. Онегин входит,
Идет меж кресел по ногам,
Двойной лорнет скосясь наводит
На ложи незнакомых дам;
Все ярусы окинул взором,
Все видел: лицами, убором
Ужасно недоволен он;
С мужчинами со всех сторон
Раскланялся, потом на сцену
В большом рассеянье взглянул,
Отворотился — и зевнул,
И молвил: «Всех пора на смену;
Балеты долго я терпел,
Но и Дидло мне надоел».
XXII
Еще амуры, черти, змеи
На сцене скачут и шумят;
Еще усталые лакеи
На шубах у подъезда спят;
Еще не перестали топать,
Сморкаться, кашлять, шикать, хлопать;
Еще снаружи и внутри
Везде блистают фонари;
Еще, прозябнув, бьются кони,
Наскуча упряжью своей,
И кучера, вокруг огней,
Бранят господ и бьют в ладони —
А уж Онегин вышел вон;
Домой одеться едет он.
XXIII
Изображу ль в картине верной
Уединенный кабинет,
Где мод воспитанник примерный
Одет, раздет и вновь одет?
Все, чем для прихоти обильной
Торгует Лондон щепетильный
И по Балтическим волнам
За лес и сало возит нам,
Все, что в Париже вкус голодный,
Полезный промысел избрав,
Изобретает для забав,
Для роскоши, для неги модной, —
Все украшало кабинет
Философа в осьмнадцать лет.
XXIV
Янтарь на трубках Цареграда,
Фарфор и бронза на столе,
И, чувств изнеженных отрада,
Духи в граненом хрустале;
Гребенки, пилочки стальные,
Прямые ножницы, кривые
И щетки тридцати родов
И для ногтей и для зубов.
Руссо (замечу мимоходом)
Не мог понять, как важный Грим
Смел чистить ногти перед ним,
Красноречивым сумасбродом.
Защитник вольности и прав
В сем случае совсем неправ.
XXV
Быть можно дельным человеком
И думать о красе ногтей:
К чему бесплодно спорить с веком?
Обычай деспот меж людей.
Второй Чадаев, мой Евгений,
Боясь ревнивых осуждений,
В своей одежде был педант
И то, что мы назвали франт.
Он три часа по крайней мере
Пред зеркалами проводил
И из уборной выходил
Подобный ветреной Венере,
Когда, надев мужской наряд,
Богиня едет в маскарад.
XXVI
В последнем вкусе туалетом
Заняв ваш любопытный взгляд,
Я мог бы пред ученым светом
Здесь описать его наряд;
Конечно б это было смело,
Описывать мое же дело:
Но панталоны, фрак, жилет,
Всех этих слов на русском нет;
А вижу я, винюсь пред вами,
Что уж и так мой бедный слог
Пестреть гораздо б меньше мог
Иноплеменными словами,
Хоть и заглядывал я встарь
В Академический словарь.
XXVII
У нас теперь не то в предмете:
Мы лучше поспешим на бал,
Куда стремглав в ямской карете
Уж мой Онегин поскакал.
Перед померкшими домами
Вдоль сонной улицы рядами
Двойные фонари карет
Веселый изливают свет
И радуги на снег наводят;
Усеян плошками кругом,
Блестит великолепный дом;
По цельным окнам тени ходят,
Мелькают профили голов
И дам и модных чудаков.
XXVIII
Вот наш герой подъехал к сеням;
Швейцара мимо он стрелой
Взлетел по мраморным ступеням,
Расправил волоса рукой,
Вошел. Полна народу зала;
Музыка уж греметь устала;
Толпа мазуркой занята;
Кругом и шум и теснота;
Бренчат кавалергарда шпоры;
Летают ножки милых дам;
По их пленительным следам
Летают пламенные взоры,
И ревом скрыпок заглушен
Ревнивый шепот модных жен.
XXIX
Во дни веселий и желаний
Я был от балов без ума:
Верней нет места для признаний
И для вручения письма.
О вы, почтенные супруги!
Вам предложу свои услуги;
Прошу мою заметить речь:
Я вас хочу предостеречь.
Вы также, маменьки, построже
За дочерьми смотрите вслед:
Держите прямо свой лорнет!
Не то… не то, избави боже!
Я это потому пишу,
Что уж давно я не грешу.
XXX
Увы, на разные забавы
Я много жизни погубил!
Но если б не страдали нравы,
Я балы б до сих пор любил.
Люблю я бешеную младость,
И тесноту, и блеск, и радость,
И дам обдуманный наряд;
Люблю их ножки; только вряд
Найдете вы в России целой
Три пары стройных женских ног.
Ах! долго я забыть не мог
Две ножки… Грустный, охладелый,
Я все их помню, и во сне
Они тревожат сердце мне.
XXXI
Когда ж и где, в какой пустыне,
Безумец, их забудешь ты?
Ах, ножки, ножки! где вы ныне?
Где мнете вешние цветы?
Взлелеяны в восточной неге,
На северном, печальном снеге
Вы не оставили следов:
Любили мягких вы ковров
Роскошное прикосновенье.
Давно ль для вас я забывал
И жажду славы и похвал,
И край отцов, и заточенье?
Исчезло счастье юных лет,
Как на лугах ваш легкий след.
XXXII
Дианы грудь, ланиты Флоры
Прелестны, милые друзья!
Однако ножка Терпсихоры
Прелестней чем-то для меня.
Она, пророчествуя взгляду
Неоцененную награду,
Влечет условною красой
Желаний своевольный рой.
Люблю ее, мой друг Эльвина,
Под длинной скатертью столов,
Весной на мураве лугов,
Зимой на чугуне камина,
На зеркальном паркете зал,
У моря на граните скал.
XXXIII
Я помню море пред грозою:
Как я завидовал волнам,
Бегущим бурной чередою
С любовью лечь к ее ногам!
Как я желал тогда с волнами
Коснуться милых ног устами!
Нет, никогда средь пылких дней
Кипящей младости моей
Я не желал с таким мученьем
Лобзать уста младых Армид,
Иль розы пламенных ланит,
Иль перси, полные томленьем;
Нет, никогда порыв страстей
Так не терзал души моей!
XXXIV
Мне памятно другое время!
В заветных иногда мечтах
Держу я счастливое стремя…
И ножку чувствую в руках;
Опять кипит воображенье,
Опять ее прикосновенье
Зажгло в увядшем сердце кровь,
Опять тоска, опять любовь!..
Но полно прославлять надменных
Болтливой лирою своей;
Они не стоят ни страстей,
Ни песен, ими вдохновенных:
Слова и взор волшебниц сих
Обманчивы… как ножки их.
XXXV
Что ж мой Онегин? Полусонный
В постелю с бала едет он:
А Петербург неугомонный
Уж барабаном пробужден.
Встает купец, идет разносчик,
На биржу тянется извозчик,
С кувшином охтенка спешит,
Под ней снег утренний хрустит.
Проснулся утра шум приятный.
Открыты ставни; трубный дым
Столбом восходит голубым,
И хлебник, немец аккуратный,
В бумажном колпаке, не раз
Уж отворял свой васисдас.
XXXVI
Но, шумом бала утомленный
И утро в полночь обратя,
Спокойно спит в тени блаженной
Забав и роскоши дитя.
Проснется за полдень, и снова
До утра жизнь его готова,
Однообразна и пестра.
И завтра то же, что вчера.
Но был ли счастлив мой Евгений,
Свободный, в цвете лучших лет,
Среди блистательных побед,
Среди вседневных наслаждений?
Вотще ли был он средь пиров
Неосторожен и здоров?
XXXVII
Нет: рано чувства в нем остыли;
Ему наскучил света шум;
Красавицы не долго были
Предмет его привычных дум;
Измены утомить успели;
Друзья и дружба надоели,
Затем, что не всегда же мог
Beef-stеаks и страсбургский пирог
Шампанской обливать бутылкой
И сыпать острые слова,
Когда болела голова;
И хоть он был повеса пылкой,
Но разлюбил он наконец
И брань, и саблю, и свинец.
XXXVIII
Недуг, которого причину
Давно бы отыскать пора,
Подобный английскому сплину,
Короче: русская хандра
Им овладела понемногу;
Он застрелиться, слава богу,
Попробовать не захотел,
Но к жизни вовсе охладел.
Как Child-Harold, угрюмый, томный
В гостиных появлялся он;
Ни сплетни света, ни бостон,
Ни милый взгляд, ни вздох нескромный,
Ничто не трогало его,
Не замечал он ничего.
XXXIX, ХL, ХLI
. . . . . . . . . . . . . . . . . .
. . . . . . . . . . . . . . . . . .
. . . . . . . . . . . . . . . . . .
ХLII
Причудницы большого света!
Всех прежде вас оставил он;
И правда то, что в наши лета
Довольно скучен высший тон;
Хоть, может быть, иная дама
Толкует Сея и Бентама,
Но вообще их разговор
Несносный, хоть невинный вздор;
К тому ж они так непорочны,
Так величавы, так умны,
Так благочестия полны,
Так осмотрительны, так точны,
Так неприступны для мужчин,
Что вид их уж рождает сплин.
XLIII
И вы, красотки молодые,
Которых позднею порой
Уносят дрожки удалые
По петербургской мостовой,
И вас покинул мой Евгений.
Отступник бурных наслаждений,
Онегин дома заперся,
Зевая, за перо взялся,
Хотел писать — но труд упорный
Ему был тошен; ничего
Не вышло из пера его,
И не попал он в цех задорный
Людей, о коих не сужу,
Затем, что к ним принадлежу.
ХLIV
И снова, преданный безделью,
Томясь душевной пустотой,
Уселся он — с похвальной целью
Себе присвоить ум чужой;
Отрядом книг уставил полку,
Читал, читал, а все без толку:
Там скука, там обман иль бред;
В том совести, в том смысла нет;
На всех различные вериги;
И устарела старина,
И старым бредит новизна.
Как женщин, он оставил книги,
И полку, с пыльной их семьей,
Задернул траурной тафтой.
ХLV
Условий света свергнув бремя,
Как он, отстав от суеты,
С ним подружился я в то время.
Мне нравились его черты,
Мечтам невольная преданность,
Неподражательная странность
И резкий, охлажденный ум.
Я был озлоблен, он угрюм;
Страстей игру мы знали оба;
Томила жизнь обоих нас;
В обоих сердца жар угас;
Обоих ожидала злоба
Слепой Фортуны и людей
На самом утре наших дней.
XLVI
Кто жил и мыслил, тот не может
В душе не презирать людей;
Кто чувствовал, того тревожит
Призрак невозвратимых дней:
Тому уж нет очарований,
Того змия воспоминаний,
Того раскаянье грызет.
Все это часто придает
Большую прелесть разговору.
Сперва Онегина язык
Меня смущал; но я привык
К его язвительному спору,
И к шутке, с желчью пополам,
И злости мрачных эпиграмм.
XLVII
Как часто летнею порою,
Когда прозрачно и светло
Ночное небо над Невою
И вод веселое стекло
Не отражает лик Дианы,
Воспомня прежних лет романы,
Воспомня прежнюю любовь,
Чувствительны, беспечны вновь,
Дыханьем ночи благосклонной
Безмолвно упивались мы!
Как в лес зеленый из тюрьмы
Перенесен колодник сонный,
Так уносились мы мечтой
К началу жизни молодой.
XLVIII
С душою, полной сожалений,
И опершися на гранит,
Стоял задумчиво Евгений,
Как описал себя пиит.
Все было тихо; лишь ночные
Перекликались часовые,
Да дрожек отдаленный стук
С Мильонной раздавался вдруг;
Лишь лодка, веслами махая,
Плыла по дремлющей реке:
И нас пленяли вдалеке
Рожок и песня удалая…
Но слаще, средь ночных забав,
Напев Торкватовых октав!
XLIX
Адриатические волны,
О Брента! нет, увижу вас
И, вдохновенья снова полный,
Услышу ваш волшебный глас!
Он свят для внуков Аполлона;
По гордой лире Альбиона
Он мне знаком, он мне родной.
Ночей Италии златой
Я негой наслажусь на воле,
С венецианкою младой,
То говорливой, то немой,
Плывя в таинственной гондоле;
С ней обретут уста мои
Язык Петрарки и любви.
L
Придет ли час моей свободы?
Пора, пора! — взываю к ней;
Брожу над морем, жду погоды,
Маню ветрила кораблей.
Под ризой бурь, с волнами споря,
По вольному распутью моря
Когда ж начну я вольный бег?
Пора покинуть скучный брег
Мне неприязненной стихии
И средь полуденных зыбей,
Под небом Африки моей,
Вздыхать о сумрачной России,
Где я страдал, где я любил,
Где сердце я похоронил.
LI
Онегин был готов со мною
Увидеть чуждые страны;
Но скоро были мы судьбою
На долгой срок разведены.
Отец его тогда скончался.
Перед Онегиным собрался
Заимодавцев жадный полк.
У каждого свой ум и толк:
Евгений, тяжбы ненавидя,
Довольный жребием своим,
Наследство предоставил им,
Большой потери в том не видя
Иль предузнав издалека
Кончину дяди старика.
LII
Вдруг получил он в самом деле
От управителя доклад,
Что дядя при смерти в постеле
И с ним проститься был бы рад.
Прочтя печальное посланье,
Евгений тотчас на свиданье
Стремглав по почте поскакал
И уж заранее зевал,
Приготовляясь, денег ради,
На вздохи, скуку и обман
(И тем я начал мой роман);
Но, прилетев в деревню дяди,
Его нашел уж на столе,
Как дань готовую земле.
LIII
Нашел он полон двор услуги;
К покойнику со всех сторон
Съезжались недруги и други,
Охотники до похорон.
Покойника похоронили.
Попы и гости ели, пили
И после важно разошлись,
Как будто делом занялись.
Вот наш Онегин — сельский житель,
Заводов, вод, лесов, земель
Хозяин полный, а досель
Порядка враг и расточитель,
И очень рад, что прежний путь
Переменил на что-нибудь.
LIV
Два дня ему казались новы
Уединенные поля,
Прохлада сумрачной дубровы,
Журчанье тихого ручья;
На третий роща, холм и поле
Его не занимали боле;
Потом уж наводили сон;
Потом увидел ясно он,
Что и в деревне скука та же,
Хоть нет ни улиц, ни дворцов,
Ни карт, ни балов, ни стихов.
Хандра ждала его на страже,
И бегала за ним она,
Как тень иль верная жена.
LV
Я был рожден для жизни мирной,
Для деревенской тишины;
В глуши звучнее голос лирный,
Живее творческие сны.
Досугам посвятясь невинным,
Брожу над озером пустынным,
И far nientе мой закон.
Я каждым утром пробужден
Для сладкой неги и свободы:
Читаю мало, долго сплю,
Летучей славы не ловлю.
Не так ли я в былые годы
Провел в бездействии, в тени
Мои счастливейшие дни?
LVI
Цветы, любовь, деревня, праздность,
Поля! я предан вам душой.
Всегда я рад заметить разность
Между Онегиным и мной,
Чтобы насмешливый читатель
Или какой-нибудь издатель
Замысловатой клеветы,
Сличая здесь мои черты,
Не повторял потом безбожно,
Что намарал я свой портрет,
Как Байрон, гордости поэт,
Как будто нам уж невозможно
Писать поэмы о другом,
Как только о себе самом.
LVII
Замечу кстати: все поэты —
Любви мечтательной друзья.
Бывало, милые предметы
Мне снились, и душа моя
Их образ тайный сохранила;
Их после муза оживила:
Так я, беспечен, воспевал
И деву гор, мой идеал,
И пленниц берегов Салгира.
Теперь от вас, мои друзья,
Вопрос нередко слышу я:
«О ком твоя вздыхает лира?
Кому, в толпе ревнивых дев,
Ты посвятил ее напев?
LVIII
Чей взор, волнуя вдохновенье,
Умильной лаской наградил
Твое задумчивое пенье?
Кого твой стих боготворил?»
И, други, никого, ей-богу!
Любви безумную тревогу
Я безотрадно испытал.
Блажен, кто с нею сочетал
Горячку рифм: он тем удвоил
Поэзии священный бред,
Петрарке шествуя вослед,
А муки сердца успокоил,
Поймал и славу между тем;
Но я, любя, был глуп и нем.
LIX
Прошла любовь, явилась муза,
И прояснился темный ум.
Свободен, вновь ищу союза
Волшебных звуков, чувств и дум;
Пишу, и сердце не тоскует,
Перо, забывшись, не рисует,
Близ неоконченных стихов,
Ни женских ножек, ни голов;
Погасший пепел уж не вспыхнет,
Я все грущу; но слез уж нет,
И скоро, скоро бури след
В душе моей совсем утихнет:
Тогда-то я начну писать
Поэму песен в двадцать пять.
LX
Я думал уж о форме плана
И как героя назову;
Покамест моего романа
Я кончил первую главу;
Пересмотрел все это строго:
Противоречий очень много,
Но их исправить не хочу.
Цензуре долг свой заплачу
И журналистам на съеденье
Плоды трудов моих отдам:
Иди же к невским берегам,
Новорожденное творенье,
И заслужи мне славы дань:
Кривые толки, шум и брань!
\chapter{} % Capitolo di Metodologia
\label{cap2} % Etichetta per riferimenti incrociati

\textit{O rus!..\\
Hor.\\ 
(О Русь!)}\\

I

Деревня, где скучал Евгений,\\
Была прелестный уголок;\\
Там друг невинных наслаждений\\
Благословить бы небо мог.\\
Господский дом уединенный,\\
Горой от ветров огражденный,\\
Стоял над речкою. Вдали\\
Пред ним пестрели и цвели\\
Луга и нивы золотые,\\
Мелькали селы; здесь и там\\
Стада бродили по лугам,\\
И сени расширял густые\\
Огромный, запущенный сад,\\
Приют задумчивых \emph{дриад.}\footnote{в греческой мифологии нимфы, покровительницы деревьев}

II

Почтенный замок был построен,\\
Как замки строиться должны:\\
\emph{Отменно}\footnote{Оценочная характеристика чего-либо как очень хорошего, превосходного.} прочен и спокоен\\
Во вкусе умной старины.\\
Везде высокие покои,\\
В гостиной \emph{штофные}\marginpar{\textit{Damasco}} обои,\\
Царей портреты на стенах,\\
И печи в пестрых изразцах.\\
Все это ныне \emph{обветшало,}\marginpar{fatiscente}\\
Не знаю, право, почему;\\
Да, впрочем, другу моему\\
В том нужды было очень мало,\\
Затем, что он равно зевал\\
Средь модных и старинных зал.\\

III

Он в том покое поселился,\\
Где деревенский старожил\\
Лет сорок с ключницей бранился,\\
В окно смотрел и мух давил.\\
Все было просто: пол дубовый,\\
Два шкафа, стол, диван пуховый,\\
Нигде ни пятнышка чернил.\\
Онегин шкафы отворил;\\
В одном нашел тетрадь расхода,\\
В другом наливок целый строй,\\
Кувшины с яблочной водой\\
И календарь осьмого года:\\
Старик, имея много дел,\\
В иные книги не глядел.\\

IV

Один среди своих владений,\\
\textit{Чтоб только время проводить,\\
Сперва задумал наш Евгений\\
Порядок новый учредить.\\
В своей глуши мудрец пустынный,\\
Ярем он барщины старинной\\
Оброком легким заменил;\\
И раб судьбу благословил.}\footnote{При барщине крестьяне бесплатно работали на земле своего хозяина-крепостника, в лучшем случае им давали месячину - продовольственный паёк. Производительность труда, сами понимаете, крайне невелика, невелики и доходы. Онегин разделил свои владения между крестьянами, которые теперь сами обрабатывали землю, крутились как могли, и платили своему хозяину оброк. На эту же тему см. тургеневский "Хорь и Калиныч" из "Записок охотника". Оброк, кстати, тоже большого дохода не приносил - в среднем по 5 рублей с крестьянского двора в год. 
Сроки барщины были неопределёнными. Да, император Павел Первый в 1797 г. попытался запретить крестьянам работать в праздники и ограничил барщину 3 днями в неделю, но фактически указ не выполнялся. По закону крестьянин должен был обрабатывать столько же помещичьей земли, сколько ему выделялось на личное хозяйство. Точнее, помещик выделял землю крестьянской общине, а делили её между собой уже на сходе. }\\
Зато в углу своем надулся,\\
Увидя в этом страшный вред,\\\
Его расчетливый сосед;\\
Другой лукаво улыбнулся,\\
И в голос все решили так,\\
Что он опаснейший чудак.\\

V

Сначала все к нему езжали;\\
Но так как с заднего крыльца\\
Обыкновенно подавали\\
Ему донского жеребца,\\
Лишь только вдоль большой дороги\\
Заслышат их домашни дроги, —\\
Поступком оскорбясь таким,\\
Все дружбу прекратили с ним.\footnote{Вначале, сразу после того, как Онегин поселился в поместье умершего дяди, все соседи стали ездить к нему в гости с визитами - познакомиться или поддержать знакомство. Так у них было принято. 
Но Онегин ни с кем из сельских соседей дружить не хотел. И даже просто поддерживать какие-либо отношения тоже не хотел. Поэтому, когда с дороги, ведущей к его дому, слышался звук приближающегося экипажа (домашних дрог, дрожек, то есть, коляски, сделанной местным крепостным умельцем), Онегин требовал верхового коня и уезжал из дома на прогулку. Причем уезжал с заднего крыльца, чтобы попасть на другую дорогу, на которой он не повстречался бы с непрошеными гостями.
Соседи поняли все правильно, обиделись и прекратили дружбу с ним.}\\
«Сосед наш неуч; сумасбродит;\\
Он фармазон; он пьет одно\\
Стаканом красное вино;\\
Он дамам к ручке не подходит;\\
Все да да нет; не скажет да-с\\
Иль нет-с». Таков был общий глас.\\

VI\footnote{характер Ленского. Изъять из Быта}

В свою деревню в ту же пору\\
Помещик новый прискакал\\
И столь же строгому разбору\\
В соседстве повод подавал:\\
По имени \emph{Владимир Ленской,}\\
С душою прямо геттингенской,\\
Красавец, в полном цвете лет,\\
Поклонник Канта и поэт.\\
Он из Германии туманной\\
Привез учености плоды:\\
Вольнолюбивые мечты,\\
Дух пылкий\marginpar{Сильно чувствующий} и довольно странный,\\
Всегда восторженную\marginpar{легко приходит в состояние восторга...} речь\\
И кудри черные до плеч.\\

VII

От хладного разврата света\\
Еще увянуть не успев,\\
Его душа была согрета\\
Приветом друга, лаской дев;\\
Он сердцем милый был невежда,\\
Его лелеяла надежда,\\
И мира новый блеск и шум\\
Еще пленяли юный ум.\\
Он забавлял мечтою сладкой\\
Сомненья сердца своего;\\
Цель жизни нашей для него\\
Была заманчивой загадкой,\\
Над ней он голову ломал\\
И чудеса подозревал.\\

VIII

Он верил, что душа родная\\
Соединиться с ним должна,\\
Что, безотрадно изнывая,\\
Его вседневно ждет она;\\
Он верил, что друзья готовы\\
За честь его приять оковы\\
И что не дрогнет их рука\\
Разбить сосуд клеветника;\\
Что есть избранные судьбами,\\
Людей священные друзья;\\
Что их бессмертная семья\\
Неотразимыми лучами\\
Когда-нибудь нас озарит\\
И мир блаженством одарит.\\

IX

Негодованье, сожаленье,\\
Ко благу чистая любовь\\
И славы сладкое мученье\\
В нем рано волновали кровь.\\
Он с лирой странствовал на свете;\\
Под небом Шиллера и Гете\\
Их поэтическим огнем\\
Душа воспламенилась в нем;\\
И муз возвышенных искусства,\\
Счастливец, он не постыдил:\\
Он в песнях гордо сохранил\\
Всегда возвышенные чувства,\\
Порывы девственной мечты\\
И прелесть важной простоты.\\

X

Он пел любовь, любви послушный,\\
И песнь его была ясна,\\
Как мысли девы простодушной,\\
Как сон младенца, как луна\\
В пустынях неба безмятежных,\\
Богиня тайн и вздохов нежных.\\
Он пел разлуку и печаль,\\
И нечто, и туманну даль,\\
И романтические розы;\\
Он пел те дальные страны,\\
Где долго в лоно тишины\\
Лились его живые слезы;\\
Он пел поблеклый жизни цвет\\
Без малого в осьмнадцать лет.\\

XI

В пустыне, где один Евгений\\
Мог оценить его дары,\\
Господ соседственных селений\\
Ему не нравились пиры;\\
Бежал он их беседы шумной.\\
Их разговор благоразумный\\
О сенокосе, о вине,\\
О псарне, о своей родне,\\
Конечно, не блистал ни чувством,\\
Ни поэтическим огнем,\\
Ни остротою, ни умом,\\
Ни общежития искусством;\\
Но разговор их милых жен\\
Гораздо меньше был умен.\\

XII

Богат, хорош собою, Ленский\\
Везде был принят как жених;\\
Таков обычай деревенский;\\
Все дочек прочили своих\\
За полурусского соседа;\\
Взойдет ли он, тотчас беседа\\
Заводит слово стороной\\
О скуке жизни холостой;\\
Зовут соседа к самовару,\\
А Дуня разливает чай;\\
Ей шепчут: «Дуня, примечай!»\marginpar{за кем-чем. Наблюдать, следить}\\
Потом приносят и гитару:\\
И запищит она (бог мой!):\\
Приди в чертог ко мне златой!..\\

XIII

Но Ленский, не имев, конечно,\\
Охоты узы\marginpar{vincoli} брака несть,\\
С Онегиным желал сердечно\\
Знакомство покороче свесть.\\
Они сошлись. Волна и камень,\\
Стихи и проза, лед и пламень\\
Не столь различны меж собой.\\
Сперва взаимной разнотой\\
Они друг другу были скучны;\\
Потом понравились; потом\\
Съезжались каждый день верхом\\
И скоро стали неразлучны.\\
Так люди (первый каюсь я)\\
От делать нечего друзья.\\

XIV

Но дружбы нет и той меж нами.\\
Все предрассудки истребя,\\
Мы почитаем всех нулями,\\
А единицами — себя.\\
Мы все глядим в Наполеоны;\\
Двуногих тварей миллионы\\
Для нас орудие одно;\\
Нам чувство дико и смешно.\\
Сноснее многих был Евгений;\\
Хоть он людей, конечно, знал\\
И вообще их презирал, —\\
Но (правил нет без исключений)\\
Иных он очень отличал\\
И вчуже чувство уважал.\\

XV

\emph{Он слушал Ленского с улыбкой.\\
Поэта пылкий разговор,\\
И ум, еще в сужденьях зыбкой,\\
И вечно вдохновенный взор, —\\
Онегину все было ново;\\
Он охладительное слово\\
В устах старался удержать\\
И думал: глупо мне мешать\\
Его минутному блаженству;\\
И без меня пора придет;\\
Пускай покамест он живет\\
Да верит мира совершенству;\\
Простим горячке юных лет\\
И юный жар и юный бред.}\\

XVI

Меж ими все рождало споры\\
И к размышлению влекло:\\
Племен минувших договоры,\\
Плоды наук, добро и зло,\\
И предрассудки вековые,\\
И гроба тайны роковые,\\
Судьба и жизнь в свою чреду,\\
Все подвергалось их суду.\\
Поэт в жару своих суждений\\
Читал, забывшись, между тем\\
Отрывки северных поэм,\\
И снисходительный Евгений,\\
Хоть их не много понимал,\\
Прилежно юноше внимал.\\

XVII

Но чаще занимали страсти\\
Умы пустынников моих.\\
Ушед от их мятежной власти,\\
Онегин говорил об них\\
С невольным вздохом сожаленья:\\
Блажен, кто ведал их волненья\\
И наконец от них отстал;\\
Блаженней тот, кто их не знал,\\
Кто охлаждал любовь — разлукой,\\
Вражду — злословием; порой\\
Зевал с друзьями и с женой,\\
Ревнивой не тревожась мукой,\\
И дедов верный капитал\\
Коварной двойке не вверял.\\

XVIII

Когда прибегнем мы под знамя\\
Благоразумной тишины,\\
Когда страстей угаснет пламя,\\
И нам становятся смешны\\
Их своевольство иль порывы\\
И запоздалые отзывы, —\\
Смиренные не без труда,\\
Мы любим слушать иногда\\
Страстей чужих язык мятежный,\\
И нам он сердце шевелит.\\
Так точно старый инвалид\\
Охотно клонит слух прилежный\\
Рассказам юных усачей,\\
Забытый в хижине своей.\\

XIX

Зато и пламенная младость\\
Не может ничего скрывать.\\
Вражду, любовь, печаль и радость\\
Она готова разболтать.\\
В любви считаясь инвалидом,\\
Онегин слушал с важным видом,\\
Как, сердца исповедь любя,\\
Поэт высказывал себя;\\
Свою доверчивую совесть\\
Он простодушно обнажал.\\
Евгений без труда узнал\\
Его любви младую повесть,\\
Обильный чувствами рассказ,\\
Давно не новыми для нас.\\

XX

Ах, он любил, как в наши лета\\
Уже не любят; как одна\\
Безумная душа поэта\\
Еще любить осуждена:\\
Всегда, везде одно мечтанье,\\
Одно привычное желанье,\\
Одна привычная печаль.\\
Ни охлаждающая даль,\\
Ни долгие лета разлуки,\\
Ни музам данные часы,\\
Ни чужеземные красы,\\
Ни шум веселий, ни науки\\
Души не изменили в нем,\\
Согретой девственным огнем.\\

XXI

Чуть отрок, Ольгою плененный,\\
Сердечных мук еще не знав,\\
Он был свидетель умиленный\\
Ее младенческих забав;\\
В тени хранительной дубравы\\
Он разделял ее забавы,\\
И детям прочили венцы\\
Друзья-соседы, их отцы.\\
В глуши, под сению смиренной,\\
Невинной прелести полна,\\
В глазах родителей, она\\
Цвела, как ландыш потаенный,\\
Незнаемый в траве глухой\\
Ни мотыльками, ни пчелой.\\

XXII

Она поэту подарила\\
Младых восторгов первый сон,\\
И мысль об ней одушевила\\
Его цевницы первый стон.\\
Простите, игры золотые!\\
Он рощи полюбил густые,\\
Уединенье, тишину,\\
И ночь, и звезды, и луну,\\
Луну, небесную лампаду,\\
Которой посвящали мы\\
Прогулки средь вечерней тьмы,\\
И слезы, тайных мук отраду…\\
Но нынче видим только в ней\\
Замену тусклых фонарей.\\

XXIII

Всегда скромна, всегда послушна,\\
Всегда как утро весела,\\
Как жизнь поэта простодушна,\\
Как поцелуй любви мила;\\
Глаза, как небо, голубые,\\
Улыбка, локоны льняные,\\
Движенья, голос, легкий стан,\\
Все в Ольге… но любой роман\\
Возьмите и найдете верно\\
Ее портрет: он очень мил,\\
Я прежде сам его любил,\\
Но надоел он мне безмерно.\\
Позвольте мне, читатель мой,\\
Заняться старшею сестрой.\\

XXIV

Ее сестра звалась Татьяна…\\
Впервые именем таким\\
Страницы нежные романа\\
Мы своевольно освятим.\\
И что ж? оно приятно, звучно;\\
Но с ним, я знаю, неразлучно\\
Воспоминанье старины\\
Иль девичьей! Мы все должны\\
Признаться: вкусу очень мало\\
У нас и в наших именах\\
(Не говорим уж о стихах);\\
Нам просвещенье не пристало,\\
И нам досталось от него\\
Жеманство, — больше ничего.\\

XXV

Итак, она звалась Татьяной.\\
Ни красотой сестры своей,\\
Ни свежестью ее румяной\\
Не привлекла б она очей.\\
Дика, печальна, молчалива,\\
\emph{Как лань лесная боязлива,\\
Она в семье своей родной}\\
Казалась девочкой чужой.\\
Она ласкаться не умела\\
К отцу, ни к матери своей;\\
Дитя сама, в толпе детей\\
Играть и прыгать не хотела\\
И часто целый день одна\\
Сидела молча у окна.\\

XXVI\footnote{[задумчивый мичтательный характер Татьяны}

Задумчивость, ее подруга\\
От самых колыбельных дней,\\
Теченье сельского досуга\\
Мечтами украшала ей.\\
Ее изнеженные пальцы\\
Не знали игл; склонясь на пяльцы,\\
Узором шелковым она\
Не оживляла полотна.\\
Охоты властвовать примета,\\
С послушной куклою дитя\\
Приготовляется шутя\\
К приличию — закону света,\\
И важно повторяет ей\\
Уроки маменьки своей.\\

XXVII

Но куклы даже в эти годы\\
Татьяна в руки не брала;\\
Про вести города, про моды\\
Беседы с нею не вела.\\
И были детские проказы\\
Ей чужды: страшные рассказы\\
Зимою в темноте ночей\\
Пленяли больше сердце ей.\\
Когда же няня собирала\\
Для Ольги на широкий луг\\
Всех маленьких ее подруг,\\
Она в горелки не играла,\\
Ей скучен был и звонкий смех,\\
И шум их ветреных утех.\\

XXVIII

Она любила на балконе\\\
Предупреждать зари восход,\\
Когда на бледном небосклоне\
Звезд исчезает хоровод,\\
И тихо край земли светлеет,\\
И, вестник утра, ветер веет,\\
И всходит постепенно день.\\
Зимой, когда ночная тень\\
Полмиром доле обладает,\\
И доле в праздной тишине,\\
При отуманенной луне,\\
Восток ленивый почивает,\\
В привычный час пробуждена\\
Вставала при свечах она.\\

XXIX

Ей рано нравились романы;\\
Они ей заменяли все;\\
Она влюблялася в обманы\\
И Ричардсона и Руссо.\\
Отец ее был добрый малый,\\
В прошедшем веке запоздалый;\\
Но в книгах не видал вреда;\\
\emph{Он, не читая никогда,\\
Их почитал пустой игрушкой}\\
И не заботился о том,\\
Какой у дочки тайный том\\
Дремал до утра под подушкой.\\
Жена ж его была сама\\
От Ричардсона без ума.\\

XXX

Она любила Ричардсона\\
Не потому, чтобы прочла,\\
Не потому, чтоб Грандисона\\
Она Ловласу предпочла;\\
Но в старину княжна Алина,\\
Ее московская кузина,\\
Твердила часто ей об них.\\
В то время был еще жених\\
Ее супруг, но по неволе;\\
Она вздыхала по другом,\\
Который сердцем и умом\\
Ей нравился гораздо боле:\\
Сей Грандисон был славный франт,\\
Игрок и гвардии сержант.\\

XXXI

Как он, она была одета\\
Всегда по моде и к лицу;\\
Но, не спросясь ее совета,\\
Девицу повезли к венцу.\\
И, чтоб ее рассеять горе,\\
Разумный муж уехал вскоре\\
В свою деревню, где она,\\
Бог знает кем окружена,\\
Рвалась и плакала сначала,\\
С супругом чуть не развелась;\\
Потом хозяйством занялась,\\
Привыкла и довольна стала.\\
\textbf{Привычка свыше нам дана:\\
Замена счастию она.}\\

XXXII

Привычка усладила горе,\\
Не отразимое ничем;\\
Открытие большое вскоре\\
Ее утешило совсем:\\
Она меж делом и досугом\\
Открыла тайну, как супругом\\
Самодержавно управлять,\\
И все тогда пошло на стать.\\
Она езжала по работам,\\
Солила на зиму грибы,\\
Вела расходы, брила лбы,\\
Ходила в баню по субботам,\\
Служанок била осердясь —\\
Все это мужа не спросясь.\\

XXXIII

Бывало, писывала кровью\\
Она в альбомы нежных дев,\\
Звала Полиною Прасковью\\
И говорила нараспев,\\
Корсет носила очень узкий,\\
И русский Н как N французский\\
Произносить умела в нос;\\
Но скоро все перевелось:\\
Корсет, альбом, княжну Алину,\\
Стишков чувствительных тетрадь\\
Она забыла: стала звать\\
Акулькой прежнюю Селину\\
И обновила наконец\\
На вате шлафор и чепец.\\

XXXIV

Но муж любил ее сердечно,\\
В ее затеи не входил,\\
Во всем ей веровал беспечно,\\
А сам в халате ел и пил;\\
Покойно жизнь его катилась;\\
Под вечер иногда сходилась\\
Соседей добрая семья,\\
Нецеремонные друзья,\\
И потужить, и позлословить,\\
И посмеяться кой о чем.\\
Проходит время; между тем\\
Прикажут Ольге чай готовить,\\
Там ужин, там и спать пора,\\
И гости едут со двора.\\

XXXV

Они хранили в жизни мирной\\
Привычки милой старины;\\
У них на масленице жирной\\
Водились русские блины;\\
Два раза в год они говели;\\
Любили круглые качели,\\
Подблюдны песни, хоровод;\\
В день Троицын, когда народ,\\
Зевая, слушает молебен,\\
Умильно на пучок зари\\
Они роняли слезки три;\\
Им квас как воздух был потребен,\\
И за столом у них гостям\\
Носили блюды по чинам.\\

XXXVI

И так они старели оба.\\
И отворились наконец\\
Перед супругом двери гроба,\
И новый он приял венец.\\
Он умер в час перед обедом,\\
Оплаканный своим соседом,\\
Детьми и верною женой\\
Чистосердечней, чем иной.\\
Он был простой и добрый барин,\\
И там, где прах его лежит,\\
Надгробный памятник гласит:\\
Смиренный грешник, Дмитрий Ларин,\\
Господний раб и бригадир,\\
Под камнем сим вкушает мир.\\

XXXVII

Своим пенатам возвращенный,\\
Владимир Ленский посетил\\
Соседа памятник смиренный,\\
И вздох он пеплу посвятил;\\
И долго сердцу грустно было.\\
«Рооr Yorick! — молвил он уныло. —\\
Он на руках меня держал.\\
Как часто в детстве я играл\\
Его Очаковской медалью!\\
Он Ольгу прочил за меня,\\
Он говорил: дождусь ли дня?..»\\
И, полный искренней печалью,\\
Владимир тут же начертал\\
Ему надгробный мадригал.\\

XXXVIII

И там же надписью печальной\\
Отца и матери, в слезах,\\
Почтил он прах патриархальный…\\
Увы! на жизненных браздах\\
Мгновенной жатвой поколенья,\\
По тайной воле провиденья,\\
Восходят, зреют и падут;\\
Другие им вослед идут…\\
Так наше ветреное племя\\
Растет, волнуется, кипит\\
И к гробу прадедов теснит.\\
Придет, придет и наше время,\\
И наши внуки в добрый час\\
Из мира вытеснят и нас!\\

XXXIX

Покамест упивайтесь ею,\\
Сей легкой жизнию, друзья!\\
Ее ничтожность разумею\\
И мало к ней привязан я;\\
Для призраков закрыл я вежды;\\
Но отдаленные надежды\\
Тревожат сердце иногда:\\
Без неприметного следа\\
Мне было б грустно мир оставить.\\
Живу, пишу не для похвал;\\
Но я бы, кажется, желал\\
Печальный жребий свой прославить,\\
Чтоб обо мне, как верный друг,\\
Напомнил хоть единый звук.\\

XL

И чье-нибудь он сердце тронет;\\
И, сохраненная судьбой,\\
Быть может, в Лете не потонет\\
Строфа, слагаемая мной;\\
Быть может (лестная надежда!),\\
Укажет будущий невежда\\
На мой прославленный портрет\\
И молвит: то-то был поэт!\\
Прими ж мои благодаренья,\\
Поклонник мирных аонид,\\
О ты, чья память сохранит\\
Мои летучие творенья,\\
Чья благосклонная рука\\
Потреплет лавры старика!\\

\chapter{} % Capitolo 3
\label{cap3} % Etichetta per riferimenti incrociati

\textit{Elle était fille, elle était amoureuse.\\
Malfilâtre}

I

«Куда? \textbf{Уж эти мне поэты!}»\\
— Прощай, Онегин, мне пора.	\\
«\textbf{Я не держу тебя}; но где ты\\
Свои проводишь вечера?»\\
— У Лариных. — «Вот это чудно.\\
Помилуй! \textbf{и тебе не трудно}\\
Там каждый вечер убивать?»\\
— Нимало. — «Не могу понять.\\
\emph{Отселе}\footnote{откуда} вижу, что такое:\\
Во-первых (слушай, прав ли я?),\\
Простая, русская семья,\\
К гостям усердие большое,\\
Варенье, вечный разговор\\
Про дождь, про лен, про скотный двор…»\\

II

— Я тут еще беды не вижу.\\
«Да скука, вот беда, мой друг».\\
— Я модный свет ваш ненавижу;\\
Милее мне домашний круг,\\
Где я могу… — «Опять\emph{эклога!}\footnote{Стихотворение в повествовательной или диалогической форме на тему о пастушеской жизни, сходное с идиллией.}\\
Да полно, милый, ради бога.\\
Ну что ж? ты едешь: очень жаль.\\
Ах, слушай, Ленский; да нельзя ль\\
Увидеть мне Филлиду эту,\\
Предмет и мыслей, и пера,\\
И слез, и рифм et cetera?..\\
Представь меня». — Ты шутишь. — «Нету».\\
— Я рад. — «Когда же?» — Хоть сейчас.\\
Они с охотой примут нас.\\

III

Поедем. —\\
Поскакали други,\\
Явились; им расточены\\
\emph{Порой}\\footnote{Иногда, иной раз} тяжелые услуги\\
Гостеприимной старины.\\
Обряд известный угощенья:\\
Несут на блюдечках варенья,\\
На столик ставят \emph{вощаной}\footnote{Натёртый или пропитанный воском; вощёный.}\\
Кувшин с \emph{брусничною водой}.\footnote{БРУСНИ́КА, Мелкие красные кисловатые ягоды этого кустарничка.}\\
. . . . . . . . . . . . . .\\
. . . . . . . . . . . . . .\\
. . . . . . . . . . . . . .\\
. . . . . . . . . . . . . .\\
. . . . . . . . . . . . . .\\
. . . . . . . . . . . . . .\\

IV

Они дорогой самой краткой\\
Домой летят во весь опор.\\
Теперь подслушаем \emph{украдкой}`\footnote{Скрытно, незаметно от других.}\\
Героев наших разговор:\\
— Ну что ж, Онегин? ты зеваешь. —\\
«Привычка, Ленский». — Но скучаешь\\
Ты как-то больше. — «Нет, равно.\\
Однако в поле уж темно;\\
Скорей! пошел, пошел, Андрюшка!\\
Какие глупые места!\\
А кстати: Ларина проста,\\
Но очень милая старушка;\\
Боюсь: брусничная вода\\
Мне не наделала б вреда.\\% fa mal di pancia ?

V

Скажи: которая Татьяна?»\\
— Да та, которая, грустна\\
И молчалива, как Светлана,\\% chi è Svetlana ?
Вошла и села у окна. —\\
«Неужто ты влюблен в меньшую?»\\
— А что? — «Я выбрал бы другую,\\
Когда б я был, как ты, поэт.\\
В чертах у Ольги жизни нет.\\
Точь-в-точь в Вандиковой Мадоне:\\ %cioè ?
Кругла, красна лицом она,\\
Как эта глупая луна\\
На этом глупом небосклоне».\\
Владимир сухо отвечал\\
И после во весь путь молчал.\\

VI

Меж тем Онегина явленье\\
У Лариных произвело\\
На всех большое впечатленье\\
И всех соседей развлекло.\\
Пошла догадка за догадкой.\\
Все стали толковать украдкой,\\
Шутить, судить не без греха,\\% interessante frase
Татьяне прочить жениха;\\
Иные даже утверждали,\\
Что свадьба слажена совсем,\\
Но остановлена затем,\\
Что модных колец не достали.\\
О свадьбе Ленского давно\\
У них уж было решено.\\

VII% come scoppia l'amore per Oneghin

Татьяна слушала с \emph{досадой}\footnote{Раздражение, неудовольствие}\\
Такие сплетни; но тайком\\
С \emph'{неизъяснимою отрадой}\footnote{невыразимый радость}\\ 
Невольно думала о том;\\
И в сердце дума заронилась;\\
Пора пришла, она влюбилась.\\
Так в землю падшее зерно\\
Весны огнем оживлено.\\
Давно ее воображенье,\\
Сгорая негой и тоской,\\
Алкало пищи роковой;\\
Давно сердечное томленье\\
Теснило ей младую грудь;\\
Душа ждала… кого-нибудь,\\

VIII

И дождалась… Открылись очи;\\
Она сказала: это он!\\
Увы! теперь и дни и ночи,\\
И жаркий одинокий сон,\\
Все полно им; все деве милой\\
Без умолку волшебной силой\\
Твердит о нем. Докучны ей\\
И звуки ласковых речей,\\
И взор заботливой прислуги.\\
В уныние погружена,\\
Гостей не слушает она\\
И проклинает их \emph{досуги,}\footnote{свободное время}\\
Их неожиданный приезд\\
И продолжительный присест.\\

IX

Теперь с каким она вниманьем\\
Читает сладостный роман,\\
С каким живым очарованьем\\
Пьет \emph{обольстительный}\footnote{заманчивый, привлекательный} обман!\\
Счастливой силою мечтанья\\
Одушевленные созданья,\\
Любовник Юлии Вольмар,\\
Малек-Адель и де Линар,\\
И Вертер, мученик мятежный,\\
И бесподобный Грандисон,\\%??
Который нам наводит сон,\\
Все для мечтательницы нежной\\
В единый образ облеклись,\\
\textbf{В одном Онегине слились.}\\

X

Воображаясь героиной?\\
Своих возлюбленных творцов,\\
Кларисой, Юлией, Дельфиной,\\
Татьяна в тишине лесов\\
\textbf{Одна с опасной книгой бродит,}\\
Она в ней ищет и находит\\
Свой тайный жар, свои мечты,\\
Плоды сердечной полноты,\\
Вздыхает и, себе присвоя\\
Чужой восторг, чужую грусть,\\
В забвенье шепчет наизусть\\
Письмо для милого героя…\\
Но наш герой, кто б ни был он,\\
Уж верно был не Грандисон.\\

XI

Свой слог на важный лад настроя,\\
Бывало, пламенный творец\\
Являл нам своего героя\\
Как совершенства образец.\\
Он одарял предмет любимый,\\
Всегда неправедно гонимый,\\
Душой чувствительной, умом\\
И привлекательным лицом.\\
Питая жар чистейшей страсти,\\
Всегда восторженный герой\\
Готов был жертвовать собой,\\
И при конце последней части\\
Всегда наказан был \emph{порок,}\footnote{vizio}\\
Добру достойный был венок.\\

XII

А нынче все умы в тумане,\\
Мораль на нас наводит сон,\\
Порок любезен — и в романе,\\
И там уж торжествует он.\\
Британской музы \emph{небылицы}\footnote{delle favole}\\
Тревожат сон \emph{отроковицы,}\footnote{Девочка-подросток в возрасте между ребенком и девушкой}\\
И стал теперь ее кумир\\
Или задумчивый Вампир,\\
Или Мельмот, бродяга мрачный,\\
Иль Вечный жид, или Корсар,\\
Или таинственный Сбогар.\\
Лорд Байрон прихотью удачной\\
Облек в унылый романтизм\\
И безнадежный эгоизм.\\

XIII\\footnote{простая русская семья, простая литература намичает путь к эволуций прозе}

Друзья мои, что ж толку в этом?\\
Быть может, волею небес,\\
Я перестану быть поэтом,\\
В меня вселится новый бес,\\
И, Фебовы\\footnote{Аполлон-Представитель высокого исскуства в классицизме} презрев угрозы,\\
Унижусь до смиренной прозы;\\
Тогда роман на старый лад\\
Займет веселый мой закат.\\
Не муки тайные злодейства\\
Я грозно в нем изображу,\\
Но просто вам перескажу\\
Преданья русского семейства,\\
Любви \emph{пленительные}\footnote{привлекательный} сны\\
Да нравы нашей старины.\\

XIV

Перескажу простые речи\\
Отца иль дяди-старика,\\
Детей условленные встречи\\
У старых лип, у ручейка;\\
Несчастной ревности мученья,\\
Разлуку, слезы примиренья,\\
Поссорю вновь, и наконец\\
Я поведу их под венец…\\
Я вспомню речи неги страстной,\\
Слова тоскующей любви,\\
Которые в минувши дни\\
У ног любовницы прекрасной\\
Мне приходили на язык,\\
От коих я теперь отвык.\\

XV\footnote{П. описывает Татьяну с ееромантическим языком}

Татьяна, милая Татьяна!\\
С тобой теперь я слезы лью;\\
Ты в руки модного тирана\\
Уж отдала судьбу свою.\\
Погибнешь, милая; но прежде\\
Ты в ослепительной надежде\\
Блаженство темное зовешь,\\
Ты негу жизни узнаешь,\\
Ты пьешь волшебный яд желаний,\\
Тебя преследуют мечты:\\
Везде воображаешь ты\\
Приюты счастливых свиданий;\\
Везде, везде перед тобой\\
Твой \emph{искуситель роковой.}`\footnote{соблазнитель неизбежный.}\\

XVI

Тоска любви Татьяну гонит,\\
И в сад идет она грустить,\\
И вдруг недвижны очи клонит,\\
И лень ей далее ступить.\\
Приподнялася грудь, ланиты\\
Мгновенным пламенем покрыты,\\
Дыханье замерло в устах,\\
И в слухе шум, и блеск в очах…\\
Настанет ночь; луна обходит\\
Дозором дальный свод небес,\\
И соловей во мгле древес\\
Напевы звучные заводит.\\
Татьяна в темноте не спит\\
И тихо с няней говорит:\\

XVII

«Не спится, няня: здесь так душно!\\
Открой окно да сядь ко мне».\\
— Что, Таня, что с тобой? — «Мне скучно,\\
Поговорим о старине».\\
— О чем же, Таня? Я, бывало,\\
Хранила в памяти не мало\\
Старинных былей, небылиц\\
Про злых духов и про девиц;\\
А нынче все мне темно, Таня:\\
Что знала, то забыла. Да,\\
Пришла худая череда!\\
Зашибло… — «Расскажи мне, няня,\\
Про ваши старые года:\\
Была ты влюблена тогда?»\\

XVIII\footnote{Т. нуждается в совет и обращается к няне}

— И, полно, Таня! В эти лета\\
Мы не слыхали про любовь;\\
А то бы согнала со света\\
Меня покойница свекровь. —\\
«Да как же ты венчалась, няня?»\\
— Так, видно, бог велел. Мой Ваня\\
Моложе был меня, мой свет,\\
А было мне тринадцать лет.\\
Недели две ходила сваха\\
К моей родне, и наконец\\
Благословил меня отец.\\
Я горько плакала со страха,\\
Мне с плачем \emph{косу расплели}\footnote{одна коса это девичья, две нже замужная женшина}\\
Да с пеньем в церковь повели.\\

XIX

И вот ввели в семью чужую…\\
Да ты не слушаешь меня… —\\
«Ах, няня, няня, я тоскую,\\
Мне тошно, милая моя:\\
Я плакать, я рыдать готова!..»\\
— Дитя мое, ты нездорова;\\
Господь помилуй и спаси!\\
Чего ты хочешь, попроси…\\
Дай окроплю святой водою,\\
Ты вся горишь… — «Я не больна:\\
Я… знаешь, няня… влюблена».\\
— Дитя мое, господь с тобою! —\\
И няня девушку с мольбой\marginnote{молитваЪ\\
Крестила дряхлою рукой.\\

XX

«Я влюблена», — шептала снова\\
Старушке с горестью она.\\
— Сердечный друг, ты нездорова.\\
«Оставь меня: я влюблена».\\
И между тем луна сияла\\
И томным светом озаряла\\
Татьяны бледные красы,\\
И распущенные власы,\\
И капли слез, и на скамейке\\
Пред героиней молодой,\\
С платком на голове седой,\\
Старушку в длинной телогрейке;\\
И все дремало в тишине\\
При вдохновительной луне.\\

XXI

И сердцем далеко носилась\\
Татьяна, смотря на луну…\\
Вдруг мысль в уме ее родилась…\\
«Поди, оставь меня одну.\\
Дай, няня, мне перо, бумагу,\\
Да стол подвинь; я скоро лягу;\\
Прости». И вот она одна.\\
Все тихо. Светит ей луна.\\
Облокотясь\marginnote'{appoggiare i gomiti sul tavolo}, Татьяна пишет\\,
И все Евгений на уме,\\
И в необдуманном письме\\
Любовь невинной девы дышит.\\
Письмо готово, сложено…\\
Татьяна! для кого ж оно?\\

XXII

Я знал красавиц недоступных,\\
Холодных, чистых, как зима,\\
Неумолимых, неподкупных,\\
Непостижимых для ума;\\
Дивился я их спеси модной,\\
Их добродетели природной,\\
И, признаюсь, от них бежал,\\
И, мнится, с ужасом читал\\
Над их бровями надпись ада:\\
Оставь надежду навсегда.\\
Внушать любовь для них беда,\\
Пугать людей для них отрада.\\
Быть может, на брегах Невы\\
Подобных дам видали вы.\\

XXIII

Среди поклонников послушных\\
Других причудниц я видал,\\
Самолюбиво равнодушных\\
Для вздохов страстных и похвал.\\
И что ж нашел я с изумленьем?\\
Они, суровым повеленьем\\
Пугая робкую любовь,\\
Ее привлечь умели вновь\\
По крайней мере сожаленьем,\\
По крайней мере звук речей\\
Казался иногда нежней,\\
И с легковерным ослепленьем\\
Опять любовник молодой\\
Бежал за милой суетой.\\

XXIV

За что ж виновнее Татьяна?\\
За то ль, что в милой простоте\\
Она не ведает обмана\\
И верит избранной мечте?\\
За то ль, что любит без искусства,\\
Послушная влеченью чувства,\\
Что так доверчива она,\\
Что от небес одарена\\
Воображением мятежным,\\
Умом и волею живой,\\
И своенравной головой,\\
И сердцем пламенным и нежным?\\
Ужели не простите ей\\
Вы легкомыслия страстей?\\

XXV

Кокетка судит хладнокровно,\\
Татьяна любит не шутя\\
И предается безусловно\\
Любви, как милое дитя.\\
Не говорит она: отложим —\\
Любви мы цену тем умножим,\\
Вернее в сети заведем;\\
Сперва тщеславие кольнем\\\
Надеждой, там недоуменьем\\
Измучим сердце, а потом\\
Ревнивым оживим огнем;\\
А то, скучая наслажденьем\\,
Невольник хитрый из оков\\
Всечасно вырваться готов.\\

XXVI

Еще предвижу затрудненья:\\
Родной земли спасая честь,\\
Я должен буду, без сомненья,\\
Письмо Татьяны перевесть.\\
Она по-русски плохо знала,\\
Журналов наших не читала\\
И выражалася с трудом\\
На языке своем родном,\\
Итак, писала по-французски…\\
Что делать! повторяю вновь:\\
Доныне \marginnote{ло сих пор}дамская любовь\\
Не изьяснялася по-русски,\\
Доныне гордый наш язык\\
К почтовой прозе не привык.\\

XXVII

Я знаю: дам хотят заставить\\
Читать по-русски. Право, страх!\\
Могу ли их себе представить\\
С «Благонамеренным» в руках!\\
Я шлюсь на вас, мои поэты;\\
Не правда ль: милые предметы,\\
Которым, за свои грехи,\\
Писали втайне вы стихи,\\
Которым сердце посвящали,\\
Не все ли, русским языком\\
Владея слабо и с трудом,\\
Его так мило `emph{искажали,}\marginnote{портить, коверкать}\\
И в их устах язык чужой\\
Не обратился ли в родной?\\

XXVIII

Не дай мне бог сойтись на бале\\
Иль при разъезде на крыльце\\
С семинаристом в желтой шале\\
Иль с академиком в чепце!\\
Как уст румяных без улыбки,\\
Без грамматической ошибки\\
Я русской речи не люблю.\\	
Быть может, на беду мою,\\
Красавиц новых поколенье,\\
Журналов вняв молящий глас,\\
К грамматике приучит нас;\\
Стихи введут в употребленье;\\
Но я… какое дело мне?\\
Я верен буду старине.\\

XXIX

Неправильный, небрежный лепет,\\
Неточный выговор речей\\
По-прежнему сердечный трепет\\
Произведут в груди моей;\\
Раскаяться во мне нет силы,\\
Мне галлицизмы будут милы,\\
Как прошлой юности грехи,\\
Как Богдановича стихи.\\
Но полно. Мне пора заняться\\
Письмом красавицы моей;\\
Я слово дал, и что ж? ей-ей\\
Теперь готов уж отказаться.\\
Я знаю: нежного Парни\\
Перо не в моде в наши дни.\\

XXX

Певец Пиров и грусти томной,\\
Когда б еще ты был со мной,\\
Я стал бы просьбою нескромной\\
Тебя тревожить, милый мой:\\
Чтоб на волшебные напевы\\
Переложил ты страстной девы\\
Иноплеменные слова.\\
Где ты? приди: свои права\\
Передаю тебе с поклоном…\\
Но посреди печальных скал,\\
Отвыкнув сердцем от похвал,\\
Один, под финским небосклоном,\\
Он бродит, и душа его\\
Не слышит горя моего.\\

XXXI

Письмо Татьяны предо мною;\\
Его я свято берегу,\\
Читаю с тайною тоскою\\
И начитаться не могу.\\
Кто ей внушал и эту нежность,\\
И слов любезную небрежность\marginnote{negligenza}?\\
Кто ей внушал умильный вздор,\\
Безумный сердца разговор,\\
И увлекательный и вредный?\\
Я не могу понять. Но вот\\
Неполный, слабый перевод,\\
С живой картины список бледный\\
Или разыгранный Фрейшиц\\
Перстами робких учениц:\\
Письмо Татьяны к Онегину\\
Я к вам пишу — чего же боле?\\
Что я могу еще сказать?\\
Теперь, я знаю, в вашей воле\\
Меня презреньем наказать.\\
Но вы, к моей несчастной доле\\
Хоть каплю жалости храня,\\
Вы не оставите меня.\\
Сначала я молчать хотела;\\
Поверьте: моего стыда\\
Вы не узнали б никогда,\\
Когда б надежду я имела\\
Хоть редко, хоть в неделю раз\\
В деревне нашей видеть вас,\\
Чтоб только слышать ваши речи,\\
Вам слово молвить, и потом\\
Все думать, думать об одном\\
И день и ночь до новой встречи.\\
Но, говорят, вы нелюдим;\\
В глуши, в деревне все вам скучно,\\
А мы… ничем мы не блестим,\\
Хоть вам и рады простодушно.\\
Зачем вы посетили нас?\\
В глуши забытого селенья\\
Я никогда не знала б вас,\\
Не знала б горького мученья.\\
Души неопытной волненья\\
Смирив со временем (как знать?),\\
По сердцу я нашла бы друга,\\
Была бы верная супруга\\
И добродетельная мать.\\
Другой!.. Нет, никому на свете\\
Не отдала бы сердца я!\\
То в вышнем суждено совете…\\
То воля неба: я твоя;\\
Вся жизнь моя была залогом\\
Свиданья верного с тобой;\\
Я знаю, ты мне послан богом,\\
До гроба ты хранитель мой…\\
Ты в сновиденьях мне являлся\\
Незримый, ты мне был уж мил,\\
Твой чудный взгляд меня томил,\\
В душе твой голос раздавался\\
Давно… нет, это был не сон!\\
Ты чуть вошел, я вмиг узнала,\\
Вся обомлела, запылала\\
И в мыслях молвила: вот он!\\
Не правда ль? я тебя слыхала:\\
Ты говорил со мной в тиши,\\
Когда я бедным помогала\\
Или молитвой услаждала\\
Тоску волнуемой души?\\
И в это самое мгновенье\\
Не ты ли, милое виденье,\\
В прозрачной темноте мелькнул,\\
Приникнул тихо к изголовью?\\
Не ты ль, с отрадой и любовью,\\
Слова надежды мне шепнул?\\
Кто ты, мой ангел ли хранитель,\\
Или коварный искуситель:\\
Мои сомненья разреши.\\
Быть может, это все пустое,\\
Обман неопытной души!\\
И суждено совсем иное…\\
Но так и быть! Судьбу мою\\
Отныне я тебе вручаю,\\
Перед тобою слезы лью,\\
Твоей защиты умоляю…\\
Вообрази: я здесь одна,\\
Никто меня не понимает,\\
Рассудок мой изнемогает,\\
И молча гибнуть я должна.\\
Я жду тебя: единым взором\\
Надежды сердца оживи\\
Иль сон тяжелый перерви,\\
Увы, заслуженным укором!\\
Кончаю! Страшно перечесть…\\
Стыдом и страхом замираю…\\
Но мне порукой ваша честь,\\
И смело ей себя вверяю…\\

XXXII

Татьяна то вздохнет, то охнет\marginnote{Воскликнуть ох, выражая какое-л. ощущение, чувство: боль, досаду, испуг и т. п.}\\
Письмо дрожит в ее руке;\\
Облатка розовая сохнет\\
На воспаленном\marginnote{infiammata} языке.\\
К плечу головушкой склонилась,\\
Соро'чка\\marginnote{intimo} легкая спустилась\\
С ее прелестного плеча…\\
Но вот уж лунного луча\\
Сиянье гаснет. Там долина\marginnote{Ровное пространство вдоль речного русла}\\
Сквозь пар яснеет. Там поток\\
Засеребрился; там рожок\\
Пастуший будит селянина.\marginnote{Крестьянин}\\
Вот утро: встали все давно,\\
Моей Татьяне все равно.\\

XXXIII

Она зари не замечает,\\
Сидит с поникшею главой\\
И на письмо не напирает\\
Своей печати вырезной.\\
Но, дверь тихонько отпирая,\\
Уж ей Филипьевна седая\\
Приносит на подносе чай.\\
«Пора, дитя мое, вставай:\\
Да ты, красавица, готова!\\
О пташка ранняя моя!\\
Вечор уж как боялась я!\\
Да, слава богу, ты здорова!\\
Тоски ночной и следу нет,\\
Лицо твое как маков цвет».\\

XXXIV

— Ах! няня, сделай одолженье. —\\
«Изволь, родная, прикажи».\\
— Не думай… право… подозренье…\\
Но видишь… ах! не откажи. —\\
«Мой друг, вот бог тебе порука».\\
— Итак, пошли тихонько внука\\
С запиской этой к О… к тому…\\
К соседу… да велеть ему,\\
Чтоб он не говорил ни слова,\\
Чтоб он не называл меня… —\\
«Кому же, милая моя?\\
Я нынче стала бестолкова.\\
Кругом соседей много есть;\\
Куда мне их и перечесть».\\

XXXV

— Как недогадлива ты, няня! —\\
«Сердечный друг, уж я стара,\\
Стара; тупеет разум, Таня;\\
А то, бывало, я востра,\\
Бывало, слово барской воли…»\\
— Ах, няня, няня! до того ли?\\
Что нужды мне в твоем уме?\\
Ты видишь, дело о письме\\
К Онегину. — «Ну, дело, дело.\\
Не гневайся, душа моя,\\
Ты знаешь, непонятна я…\\
Да что ж ты снова побледнела?»\\
— Так, няня, право ничего.\\
Пошли же внука своего.\\

XXXVI

Но день протек, и нет ответа.\\
Другой настал: все нет как нет.\\
Бледна, как тень, с утра одета,\\
Татьяна ждет: когда ж ответ?\\
Приехал Ольгин обожатель.\\
«Скажите: где же ваш приятель? —\\
Ему вопрос хозяйки был. —\\
Он что-то нас совсем забыл».\\
Татьяна, вспыхнув, задрожала.\\
— Сегодня быть он обещал, —\\
Старушке Ленский отвечал, —\\
Да, видно, почта задержала. —\\
Татьяна потупила взор,\\
Как будто слыша злой укор.\\

XXXVII

Смеркалось; на столе, блистая,\\
Шипел вечерний самовар,\\
Китайский чайник нагревая;\\
Под ним клубился легкий пар.\\
Разлитый Ольгиной рукою,\\
По чашкам темною струею\\
Уже душистый чай бежал,\\
И сливки мальчик подавал;\\
Татьяна пред окном стояла,\\
На стекла хладные дыша,\\
Задумавшись, моя душа,\\
Прелестным пальчиком писала\\
На отуманенном стекле\\
Заветный вензель\marginnote{Особенно ценимый Начальные буквы имени и фамилии} О да Е.\\

XXXVIII

И между тем душа в ней ныла,
И слез был полон томный взор.
Вдруг топот!.. кровь ее застыла.
Вот ближе! скачут… и на двор
Евгений! «Ах!» — и легче тени
Татьяна прыг в другие сени,
С крыльца на двор, и прямо в сад,
Летит, летит; взглянуть назад
Не смеет; мигом обежала
Куртины, мостики, лужок,
Аллею к озеру, лесок,
Кусты сирен переломала,
По цветникам летя к ручью.
И, задыхаясь, на скамью
XXXIX
Упала…
«Здесь он! здесь Евгений!
О боже! что подумал он!»
В ней сердце, полное мучений,
Хранит надежды темный сон;
Она дрожит и жаром пышет,
И ждет: нейдет ли? Но не слышит.
В саду служанки, на грядах,
Сбирали ягоду в кустах
И хором по наказу пели
(Наказ, основанный на том,
Чтоб барской ягоды тайком
Уста лукавые не ели
И пеньем были заняты:
Затея сельской остроты!)
Песня девушек
Девицы, красавицы,
Душеньки, подруженьки,
Разыграйтесь девицы,
Разгуляйтесь, милые!
Затяните песенку,
Песенку заветную,
Заманите молодца
К хороводу нашему,
Как заманим молодца,
Как завидим издали,
Разбежимтесь, милые,
Закидаем вишеньем,
Вишеньем, малиною,
Красною смородиной.
Не ходи подслушивать
Песенки заветные,
Не ходи подсматривать
Игры наши девичьи.
ХL
Они поют, и, с небреженьем
Внимая звонкий голос их,
Ждала Татьяна с нетерпеньем,
Чтоб трепет сердца в ней затих,
Чтобы прошло ланит пыланье.
Но в персях то же трепетанье,
И не проходит жар ланит,
Но ярче, ярче лишь горит…
Так бедный мотылек и блещет
И бьется радужным крылом,
Плененный школьным шалуном;
Так зайчик в озими трепещет,
Увидя вдруг издалека
В кусты припадшего стрелка.
ХLI
Но наконец она вздохнула
И встала со скамьи своей;
Пошла, но только повернула
В аллею, прямо перед ней,
Блистая взорами, Евгений
Стоит подобно грозной тени,
И, как огнем обожжена,
Остановилася она.
Но следствия нежданной встречи
Сегодня, милые друзья,
Пересказать не в силах я;
Мне должно после долгой речи
И погулять и отдохнуть:
Докончу после как-нибудь.

\section{Analisi cap.3}
Строфа I - II   П. вводит в поэтическое произведение буднучную прозаичесукую беседу



\end{document}
