\chapter{} % Capitolo 3
\label{cap3} % Etichetta per riferimenti incrociati

\textit{Elle était fille, elle était amoureuse.\\
Malfilâtre}

I

«Куда? \textbf{Уж эти мне поэты!}»\\
— Прощай, Онегин, мне пора.	\\
«\textbf{Я не держу тебя}; но где ты\\
Свои проводишь вечера?»\\
— У Лариных. — «Вот это чудно.\\
Помилуй! \textbf{и тебе не трудно}\\
Там каждый вечер убивать?»\\
— Нимало. — «Не могу понять.\\
\emph{Отселе}\footnote{откуда} вижу, что такое:\\
Во-первых (слушай, прав ли я?),\\
Простая, русская семья,\\
К гостям усердие большое,\\
Варенье, вечный разговор\\
Про дождь, про лен, про скотный двор…»\\

II

— Я тут еще беды не вижу.\\
«Да скука, вот беда, мой друг».\\
— Я модный свет ваш ненавижу;\\
Милее мне домашний круг,\\
Где я могу… — «Опять\emph{эклога!}\footnote{Стихотворение в повествовательной или диалогической форме на тему о пастушеской жизни, сходное с идиллией.}\\
Да полно, милый, ради бога.\\
Ну что ж? ты едешь: очень жаль.\\
Ах, слушай, Ленский; да нельзя ль\\
Увидеть мне Филлиду эту,\\
Предмет и мыслей, и пера,\\
И слез, и рифм et cetera?..\\
Представь меня». — Ты шутишь. — «Нету».\\
— Я рад. — «Когда же?» — Хоть сейчас.\\
Они с охотой примут нас.\\

III

Поедем. —\\
Поскакали други,\\
Явились; им расточены\\
\emph{Порой}\\footnote{Иногда, иной раз} тяжелые услуги\\
Гостеприимной старины.\\
Обряд известный угощенья:\\
Несут на блюдечках варенья,\\
На столик ставят \emph{вощаной}\footnote{Натёртый или пропитанный воском; вощёный.}\\
Кувшин с \emph{брусничною водой}.\footnote{БРУСНИ́КА, Мелкие красные кисловатые ягоды этого кустарничка.}\\
. . . . . . . . . . . . . .\\
. . . . . . . . . . . . . .\\
. . . . . . . . . . . . . .\\
. . . . . . . . . . . . . .\\
. . . . . . . . . . . . . .\\
. . . . . . . . . . . . . .\\

IV

Они дорогой самой краткой\\
Домой летят во весь опор.\\
Теперь подслушаем \emph{украдкой}`\footnote{Скрытно, незаметно от других.}\\
Героев наших разговор:\\
— Ну что ж, Онегин? ты зеваешь. —\\
«Привычка, Ленский». — Но скучаешь\\
Ты как-то больше. — «Нет, равно.\\
Однако в поле уж темно;\\
Скорей! пошел, пошел, Андрюшка!\\
Какие глупые места!\\
А кстати: Ларина проста,\\
Но очень милая старушка;\\
Боюсь: брусничная вода\\
Мне не наделала б вреда.\\% fa mal di pancia ?

V

Скажи: которая Татьяна?»\\
— Да та, которая, грустна\\
И молчалива, как Светлана,\\% chi è Svetlana ?
Вошла и села у окна. —\\
«Неужто ты влюблен в меньшую?»\\
— А что? — «Я выбрал бы другую,\\
Когда б я был, как ты, поэт.\\
В чертах у Ольги жизни нет.\\
Точь-в-точь в Вандиковой Мадоне:\\ %cioè ?
Кругла, красна лицом она,\\
Как эта глупая луна\\
На этом глупом небосклоне».\\
Владимир сухо отвечал\\
И после во весь путь молчал.\\

VI

Меж тем Онегина явленье\\
У Лариных произвело\\
На всех большое впечатленье\\
И всех соседей развлекло.\\
Пошла догадка за догадкой.\\
Все стали толковать украдкой,\\
Шутить, судить не без греха,\\% interessante frase
Татьяне прочить жениха;\\
Иные даже утверждали,\\
Что свадьба слажена совсем,\\
Но остановлена затем,\\
Что модных колец не достали.\\
О свадьбе Ленского давно\\
У них уж было решено.\\

VII% come scoppia l'amore per Oneghin

Татьяна слушала с \emph{досадой}\footnote{Раздражение, неудовольствие}\\
Такие сплетни; но тайком\\
С \emph'{неизъяснимою отрадой}\footnote{невыразимый радость}\\ 
Невольно думала о том;\\
И в сердце дума заронилась;\\
Пора пришла, она влюбилась.\\
Так в землю падшее зерно\\
Весны огнем оживлено.\\
Давно ее воображенье,\\
Сгорая негой и тоской,\\
Алкало пищи роковой;\\
Давно сердечное томленье\\
Теснило ей младую грудь;\\
Душа ждала… кого-нибудь,\\

VIII

И дождалась… Открылись очи;\\
Она сказала: это он!\\
Увы! теперь и дни и ночи,\\
И жаркий одинокий сон,\\
Все полно им; все деве милой\\
Без умолку волшебной силой\\
Твердит о нем. Докучны ей\\
И звуки ласковых речей,\\
И взор заботливой прислуги.\\
В уныние погружена,\\
Гостей не слушает она\\
И проклинает их \emph{досуги,}\footnote{свободное время}\\
Их неожиданный приезд\\
И продолжительный присест.\\

IX

Теперь с каким она вниманьем\\
Читает сладостный роман,\\
С каким живым очарованьем\\
Пьет \emph{обольстительный}\footnote{заманчивый, привлекательный} обман!\\
Счастливой силою мечтанья\\
Одушевленные созданья,\\
Любовник Юлии Вольмар,\\
Малек-Адель и де Линар,\\
И Вертер, мученик мятежный,\\
И бесподобный Грандисон,\\%??
Который нам наводит сон,\\
Все для мечтательницы нежной\\
В единый образ облеклись,\\
\textbf{В одном Онегине слились.}\\

X

Воображаясь героиной?\\
Своих возлюбленных творцов,\\
Кларисой, Юлией, Дельфиной,\\
Татьяна в тишине лесов\\
\textbf{Одна с опасной книгой бродит,}\\
Она в ней ищет и находит\\
Свой тайный жар, свои мечты,\\
Плоды сердечной полноты,\\
Вздыхает и, себе присвоя\\
Чужой восторг, чужую грусть,\\
В забвенье шепчет наизусть\\
Письмо для милого героя…\\
Но наш герой, кто б ни был он,\\
Уж верно был не Грандисон.\\

XI

Свой слог на важный лад настроя,\\
Бывало, пламенный творец\\
Являл нам своего героя\\
Как совершенства образец.\\
Он одарял предмет любимый,\\
Всегда неправедно гонимый,\\
Душой чувствительной, умом\\
И привлекательным лицом.\\
Питая жар чистейшей страсти,\\
Всегда восторженный герой\\
Готов был жертвовать собой,\\
И при конце последней части\\
Всегда наказан был \emph{порок,}\footnote{vizio}\\
Добру достойный был венок.\\

XII

А нынче все умы в тумане,\\
Мораль на нас наводит сон,\\
Порок любезен — и в романе,\\
И там уж торжествует он.\\
Британской музы \emph{небылицы}\footnote{delle favole}\\
Тревожат сон \emph{отроковицы,}\footnote{Девочка-подросток в возрасте между ребенком и девушкой}\\
И стал теперь ее кумир\\
Или задумчивый Вампир,\\
Или Мельмот, бродяга мрачный,\\
Иль Вечный жид, или Корсар,\\
Или таинственный Сбогар.\\
Лорд Байрон прихотью удачной\\
Облек в унылый романтизм\\
И безнадежный эгоизм.\\

XIII\\footnote{простая русская семья, простая литература намичает путь к эволуций прозе}

Друзья мои, что ж толку в этом?\\
Быть может, волею небес,\\
Я перестану быть поэтом,\\
В меня вселится новый бес,\\
И, Фебовы\\footnote{Аполлон-Представитель высокого исскуства в классицизме} презрев угрозы,\\
Унижусь до смиренной прозы;\\
Тогда роман на старый лад\\
Займет веселый мой закат.\\
Не муки тайные злодейства\\
Я грозно в нем изображу,\\
Но просто вам перескажу\\
Преданья русского семейства,\\
Любви \emph{пленительные}\footnote{привлекательный} сны\\
Да нравы нашей старины.\\

XIV

Перескажу простые речи\\
Отца иль дяди-старика,\\
Детей условленные встречи\\
У старых лип, у ручейка;\\
Несчастной ревности мученья,\\
Разлуку, слезы примиренья,\\
Поссорю вновь, и наконец\\
Я поведу их под венец…\\
Я вспомню речи неги страстной,\\
Слова тоскующей любви,\\
Которые в минувши дни\\
У ног любовницы прекрасной\\
Мне приходили на язык,\\
От коих я теперь отвык.\\

XV\footnote{П. описывает Татьяну с ееромантическим языком}

Татьяна, милая Татьяна!\\
С тобой теперь я слезы лью;\\
Ты в руки модного тирана\\
Уж отдала судьбу свою.\\
Погибнешь, милая; но прежде\\
Ты в ослепительной надежде\\
Блаженство темное зовешь,\\
Ты негу жизни узнаешь,\\
Ты пьешь волшебный яд желаний,\\
Тебя преследуют мечты:\\
Везде воображаешь ты\\
Приюты счастливых свиданий;\\
Везде, везде перед тобой\\
Твой \emph{искуситель роковой.}`\footnote{соблазнитель неизбежный.}\\

XVI

Тоска любви Татьяну гонит,\\
И в сад идет она грустить,\\
И вдруг недвижны очи клонит,\\
И лень ей далее ступить.\\
Приподнялася грудь, ланиты\\
Мгновенным пламенем покрыты,\\
Дыханье замерло в устах,\\
И в слухе шум, и блеск в очах…\\
Настанет ночь; луна обходит\\
Дозором дальный свод небес,\\
И соловей во мгле древес\\
Напевы звучные заводит.\\
Татьяна в темноте не спит\\
И тихо с няней говорит:\\

XVII

«Не спится, няня: здесь так душно!\\
Открой окно да сядь ко мне».\\
— Что, Таня, что с тобой? — «Мне скучно,\\
Поговорим о старине».\\
— О чем же, Таня? Я, бывало,\\
Хранила в памяти не мало\\
Старинных былей, небылиц\\
Про злых духов и про девиц;\\
А нынче все мне темно, Таня:\\
Что знала, то забыла. Да,\\
Пришла худая череда!\\
Зашибло… — «Расскажи мне, няня,\\
Про ваши старые года:\\
Была ты влюблена тогда?»\\

XVIII\footnote{Т. нуждается в совет и обращается к няне}

— И, полно, Таня! В эти лета\\
Мы не слыхали про любовь;\\
А то бы согнала со света\\
Меня покойница свекровь. —\\
«Да как же ты венчалась, няня?»\\
— Так, видно, бог велел. Мой Ваня\\
Моложе был меня, мой свет,\\
А было мне тринадцать лет.\\
Недели две ходила сваха\\
К моей родне, и наконец\\
Благословил меня отец.\\
Я горько плакала со страха,\\
Мне с плачем \emph{косу расплели}\footnote{одна коса это девичья, две нже замужная женшина}\\
Да с пеньем в церковь повели.\\

XIX

И вот ввели в семью чужую…\\
Да ты не слушаешь меня… —\\
«Ах, няня, няня, я тоскую,\\
Мне тошно, милая моя:\\
Я плакать, я рыдать готова!..»\\
— Дитя мое, ты нездорова;\\
Господь помилуй и спаси!\\
Чего ты хочешь, попроси…\\
Дай окроплю святой водою,\\
Ты вся горишь… — «Я не больна:\\
Я… знаешь, няня… влюблена».\\
— Дитя мое, господь с тобою! —\\
И няня девушку с мольбой\marginnote{молитваЪ\\
Крестила дряхлою рукой.\\

XX

«Я влюблена», — шептала снова\\
Старушке с горестью она.\\
— Сердечный друг, ты нездорова.\\
«Оставь меня: я влюблена».\\
И между тем луна сияла\\
И томным светом озаряла\\
Татьяны бледные красы,\\
И распущенные власы,\\
И капли слез, и на скамейке\\
Пред героиней молодой,\\
С платком на голове седой,\\
Старушку в длинной телогрейке;\\
И все дремало в тишине\\
При вдохновительной луне.\\

XXI

И сердцем далеко носилась\\
Татьяна, смотря на луну…\\
Вдруг мысль в уме ее родилась…\\
«Поди, оставь меня одну.\\
Дай, няня, мне перо, бумагу,\\
Да стол подвинь; я скоро лягу;\\
Прости». И вот она одна.\\
Все тихо. Светит ей луна.\\
Облокотясь\marginnote'{appoggiare i gomiti sul tavolo}, Татьяна пишет\\,
И все Евгений на уме,\\
И в необдуманном письме\\
Любовь невинной девы дышит.\\
Письмо готово, сложено…\\
Татьяна! для кого ж оно?\\

XXII

Я знал красавиц недоступных,\\
Холодных, чистых, как зима,\\
Неумолимых, неподкупных,\\
Непостижимых для ума;\\
Дивился я их спеси модной,\\
Их добродетели природной,\\
И, признаюсь, от них бежал,\\
И, мнится, с ужасом читал\\
Над их бровями надпись ада:\\
Оставь надежду навсегда.\\
Внушать любовь для них беда,\\
Пугать людей для них отрада.\\
Быть может, на брегах Невы\\
Подобных дам видали вы.\\

XXIII

Среди поклонников послушных\\
Других причудниц я видал,\\
Самолюбиво равнодушных\\
Для вздохов страстных и похвал.\\
И что ж нашел я с изумленьем?\\
Они, суровым повеленьем\\
Пугая робкую любовь,\\
Ее привлечь умели вновь\\
По крайней мере сожаленьем,\\
По крайней мере звук речей\\
Казался иногда нежней,\\
И с легковерным ослепленьем\\
Опять любовник молодой\\
Бежал за милой суетой.\\

XXIV

За что ж виновнее Татьяна?\\
За то ль, что в милой простоте\\
Она не ведает обмана\\
И верит избранной мечте?\\
За то ль, что любит без искусства,\\
Послушная влеченью чувства,\\
Что так доверчива она,\\
Что от небес одарена\\
Воображением мятежным,\\
Умом и волею живой,\\
И своенравной головой,\\
И сердцем пламенным и нежным?\\
Ужели не простите ей\\
Вы легкомыслия страстей?\\

XXV

Кокетка судит хладнокровно,\\
Татьяна любит не шутя\\
И предается безусловно\\
Любви, как милое дитя.\\
Не говорит она: отложим —\\
Любви мы цену тем умножим,\\
Вернее в сети заведем;\\
Сперва тщеславие кольнем\\\
Надеждой, там недоуменьем\\
Измучим сердце, а потом\\
Ревнивым оживим огнем;\\
А то, скучая наслажденьем\\,
Невольник хитрый из оков\\
Всечасно вырваться готов.\\

XXVI

Еще предвижу затрудненья:\\
Родной земли спасая честь,\\
Я должен буду, без сомненья,\\
Письмо Татьяны перевесть.\\
Она по-русски плохо знала,\\
Журналов наших не читала\\
И выражалася с трудом\\
На языке своем родном,\\
Итак, писала по-французски…\\
Что делать! повторяю вновь:\\
Доныне \marginnote{ло сих пор}дамская любовь\\
Не изьяснялася по-русски,\\
Доныне гордый наш язык\\
К почтовой прозе не привык.\\

XXVII

Я знаю: дам хотят заставить\\
Читать по-русски. Право, страх!\\
Могу ли их себе представить\\
С «Благонамеренным» в руках!\\
Я шлюсь на вас, мои поэты;\\
Не правда ль: милые предметы,\\
Которым, за свои грехи,\\
Писали втайне вы стихи,\\
Которым сердце посвящали,\\
Не все ли, русским языком\\
Владея слабо и с трудом,\\
Его так мило `emph{искажали,}\marginnote{портить, коверкать}\\
И в их устах язык чужой\\
Не обратился ли в родной?\\

XXVIII

Не дай мне бог сойтись на бале\\
Иль при разъезде на крыльце\\
С семинаристом в желтой шале\\
Иль с академиком в чепце!\\
Как уст румяных без улыбки,\\
Без грамматической ошибки\\
Я русской речи не люблю.\\	
Быть может, на беду мою,\\
Красавиц новых поколенье,\\
Журналов вняв молящий глас,\\
К грамматике приучит нас;\\
Стихи введут в употребленье;\\
Но я… какое дело мне?\\
Я верен буду старине.\\

XXIX

Неправильный, небрежный лепет,\\
Неточный выговор речей\\
По-прежнему сердечный трепет\\
Произведут в груди моей;\\
Раскаяться во мне нет силы,\\
Мне галлицизмы будут милы,\\
Как прошлой юности грехи,\\
Как Богдановича стихи.\\
Но полно. Мне пора заняться\\
Письмом красавицы моей;\\
Я слово дал, и что ж? ей-ей\\
Теперь готов уж отказаться.\\
Я знаю: нежного Парни\\
Перо не в моде в наши дни.\\

XXX

Певец Пиров и грусти томной,\\
Когда б еще ты был со мной,\\
Я стал бы просьбою нескромной\\
Тебя тревожить, милый мой:\\
Чтоб на волшебные напевы\\
Переложил ты страстной девы\\
Иноплеменные слова.\\
Где ты? приди: свои права\\
Передаю тебе с поклоном…\\
Но посреди печальных скал,\\
Отвыкнув сердцем от похвал,\\
Один, под финским небосклоном,\\
Он бродит, и душа его\\
Не слышит горя моего.\\

XXXI

Письмо Татьяны предо мною;\\
Его я свято берегу,\\
Читаю с тайною тоскою\\
И начитаться не могу.\\
Кто ей внушал и эту нежность,\\
И слов любезную небрежность\marginnote{negligenza}?\\
Кто ей внушал умильный вздор,\\
Безумный сердца разговор,\\
И увлекательный и вредный?\\
Я не могу понять. Но вот\\
Неполный, слабый перевод,\\
С живой картины список бледный\\
Или разыгранный Фрейшиц\\
Перстами робких учениц:\\
Письмо Татьяны к Онегину\\
Я к вам пишу — чего же боле?\\
Что я могу еще сказать?\\
Теперь, я знаю, в вашей воле\\
Меня презреньем наказать.\\
Но вы, к моей несчастной доле\\
Хоть каплю жалости храня,\\
Вы не оставите меня.\\
Сначала я молчать хотела;\\
Поверьте: моего стыда\\
Вы не узнали б никогда,\\
Когда б надежду я имела\\
Хоть редко, хоть в неделю раз\\
В деревне нашей видеть вас,\\
Чтоб только слышать ваши речи,\\
Вам слово молвить, и потом\\
Все думать, думать об одном\\
И день и ночь до новой встречи.\\
Но, говорят, вы нелюдим;\\
В глуши, в деревне все вам скучно,\\
А мы… ничем мы не блестим,\\
Хоть вам и рады простодушно.\\
Зачем вы посетили нас?\\
В глуши забытого селенья\\
Я никогда не знала б вас,\\
Не знала б горького мученья.\\
Души неопытной волненья\\
Смирив со временем (как знать?),\\
По сердцу я нашла бы друга,\\
Была бы верная супруга\\
И добродетельная мать.\\
Другой!.. Нет, никому на свете\\
Не отдала бы сердца я!\\
То в вышнем суждено совете…\\
То воля неба: я твоя;\\
Вся жизнь моя была залогом\\
Свиданья верного с тобой;\\
Я знаю, ты мне послан богом,\\
До гроба ты хранитель мой…\\
Ты в сновиденьях мне являлся\\
Незримый, ты мне был уж мил,\\
Твой чудный взгляд меня томил,\\
В душе твой голос раздавался\\
Давно… нет, это был не сон!\\
Ты чуть вошел, я вмиг узнала,\\
Вся обомлела, запылала\\
И в мыслях молвила: вот он!\\
Не правда ль? я тебя слыхала:\\
Ты говорил со мной в тиши,\\
Когда я бедным помогала\\
Или молитвой услаждала\\
Тоску волнуемой души?\\
И в это самое мгновенье\\
Не ты ли, милое виденье,\\
В прозрачной темноте мелькнул,\\
Приникнул тихо к изголовью?\\
Не ты ль, с отрадой и любовью,\\
Слова надежды мне шепнул?\\
Кто ты, мой ангел ли хранитель,\\
Или коварный искуситель:\\
Мои сомненья разреши.\\
Быть может, это все пустое,\\
Обман неопытной души!\\
И суждено совсем иное…\\
Но так и быть! Судьбу мою\\
Отныне я тебе вручаю,\\
Перед тобою слезы лью,\\
Твоей защиты умоляю…\\
Вообрази: я здесь одна,\\
Никто меня не понимает,\\
Рассудок мой изнемогает,\\
И молча гибнуть я должна.\\
Я жду тебя: единым взором\\
Надежды сердца оживи\\
Иль сон тяжелый перерви,\\
Увы, заслуженным укором!\\
Кончаю! Страшно перечесть…\\
Стыдом и страхом замираю…\\
Но мне порукой ваша честь,\\
И смело ей себя вверяю…\\

XXXII

Татьяна то вздохнет, то охнет\marginnote{Воскликнуть ох, выражая какое-л. ощущение, чувство: боль, досаду, испуг и т. п.}\\
Письмо дрожит в ее руке;\\
Облатка розовая сохнет\\
На воспаленном\marginnote{infiammata} языке.\\
К плечу головушкой склонилась,\\
Соро'чка\\marginnote{intimo} легкая спустилась\\
С ее прелестного плеча…\\
Но вот уж лунного луча\\
Сиянье гаснет. Там долина\marginnote{Ровное пространство вдоль речного русла}\\
Сквозь пар яснеет. Там поток\\
Засеребрился; там рожок\\
Пастуший будит селянина.\marginnote{Крестьянин}\\
Вот утро: встали все давно,\\
Моей Татьяне все равно.\\

XXXIII

Она зари не замечает,\\
Сидит с поникшею главой\\
И на письмо не напирает\\
Своей печати вырезной.\\
Но, дверь тихонько отпирая,\\
Уж ей Филипьевна седая\\
Приносит на подносе чай.\\
«Пора, дитя мое, вставай:\\
Да ты, красавица, готова!\\
О пташка ранняя моя!\\
Вечор уж как боялась я!\\
Да, слава богу, ты здорова!\\
Тоски ночной и следу нет,\\
Лицо твое как маков цвет».\\

XXXIV

— Ах! няня, сделай одолженье. —\\
«Изволь, родная, прикажи».\\
— Не думай… право… подозренье…\\
Но видишь… ах! не откажи. —\\
«Мой друг, вот бог тебе порука».\\
— Итак, пошли тихонько внука\\
С запиской этой к О… к тому…\\
К соседу… да велеть ему,\\
Чтоб он не говорил ни слова,\\
Чтоб он не называл меня… —\\
«Кому же, милая моя?\\
Я нынче стала бестолкова.\\
Кругом соседей много есть;\\
Куда мне их и перечесть».\\

XXXV

— Как недогадлива ты, няня! —\\
«Сердечный друг, уж я стара,\\
Стара; тупеет разум, Таня;\\
А то, бывало, я востра,\\
Бывало, слово барской воли…»\\
— Ах, няня, няня! до того ли?\\
Что нужды мне в твоем уме?\\
Ты видишь, дело о письме\\
К Онегину. — «Ну, дело, дело.\\
Не гневайся, душа моя,\\
Ты знаешь, непонятна я…\\
Да что ж ты снова побледнела?»\\
— Так, няня, право ничего.\\
Пошли же внука своего.\\

XXXVI

Но день протек, и нет ответа.\\
Другой настал: все нет как нет.\\
Бледна, как тень, с утра одета,\\
Татьяна ждет: когда ж ответ?\\
Приехал Ольгин обожатель.\\
«Скажите: где же ваш приятель? —\\
Ему вопрос хозяйки был. —\\
Он что-то нас совсем забыл».\\
Татьяна, вспыхнув, задрожала.\\
— Сегодня быть он обещал, —\\
Старушке Ленский отвечал, —\\
Да, видно, почта задержала. —\\
Татьяна потупила взор,\\
Как будто слыша злой укор.\\

XXXVII

Смеркалось; на столе, блистая,\\
Шипел вечерний самовар,\\
Китайский чайник нагревая;\\
Под ним клубился легкий пар.\\
Разлитый Ольгиной рукою,\\
По чашкам темною струею\\
Уже душистый чай бежал,\\
И сливки мальчик подавал;\\
Татьяна пред окном стояла,\\
На стекла хладные дыша,\\
Задумавшись, моя душа,\\
Прелестным пальчиком писала\\
На отуманенном стекле\\
Заветный вензель\marginnote{Особенно ценимый Начальные буквы имени и фамилии} О да Е.\\

XXXVIII

И между тем душа в ней ныла,
И слез был полон томный взор.
Вдруг топот!.. кровь ее застыла.
Вот ближе! скачут… и на двор
Евгений! «Ах!» — и легче тени
Татьяна прыг в другие сени,
С крыльца на двор, и прямо в сад,
Летит, летит; взглянуть назад
Не смеет; мигом обежала
Куртины, мостики, лужок,
Аллею к озеру, лесок,
Кусты сирен переломала,
По цветникам летя к ручью.
И, задыхаясь, на скамью
XXXIX
Упала…
«Здесь он! здесь Евгений!
О боже! что подумал он!»
В ней сердце, полное мучений,
Хранит надежды темный сон;
Она дрожит и жаром пышет,
И ждет: нейдет ли? Но не слышит.
В саду служанки, на грядах,
Сбирали ягоду в кустах
И хором по наказу пели
(Наказ, основанный на том,
Чтоб барской ягоды тайком
Уста лукавые не ели
И пеньем были заняты:
Затея сельской остроты!)
Песня девушек
Девицы, красавицы,
Душеньки, подруженьки,
Разыграйтесь девицы,
Разгуляйтесь, милые!
Затяните песенку,
Песенку заветную,
Заманите молодца
К хороводу нашему,
Как заманим молодца,
Как завидим издали,
Разбежимтесь, милые,
Закидаем вишеньем,
Вишеньем, малиною,
Красною смородиной.
Не ходи подслушивать
Песенки заветные,
Не ходи подсматривать
Игры наши девичьи.
ХL
Они поют, и, с небреженьем
Внимая звонкий голос их,
Ждала Татьяна с нетерпеньем,
Чтоб трепет сердца в ней затих,
Чтобы прошло ланит пыланье.
Но в персях то же трепетанье,
И не проходит жар ланит,
Но ярче, ярче лишь горит…
Так бедный мотылек и блещет
И бьется радужным крылом,
Плененный школьным шалуном;
Так зайчик в озими трепещет,
Увидя вдруг издалека
В кусты припадшего стрелка.
ХLI
Но наконец она вздохнула
И встала со скамьи своей;
Пошла, но только повернула
В аллею, прямо перед ней,
Блистая взорами, Евгений
Стоит подобно грозной тени,
И, как огнем обожжена,
Остановилася она.
Но следствия нежданной встречи
Сегодня, милые друзья,
Пересказать не в силах я;
Мне должно после долгой речи
И погулять и отдохнуть:
Докончу после как-нибудь.

\section{Analisi cap.3}
Строфа I - II   П. вводит в поэтическое произведение буднучную прозаичесукую беседу