\chapter{Capitolo 2 I grafici}

\section{Il grafico a candele}
Il grafico a candele è come una mappa, più informazioni fornisce, maggiori sono le possibilità di raggiungere la destinazione in sicurezza.  Saper leggere la mappa è fondamentale per aumentare la propria statistica.
Il grafico a candele rappresenta graficamente la forza della domanda e dell'offerta, l'andamento del mmercato, importanti indizi di inversione di trend.

\subsection{La psicologia del mercato.}
L'aspetto psicologico del mercato si riflette nel grafico a candele.
E' importante considerare le seguenti 3 regole:
\paragraph{regola nr 1}
Quando vedi una tendenza rialzista/ribassista,considerala un'opportunità di vendita/acquisto.
\paragraph{regola nr 2}
Quando senti notizie rialziste o ribassiste, non fare trading su di loro.
\paragraph{regola nr 3}
Non dire agli altri ciò che stai per fare nel mercato; prendi le tue decisioni solo guardando il mercato.

\subsection{La candela}
Composta da una sezione rettangolare e due linee sottili sopra o sotto registra il movimento del prezzo rispetto alla sessione temporale del grafico.
La sezione rettangolare è chiamata corpo reale e rappresenta l'intervallo di prezzo tra apertura e chiusura della sessione temporale, le singole linee sopra e sotto il corpo questa
sezione. Vediamo perché questi sono chiamati grafici a candela; le singole
linee spesso sembrano candele con i loro stoppini. La parte rettangolare della
linea delle candele è chiamata corpo reale. Rappresenta l'intervallo tra
l'apertura e la chiusura della sessione. Quando il corpo reale è nero (ad esempio, riempito),
mostra che la chiusura della sessione è stata inferiore all'apertura. Se il corpo reale è bianco (cioè, vuoto), significa che la chiusura è stata superiore all'apertura.
 
\section{le medie mobili}