\input{stfi_2022arx.utf8.sty}



% Вслучае необходимости, здесь можно вставить ТОЛЬКО самые необходимые пакеты 
% (не надо подключать все подряд)
\usepackage{graphicx}

\begin{document}

\Title%%%%	Названия статьи (все название статьи КАПСОМ писать не нужно)
	{Капитанская Дочка} % Краткое Название статьи на русском языке для колонтитулов
	{Капитанская Дочка} % Название статьи на русском языке 


\Abstract{Главная русская идиллия: история любви и спасения на фоне беспощадного бунта. O том, что любовь и милость важнее справедливого возмездия, а добрый нрав и чистое сердце спасают даже в самые жестокие времена.}%

%%%	Информация о первом авторе:
%%% В контактной информации об авторе указывается:
%%% Электронная почта
%%% Фамилия, Имя, Отчество (обязательно полностью), ученая степень (по желанию),
%%% ученое звание (по желанию), должность (обязательно), место работы (обязательно)
%%% не обязательно подробно (кафедра, факультет, отдел), достаточно лишь название организации,
%%% адрес организации (обязательно) -  улица, дом, город, индекс, страна, ,.
%%% Не нужно указывать домашний адрес, нужно указывать рабочий адрес!!!

\Author%
{Fabio Rizzi} % обратите внимание, как оформлены инициалы и пробел перед фамилией
{\textbf{Иванов Иван Иванович}, к.ф.-м.н., доцент,  первый университет, ул. Такая, д. 00, г. Первый, 000000, Россия.}
{email@mail.ru} % E-mail
{I.\,I.~Ivanov}% обратите внимание, как оформлены инициалы и пробел перед фамилией
{\textbf{Ivanov Ivan Ivanovich}, Ph.D., Associate Professor, First University, Takaya st., 00, First-City, 000000, Russia.}
{a} % индексы для аффилиации
%{a,b} если два места работы, то пишется два индекса для каждой аффилиации
{1} %первый автор, поэтому пишем номер: 1

%	второй автор, а затем и последующие авторы аналогично первому (если есть):
\Author%
{П.\,П.~Петров}% обратите внимание, как оформлены инициалы и пробел перед фамилией
{\textbf{Петров Петр Петрович}, к.ф.-м.н., профессор,  второй университет, ул. Другая, д. 01, г. Новый, 012340, Россия; доцент, первый университет, ул. Такая, д. 00, г. Первый, 000000, Россия.}
{email2@inbox.ru} % E-mail второго автора
{P.\,P.~Petrov} % обратите внимание, как оформлены инициалы и пробел перед фамилией
{\textbf{Petrov Petr Petrovich}, Ph.D., Professor, Second University, Another st., 01, New-City, 012340, Russia; Associate Professor, First University, Takaya st., 00, First-City, 000000, Russia.}
{b,a} %если два места работы, то пишется два индекса для каждой аффилиации
{2} %второй автор, поэтому пишем номер: 2 (и так для каждого автора)

%%%	Список аффилиаций:
\affili {a} % символ индекса первой аффилиации
				{первый университет, г. Первый, 000000, Россия.}% на русском языке
				{First University, First-City, 000000, Russia.}% на английском языке
\affili {b} % символ индекса  второй аффилиации
				{второй университет, г. Новый, 012340, Россия.} % на русском языке
				{Second University,  New-City, 012340, Russia.} % на английском языке

%%%	Проставляет редактор!!!!!!:
\DOI {00.000000/issn2226-8812.0000.0.0-00}

%%% Здесь могут быть "свои" макрокоманды, например так
\newcommand{\pX}{{\mathcal X}}
\let\msf=\mathsf
\newcommand{\var}{\mathop{\sf Var}}
%%% однако их количество не должно быть большим.
%%% здесь нельзя вставлять сокращения, которые не используются.



%%% Шапка оформления НЕ ТРОГАТЬ!!! %%%
%%%%%%%%%%%%%%%%%%%%%%%%%%%%%%%%%%%%%%
	\Header
	\captionsenglish
	\SubHeader
	\captionsrussian
%%%%%%%%%%%%%%%%%%%%%%%%%%%%%%%%%%%%%%

\section*{Введение}

Текст статьи Текст статьи Текст статьи Текст статьи Текст статьи Текст статьи Текст статьи Текст статьи Текст статьи Текст статьи Текст статьи Текст статьи Текст статьи Текст статьи 

Текст статьи Текст статьи Текст статьи Текст статьи Текст статьи Текст статьи Текст статьи Текст статьи Текст статьи Текст статьи Текст статьи Текст статьи Текст статьи Текст статьи Текст статьи Текст статьи Текст статьи Текст статьи Текст статьи Текст статьи Текст статьи Текст статьи Текст статьи Текст статьи Текст статьи Текст статьи Текст статьи Текст статьи Текст статьи Текст статьи Текст статьи Текст статьи Текст статьи Текст статьи 

Продолжение текста статьи Продолжение текста статьи Продолжение текста статьи Продолжение текста статьи Продолжение текста статьи Продолжение текста статьи Продолжение текста статьи Продолжение текста статьи Продолжение текста статьи Продолжение текста статьи Продолжение текста статьи Продолжение текста статьи Продолжение текста статьи Продолжение текста статьи Продолжение текста статьи Продолжение текста статьи Продолжение текста статьи Продолжение текста статьи Продолжение текста статьи 

\section{Название первого раздела}


Обратите внимание и проверьте, что в контактной информации об авторе указывается:
\begin{itemize}
	\item Фамилия, Имя, Отчество (обязательно полностью);
	\item ученая степень (по желанию);
	\item ученое звание (по желанию);
	\item должность (обязательно);
	\item место работы (обязательно);
не обязательно подробно (кафедра, факультет, отдел), достаточно лишь название организации;
	\item адрес организации (обязательно) -  улица, дом, город, индекс, страна;
	\item Не нужно указывать домашний адрес, нужно указывать рабочий адрес;
	\item Электронная почта;
	\item Все данные продублированы на английском языке.
\end{itemize}

Пример формулы 
\begin{equation}
\frac{\partial A}{\partial b}=\frac{1}{\sin \alpha } + 1
\end{equation}

Пример формулы 
\begin{equation}
\frac{\partial A}{\partial b}=\frac{1}{\sin \alpha } + 1
\end{equation}


\section{Название второго раздела}
Продолжение текста статьи Продолжение текста статьи Продолжение текста статьи Продолжение текста статьи Продолжение текста статьи Продолжение текста статьи Продолжение текста статьи Продолжение текста статьи Продолжение текста статьи Продолжение текста статьи Продолжение текста статьи Продолжение текста статьи... 

Пример формулы 
\begin{equation}
\frac{\partial A}{\partial b}=\frac{1}{\sin \alpha } + 1
\end{equation}


Пример формулы 
\begin{equation}
\frac{\partial A}{\partial b}=\frac{1}{\sin \alpha } + 1
\end{equation}

\noindent 
Продолжение текста статьи $H(t)=\dot{a}(t)/a(t)$ Продолжение текста статьи Продолжение текста статьи Продолжение текста статьи Продолжение текста статьи Продолжение текста статьи Продолжение текста статьи Продолжение текста статьи Продолжение текста статьи Продолжение текста статьи Продолжение текста статьи Продолжение текста статьи Продолжение текста статьи Продолжение текста статьи Продолжение текста статьи Продолжение текста статьи Продолжение текста статьи Продолжение текста статьи Продолжение текста статьи 

Пример оформления одиночного рисунка

\begin{figure}[ht]
\centerline{\includegraphics[width=3.83in,height=2.38in]{ris_01_a.png}}
\caption{название рисунка}
\label{fig1}
\end{figure}

Продолжение текста статьи Продолжение текста статьи Продолжение текста статьи Продолжение текста статьи Продолжение текста статьи Продолжение текста статьи Продолжение текста статьи Продолжение текста статьи Продолжение текста статьи Продолжение текста статьи Продолжение текста статьи Продолжение текста статьи Продолжение текста статьи Продолжение текста статьи Продолжение текста статьи Продолжение текста статьи Продолжение текста статьи Продолжение текста статьи Продолжение текста статьи 

Пример размещения двух рисунков
\begin{figure}[ht!]
\begin{minipage}
[b]{0.47\linewidth}
\center{\includegraphics[width=1\linewidth]{ris_01_a.png}} a) \\
\end{minipage}
\hfill
\begin{minipage}[b]{0.5\linewidth}
\center{\includegraphics[width=1\linewidth]{ris_01_b.png}} b) \\
\end{minipage}
\caption{а) Пространственное распределение, b) График повторяемости в энергетической}\label{fig:RC2}
\end{figure}

Продолжение текста статьи Продолжение текста статьи Продолжение текста статьи Продолжение текста статьи Продолжение текста статьи Продолжение текста статьи Продолжение текста статьи Продолжение текста статьи Продолжение текста статьи Продолжение текста статьи Продолжение текста статьи Продолжение текста статьи Продолжение текста статьи Продолжение текста статьи Продолжение текста статьи Продолжение текста статьи Продолжение текста статьи Продолжение текста статьи Продолжение текста статьи 

\section*{Заключение}

Продолжение текста статьи Продолжение текста статьи Продолжение текста статьи Продолжение текста статьи Продолжение текста статьи Продолжение текста статьи Продолжение текста статьи Продолжение текста статьи Продолжение текста статьи Продолжение текста статьи Продолжение текста статьи Продолжение текста статьи Продолжение текста статьи Продолжение текста статьи Продолжение текста статьи Продолжение текста статьи Продолжение текста статьи Продолжение текста статьи Продолжение текста статьи 


%%%%%%%%%%%%%%%%%%%%%%%%%%%%%%%%%%%%%%%%%%%%%%%%%%%%%%%%%%%%%%%%%%%%%%%%%%%%%%%%%%%%%%%%%%%

\begin{thebibliography}{99}

\bibitem{Singe}
Синг Дж. 
Общая теория относительности. 
М.: ИЛ, 1963. С. 247--248 

\bibitem{Zakh}
Захаров В.Д. 
Гравитационные волны в теории тяготения Эйнштейна (Современные проблемы физики). 
М.: Наука, 1972. С. 119.

\bibitem{Kennel1} 
Kennel C.F., Petschek H.E. 
Limit on stably trapped particle fluxes. 
{\it J. Geophys. Res.} 1966. V. 71. № 1. 1--14~pp.

\bibitem{Lyons2} 
Lyons L.R., Williams D.J. 
Quantitative aspects of magnetospheric physics. 
N.Y.: Springer, 1984. 312 p.

\bibitem{Nishida1} 
Nishida A. Geomagnetic diagnosis of the magnetosphere. 
N.Y.: Springer-Verlag, 1978. 301 p.

\end{thebibliography}

%%%%%%%%%%%%%%%%%%%%%%%%%%%%%%%%%%%%%%%%%%%%%%%%%%%%%%%%%%%

\begin{otherlanguage}{english}

\begin{thebibliography}{99}

\bibitem{Synge}
Synge J.L. 
{\it Relativity: The General Theory}. 
Amsterdam: North-Holland Publishing Company, 1960.  

%\bibitem{Zakh}
Zakharov V.D. 
{\it Gravitatsionnyye volny v teorii tyagoteniya Eynshteyna (Sovremennyye problemy fiziki)}. 
Moscow: Nauka Publ., 1972. 119 p. (in Russ.)

\bibitem{Kennel} 
Kennel C.F., Petschek H.E. 
Limit on stably trapped particle fluxes. 
{\it J. Geophys. Res.}, 1966, vol. 71, no. 1, pp. 1--14.

\bibitem{Lyons} 
Lyons L.R., Williams D.J. 
{\it Quantitative aspects of magnetospheric physics.} 
N.Y.: Springer, 1984. 312 p.

\bibitem{Nishida} 
Nishida A. 
{\it Geomagnetic diagnosis of the magnetosphere.} 
N.Y.: Springer-Verlag, 1978. 301 p.


\end{thebibliography}

\vspace{10pt}
\begin{otherlanguage}{russian}
\Footer
\end{otherlanguage}

\vspace{10pt}
\FooterSub
\end{otherlanguage}



\end{document}

