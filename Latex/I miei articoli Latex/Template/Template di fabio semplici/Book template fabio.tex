\documentclass[12pt]{book}

% Lingua e pacchetti principali
\usepackage[utf8]{inputenc}  % Codifica dei caratteri
\usepackage[english]{babel}  % Lingua italiana (puoi cambiarla con 'english' se preferisci l'inglese)
\usepackage{amsmath, amssymb}  % Pacchetti matematici
\usepackage{graphicx}  % Inserimento immagini
\usepackage{fancyhdr}  % Per intestazioni personalizzate
\usepackage{hyperref}  % Per i collegamenti ipertestuali
\usepackage{geometry}  % Personalizzazione dei margini

% Personalizzazione dei margini
\geometry{top=3cm,bottom=3cm,left=2.5cm,right=2.5cm}

% Inizio del documento
\begin{document}

% Frontespizio (titolo, autore, data, etc.)
\title{Il Mio Libro}
\author{Nome Autore}
\date{\today}
\maketitle  % Crea il frontespizio

% Sommario
\tableofcontents
\newpage

% Capitolo 1
\chapter{Introduzione}
Benvenuti nel mio libro! Questo è un esempio semplice di un template in LaTeX per un libro. Qui puoi inserire il contenuto del tuo primo capitolo.

% Capitolo 2
\chapter{Secondo Capitolo}
In questo capitolo parliamo di qualcosa di più complesso. Puoi includere anche formule matematiche come:

\begin{equation}
E = mc^2
\end{equation}

% Sezione del Capitolo
\section{Una Sezione}
Questa è una sezione all'interno del capitolo. Puoi aggiungere sottosezioni e paragrafi come desideri.

% Capitolo 3
\chapter{Conclusioni}
Nel capitolo finale, puoi riassumere tutto ciò che hai trattato nel libro. Questo è solo un esempio di base.

% Bibliografia
\newpage
\chapter*{Bibliografia}
\addcontentsline{toc}{chapter}{Bibliografia}
\begin{thebibliography}{9}
\bibitem{latex} Donald E. Knuth, \textit{The TeXbook}, Addison-Wesley, 1986.
\end{thebibliography}

\end{document}
