\documentclass[12pt]{article} % Classe di documento "article" per un report

% Pacchetti necessari
\usepackage[utf8]{inputenc}    % Codifica dei caratteri
\usepackage[italian]{babel}    % Lingua italiana
\usepackage{amsmath}           % Per le formule matematiche
\usepackage{graphicx}          % Per le immagini
\usepackage{geometry}          % Per i margini personalizzati
\usepackage{fancyhdr}          % Per intestazioni e piè di pagina personalizzati
\usepackage{hyperref}          % Per i collegamenti ipertestuali

% Impostazione dei margini
\geometry{top=2cm, bottom=2cm, left=2.5cm, right=2.5cm}

% Intestazioni personalizzate
\pagestyle{fancy}
\fancyhf{}
\fancyhead[L]{Rapporto di Esempio}
\fancyhead[C]{}
\fancyhead[R]{\thepage}

\title{Report Semplice in LaTeX}
\author{Nome Autore}
\date{\today}

\begin{document}

\maketitle  % Crea la pagina del titolo

\tableofcontents  % Crea il sommario
\newpage  % Nuova pagina

% Introduzione
\section{Introduzione}
Questo è un esempio di report scritto in LaTeX. Il LaTeX è un sistema di preparazione di documenti che permette una gestione avanzata dei contenuti e una qualità tipografica eccellente. In questo esempio, vedremo come scrivere un report con sezioni, formule matematiche, e tabelle.

% Sezione 1
\section{Sezione 1: Concetti di Base}
In questa sezione esploreremo alcuni concetti di base. LaTeX è principalmente utilizzato per la scrittura di documenti scientifici e matematici, ma può essere utilizzato anche per altri tipi di documenti.

Ecco un esempio di una formula matematica:

\begin{equation}
E = mc^2
\end{equation}

Questa è la famosa equazione di Einstein, che esprime la relazione tra energia, massa e velocità della luce.

% Sezione 2
\section{Sezione 2: Tabelle}
Le tabelle sono facili da creare in LaTeX. Ecco un esempio di tabella:

\begin{table}[h]
\centering
\begin{tabular}{|c|c|c|}
\hline
\textbf{Parametro} & \textbf{Valore} & \textbf{Unità} \\
\hline
Massa & 5.972 & $10^{24}$ kg \\
Velocità della luce & 299,792,458 & m/s \\
\hline
\end{tabular}
\caption{Tabella di esempio}
\end{table}

% Sezione 3
\section{Sezione 3: Immagini}
LaTeX permette anche di includere immagini nel documento. Ecco come inserire un'immagine:

\begin{figure}[h]
\centering
\includegraphics[width=0.5\textwidth]{esempio.jpg}  % Sostituisci 'esempio.jpg' con il percorso del tuo file immagine
\caption{Esempio di immagine}
\end{figure}

% Conclusioni
\section{Conclusioni}
Questo è un esempio di un report molto semplice creato in LaTeX. È possibile personalizzare ulteriormente il documento con diversi pacchetti e comandi per aggiungere grafici, citazioni, indici, e molto altro.

\end{document}
