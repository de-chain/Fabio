\chapter{capitolo10}\label{section:cap10}


Властитель слабый и лукавый,\\
Плешивый щеголь, враг труда,\\
Нечаянно пригретый славой,\\
Над нами царствовал тогда.

II

Его мы очень смирным знали,\\
Когда не наши повара\\
Орла двуглавого щипали\footnote{ижздивательство над Россиеи}\\
У Бонапартова шатра.

III\footnote{почему Франция не победила}

Гроза двенадцатого года\\
Настала — кто тут нам помог?\\
Остервенение\footnote{патриоттизм} народа,\\
Барклай, зима иль русский бог?

IV

Но бог помог — стал ропот ниже,\\
И скоро силою вещей\footnote{исторические причины не заслуги Александра}\\
Мы очутилися в Париже,\\
А русский царь главой царей.

V

И чем жирнее, тем тяжеле.\\
О русский глупый наш народ,\\
Скажи, зачем ты в самом деле

VI

Авось, о Шиболет народный,
Тебе б я оду посвятил,
Но стихоплет великородный
Меня уже предупредил
Моря достались Албиону

VII

Авось, аренды забывая,\\
Ханжа запрется в монастырь,\\
Авось\footnote{yt,jkmifz yfltòlf} по манью Николая\\
Семействам возвратит Сибирь\\
Авось дороги нам исправят

VIII

Сей муж судьбы, сей странник бранный,\\
Пред кем унизились цари,ёё
Сей всадник, папою венчанный,ёё
Исчезнувший как тень зари,ёё
Измучен казнию покоя

IX

Тряслися грозно Пиренеи,\\
Волкан Неаполя пылал\\
Безрукий князь друзьям Мореи\\
Из Кишинева уж мигал.\\
Кинжал Л, тень Б

Х

Я всех уйму с моим народом, —\\
Наш царь в конгрессе говорил,\\
А про тебя и в ус не дует,\\
Ты александровский холоп\footnote{Аракчеев}

XI\footnote{Семеновскии полк}

Потешный полк Петра Титана,\\
Дружина старых усачей,\\
Предавших некогда тирана\\
Свирепой шайке палачей.

XII

Россия присмирела снова,\\
И пуще царь пошел кутить,\\
Но искра пламени иного\footnote{декабристы}\\
Уже издавна, может быть,

XIII\footnote{русские завтраки у Релеева}

У них свои бывали сходки,\footnote{собрание}\\
Они за чашею вина,\\
Они за рюмкой русской водки

XIV

Витийством резким знамениты,\\
Сбирались члены сей семьи\\
У беспокойного Никиты,\footnote{Муравьев}\\
У осторожного Ильи.\footnote{Долгоруков}

XV

Друг Марса, Вакха и Венеры,\\
Тут Лунин\footnote{деятел} дерзко предлагал\\
Свои решительные меры\footnote{проект царубиство}\\
И вдохновенно бормотал.\\
Читал свои Ноэли\footnote{рождественские куплеты сатира} Пушкин,\\
Меланхолический Якушкин,\\
Казалось, молча обнажал\\
Цареубийственный кинжал.\\
Одну Россию в мире видя,\\
Преследуя свой идеал,\\
Хромой Тургенев им внимал\\
И, плети рабства ненавидя,\\
Предвидел в сей толпе дворян\\
Освободителей крестьян.

XVI

Так было над Невою льдистой,\\
Но там, где ранее весна\\
Блестит над Каменкой тенистой\\
И над холмами Тульчина,\\
Где Витгенштейновы дружины\\
Днепром подмытые равнины\\
И степи Буга облегли,\\
Дела иные уж пошли.\\
Там Пестель  для тиранов\\
И рать  набирал\\
Холоднокровный генерал,\\
И Муравьев, его склоняя,\\
И полон дерзости и сил,\\
Минуты вспышки торопил.

XVII

Сначала эти заговоры\\
Между Лафитом и Клико\\
Лишь были дружеские споры,\\
И не входила глубоко\\
В сердца мятежная наука,\\
Все это было только скука,\\
Безделье молодых умов,\\
Забавы взрослых шалунов,\\
Казалось\\                                                                                                                                             
Узлы к узлам\\                                                                                                                                              
И постепенно сетью тайной\\
Россия\\                                                                                                                                              
Наш царь дремал                                                                                                                                              
