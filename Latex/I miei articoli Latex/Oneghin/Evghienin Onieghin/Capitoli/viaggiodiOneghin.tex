\chapter{viaggiodiOneghin}\label{section:Viaggio}

\textbf{Отрывки из путешествия Онегина}

Последняя глава «Евгения Онегина» издана была особо, с следующим предисловием:\\
«Пропущенные строфы подавали неоднократно повод к порицанию\footnote{обвинение} и насмешкам (впрочем, весьма справедливым и остроумным).\\
Автор чистосердечно признается, что он выпустил из своего романа целую главу, в коей описано было путешествие Онегина по России. От него зависело означить сию выпущенную главу точками или цифром; но во избежание соблазна решился он лучше выставить, вместо девятого нумера, осьмой над последней главою Евгения Онегина и пожертвовать одною из окончательных строф:

Пора: перо покоя просит;\\
Я девять песен написал;\\
На берег радостный выносит\\
Мою ладью девятый вал —\\
Хвала вам, девяти каменам, и проч.

П.А.Катенин (коему прекрасный поэтический талант не мешает быть и тонким критиком) заметил нам, что сие исключение, может быть и выгодное для читателей, вредит, однако ж, плану целого сочинения; ибо чрез то переход от Татьяны, уездной барышни, к Татьяне, знатной даме, становится слишком неожиданным и необъясненным. — Замечание, обличающее опытного художника. Автор сам чувствовал справедливость оного, но решился выпустить эту главу по причинам, важным для него, а не для публики. Некоторые отрывки были напечатаны; мы здесь их помещаем, присовокупив\footnote{aggiungendo} к ним еще несколько строф.

Е. Онегин из Москвы едет в Нижний Новгород:\\
. . . . . . . перед ним\\
Макарьев суетно хлопочет,\\
Кипит обилием своим.\\
Сюда жемчуг привез индеец,\\
Поддельны вины европеец,\\
Табун бракованных коней\\
Пригнал заводчик из степей,\\
Игрок привез свои колоды\\
И горсть услужливых костей,\\
Помещик — спелых дочерей,\\
А дочки — прошлогодни моды.\\
Всяк суетится, лжет за двух,\\
И всюду меркантильный дух.

*

Тоска!..\\
Онегин едет в Астрахань и оттуда на Кавказ.\\
Он видит: Терек своенравный\\
Крутые роет берега;\\
Пред ним парит орел державный,\\
Стоит олень, склонив рога;\\
Верблюд лежит в тени утеса,\\
В лугах несется конь черкеса,\\
И вкруг кочующих шатров\\
Пасутся овцы калмыков,\\
Вдали — кавказские громады:\\
К ним путь открыт. Пробилась брань\\
За их естественную грань,\\
Чрез их опасные преграды;\\
Брега Арагвы и Куры\\
Узрели русские шатры.\\

*

Уже пустыни сторож вечный,\\
Стесненный холмами вокруг,\\
Стоит Бешту остроконечный\\
И зеленеющий Машук,\\
Машук, податель струй целебных;\\
Вокруг ручьев его волшебных\\
Больных теснится бледный рой;\\
Кто жертва чести боевой,\\
Кто почечуя, кто Киприды;\\
Страдалец мыслит жизни нить\\
В волнах чудесных укрепить,\\
Кокетка злых годов обиды\\
На дне оставить, а старик\\
Помолодеть — хотя на миг.

*

Питая горьки размышленья,\\
Среди печальной их семьи,\\
Онегин взором сожаленья\\
Глядит на дымные струи\\
И мыслит, грустью отуманен:\\
Зачем я пулей в грудь не ранен?\\
Зачем не хилый я старик,\\
Как этот бедный откупщик?\\
Зачем, как тульский заседатель,\\
Я не лежу в параличе?\\
Зачем не чувствую в плече\\
Хоть ревматизма? — ах, создатель!\\
Я молод, жизнь во мне крепка;\\
Чего мне ждать? тоска, тоска!..\\
Онегин посещает потом Тавриду:\\
Воображенью край священный:\\
С Атридом спорил там Пилад,\\
Там закололся Митридат,\\
Там пел Мицкевич вдохновенный\\
И посреди прибрежных скал\\
Свою Литву воспоминал.

*

Прекрасны вы, брега Тавриды,\\
Когда вас видишь с корабля\\
При свете утренней Киприды,\\
Как вас впервой увидел я;\\
Вы мне предстали в блеске брачном:\\
На небе синем и прозрачном\\
Сияли груды ваших гор,\\
Долин, деревьев, сёл узор\\
Разостлан был передо мною.\\
А там, меж хижинок татар...\\
Какой во мне проснулся жар!\\
Какой волшебною тоскою\\
Стеснялась пламенная грудь!\\
Но, муза! прошлое забудь.

*\footnote{старое творчество}

Какие б чувства ни таились\\
Тогда во мне — теперь их нет:\\
Они прошли иль изменились...\\
Мир вам, тревоги прошлых лет!\\
В ту пору мне казались нужны\\
Пустыни, волн края жемчужны,\\
И моря шум, и груды скал,\footnote{3 ideali: natura, donne, se stesso}\\
И гордой девы идеал,\\
И безыменные страданья...\\
Другие дни, другие сны;\\
Смирились вы, моей весны\\
Высокопарные мечтанья,\\
И в поэтический бокал\\
Воды я много подмешал.

*\footnote{новое творчество}

Иные нужны мне картины:\\
Люблю песчаный косогор,\\
Перед избушкой две рябины,\\
Калитку, сломанный забор,\\
На небе серенькие тучи,\\
Перед гумном соломы кучи\\
Да пруд под сенью ив густых,\\
Раздолье уток молодых;\\
Теперь мила мне балалайка\\
Да пьяный топот трепака\\
Перед порогом кабака.\\
Мой идеал теперь — хозяйка,\\
Мои желания — покой,\\
Да щей горшок\footnote{незавысимая жизнь}, да сам большой\footnote{большая семья}.

*

Порой дождливою намедни\\
Я, завернув на скотный двор...\\
Тьфу! прозаические бредни,\\
Фламандской школы\footnote{бытовые картины} пестрый сор!\\
Таков ли был я, расцветая?
Скажи, фонтан Бахчисарая!
Такие ль мысли мне на ум
Навел твой бесконечный шум,
Когда безмолвно пред тобою
Зарему я воображал
Средь пышных, опустелых зал...
Спустя три года, вслед за мною,
Скитаясь в той же стороне,
Онегин вспомнил обо мне.

*

Я жил тогда в Одессе пыльной...\\
Там долго ясны небеса,\\
Там хлопотливо торг обильный\\
Свои подъемлет паруса;\\
Там все Европой дышит, веет,
Все блещет югом и пестреет
Разнообразностью живой.
Язык Италии златой
Звучит по улице веселой,
Где ходит гордый славянин,
Француз, испанец, армянин,
И грек, и молдаван тяжелый,
И сын египетской земли,
Корсар в отставке, Морали.
*
Одессу звучными стихами
Наш друг Туманский описал,
Но он пристрастными глазами
В то время на нее взирал.
Приехав, он прямым поэтом
Пошел бродить с своим лорнетом
Один над морем — и потом
Очаровательным пером
Сады одесские прославил.
Все хорошо, но дело в том,
Что степь нагая там кругом;
Кой-где недавный труд заставил
Младые ветви в знойный день
Давать насильственную тень.

*

А где, бишь, мой рассказ несвязный?\\
В Одессе пыльной, я сказал.\\
Я б мог сказать: в Одессе грязной —\\
И тут бы, право, не солгал.\\
В году недель пять-шесть Одесса,\\
По воле бурного Зевеса,\\
Потоплена, запружена,\\
В густой грязи погружена.\\
Все домы на аршин загрязнут,\\
Лишь на ходулях пешеход\\
По улице дерзает вброд;\\
Кареты, люди тонут, вязнут,\\
И в дрожках вол, рога склоня,\\
Сменяет хилого коня.

*

Но уж дробит каменья молот,\\
И скоро звонкой мостовой\\
Покроется спасенный город,\\
Как будто кованой броней.\\
Однако в сей Одессе влажной\\
Еще есть недостаток важный;\\
Чего б вы думали? — воды.\\
Потребны тяжкие труды...\\
Что ж? это небольшое горе,\\
Особенно, когда вино\\
Без пошлины привезено.\\
Но солнце южное, но море...\\
Чего ж вам более, друзья?\\
Благословенные края!

*

Бывало, пушка зоревая\\
Лишь только грянет с корабля,\\
С крутого берега сбегая,\\
Уж к морю отправляюсь я.\\
Потом за трубкой раскаленной,\\
Волной соленой оживленный,\\
Как мусульман в своем раю,\\
С восточной гущей кофе пью.\\
Иду гулять. Уж благосклонный\\
Открыт Casino; чашек звон\\
Там раздается; на балкон\\
Маркёр выходит полусонный\\
С метлой в руках, и у крыльца\\
Уже сошлися два купца.

*

Глядишь — и площадь запестрела.
Все оживилось; здесь и там
Бегут за делом и без дела,
Однако больше по делам.
Дитя расчета и отваги,
Идет купец взглянуть на флаги,
Проведать, шлют ли небеса
Ему знакомы паруса.
Какие новые товары
Вступили нынче в карантин?
Пришли ли бочки жданных вин?
И что чума? и где пожары?
И нет ли голода, войны
Или подобной новизны?
*
Но мы, ребята без печали,
Среди заботливых купцов,
Мы только устриц ожидали
От цареградских берегов.
Что устрицы? пришли! О радость!
Летит обжорливая младость
Глотать из раковин морских
Затворниц жирных и живых,
Слегка обрызгнутых лимоном.
Шум, споры — легкое вино
Из погребов принесено
На стол услужливым Отоном;
Часы летят, а грозный счет
Меж тем невидимо растет.

*

Но уж темнеет вечер синий,\\
Пора нам в оперу скорей:\\
Там упоительный Россини,\\
Европы баловень — Орфей.\\
Не внемля критике суровой,\\
Он вечно тот же, вечно новый,\\
Он звуки льет — они кипят,\\
Они текут, они горят,\\
Как поцелуи молодые,\\
Все в неге, в пламени любви,\\
Как зашипевшего аи\\
Струя и брызги золотые...\\
Но, господа, позволено ль\\
С вином равнять do-re-mi-sol?

*

А только ль там очарований?\\
А разыскательный лорнет?\\
А закулисные свиданья?\\
А prima donna? а балет?\\
А ложа, где, красой блистая,\\
Негоцианка молодая,\\
Самолюбива и томна,\\
Толпой рабов окружена?\\
Она и внемлет и не внемлет\\
И каватине, и мольбам,\\
И шутке с лестью пополам...\\
А муж — в углу за нею дремлет,\\
Впросонках фора закричит,\\
Зевнет и — снова захрапит.

*

Финал гремит; пустеет зала;\\
Шумя, торопится разъезд;\\
Толпа на площадь побежала\\
При блеске фонарей и звезд,\\
Сыны Авзонии\footnote{Италия} счастливой\\
Слегка поют мотив игривый,\\
Его невольно затвердив,\\
А мы ревем речитатив.\\
Но поздно. Тихо спит Одесса;\\
И бездыханна и тепла\\
Немая ночь. Луна взошла,\\
Прозрачно-легкая завеса\\
Объемлет небо. Все молчит;\\
Лишь море Черное шумит...\\

*

Итак, я жил тогда в Одессе...\\