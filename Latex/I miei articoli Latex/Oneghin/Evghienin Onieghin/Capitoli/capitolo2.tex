\chapter{} % Capitolo di Metodologia
\label{cap2} % Etichetta per riferimenti incrociati

\textit{O rus!..\\
Hor.\\ 
(О Русь!)}\\

I

Деревня, где скучал Евгений,\\
Была прелестный уголок;\\
Там друг невинных наслаждений\\
Благословить бы небо мог.\\
Господский дом уединенный,\\
Горой от ветров огражденный,\\
Стоял над речкою. Вдали\\
Пред ним пестрели и цвели\\
Луга и нивы золотые,\\
Мелькали селы; здесь и там\\
Стада бродили по лугам,\\
И сени расширял густые\\
Огромный, запущенный сад,\\
Приют задумчивых \emph{дриад.}\footnote{в греческой мифологии нимфы, покровительницы деревьев}

II

Почтенный замок был построен,\\
Как замки строиться должны:\\
\emph{Отменно}\footnote{Оценочная характеристика чего-либо как очень хорошего, превосходного.} прочен и спокоен\\
Во вкусе умной старины.\\
Везде высокие покои,\\
В гостиной \emph{штофные}\marginpar{\textit{Damasco}} обои,\\
Царей портреты на стенах,\\
И печи в пестрых изразцах.\\
Все это ныне \emph{обветшало,}\marginpar{fatiscente}\\
Не знаю, право, почему;\\
Да, впрочем, другу моему\\
В том нужды было очень мало,\\
Затем, что он равно зевал\\
Средь модных и старинных зал.\\

III

Он в том покое поселился,\\
Где деревенский старожил\\
Лет сорок с ключницей бранился,\\
В окно смотрел и мух давил.\\
Все было просто: пол дубовый,\\
Два шкафа, стол, диван пуховый,\\
Нигде ни пятнышка чернил.\\
Онегин шкафы отворил;\\
В одном нашел тетрадь расхода,\\
В другом наливок целый строй,\\
Кувшины с яблочной водой\\
И календарь осьмого года:\\
Старик, имея много дел,\\
В иные книги не глядел.\\

IV

Один среди своих владений,\\
\textit{Чтоб только время проводить,\\
Сперва задумал наш Евгений\\
Порядок новый учредить.\\
В своей глуши мудрец пустынный,\\
Ярем он барщины старинной\\
Оброком легким заменил;\\
И раб судьбу благословил.}\footnote{При барщине крестьяне бесплатно работали на земле своего хозяина-крепостника, в лучшем случае им давали месячину - продовольственный паёк. Производительность труда, сами понимаете, крайне невелика, невелики и доходы. Онегин разделил свои владения между крестьянами, которые теперь сами обрабатывали землю, крутились как могли, и платили своему хозяину оброк. На эту же тему см. тургеневский "Хорь и Калиныч" из "Записок охотника". Оброк, кстати, тоже большого дохода не приносил - в среднем по 5 рублей с крестьянского двора в год. 
Сроки барщины были неопределёнными. Да, император Павел Первый в 1797 г. попытался запретить крестьянам работать в праздники и ограничил барщину 3 днями в неделю, но фактически указ не выполнялся. По закону крестьянин должен был обрабатывать столько же помещичьей земли, сколько ему выделялось на личное хозяйство. Точнее, помещик выделял землю крестьянской общине, а делили её между собой уже на сходе. }\\
Зато в углу своем надулся,\\
Увидя в этом страшный вред,\\\
Его расчетливый сосед;\\
Другой лукаво улыбнулся,\\
И в голос все решили так,\\
Что он опаснейший чудак.\\

V

Сначала все к нему езжали;\\
Но так как с заднего крыльца\\
Обыкновенно подавали\\
Ему донского жеребца,\\
Лишь только вдоль большой дороги\\
Заслышат их домашни дроги, —\\
Поступком оскорбясь таким,\\
Все дружбу прекратили с ним.\footnote{Вначале, сразу после того, как Онегин поселился в поместье умершего дяди, все соседи стали ездить к нему в гости с визитами - познакомиться или поддержать знакомство. Так у них было принято. 
Но Онегин ни с кем из сельских соседей дружить не хотел. И даже просто поддерживать какие-либо отношения тоже не хотел. Поэтому, когда с дороги, ведущей к его дому, слышался звук приближающегося экипажа (домашних дрог, дрожек, то есть, коляски, сделанной местным крепостным умельцем), Онегин требовал верхового коня и уезжал из дома на прогулку. Причем уезжал с заднего крыльца, чтобы попасть на другую дорогу, на которой он не повстречался бы с непрошеными гостями.
Соседи поняли все правильно, обиделись и прекратили дружбу с ним.}\\
«Сосед наш неуч; сумасбродит;\\
Он фармазон; он пьет одно\\
Стаканом красное вино;\\
Он дамам к ручке не подходит;\\
Все да да нет; не скажет да-с\\
Иль нет-с». Таков был общий глас.\\

VI\footnote{характер Ленского. Изъять из Быта}

В свою деревню в ту же пору\\
Помещик новый прискакал\\
И столь же строгому разбору\\
В соседстве повод подавал:\\
По имени \emph{Владимир Ленской,}\\
С душою прямо геттингенской,\\
Красавец, в полном цвете лет,\\
Поклонник Канта и поэт.\\
Он из Германии туманной\\
Привез учености плоды:\\
Вольнолюбивые мечты,\\
Дух пылкий\marginpar{Сильно чувствующий} и довольно странный,\\
Всегда восторженную\marginpar{легко приходит в состояние восторга...} речь\\
И кудри черные до плеч.\\

VII

От хладного разврата света\\
Еще увянуть не успев,\\
Его душа была согрета\\
Приветом друга, лаской дев;\\
Он сердцем милый был невежда,\\
Его лелеяла надежда,\\
И мира новый блеск и шум\\
Еще пленяли юный ум.\\
Он забавлял мечтою сладкой\\
Сомненья сердца своего;\\
Цель жизни нашей для него\\
Была заманчивой загадкой,\\
Над ней он голову ломал\\
И чудеса подозревал.\\

VIII

Он верил, что душа родная\\
Соединиться с ним должна,\\
Что, безотрадно изнывая,\\
Его вседневно ждет она;\\
Он верил, что друзья готовы\\
За честь его приять оковы\\
И что не дрогнет их рука\\
Разбить сосуд клеветника;\\
Что есть избранные судьбами,\\
Людей священные друзья;\\
Что их бессмертная семья\\
Неотразимыми лучами\\
Когда-нибудь нас озарит\\
И мир блаженством одарит.\\

IX

Негодованье, сожаленье,\\
Ко благу чистая любовь\\
И славы сладкое мученье\\
В нем рано волновали кровь.\\
Он с лирой странствовал на свете;\\
Под небом Шиллера и Гете\\
Их поэтическим огнем\\
Душа воспламенилась в нем;\\
И муз возвышенных искусства,\\
Счастливец, он не постыдил:\\
Он в песнях гордо сохранил\\
Всегда возвышенные чувства,\\
Порывы девственной мечты\\
И прелесть важной простоты.\\

X

Он пел любовь, любви послушный,\\
И песнь его была ясна,\\
Как мысли девы простодушной,\\
Как сон младенца, как луна\\
В пустынях неба безмятежных,\\
Богиня тайн и вздохов нежных.\\
Он пел разлуку и печаль,\\
И нечто, и туманну даль,\\
И романтические розы;\\
Он пел те дальные страны,\\
Где долго в лоно тишины\\
Лились его живые слезы;\\
Он пел поблеклый жизни цвет\\
Без малого в осьмнадцать лет.\\

XI

В пустыне, где один Евгений\\
Мог оценить его дары,\\
Господ соседственных селений\\
Ему не нравились пиры;\\
Бежал он их беседы шумной.\\
Их разговор благоразумный\\
О сенокосе, о вине,\\
О псарне, о своей родне,\\
Конечно, не блистал ни чувством,\\
Ни поэтическим огнем,\\
Ни остротою, ни умом,\\
Ни общежития искусством;\\
Но разговор их милых жен\\
Гораздо меньше был умен.\\

XII

Богат, хорош собою, Ленский\\
Везде был принят как жених;\\
Таков обычай деревенский;\\
Все дочек прочили своих\\
За полурусского соседа;\\
Взойдет ли он, тотчас беседа\\
Заводит слово стороной\\
О скуке жизни холостой;\\
Зовут соседа к самовару,\\
А Дуня разливает чай;\\
Ей шепчут: «Дуня, примечай!»\marginpar{за кем-чем. Наблюдать, следить}\\
Потом приносят и гитару:\\
И запищит она (бог мой!):\\
Приди в чертог ко мне златой!..\\

XIII

Но Ленский, не имев, конечно,\\
Охоты узы\marginpar{vincoli} брака несть,\\
С Онегиным желал сердечно\\
Знакомство покороче свесть.\\
Они сошлись. Волна и камень,\\
Стихи и проза, лед и пламень\\
Не столь различны меж собой.\\
Сперва взаимной разнотой\\
Они друг другу были скучны;\\
Потом понравились; потом\\
Съезжались каждый день верхом\\
И скоро стали неразлучны.\\
Так люди (первый каюсь я)\\
От делать нечего друзья.\\

XIV

Но дружбы нет и той меж нами.\\
Все предрассудки истребя,\\
Мы почитаем всех нулями,\\
А единицами — себя.\\
Мы все глядим в Наполеоны;\\
Двуногих тварей миллионы\\
Для нас орудие одно;\\
Нам чувство дико и смешно.\\
Сноснее многих был Евгений;\\
Хоть он людей, конечно, знал\\
И вообще их презирал, —\\
Но (правил нет без исключений)\\
Иных он очень отличал\\
И вчуже чувство уважал.\\

XV

\emph{Он слушал Ленского с улыбкой.\\
Поэта пылкий разговор,\\
И ум, еще в сужденьях зыбкой,\\
И вечно вдохновенный взор, —\\
Онегину все было ново;\\
Он охладительное слово\\
В устах старался удержать\\
И думал: глупо мне мешать\\
Его минутному блаженству;\\
И без меня пора придет;\\
Пускай покамест он живет\\
Да верит мира совершенству;\\
Простим горячке юных лет\\
И юный жар и юный бред.}\\

XVI

Меж ими все рождало споры\\
И к размышлению влекло:\\
Племен минувших договоры,\\
Плоды наук, добро и зло,\\
И предрассудки вековые,\\
И гроба тайны роковые,\\
Судьба и жизнь в свою чреду,\\
Все подвергалось их суду.\\
Поэт в жару своих суждений\\
Читал, забывшись, между тем\\
Отрывки северных поэм,\\
И снисходительный Евгений,\\
Хоть их не много понимал,\\
Прилежно юноше внимал.\\

XVII

Но чаще занимали страсти\\
Умы пустынников моих.\\
Ушед от их мятежной власти,\\
Онегин говорил об них\\
С невольным вздохом сожаленья:\\
Блажен, кто ведал их волненья\\
И наконец от них отстал;\\
Блаженней тот, кто их не знал,\\
Кто охлаждал любовь — разлукой,\\
Вражду — злословием; порой\\
Зевал с друзьями и с женой,\\
Ревнивой не тревожась мукой,\\
И дедов верный капитал\\
Коварной двойке не вверял.\\

XVIII

Когда прибегнем мы под знамя\\
Благоразумной тишины,\\
Когда страстей угаснет пламя,\\
И нам становятся смешны\\
Их своевольство иль порывы\\
И запоздалые отзывы, —\\
Смиренные не без труда,\\
Мы любим слушать иногда\\
Страстей чужих язык мятежный,\\
И нам он сердце шевелит.\\
Так точно старый инвалид\\
Охотно клонит слух прилежный\\
Рассказам юных усачей,\\
Забытый в хижине своей.\\

XIX

Зато и пламенная младость\\
Не может ничего скрывать.\\
Вражду, любовь, печаль и радость\\
Она готова разболтать.\\
В любви считаясь инвалидом,\\
Онегин слушал с важным видом,\\
Как, сердца исповедь любя,\\
Поэт высказывал себя;\\
Свою доверчивую совесть\\
Он простодушно обнажал.\\
Евгений без труда узнал\\
Его любви младую повесть,\\
Обильный чувствами рассказ,\\
Давно не новыми для нас.\\

XX

Ах, он любил, как в наши лета\\
Уже не любят; как одна\\
Безумная душа поэта\\
Еще любить осуждена:\\
Всегда, везде одно мечтанье,\\
Одно привычное желанье,\\
Одна привычная печаль.\\
Ни охлаждающая даль,\\
Ни долгие лета разлуки,\\
Ни музам данные часы,\\
Ни чужеземные красы,\\
Ни шум веселий, ни науки\\
Души не изменили в нем,\\
Согретой девственным огнем.\\

XXI

Чуть отрок, Ольгою плененный,\\
Сердечных мук еще не знав,\\
Он был свидетель умиленный\\
Ее младенческих забав;\\
В тени хранительной дубравы\\
Он разделял ее забавы,\\
И детям прочили венцы\\
Друзья-соседы, их отцы.\\
В глуши, под сению смиренной,\\
Невинной прелести полна,\\
В глазах родителей, она\\
Цвела, как ландыш потаенный,\\
Незнаемый в траве глухой\\
Ни мотыльками, ни пчелой.\\

XXII

Она поэту подарила\\
Младых восторгов первый сон,\\
И мысль об ней одушевила\\
Его цевницы первый стон.\\
Простите, игры золотые!\\
Он рощи полюбил густые,\\
Уединенье, тишину,\\
И ночь, и звезды, и луну,\\
Луну, небесную лампаду,\\
Которой посвящали мы\\
Прогулки средь вечерней тьмы,\\
И слезы, тайных мук отраду…\\
Но нынче видим только в ней\\
Замену тусклых фонарей.\\

XXIII

Всегда скромна, всегда послушна,\\
Всегда как утро весела,\\
Как жизнь поэта простодушна,\\
Как поцелуй любви мила;\\
Глаза, как небо, голубые,\\
Улыбка, локоны льняные,\\
Движенья, голос, легкий стан,\\
Все в Ольге… но любой роман\\
Возьмите и найдете верно\\
Ее портрет: он очень мил,\\
Я прежде сам его любил,\\
Но надоел он мне безмерно.\\
Позвольте мне, читатель мой,\\
Заняться старшею сестрой.\\

XXIV

Ее сестра звалась Татьяна…\\
Впервые именем таким\\
Страницы нежные романа\\
Мы своевольно освятим.\\
И что ж? оно приятно, звучно;\\
Но с ним, я знаю, неразлучно\\
Воспоминанье старины\\
Иль девичьей! Мы все должны\\
Признаться: вкусу очень мало\\
У нас и в наших именах\\
(Не говорим уж о стихах);\\
Нам просвещенье не пристало,\\
И нам досталось от него\\
Жеманство, — больше ничего.\\

XXV

Итак, она звалась Татьяной.\\
Ни красотой сестры своей,\\
Ни свежестью ее румяной\\
Не привлекла б она очей.\\
Дика, печальна, молчалива,\\
\emph{Как лань лесная боязлива,\\
Она в семье своей родной}\\
Казалась девочкой чужой.\\
Она ласкаться не умела\\
К отцу, ни к матери своей;\\
Дитя сама, в толпе детей\\
Играть и прыгать не хотела\\
И часто целый день одна\\
Сидела молча у окна.\\

XXVI\footnote{[задумчивый мичтательный характер Татьяны}

Задумчивость, ее подруга\\
От самых колыбельных дней,\\
Теченье сельского досуга\\
Мечтами украшала ей.\\
Ее изнеженные пальцы\\
Не знали игл; склонясь на пяльцы,\\
Узором шелковым она\
Не оживляла полотна.\\
Охоты властвовать примета,\\
С послушной куклою дитя\\
Приготовляется шутя\\
К приличию — закону света,\\
И важно повторяет ей\\
Уроки маменьки своей.\\

XXVII

Но куклы даже в эти годы\\
Татьяна в руки не брала;\\
Про вести города, про моды\\
Беседы с нею не вела.\\
И были детские проказы\\
Ей чужды: страшные рассказы\\
Зимою в темноте ночей\\
Пленяли больше сердце ей.\\
Когда же няня собирала\\
Для Ольги на широкий луг\\
Всех маленьких ее подруг,\\
Она в горелки не играла,\\
Ей скучен был и звонкий смех,\\
И шум их ветреных утех.\\

XXVIII

Она любила на балконе\\\
Предупреждать зари восход,\\
Когда на бледном небосклоне\
Звезд исчезает хоровод,\\
И тихо край земли светлеет,\\
И, вестник утра, ветер веет,\\
И всходит постепенно день.\\
Зимой, когда ночная тень\\
Полмиром доле обладает,\\
И доле в праздной тишине,\\
При отуманенной луне,\\
Восток ленивый почивает,\\
В привычный час пробуждена\\
Вставала при свечах она.\\

XXIX

Ей рано нравились романы;\\
Они ей заменяли все;\\
Она влюблялася в обманы\\
И Ричардсона и Руссо.\\
Отец ее был добрый малый,\\
В прошедшем веке запоздалый;\\
Но в книгах не видал вреда;\\
\emph{Он, не читая никогда,\\
Их почитал пустой игрушкой}\\
И не заботился о том,\\
Какой у дочки тайный том\\
Дремал до утра под подушкой.\\
Жена ж его была сама\\
От Ричардсона без ума.\\

XXX

Она любила Ричардсона\\
Не потому, чтобы прочла,\\
Не потому, чтоб Грандисона\\
Она Ловласу предпочла;\\
Но в старину княжна Алина,\\
Ее московская кузина,\\
Твердила часто ей об них.\\
В то время был еще жених\\
Ее супруг, но по неволе;\\
Она вздыхала по другом,\\
Который сердцем и умом\\
Ей нравился гораздо боле:\\
Сей Грандисон был славный франт,\\
Игрок и гвардии сержант.\\

XXXI

Как он, она была одета\\
Всегда по моде и к лицу;\\
Но, не спросясь ее совета,\\
Девицу повезли к венцу.\\
И, чтоб ее рассеять горе,\\
Разумный муж уехал вскоре\\
В свою деревню, где она,\\
Бог знает кем окружена,\\
Рвалась и плакала сначала,\\
С супругом чуть не развелась;\\
Потом хозяйством занялась,\\
Привыкла и довольна стала.\\
\textbf{Привычка свыше нам дана:\\
Замена счастию она.}\\

XXXII

Привычка усладила горе,\\
Не отразимое ничем;\\
Открытие большое вскоре\\
Ее утешило совсем:\\
Она меж делом и досугом\\
Открыла тайну, как супругом\\
Самодержавно управлять,\\
И все тогда пошло на стать.\\
Она езжала по работам,\\
Солила на зиму грибы,\\
Вела расходы, брила лбы,\\
Ходила в баню по субботам,\\
Служанок била осердясь —\\
Все это мужа не спросясь.\\

XXXIII

Бывало, писывала кровью\\
Она в альбомы нежных дев,\\
Звала Полиною Прасковью\\
И говорила нараспев,\\
Корсет носила очень узкий,\\
И русский Н как N французский\\
Произносить умела в нос;\\
Но скоро все перевелось:\\
Корсет, альбом, княжну Алину,\\
Стишков чувствительных тетрадь\\
Она забыла: стала звать\\
Акулькой прежнюю Селину\\
И обновила наконец\\
На вате шлафор и чепец.\\

XXXIV

Но муж любил ее сердечно,\\
В ее затеи не входил,\\
Во всем ей веровал беспечно,\\
А сам в халате ел и пил;\\
Покойно жизнь его катилась;\\
Под вечер иногда сходилась\\
Соседей добрая семья,\\
Нецеремонные друзья,\\
И потужить, и позлословить,\\
И посмеяться кой о чем.\\
Проходит время; между тем\\
Прикажут Ольге чай готовить,\\
Там ужин, там и спать пора,\\
И гости едут со двора.\\

XXXV

Они хранили в жизни мирной\\
Привычки милой старины;\\
У них на масленице жирной\\
Водились русские блины;\\
Два раза в год они говели;\\
Любили круглые качели,\\
Подблюдны песни, хоровод;\\
В день Троицын, когда народ,\\
Зевая, слушает молебен,\\
Умильно на пучок зари\\
Они роняли слезки три;\\
Им квас как воздух был потребен,\\
И за столом у них гостям\\
Носили блюды по чинам.\\

XXXVI

И так они старели оба.\\
И отворились наконец\\
Перед супругом двери гроба,\
И новый он приял венец.\\
Он умер в час перед обедом,\\
Оплаканный своим соседом,\\
Детьми и верною женой\\
Чистосердечней, чем иной.\\
Он был простой и добрый барин,\\
И там, где прах его лежит,\\
Надгробный памятник гласит:\\
Смиренный грешник, Дмитрий Ларин,\\
Господний раб и бригадир,\\
Под камнем сим вкушает мир.\\

XXXVII

Своим пенатам возвращенный,\\
Владимир Ленский посетил\\
Соседа памятник смиренный,\\
И вздох он пеплу посвятил;\\
И долго сердцу грустно было.\\
«Рооr Yorick! — молвил он уныло. —\\
Он на руках меня держал.\\
Как часто в детстве я играл\\
Его Очаковской медалью!\\
Он Ольгу прочил за меня,\\
Он говорил: дождусь ли дня?..»\\
И, полный искренней печалью,\\
Владимир тут же начертал\\
Ему надгробный мадригал.\\

XXXVIII

И там же надписью печальной\\
Отца и матери, в слезах,\\
Почтил он прах патриархальный…\\
Увы! на жизненных браздах\\
Мгновенной жатвой поколенья,\\
По тайной воле провиденья,\\
Восходят, зреют и падут;\\
Другие им вослед идут…\\
Так наше ветреное племя\\
Растет, волнуется, кипит\\
И к гробу прадедов теснит.\\
Придет, придет и наше время,\\
И наши внуки в добрый час\\
Из мира вытеснят и нас!\\

XXXIX

Покамест упивайтесь ею,\\
Сей легкой жизнию, друзья!\\
Ее ничтожность разумею\\
И мало к ней привязан я;\\
Для призраков закрыл я вежды;\\
Но отдаленные надежды\\
Тревожат сердце иногда:\\
Без неприметного следа\\
Мне было б грустно мир оставить.\\
Живу, пишу не для похвал;\\
Но я бы, кажется, желал\\
Печальный жребий свой прославить,\\
Чтоб обо мне, как верный друг,\\
Напомнил хоть единый звук.\\

XL

И чье-нибудь он сердце тронет;\\
И, сохраненная судьбой,\\
Быть может, в Лете не потонет\\
Строфа, слагаемая мной;\\
Быть может (лестная надежда!),\\
Укажет будущий невежда\\
На мой прославленный портрет\\
И молвит: то-то был поэт!\\
Прими ж мои благодаренья,\\
Поклонник мирных аонид,\\
О ты, чья память сохранит\\
Мои летучие творенья,\\
Чья благосклонная рука\\
Потреплет лавры старика!\\
