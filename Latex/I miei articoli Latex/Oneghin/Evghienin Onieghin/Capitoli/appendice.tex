\appendix % Inizia l'appendice
\chapter{Appendice}
\section{8 глава}
восьмой главе П отказался от использованного им в предшествующей главе метода прямых характеристик героя и представляет его читателю в столкновении различных, взаимопротиворечащих точек зрения, из которых ни одна в отдельности не может быть 
отождествлена с авторской.
\section{Fare thee Well}
\label{sec:fare} 
Начало стихотворения Байрона «Fare thee well» из цикла Poems of separations («Стихи о разводе»), 1816.\\
(«Прощай, и если навсегда, то навсегда прощай»).
\section{В те дни...}
\label{sec:musa}
это история своей Музы, смена периодов творчества, читательской аудитории, жизненных 
обстоятельств, образующая единую эволюцию
\label{apuleia}
Римский писатель Изобилующий фантастическими и эротическими эпизодами роман Апулея 
«Золотой осел» был популярен в XVIII в. П читал его по-французский.
\label{soiuz}
Слово «союз» могло читаться двузначно: и как дружеское сообщество
(«Друзья мои, прекрасен наш союз!» — «19 Октября», II, 1, 425), и как политическое общество
(Союз Благоденствия). 
\label{societa}
Положительная оценка света в этих стихах звучит неожиданно после резко сатирических 
картин в предшествующих строфах. /П, сохраняя сатирическое отношение к свету, видит теперь и другую его сторону: быт 
образованного и духовно-утонченного общества обладает ценностью как часть национальной культуры. 
Он проникнут столь недостающим русскому обществу, особенно после 14 декабря 1825 г., уважением человека к себе 
(«холод гордости спокойной»). Более того, П видит в нем больше истинного демократизма, чем в неуклюжем этикете или псевдонародной грубости образованного мещанства, говорящего со страниц русских журналов от имени демократии. Люди света проще и потому ближе 
к народу.
\label{oneghin}
Строфа, представляя собой резкое-осуждение. Резкое осуждение Онегина отнюдь не выражает 
окончательного суда автора.
Онегина.
\label{giudizio}
Строфа диалогически противопоставлена предшествующей. Мнение, высказанное в строфе VIII, приписывается «самолюбивой ничтожности», что знаменует езкий перелом в отношении автора к Онегину.
\label{intelligenza}
исторические корни раскрываются у И. С. Аксакова: «Говорить снова о перевороте Петра, 
нарушившем правильность нашего органического развития, было бы излишним повторением. 
Мы могли бы кстати, говоря об уме, припомнить слово, приписываемое Кикину и хорошо 
характеризующее наше умственное развитие. Предание рассказывает, что Кикин на вопрос Петра, отчего Кикин 
его не любит, отвечал: «Русский ум любит простор, а от тебя ему тесно»
Согласно одной, Петр I, якобы, спросил А. Кикина в застенке: 
«Как ты, умный человек, мог пойти против меня?» — 
и получил ответ: «Какой я умный! Ум любит простор, а у тебя ему тесно». Современный исследователь, ком
 ментируя этот эпизод, отметил, что в нем «государственному абсолютизму, воплотившемуся в лице Петра был  противопоставлен принцип свободы личности»  
 Кикин Александр Васильевич — крупный политический деятель эпохи Петра I, участник «заговора» царевича Алексея. Колесован в 1718 г
 \label{felicita}
Права человека на обыденное, простое счастье. Слова Шатобриана: «Нет счастья вне 
проторенных дорог» П  написал в письме к Н. И. Кривцову: «Молодость моя прошла 
шумно и бесплодно. До сих пор я жил иначе, как обыкновенно живут. 
Счастья мне не было. II n’est de bonheur que dans les voies communes. 
Мне за 30 лет. 
В тридцать лет люди обыкновенно женятся — я поступаю. как люди, и вероятно не буду в том раскаиваться» (XIV, 150—151). 
Цитата эта интересна противопоставлением оставленного пути романтической молодости («жил иначе как обыкновенно живут») 
новой жизненной дороге («поступаю как люди»).
350
\label{villaggio}
Степной иногда употребляется у П в значении «сельский», 
как антоним понятия «цивилизованный».
Татьяна приехала не из степной полосы России, а из северо-западной (см. с. 325), 
и стих следует понимать: «из глуши простых, бедных селений».
\label{spagna}
В 1824 г когда происходит встреча Онегина и Татьяны в Петербурге, 
Россия не поддерживала дипломатических отношений с Испанией, прерванных во время испанской революции. Испанский посол Хуан Мигуэль Паэс де ла Кадена появился в Петербурге в 1825 г. П познакомился с ним, 
видимо, в 1832 г. и записал с его слов рассказ секретаря Наполеона Бурьена о 18 брюмера.
Анахронизм появления этого персонажа совпадает с общей тенденцией П к изображению фона седьмой- восьмой глав на основании реальных впечатлений последекабристской эпохи.
\label{serata}
Нетерпение Онегина выразилось в том, что он выехал не только без опоздания,
но и в максимально возможный ранний срок. Съезд гостей начинался после десяти вечера. 
Ростовы, приглашенные на бал к «екатерининскому вельможе», «в одиннадцать часов разместились 
по каретам и поехали» («Война и мир», т. II, ч. III, гл. 14). От Таврического сада до Английской набережной они ехали не менее получаса, но прибыли еще до появления государя и начала бала.
Онегин приезжает до появления гостей — «Татьяну он одну находит» (XXII, 6).
Поскольку князь N дает не бал, а вечер, хозяева не встречают гостей при входе в зал, а запросто принимают 
в гостиной. 
\label{stupido}
 Татяна не хочет отличаться других обществах по этому приглашает все.
\label{scherzo}
Утонченная вежливость светского обращения и стиль тонкого остроумия беседы культивировались в XVIII в. 
Особый смысл они получили в 1790-е гг., когда приобрели политический оттенок;
связанные с французскими эмигрантами круги петербургского общества демонстрировали сохранение в столице России истинно 
«версальского» тона, уже не существовавшего на его родине. XIX в. внес изменения в нормы светского 
поведения. С одной стороны, входила в моду «английская» манера — серьезные «мужские» разговоры
и отрывистая речь сменяют утонченную и интонационно от работанную беседу с дамами. 
С другой — «солдатские» манеры наполеоновских генералов, по мере того как 
Франция становится признанным дипломатическим партнером, все более входят в стиль и даже моду 
в европейских салонах.
В России они появились вместе с французским послом Коленкуром после Тильзита. 
Характеризуя стиль поведения людей империи Наполеона, мемуарист Ф. Головкин писал: «Изящные манеры, образование, скорее блестящее, чем основательное, и чрезвычайная ветряность характера» оказались совершенно чуждыми тому миру, где культивировались 
«положительные таланты и дурные манеры, как главные условия карьеры»
(Головкин Ф. Двор и царствование Павла I. М., 1912, с. 336). Общее изменение светского тона коснулось и России, 
особенно резко сказавшись в поведении передовой молодежи (см.: Лотман, Декабрист в повседневной жизни, с. 30 — 31). «Тонкость» светского обращения XVIII в. стала восприниматься как архаическая и смешная. 
\label{giornale}
— Журнальная картинка — гравюра с изображением последних мод. 
Такие иллюстрации, раскрашенные от руки, прилагались для увеличения подписки к ряду русских 
журналов. 
\label{cravatta}
Среди франтов 1820-х гг. было принято носить батистовые шейные платки.С
легка крахмалить такие платки ввел в моду знаменитый денди Джордж Брэммель.   
Слегка крахмалить галстук — признак дендизма. 
Перекрахмалить — переусердствовать по части моды, что само по себе 
противоречило неписанным нормам хорошего тона и было vulgar.
\label{eva}
Имеется в виду библейский миф о сатане, в образе змия-искусителя пробравшемся 
в рай, соблазнившем первую женщину Еву, уговорив ее вкусить от запретного древа добра и зла. 