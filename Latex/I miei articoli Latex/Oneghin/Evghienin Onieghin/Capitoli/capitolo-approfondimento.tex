\chapter{Capitolo approfondimento}\label{section:app}% Etichetta per riferimenti incrociati


Последняя ария Ленского
\begin{verse}
Пушкин активно присутствует в "Евгении Онегине". 
Помимо всех героев романа есть персонаж "автор". Рассказчик. Он даёт оценки, называет Онегина добрый мой приятель, подтрунивает над Ленским. 
Помните, в ночь перед дуэлью Ленский предаётся сочинительству? Вот как рассказывает об этом Пушкин: "Берет перо; его стихи,\\
Полны любовной чепухи,\\
Звучат и льются. Их читает\\
Он вслух, в лирическом жару,\\
Как Дельвиг пьяный на пиру.\\
Стихи на случай сохранились;\\
Я их имею; вот они:"

Далее текст от Ленского. И снова оценка рассказчика: 
"Так он писал темно и вяло\\
(Что романтизмом мы зовем,\\
Хоть романтизма тут ни мало\\
Не вижу я; да что нам в том?)\\
И наконец перед зарею,\\
Склонясь усталой головою,\\
На модном слове идеал\\
Тихонько Ленский задремал;".

Потом Ленского будят и он едет стреляться. Да, для Пушкина дуэль — обычное дело, он сам стрелялся много раз.Можно поспать.

Чайковский переосмыслил стихи Ленского, подняв их на новую высоту благодаря музыке.\\
Он написал оперу уже после смерти автора романа. Убийство поэта Ленского в опере ассоциируется у слушателя с убийством поэта Пушкина. Оценок рассказчика до ("чепуха") и после ("вяло") стихов Ленского в либретто нет. Исчезла ирония Пушкина над романтичным поэтом, вирши стали подлинно трагическими.\\
Да, Ленский, с точки зрения Пушкина, не самый сильный поэт, но гениальная музыка Чайковского вложила столько чувств в предсмертную арию!\\
В оперу стихи вошли почти без изменений, лишь некоторые слова повторяются, вместо "И думать: он меня любил" в повторе "Ах, Ольга, я тебя любил!" (Ленский мечтает как Ольга придёт к нему на могилу).\\
То, что Ленский обращается к поэтическим образам типа "стрелой пронзённый" (прекрасно зная, что дуэль на пистолетах), лишь говорит о его возбуждении перед смертью — и возвышенности духа, если хотите... Итак, ария Ленского.\\
В опере он поёт прямо на месте дуэли.

Слова Ленского — музыка Чайковского.

(сначала короткое вступление)

Куда, куда, куда вы удалились,\\
Весны моей златые дни?\\

(медленный лирический напев, в стиле романса)

Что день грядущий мне готовит?\\
Его мой взор напрасно ловит,\\
В глубокой мгле таится он.\\
Нет нужды; прав судьбы закон.\\
Паду ли я, стрелой пронзенный,\\
Иль мимо пролетит она,\\
Всё благо: бдения и сна\\
Приходит час определенный,\\
Благословен и день забот,\\
Благословен и тьмы приход!

(мажорная мелодия, несмотря на "гробницы")

Блеснет заутра луч денницы\\
И заиграет яркий день;\\
А я — быть может, я гробницы\\
Сойду в таинственную сень,\\
И память юного поэта\\
Поглотит медленная Лета,

(вновь минор)

Забудет мир меня; но ты, ты, Ольга...\\
Скажи, придешь ли, дева красоты,\\
Слезу пролить над ранней урной\\
И думать: он меня любил!\\
Он мне единой посвятил\\
Рассвет печальный жизни бурной!

(эмоционально)

Ах, Ольга, я тебя любил!\\		
Тебе единой посвятил\\
Рассвет печальный жизни бурной!\\
Ах, Ольга, я тебя любил!\\

(музыка усиливается, достигая кульминации)

Сердечный друг, желанный друг.\\
Приди, приди! Желанный друг, приди, я твой супруг!\\
Приди, я твой супруг!\\
Приди, приди! Я жду тебя, желанный друг.\\
Приди, приди; я твой супруг!\\

(и снова безнадёжно первые две строчки)

Куда, куда, куда вы удалились,\\
Весны моей, весны моей златые дни?

\ref{https://www.youtube.com/watch?v=B_w0lTJDOnE}


\ref{https://ru.wikisource.org/wiki}
\textbf{Картина третья}
Уединенный уголок сада при усадьбе Лариных. Дворовые девушки с песнями собирают ягоды.

11. Хор девушек.

ДЕВУШКИ.\\
Девицы, красавицы,\\
душеньки, подруженьки!\\
Разыграйтесь, девицы,\\
разгуляйтесь, милые!\\
Затяните песенку,\\
песенку заветную.\\
Заманите молодца\\
к хороводу нашему!\\
Как заманим молодца,\\
как завидим издали,\\
разбежимтесь, милые,\\
закидаем вишеньем,\\
вишеньем, малиною\\
красною смородиной!\\
Не ходи подслушивать\\
песенки заветные,\\
не ходи подсматривать\\
игры наши девичьи!

Девушки уходят в глубину сада. Вбегает взволнованная Татьяна и в изнеможении падает на скамью.

ТАТЬЯНА.\\
Здесь он, здесь Евгений!\\
О Боже! О Боже! Что подумал он!..\\
Что скажет он?..\\
Ах, для чего,\\
Стенанью вняв души больной,\\
Не совладав сама с собой,\\
Ему письмо я написала!\\
Да, сердце мне теперь сказало,\\

Что насмеется надо мной\\
Мой соблазнитель роковой!\\
О, Боже мой! как я несчастна,\\
Как я жалка!

(Слышен шум шагов. Татьяна прислушивается.)\\

Шаги... все ближе...\\
Да, это он, это он!\\

(Появляется Онегин.)

ОНЕГИН.\\
Вы мне писали,\\
Не отпирайтесь. Я прочел\\
Души доверчивой признанья,\\
Любви невинной излиянья;\\
Мне ваша искренность мила!\\
Она в волненье привела\\
Давно умолкнувшие чувства.\\
Но вас хвалить я не хочу;\\
Я за нее вам отплачу\\
Признаньем также без искусства.\\
Примите ж исповедь мою,\\
Себя на суд вам отдаю!

ТАТЬЯНА.(про себя)\\
О Боже! Как, обидно и как больно!\\

ОНЕГИН.\\
Когда бы жизнь домашним кругом\\
Я ограничить захотел,\\
Когда б мне быть отцом, супругом\\
Приятный жребий повелел,\\
То верно б, кроме вас одной,\\
Невесты не искал иной.\\
Но я не создан для блаженства,\\
Ему чужда душа моя.\\
Напрасны ваши совершенства,\\
Их не достоин вовсе я.\\
Поверьте, (совесть в том порукой),\\
Супружество нам будет мукой.\\
Я сколько ни любил бы вас,\\
Привыкнув, разлюблю тотчас.\\
Судите ж вы, какие розы\\
Нам заготовил Гименей,\\
И, может быть, на много дней!\\
Мечтам и годам нет возврата!\\
Ах, нет возврата;\\
Не обновлю души моей!\\
Я вас люблю любовью брата,\\
Любовью брата,\\
Иль, может быть, еще нежней!\\
Иль, может быть еще,\\
Еще нежней!\\
Послушайте ж меня без гнева,\\
Сменит не раз младая дева\\
Мечтами легкие мечты.\\

ДЕВУШКИ (за сценой).\\
Девицы, красавицы, душеньки, подруженьки!\\
Разыграйтесь, девицы, разгуляйтесь, милые.

ОНЕГИН.\\
Учитесь властвовать собой; ...\\
Не всякий вас, как я, поймет.\\
К беде неопытность ведет!

Онегин подает руку Татьяне и они уходят по направлению к дому. Девушки продолжают петь, постепенно удаляясь.

Как заманим молодца,\\
как завидим издали,\\
разбежимтесь, милые,\\
закидаем вишеньем.\\
Не ходи подслушивать,\\
не ходи подсматривать\\
игры наши девичьи!


\ref{https://yandex.ru/video/preview/154295535324376015}

\textbf{Музыка П. Чайковского Ария Гремина из оперы «Евгений Онегин»}

Любви все возрасты покорны,\\
Её порывы благотворны.\\
И юноше в расцвете лет, едва увидевшему свет,\\
И закалённому судьбой бойцу с седою головой.\\

Онегин, я скрывать не стану -\\
Безумно я люблю Татьяну!\\
Тоскливо жизнь моя текла,\\
Она явилась и зажгла,\\
Как солнца луч среди ненастья,\\
Мне жизнь и молодость, да, молодость и счастье!

Среди лукавых, малодушных,\\
Шальных, балованных детей,\\
Злодеев и смешных, и скучных,\\
Тупых, привязчивых судей;\\
Среди кокеток богомольных,\\
Среди холопов добровольных,\\
Среди вседневных модных сцен,\\
Учтивых ласковых измен,\\
Среди холодных приговоров\\
Жестокосердной суеты,\\
Среди досадной пустоты\\
Расчётов дум и разговоров, -\\
Она блистает, как звезда\\
Во мраке ночи, в небе чистом,\\
И мне является всегда\\
В сияньи ангела, в сияньи ангела лучистом!

Любви все возрасты покорны,\\
Её порывы благотворны\\\
И юноше в расцвете лет, едва увидевшему свет,\\
И закалённому судьбой бойцу с седою головой.

Онегин, я скрывать не стану,\\
Безумно я люблю Татьяну.\\
Тоскливо жизнь моя текла,\\
Она явилась и зажгла.\\
Как солнца луч среди ненастья\\
И жизнь, и молодость, да молодость и счастье.\\
И жизнь, и молодость, и счастье!\\
