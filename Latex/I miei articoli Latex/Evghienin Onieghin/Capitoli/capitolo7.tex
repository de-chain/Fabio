\chapter{Capitolo 7}\label{section:cap7}% Etichetta per riferimenti incrociati

\begin{verse}

\textit{Москва, России дочь любима,\\
Где равную тебе сыскать?\\
Дмитриев.\\
Как не любить родной Москвы?\\
Баратынский.\\
Гоненье на Москву! что значит видеть свет!\\
Где ж лучше?\\
Где нас нет.\\
Грибоедов.}

I

Гонимы вешними лучами,\\
С окрестных гор уже снега\\
Сбежали мутными ручьями\\
На потопленные луга.\\
Улыбкой ясною природа\\
Сквозь сон встречает утро года;\\
Синея блещут небеса.\\
Еще прозрачные, леса\\
Как будто пухом зеленеют.\\
Пчела за данью полевой\\
Летит из кельи восковой.\\
Долины сохнут и пестреют;\\
Стада шумят, и соловей\\
Уж пел в безмолвии ночей.

II

Как грустно мне твое явленье,\\
Весна, весна! пора любви!\\
Какое томное волненье\\
В моей душе, в моей крови!\\
С каким тяжелым умиленьем\\
Я наслаждаюсь дуновеньем\\
В лицо мне веющей весны\\
На лоне сельской тишины!\\
Или мне чуждо наслажденье,\\
И все, что радует, живит,\\
Все, что ликует и блестит\\
Наводит скуку и томленье\\
На душу мертвую давно\\
И все ей кажется темно?

III

Или, не радуясь возврату\\
Погибших осенью листов,\\
Мы помним горькую утрату,\\
Внимая новый шум лесов;\\
Или с природой оживленной\\
Сближаем думою смущенной\\
Мы увяданье наших лет,\\
Которым возрожденья нет?\\
Быть может, в мысли нам приходит\\
Средь поэтического сна\\
Иная, старая весна\\
И в трепет сердце нам приводит\\
Мечтой о дальной стороне,\\
О чудной ночи, о луне…

IV

Вот время: добрые ленивцы,\\
Эпикурейцы-мудрецы,\\
Вы, равнодушные счастливцы,\\
Вы, школы Левшина птенцы,\\
Вы, деревенские Приамы,\\
И вы, чувствительные дамы,\\
Весна в деревню вас зовет,\\
Пора тепла, цветов, работ,\\
Пора гуляний вдохновенных\\
И соблазнительных ночей.\\
В поля, друзья! скорей, скорей,\\
В каретах, тяжко нагруженных,\\
На долгих иль на почтовых\\
Тянитесь из застав градских.

V

И вы, читатель благосклонный,\\
В своей коляске выписной\\
Оставьте град неугомонный,\\
Где веселились вы зимой;\\
С моею музой своенравной\\
Пойдемте слушать шум дубравный\\
Над безыменною рекой\\
В деревне, где Евгений мой,\\
Отшельник праздный и унылый,\\
Еще недавно жил зимой\\
В соседстве Тани молодой,\\
Моей мечтательницы милой,\\
Но где его теперь уж нет…\\
Где грустный он оставил след.

VI

Меж гор, лежащих полукругом,\\
Пойдем туда, где ручеек,\\
Виясь, бежит зеленым лугом\\
К реке сквозь липовый лесок.\\
Там соловей, весны любовник,\\
Всю ночь поет; цветет шиповник,\\
И слышен говор ключевой,\\
Там виден камень гробовой\\
В тени двух сосен устарелых.\\
Пришельцу надпись говорит:\\
«Владимир Ленский здесь лежит,\\
Погибший рано смертью смелых,\\
В такой-то год, таких-то лет.\\
Покойся, юноша-поэт!»

VII

На ветви сосны преклоненной,\\
Бывало, ранний ветерок\\
Над этой урною смиренной\\
Качал таинственный венок.\\
Бывало, в поздние досуги\\
Сюда ходили две подруги,\\
И на могиле при луне,\\
Обнявшись, плакали оне.\\
Но ныне… памятник унылый\\
Забыт. К нему привычный след\\
Заглох. Венка на ветви нет;\\
Один, под ним, седой и хилый\\
Пастух по-прежнему поет\\
И обувь бедную плетет.

VIII. IX. X

Мой бедный Ленский! изнывая,\\
Не долго плакала она.\\
Увы! невеста молодая\\
Своей печали неверна.\\
Другой увлек ее вниманье,\\
Другой успел ее страданье\\
Любовной лестью усыпить,\\
Улан умел ее пленить,\\
Улан любим ее душою…\\
И вот уж с ним пред алтарем\\
Она стыдливо под венцом\\
Стоит с поникшей головою,\\
С огнем в потупленных очах,\\
С улыбкой легкой на устах.

XI

Мой бедный Ленский! за могилой\\
В пределах вечности глухой\\
Смутился ли, певец унылый,\\
Измены вестью роковой,\\
Или над Летой усыпленный\\
Поэт, бесчувствием блаженный,\\
Уж не смущается ничем,\\
И мир ему закрыт и нем?..\\
Так! равнодушное забвенье\\
За гробом ожидает нас.\\
Врагов, друзей, любовниц глас\\
Вдруг молкнет. Про одно именье\\
Наследников сердитый хор\\
Заводит непристойный спор.

XII

И скоро звонкий голос Оли\\
В семействе Лариных умолк.\\
Улан, своей невольник доли,\\
Был должен ехать с нею в полк.\\
Слезами горько обливаясь,\\
Старушка, с дочерью прощаясь,\\
Казалось, чуть жива была,\\
Но Таня плакать не могла;\\
Лишь смертной бледностью покрылось\\
Ее печальное лицо.\\
Когда все вышли на крыльцо,\\
И все, прощаясь, суетилось\\
Вокруг кареты молодых,\\
Татьяна проводила их.

XIII

И долго, будто сквозь тумана,\\
Она глядела им вослед…\\
И вот одна, одна Татьяна!\\
Увы! подруга стольких лет,\\
Ее голубка молодая,\\
Ее наперсница родная,\\
Судьбою вдаль занесена,\\
С ней навсегда разлучена.\\
Как тень она без цели бродит,\\
То смотрит в опустелый сад…\\
Нигде, ни в чем ей нет отрад,\\
И облегченья не находит\\
Она подавленным слезам,\\
И сердце рвется пополам.

XIV

И в одиночестве жестоком\\
Сильнее страсть ее горит,\\
И об Онегине далеком\\
Ей сердце громче говорит.\\
Она его не будет видеть;\\
Она должна в нем ненавидеть\\
Убийцу брата своего;\\
Поэт погиб… но уж его\\
Никто не помнит, уж другому\\
Его невеста отдалась.\\
Поэта память пронеслась\\
Как дым по небу голубому,\\
О нем два сердца, может быть,\\
Еще грустят… На что грустить?..

XV

Был вечер. Небо меркло. Воды\\
Струились тихо. Жук жужжал.\\
Уж расходились хороводы;\\
Уж за рекой, дымясь, пылал\\
Огонь рыбачий. В поле чистом,\\
Луны при свете серебристом,\\
В свои мечты погружена,\\
Татьяна долго шла одна.\\
Шла, шла. И вдруг перед собою\\
С холма господский видит дом,\\
Селенье, рощу под холмом\\
И сад над светлою рекою.\\
Она глядит — и сердце в ней\\
Забилось чаще и сильней.

XVI

Ее сомнения смущают:\\
«Пойду ль вперед, пойду ль назад?..\\
Его здесь нет. Меня не знают…\\
Взгляну на дом, на этот сад».\\
И вот с холма Татьяна сходит,\\
Едва дыша; кругом обводит\\
Недоуменья полный взор…\\
И входит на пустынный двор.\\
К ней, лая, кинулись собаки.\\
На крик испуганный ея\\
Ребят дворовая семья\\
Сбежалась шумно. Не без драки\\
Мальчишки разогнали псов,\\
Взяв барышню под свой покров.

XVII

«Увидеть барской дом нельзя ли?»\\
Спросила Таня. Поскорей\\
К Анисье дети побежали\\
У ней ключи взять от сеней;\\
Анисья тотчас к ней явилась,\\
И дверь пред ними отворилась,\\
И Таня входит в дом пустой,\\
Где жил недавно наш герой.\\
Она глядит: забытый в зале\\
Кий на бильярде отдыхал,\\
На смятом канапе лежал\\
Манежный хлыстик. Таня дале;\\
Старушка ей: «А вот камин;\\
Здесь барин сиживал один.

XVIII

Здесь с ним обедывал зимою\\
Покойный Ленский, наш сосед.\\
Сюда пожалуйте, за мною.\\
Вот это барский кабинет;\\
Здесь почивал он, кофей кушал,\\
Приказчика доклады слушал\\
И книжку поутру читал…\\
И старый барин здесь живал;\\
Со мной, бывало, в воскресенье,\\
Здесь под окном, надев очки,\\
Играть изволил в дурачки.\\
Дай бог душе его спасенье,\\
А косточкам его покой\\
В могиле, в мать-земле сырой!»

XIX

Татьяна взором умиленным\\
Вокруг себя на все глядит,\\
И все ей кажется бесценным,\\
Все душу томную живит\\
Полумучительной отрадой:\\
И стол с померкшею лампадой,\\
И груда книг, и под окном\\
Кровать, покрытая ковром,\\
И вид в окно сквозь сумрак лунный,\\
И этот бледный полусвет,\\
И лорда Байрона портрет,\\
И столбик с куклою чугунной\\
Под шляпой с пасмурным челом,\\
С руками, сжатыми крестом.

XX

Татьяна долго в келье модной\\
Как очарована стоит.\\
Но поздно. Ветер встал холодный.\\
Темно в долине. Роща спит\\
Над отуманенной рекою;\\
Луна сокрылась за горою,\\
И пилигримке молодой\\
Пора, давно пора домой.\\
И Таня, скрыв свое волненье,\\
Не без того, чтоб не вздохнуть,\\
Пускается в обратный путь.\\
Но прежде просит позволенья\\
Пустынный замок навещать,\\
Чтоб книжки здесь одной читать.

XXI

Татьяна с ключницей простилась\\
За воротами. Через день\\
Уж утром рано вновь явилась\\
Она в оставленную сень.\\
И в молчаливом кабинете,\\
Забыв на время все на свете,\\
Осталась наконец одна,\\
И долго плакала она.\\
Потом за книги принялася.\\
Сперва ей было не до них,\\
Но показался выбор их\\
Ей странен. Чтенью предалася\\
Татьяна жадною душой;\\
И ей открылся мир иной.

XXII

Хотя мы знаем, что Евгений\\
Издавна чтенье разлюбил,\\
Однако ж несколько творений\\
Он из опалы исключил:\\
Певца Гяура и Жуана\\
Да с ним еще два-три романа,\\
В которых отразился век\\
И современный человек\\
Изображен довольно верно\\
С его безнравственной душой,\\
Себялюбивой и сухой,\\
Мечтанью преданной безмерно,\\
С его озлобленным умом,\\
Кипящим в действии пустом.

XXIII

Хранили многие страницы\\
Отметку резкую ногтей;\\
Глаза внимательной девицы\\
Устремлены на них живей.\\
Татьяна видит с трепетаньем,\\
Какою мыслью, замечаньем\\
Бывал Онегин поражен,\\
В чем молча соглашался он.\\
На их полях она встречает\\
Черты его карандаша.\\
Везде Онегина душа\\
Себя невольно выражает\\
То кратким словом, то крестом,\\
То вопросительным крючком.

XXIV

И начинает понемногу\\
Моя Татьяна понимать\\
Теперь яснее — слава богу —\\
Того, по ком она вздыхать\\
Осуждена судьбою властной:\\
Чудак печальный и опасный,\\
Созданье ада иль небес,\\
Сей ангел, сей надменный бес,\\
Что ж он? Ужели подражанье,\\
Ничтожный призрак, иль еще\\
Москвич в Гарольдовом плаще,\\
Чужих причуд истолкованье,\\
Слов модных полный лексикон?..\\
Уж не пародия ли он?

XXV

Ужель загадку разрешила?\\
Ужели слово найдено?\\
Часы бегут; она забыла,\\
Что дома ждут ее давно,\\
Где собралися два соседа\\
И где об ней идет беседа.\\
— Как быть? Татьяна не дитя,\\
Старушка молвила кряхтя.\\
Ведь Оленька ее моложе.\\
Пристроить девушку, ей-ей,\\
Пора; а что мне делать с ней?\\
Всем наотрез одно и то же:\\
Нейду. И все грустит она,\\
Да бродит по лесам одна.

XXVI

«Не влюблена ль она?» — В кого же?\\
Буянов сватался: отказ.\\
Ивану Петушкову — тоже.\\
Гусар Пыхтин гостил у нас;\\
Уж как он Танею прельщался,\\
Как мелким бесом рассыпался!\\
Я думала: пойдет авось;\\
Куда! и снова дело врозь.\\
«Что ж, матушка? за чем же стало?\\
В Москву, на ярманку невест!\\
Там, слышно, много праздных мест».\\
— Ох, мой отец! доходу мало.\\
«Довольно для одной зимы,\\
Не то уж дам хоть я взаймы».

XXVII

Старушка очень полюбила\\
Совет разумный и благой;\\
Сочлась — и тут же положила\\
В Москву отправиться зимой.\\
И Таня слышит новость эту.\\
На суд взыскательному свету\\
Представить ясные черты\\
Провинциальной простоты,\\
И запоздалые наряды,\\
И запоздалый склад речей;\\
Московских франтов и цирцей\\
Привлечь насмешливые взгляды!..\\
О страх! нет, лучше и верней\\
В глуши лесов остаться ей.

XXVIII

Вставая с первыми лучами,\\
Теперь она в поля спешит\\
И, умиленными очами\\
Их озирая, говорит:\\
«Простите, мирные долины,\\
И вы, знакомых гор вершины,\\
И вы, знакомые леса;\\
Прости, небесная краса,\\
Прости, веселая природа;\\
Меняю милый, тихий свет\\
На шум блистательных сует…\\
Прости ж и ты, моя свобода!\\
Куда, зачем стремлюся я?\\
Что мне сулит судьба моя?»

XXIX

Ее прогулки длятся доле.\\
Теперь то холмик, то ручей\\
Остановляют поневоле\\
Татьяну прелестью своей.\\
Она, как с давними друзьями,\\
С своими рощами, лугами\\
Еще беседовать спешит.\\
Но лето быстрое летит.\\
Настала осень золотая.\\
Природа трепетна, бледна,\\
Как жертва, пышно убрана…\\
Вот север, тучи нагоняя,\\
Дохнул, завыл — и вот сама\\
Идет волшебница зима.

XXX

Пришла, рассыпалась; клоками\\
Повисла на суках дубов;\\
Легла волнистыми коврами\\
Среди полей, вокруг холмов;\\
Брега с недвижною рекою\\
Сравняла пухлой пеленою;\\
Блеснул мороз. И рады мы\\
Проказам матушки зимы.\\
Не радо ей лишь сердце Тани.\\
Нейдет она зиму встречать,\\
Морозной пылью подышать\\
И первым снегом с кровли бани\\
Умыть лицо, плеча и грудь:\\
Татьяне страшен зимний путь.

XXXI

Отъезда день давно просрочен,\\
Проходит и последний срок.\\
Осмотрен, вновь обит, упрочен\\
Забвенью брошенный возок.\\
Обоз обычный, три кибитки\\
Везут домашние пожитки,\\
Кастрюльки, стулья, сундуки,\\
Варенье в банках, тюфяки,\\
Перины, клетки с петухами,\\
Горшки, тазы et cetera,\\
Ну, много всякого добра.\\
И вот в избе между слугами\\
Поднялся шум, прощальный плач:\\
Ведут на двор осьмнадцать кляч,

XXXII

В возок боярский их впрягают,\\
Готовят завтрак повара,\\
Горой кибитки нагружают,\\
Бранятся бабы, кучера.\\
На кляче тощей и косматой\\
Сидит форейтор бородатый,\\
Сбежалась челядь у ворот\\
Прощаться с барами. И вот\\
Уселись, и возок почтенный,\\
Скользя, ползет за ворота.\\
«Простите, мирные места!\\
Прости, приют уединенный!\\
Увижу ль вас?..» И слез ручей\\
У Тани льется из очей.

XXXIII

Когда благому просвещенью\\
Отдвинем более границ,\\
Современем (по расчисленью\\
Философических таблиц,\\
Лет чрез пятьсот) дороги, верно,\\
У нас изменятся безмерно:\\
Шоссе Россию здесь и тут,\\
Соединив, пересекут.\\
Мосты чугунные чрез воды\\
Шагнут широкою дугой,\\
Раздвинем горы, под водой\\
Пророем дерзостные своды,\\
И заведет крещеный мир\\
На каждой станции трактир.

XXXIV

Теперь у нас дороги плохи,\\
Мосты забытые гниют,\\
На станциях клопы да блохи\\
Заснуть минуты не дают;\\
Трактиров нет. В избе холодной\\
Высокопарный, но голодный\\
Для виду прейскурант висит\\
И тщетный дразнит аппетит,\\
Меж тем как сельские циклопы\\
Перед медлительным огнем\\
Российским лечат молотком\\
Изделье легкое Европы,\\
Благословляя колеи\\
И рвы отеческой земли.

XXXV

Зато зимы порой холодной\\
Езда приятна и легка.\\
Как стих без мысли в песне модной,\\
Дорога зимняя гладка.\\
Автомедоны наши бойки,\\
Неутомимы наши тройки,\\
И версты, теша праздный взор,\\
В глазах мелькают, как забор.\\
К несчастью, Ларина тащилась,\\
Боясь прогонов дорогих,\\
Не на почтовых, на своих,\\
И наша дева насладилась\\
Дорожной скукою вполне:\\
Семь суток ехали оне.

XXXVI

Но вот уж близко. Перед ними\\
Уж белокаменной Москвы\\
Как жар, крестами золотыми\\
Горят старинные главы.\\
Ах, братцы! как я был доволен,\\
Когда церквей и колоколен,\\
Садов, чертогов полукруг\\
Открылся предо мною вдруг!\\
Как часто в горестной разлуке,\\
В моей блуждающей судьбе,\\
Москва, я думал о тебе!\\
Москва… как много в этом звуке\\
Для сердца русского слилось!\\
Как много в нем отозвалось!

XXXVII

Вот, окружен своей дубравой,\\
\hyperref[ch:дд]{Петровский Замок}. Мрачно он\\
Недавнею гордится славой.\\
Напрасно ждал Наполеон,\\
Последним счастьем упоенный,\\
Москвы коленопреклоненной\\
С ключами старого Кремля:\\
Нет, не пошла Москва моя\\
К нему с повинной головою.\\
Не праздник, не приемный дар,\\
Она готовила пожар\\
Нетерпеливому герою.\\
Отселе, в думу погружен,\\
Глядел на грозный пламень он.

XXXVIII\ftn{П. показывает две Москве Героическая и боярская}

Прощай, свидетель падшей славы,\\
Петровский замок. Ну! не стой,\\
Пошел! Уже столпы заставы\ftn{граница Москвы}\\
Белеют: вот уж по Тверской\\
Возок несется чрез ухабы.\\
Мелькают мимо будки, бабы,\\
Мальчишки\ftn{разносчики}, лавки, фонари,\\
Дворцы, сады, монастыри,\\
Бухарцы,\ftn{из города Бухара, эти продавцы восточных женских шали} сани, огороды,\ftn{много дворянины имели свою усадбу в Москве с огородом, конюшнои...}\\
Купцы, лачужки, мужики,\\
Бульвары, башни, казаки,\\
Аптеки, магазины моды,\\
Балконы, львы на воротах\ftn{герп}\\
И стаи галок\ftn{птичка} на крестах.

XXXIX. ХL

В сей утомительной прогулке\\
Проходит час-другой, и вот\\
У Харитонья в переулке\\
Возок пред домом у ворот\\
Остановился. К старой тетке,\\
Четвертый год больной в чахотке,\\
Они приехали теперь.\\
Им настежь отворяет дверь,\\
В очках, в изорванном кафтане,\\
С чулком в руке, седой калмык.\\
Встречает их в гостиной крик\\
Княжны, простертой на диване.\\
Старушки с плачем обнялись,\\
И восклицанья полились.

ХLI

— Княжна, mon аngе! —\\
«Раchеttе!» — Алина! —\\
«Кто б мог подумать? Как давно!\\
Надолго ль? Милая! Кузина!\\
Садись — как это мудрено!\\
Ей-богу, сцена из романа…»\\
— А это дочь моя, Татьяна. —\\
«Ах, Таня! подойди ко мне —\\
Как будто брежу я во сне…\\
Кузина, помнишь Грандисона?»\\
— Как, Грандисон?.. а, Грандисон!\\
Да, помню, помню. Где же он? —\\
«В Москве, живет у Симеона;\\
Меня в сочельник навестил;\\
Недавно сына он женил.

ХLII

А тот… но после все расскажем,\\
Не правда ль? Всей ее родне\\
Мы Таню завтра же покажем.\\
Жаль, разъезжать нет мочи мне;\\
Едва, едва таскаю ноги.\\
Но вы замучены с дороги;\\
Пойдемте вместе отдохнуть…\\
Ох, силы нет… устала грудь…\\
Мне тяжела теперь и радость,\\
Не только грусть… душа моя,\\
Уж никуда не годна я…\\
Под старость жизнь такая гадость…»\\
И тут, совсем утомлена,\\
В слезах раскашлялась она.

XLIII

Больной и ласки и веселье\\
Татьяну трогают; но ей\\
Нехорошо на новоселье,\\
Привыкшей к горнице своей.\\
Под занавескою шелковой\\
Не спится ей в постеле новой,\\
И ранний звон колоколов,\\
Предтеча утренних трудов,\\
Ее с постели подымает.\\
Садится Таня у окна.\\
Редеет сумрак; но она\\
Своих полей не различает:\\
Пред нею незнакомый двор,\\
Конюшня, кухня и забор.

XLIV

И вот: по родственным обедам\\
Развозят Таню каждый день\\
Представить бабушкам и дедам\\
Ее рассеянную лень.\\
Родне, прибывшей издалеча,\\
Повсюду ласковая встреча,\\
И восклицанья, и хлеб-соль.\\
«Как Таня выросла! Давно ль\\
Я, кажется, тебя крестила?\\
А я так на руки брала!\\
А я так за уши драла!\\
А я так пряником кормила!»\\
И хором бабушки твердят:\\
«Как наши годы-то летят!»

XLV

Но в них не видно перемены;\\
Все в них на старый образец:\\
У тетушки княжны Елены\\
Все тот же тюлевый чепец;\\
Все белится Лукерья Львовна,\\
Все то же лжет Любовь Петровна,\\
Иван Петрович так же глуп,\\
Семен Петрович так же скуп,\\
У Пелагеи Николавны\\
Все тот же друг мосье Финмуш,\\
И тот же шпиц, и тот же муж;\\
А он, все клуба член исправный,\\
Все так же смирен, так же глух\\
И так же ест и пьет за двух.

XLVI

Их дочки Таню обнимают.\\
Младые грации Москвы\\
Сначала молча озирают\\
Татьяну с ног до головы;\\
Ее находят что-то странной,\\
Провинциальной и жеманной,\\
И что-то бледной и худой,\\
А впрочем очень недурной;\\
Потом, покорствуя природе,\\
Дружатся с ней, к себе ведут,\\
Целуют, нежно руки жмут,\\
Взбивают кудри ей по моде\\
И поверяют нараспев\\
Сердечны тайны, тайны дев,

XLVII

Чужие и свои победы,\\
Надежды, шалости, мечты.\\
Текут невинные беседы\\
С прикрасой легкой клеветы.\\
Потом, в отплату лепетанья,\\
Ее сердечного признанья\\
Умильно требуют оне.\\
Но Таня, точно как во сне,\\
Их речи слышит без участья,\\
Не понимает ничего,\\
И тайну сердца своего,\\
Заветный клад и слез и счастья,\\
Хранит безмолвно между тем\\
И им не делится ни с кем.

XLVIII

Татьяна вслушаться желает\\
В беседы, в общий разговор;\\
Но всех в гостиной занимает\\
Такой бессвязный, пошлый вздор;\\
Все в них так бледно, равнодушно;\\
Они клевещут даже скучно;\\
В бесплодной сухости речей,\\
Расспросов, сплетен и вестей\\
Не вспыхнет мысли в целы сутки,\\
Хоть невзначай, хоть наобум;\\
Не улыбнется томный ум,\\
Не дрогнет сердце, хоть для шутки.\\
И даже глупости смешной\\
В тебе не встретишь, свет пустой.

XLIX

Архивны юноши толпою\\
На Таню чопорно глядят\\
И про нее между собою\\
Неблагосклонно говорят.\\
Один какой-то шут печальный\\
Ее находит идеальной\\
И, прислонившись у дверей,\\
Элегию готовит ей.\\
У скучной тетки Таню встретя,\\
К ней как-то Вяземский подсел\\
И душу ей занять успел.\\
И, близ него ее заметя,\\
Об ней, поправя свой парик,\\
Осведомляется старик.

L

Но там, где Мельпомены бурной\\
Протяжный раздается вой,\\
Где машет мантией мишурной\\
Она пред хладною толпой,\\
Где Талия тихонько дремлет\\
И плескам дружеским не внемлет,\\
Где Терпсихоре лишь одной\\
Дивится зритель молодой\\
(Что было также в прежни леты,\\
Во время ваше и мое),\\
Не обратились на нее\\
Ни дам ревнивые лорнеты,\\
Ни трубки модных знатоков\\
Из лож и кресельных рядов.

LI

Ее привозят и в Собранье.\\
Там теснота, волненье, жар,\\
Музыки грохот, свеч блистанье,\\
Мельканье, вихорь быстрых пар,\\
Красавиц легкие уборы,\\
Людьми пестреющие хоры,\\
Невест обширный полукруг,\\
Все чувства поражает вдруг.\\
Здесь кажут франты записные\\
Свое нахальство, свой жилет\\
И невнимательный лорнет.\\
Сюда гусары отпускные\\
Спешат явиться, прогреметь,\\
Блеснуть, пленить и улететь.

LII

У ночи много звезд прелестных,\\
Красавиц много на Москве.\\
Но ярче всех подруг небесных\\
Луна в воздушной синеве.\\
Но та, которую не смею\\
Тревожить лирою моею,\\
Как величавая луна,\\
Средь жен и дев блестит одна.\\
С какою гордостью небесной\\
Земли касается она!\\
Как негой грудь ее полна!\\
Как томен взор ее чудесный!..\\
Но полно, полно; перестань:\\
Ты заплатил безумству дань.

LIII

Шум, хохот, беготня, поклоны,\\
Галоп, мазурка, вальс… Меж тем,\\
Между двух теток у колонны,\\
Не замечаема никем,\\
Татьяна смотрит и не видит,\\
Волненье света ненавидит;\\
Ей душно здесь… она мечтой\\
Стремится к жизни полевой,\\
В деревню, к бедным поселянам,\\
В уединенный уголок,\\
Где льется светлый ручеек,\\
К своим цветам, к своим романам\\
И в сумрак липовых аллей,\\
Туда, где он являлся ей.

LIV

Так мысль ее далече\\
Забыт и свет и шумный бал,\\
А глаз меж тем с нее не сводит\\
Какой-то важный генерал.\\
Друг другу тетушки мигнули\\
И локтем Таню враз толкнули,\\
И каждая шепнула ей:\\
— Взгляни налево поскорей. —\\
«Налево? где? что там такое?»\\
— Ну, что бы ни было, гляди…\\
В той кучке, видишь? впереди,\\
Там, где еще в мундирах двое…\\
Вот отошел… вот боком стал… —\\
«Кто? толстый этот генерал?»

LV

Но здесь с победою поздравим\\
Татьяну милую мою\\
И в сторону свой путь направим,\\
Чтоб не забыть, о ком пою…\\
Да кстати, здесь о том два слова:\\
Пою приятеля младого\\
И множество его причуд.\\
Благослови мой долгий труд,\\
О ты, эпическая муза!\\
И, верный посох мне вручив,\\
Не дай блуждать мне вкось и вкрив.\\
Довольно. С плеч долой обуза!\\
Я классицизму отдал честь:\\
Хоть поздно, а вступленье есть.
\end{verse}
\chapter{Седмая глава}  
Время работы П над серединой седьмой главы (апрель 1828 г., подробнее см. в разделе «Хро­нология работы Пушкина над романом») совпадало со сложными процессами в творчестве П. В сознании поэта боролись две —в этот период противоположные и не находившие синтеза —тенденции. Первая из них — стремление к историзму, которое толкало П к принятию объективного хода исторических событий в том виде, в каком они даны в реальной действительности. 
С этих позиций требования, предъявляемые отдельной лич­
ностью к истории, третировались как «романтизм» и
«эгоизм». Не лишенные оттенка «примирения с дейст­
вительностью», такие настроения давали, однако, мощ­
ный толчок реалистическому и историческому сознанию и
определили целый ряд антиромантических выступлений
в творчестве П этих лет (от «Полтавы» и «Стансов»
(1826) до заметки о драмах Байрона). Однако пока еще
подспудно, в черновиках и глубинах сознания зрела
мысль о непреходящей ценности человеческой личности
и о необходимости мерить исторический прогресс
счастьем и правами отдельного человека. «Герой, будь
прежде человек» (1826) (VI, 411). «И нас они (домаш­
ние божества. —Ю. Л.У науке первой учат — Чтить
самого себя» (1829) (III, 1, 193), «Оставь герою сердце!
Что же Он будет без него? Тиран...» (1830) (III, 1, 253) —
такова цепь высказываний, которая закономерно при­
ведет к «Медному всаднику» и «Капитанской дочке».
\chapter{Петровский Замок}\label{ch:дд}
Дворцовый ансамбль — первая самостоятельная работа Матвея Казакова.\\
Зодчий, впоследствии перестроивший центр Москвы в палладианском стиле,
для дворца на окраине города выбрал псевдоготику, объединив мотивы древнерусской,
готической и восточной архитектуры.\\
 Расположилось величественное здание со стрельчатыми окнами на землях Высоко-Петровского монастыря. За что дворец и стал именоваться Петровским. Второе название — Путевой — объясняется его предназначением: для отдыха по пути из Петербурга в Москву. Дорога неблизкая — 700 километров.
 
Прообразом здания послужил временный город-театр в турецком стиле, построенный Василием Баженовым на Ходынском поле для торжеств по случаю подписания Кучук-Кайнаджирского мирного договора.\\
 Императрице работа понравилась, и она захотела похожий дворец. По задумке архитектора центральный купол символизировал православный храм, возвышаясь над башнями-минаретами, встроенными в боковые флигели. Богатство и величие Российской империи подчеркивали многочисленные лепные украшения, а пейзажный парк с гротами и беседками добавлял резиденции уюта.
Останавливалась Екатерина II в узорчатом замке лишь однажды, в 1787 году.\\
Спустя 10 лет путевую резиденцию посетил перед коронацией Павел I. А впоследствии и Николай II.\\
В 1812 году императорский замок на время стал ставкой Наполеона. Бонапарт покинул охваченный огнем Кремль и наблюдал из окна дворца за тем, как полыхает Москва.

Вот, окружен своей дубравой,\\
Петровский замок. Мрачно он\\ 
Недавнею гордится славой.\\
Напрасно ждал Наполеон,\\
Последним счастьем упоенный,\\
Москвы коленопреклоненной\\
С ключами старого Кремля:\\
Нет, не пошла Москва моя\\
К нему с повинной головою.\\
Не праздник, не приемный дар,\\
Она готовила пожар\\
Нетерпеливому герою.\\
Отселе, в думу погружен,\\
Глядел на грозный пламень он.

Так описывал те события Пушкин в «Евгении Онегине».\\
Но лагерь французской армии размещался здесь недолго, всего четыре дня.\\
Что не помешало солдатам и офицерам разграбить дворец и практически уничтожить парк. Выстояли только стены.\\
После бегства французов в боковых крыльях был устроен временный госпиталь.




