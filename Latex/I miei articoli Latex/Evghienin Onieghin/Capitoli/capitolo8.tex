\chapter{Capitolo 8}\label{section:cap8}% Etichetta per riferimenti incrociati

\begin{verse}

\textit{Fare thee well, and if for ever\\
Still for ever fare thee well.\\
\hyperref[sec:faretheewell]{Byron.}}

I

В те дни, когда в садах Лицея
Я безмятежно расцветал,
Читал охотно Апулея,
А Цицерона не читал,
В те дни в таинственных долинах,
Весной, при кликах лебединых,
Близ вод, сиявших в тишине,
Являться муза стала мне.
Моя студенческая келья
Вдруг озарилась: муза в ней
Открыла пир младых затей,
Воспела детские веселья,
И славу нашей старины,
И сердца трепетные сны.
II
И свет ее с улыбкой встретил;
Успех нас первый окрылил;
Старик Державин нас заметил
И в гроб сходя, благословил.
. . . . . . . . . . . . . . .
. . . . . . . . . . . . . . .
. . . . . . . . . . . . . . .
. . . . . . . . . . . . . . .
. . . . . . . . . . . . . . .
. . . . . . . . . . . . . . .
. . . . . . . . . . . . . . .
. . . . . . . . . . . . . . .
. . . . . . . . . . . . . . .
. . . . . . . . . . . . . . .
III
И я, в закон себе вменяя
Страстей единый произвол,
С толпою чувства разделяя,
Я музу резвую привел
На шум пиров и буйных споров,
Грозы полуночных дозоров;
И к ним в безумные пиры
Она несла свои дары
И как вакханочка резвилась,
За чашей пела для гостей,
И молодежь минувших дней
За нею буйно волочилась,
А я гордился меж друзей
Подругой ветреной моей.
IV
Но я отстал от их союза
И вдаль бежал… Она за мной.
Как часто ласковая муза
Мне услаждала путь немой
Волшебством тайного рассказа!
Как часто по скалам Кавказа
Она Ленорой, при луне,
Со мной скакала на коне!
Как часто по брегам Тавриды
Она меня во мгле ночной
Водила слушать шум морской,
Немолчный шепот Нереиды,
Глубокий, вечный хор валов,
Хвалебный гимн отцу миров.
V
И, позабыв столицы дальной
И блеск и шумные пиры,
В глуши Молдавии печальной
Она смиренные шатры
Племен бродящих посещала,
И между ими одичала,
И позабыла речь богов
Для скудных, странных языков,
Для песен степи, ей любезной…
Вдруг изменилось все кругом,
И вот она в саду моем
Явилась барышней уездной,
С печальной думою в очах,
С французской книжкою в руках.
VI
И ныне музу я впервые
На светский раут привожу;
На прелести ее степные
С ревнивой робостью гляжу.
Сквозь тесный ряд аристократов,
Военных франтов, дипломатов
И гордых дам она скользит;
Вот села тихо и глядит,
Любуясь шумной теснотою,
Мельканьем платьев и речей,
Явленьем медленным гостей
Перед хозяйкой молодою
И темной рамою мужчин
Вкруг дам как около картин.
VII
Ей нравится порядок стройный
Олигархических бесед,
И холод гордости спокойной,
И эта смесь чинов и лет.
Но это кто в толпе избранной
Стоит безмолвный и туманный?
Для всех он кажется чужим.
Мелькают лица перед ним
Как ряд докучных привидений.
Что, сплин иль страждущая спесь
В его лице? Зачем он здесь?
Кто он таков? Ужель Евгений?
Ужели он?.. Так, точно он.
— Давно ли к нам он занесен?
VIII
Все тот же ль он иль усмирился?
Иль корчит также чудака?
Скажите: чем он возвратился?
Что нам представит он пока?
Чем ныне явится? Мельмотом,
Космополитом, патриотом,
Гарольдом, квакером, ханжой,
Иль маской щегольнет иной,
Иль просто будет добрый малый,
Как вы да я, как целый свет?
По крайней мере мой совет:
Отстать от моды обветшалой.
Довольно он морочил свет…
— Знаком он вам? — И да и нет.
IX
— Зачем же так неблагосклонно
Вы отзываетесь о нем?
За то ль, что мы неугомонно
Хлопочем, судим обо всем,
Что пылких душ неосторожность
Самолюбивую ничтожность
Иль оскорбляет, иль смешит,
Что ум, любя простор, теснит,
Что слишком часто разговоры
Принять мы рады за дела,
Что глупость ветрена и зла,
Что важным людям важны вздоры
И что посредственность одна
Нам по плечу и не странна?
X
Блажен, кто смолоду был молод,
Блажен, кто вовремя созрел,
Кто постепенно жизни холод
С летами вытерпеть умел;
Кто странным снам не предавался,
Кто черни светской не чуждался,
Кто в двадцать лет был франт иль хват,
А в тридцать выгодно женат;
Кто в пятьдесят освободился
От частных и других долгов,
Кто славы, денег и чинов
Спокойно в очередь добился,
О ком твердили целый век:
N. N. прекрасный человек.
XI
Но грустно думать, что напрасно
Была нам молодость дана,
Что изменяли ей всечасно,
Что обманула нас она;
Что наши лучшие желанья,
Что наши свежие мечтанья
Истлели быстрой чередой,
Как листья осенью гнилой.
Несносно видеть пред собою
Одних обедов длинный ряд,
Глядеть на жизнь, как на обряд,
И вслед за чинною толпою
Идти, не разделяя с ней
Ни общих мнений, ни страстей.
XII
Предметом став суждений шумных,
Несносно (согласитесь в том)
Между людей благоразумных
Прослыть притворным чудаком,
Или печальным сумасбродом,
Иль сатаническим уродом,
Иль даже демоном моим.
Онегин (вновь займуся им),
Убив на поединке друга,
Дожив без цели, без трудов
До двадцати шести годов,
Томясь в бездействии досуга
Без службы, без жены, без дел,
Ничем заняться не умел.
XIII
Им овладело беспокойство,
Охота к перемене мест
(Весьма мучительное свойство,
Немногих добровольный крест).
Оставил он свое селенье,
Лесов и нив уединенье,
Где окровавленная тень
Ему являлась каждый день,
И начал странствия без цели,
Доступный чувству одному;
И путешествия ему,
Как все на свете, надоели;
Он возвратился и попал,
Как Чацкий, с корабля на бал.
XIV
Но вот толпа заколебалась,
По зале шепот пробежал…
К хозяйке дама приближалась,
За нею важный генерал.
Она была нетороплива,
Не холодна, не говорлива,
Без взора наглого для всех,
Без притязаний на успех,
Без этих маленьких ужимок,
Без подражательных затей…
Все тихо, просто было в ней,
Она казалась верный снимок
Du comme il faut… (Шишков, прости:
Не знаю, как перевести.)
XV
К ней дамы подвигались ближе;
Старушки улыбались ей;
Мужчины кланялися ниже,
Ловили взор ее очей;
Девицы проходили тише
Пред ней по зале, и всех выше
И нос и плечи подымал
Вошедший с нею генерал.
Никто б не мог ее прекрасной
Назвать; но с головы до ног
Никто бы в ней найти не мог
Того, что модой самовластной
В высоком лондонском кругу
Зовется vulgаr. (Не могу…
XVI
Люблю я очень это слово,
Но не могу перевести;
Оно у нас покамест ново,
И вряд ли быть ему в чести.
Оно б годилось в эпиграмме…)
Но обращаюсь к нашей даме.
Беспечной прелестью мила,
Она сидела у стола
С блестящей Ниной Воронскою,
Сей Клеопатрою Невы;
И верно б согласились вы,
Что Нина мраморной красою
Затмить соседку не могла,
Хоть ослепительна была.
XVII
«Ужели, — думает Евгений: —
Ужель она? Но точно… Нет…
Как! из глуши степных селений…»
И неотвязчивый лорнет
Он обращает поминутно
На ту, чей вид напомнил смутно
Ему забытые черты.
«Скажи мне, князь, не знаешь ты,
Кто там в малиновом берете
С послом испанским говорит?»
Князь на Онегина глядит.
— Ага! давно ж ты не был в свете.
Постой, тебя представлю я. —
«Да кто ж она?» — Жена моя. —
XVIII
«Так ты женат! не знал я ране!
Давно ли?» — Около двух лет. —
«На ком?» — На Лариной. — «Татьяне!»
— Ты ей знаком? — «Я им сосед».
— О, так пойдем же. — Князь подходит
К своей жене и ей подводит
Родню и друга своего.
Княгиня смотрит на него…
И что ей душу ни смутило,
Как сильно ни была она
Удивлена, поражена,
Но ей ничто не изменило:
В ней сохранился тот же тон,
Был так же тих ее поклон.
XIX
Ей-ей! не то, чтоб содрогнулась
Иль стала вдруг бледна, красна…
У ней и бровь не шевельнулась;
Не сжала даже губ она.
Хоть он глядел нельзя прилежней,
Но и следов Татьяны прежней
Не мог Онегин обрести.
С ней речь хотел он завести
И — и не мог. Она спросила,
Давно ль он здесь, откуда он
И не из их ли уж сторон?
Потом к супругу обратила
Усталый взгляд; скользнула вон…
И недвижим остался он.
XX
Ужель та самая Татьяна,
Которой он наедине,
В начале нашего романа,
В глухой, далекой стороне,
В благом пылу нравоученья,
Читал когда-то наставленья,
Та, от которой он хранит
Письмо, где сердце говорит,
Где все наруже, все на воле,
Та девочка… иль это сон?..
Та девочка, которой он
Пренебрегал в смиренной доле,
Ужели с ним сейчас была
Так равнодушна, так смела?
XXI
Он оставляет раут тесный,
Домой задумчив едет он;
Мечтой то грустной, то прелестной
Его встревожен поздний сон.
Проснулся он; ему приносят
Письмо: князь N покорно просит
Его на вечер. «Боже! к ней!..
О буду, буду!» и скорей
Марает он ответ учтивый.
Что с ним? в каком он странном сне!
Что шевельнулось в глубине
Души холодной и ленивой?
Досада? суетность? иль вновь
Забота юности — любовь?
XXII
Онегин вновь часы считает,
Вновь не дождется дню конца.
Но десять бьет; он выезжает,
Он полетел, он у крыльца,
Он с трепетом к княгине входит;
Татьяну он одну находит,
И вместе несколько минут
Они сидят. Слова нейдут
Из уст Онегина. Угрюмый,
Неловкий, он едва-едва
Ей отвечает. Голова
Его полна упрямой думой.
Упрямо смотрит он: она
Сидит покойна и вольна.
XXIII
Приходит муж. Он прерывает
Сей неприятный tete-a-tete;
С Онегиным он вспоминает
Проказы, шутки прежних лет.
Они смеются. Входят гости.
Вот крупной солью светской злости
Стал оживляться разговор;
Перед хозяйкой легкий вздор
Сверкал без глупого жеманства,
И прерывал его меж тем
Разумный толк без пошлых тем,
Без вечных истин, без педантства,
И не пугал ничьих ушей
Свободной живостью своей.
XXIV
Тут был, однако, цвет столицы,
И знать, и моды образцы,
Везде встречаемые лицы,
Необходимые глупцы;
Тут были дамы пожилые
В чепцах и в розах, с виду злые;
Тут было несколько девиц,
Не улыбающихся лиц;
Тут был посланник, говоривший
О государственных делах;
Тут был в душистых сединах
Старик, по-старому шутивший:
Отменно тонко и умно,
Что нынче несколько смешно.
XXV
Тут был на эпиграммы падкий,
На все сердитый господин:
На чай хозяйский слишком сладкий,
На плоскость дам, на тон мужчин,
На толки про роман туманный,
На вензель, двум сестрицам данный,
На ложь журналов, на войну,
На снег и на свою жену.
. . . . . . . . . . . . . . .
. . . . . . . . . . . . . . .
. . . . . . . . . . . . . . .
. . . . . . . . . . . . . . .
. . . . . . . . . . . . . . .
. . . . . . . . . . . . . . .
XXVI
Тут был Проласов, заслуживший
Известность низостью души,
Во всех альбомах притупивший,
St.-Рriest, твои карандаши;
В дверях другой диктатор бальный
Стоял картинкою журнальной,
Румян, как вербный херувим,
Затянут, нем и недвижим,
И путешественник залетный,
Перекрахмаленный нахал,
В гостях улыбку возбуждал
Своей осанкою заботной,
И молча обмененный взор
Ему был общий приговор.
XXVII
Но мой Онегин вечер целый
Татьяной занят был одной,
Не этой девочкой несмелой,
Влюбленной, бедной и простой,
Но равнодушною княгиней,
Но неприступною богиней
Роскошной, царственной Невы.
О люди! все похожи вы
На прародительницу Эву:
Что вам дано, то не влечет,
Вас непрестанно змий зовет
К себе, к таинственному древу;
Запретный плод вам подавай:
А без того вам рай не рай.
XXVIII
Как изменилася Татьяна!
Как твердо в роль свою вошла!
Как утеснительного сана
Приемы скоро приняла!
Кто б смел искать девчонки нежной
В сей величавой, в сей небрежной
Законодательнице зал?
И он ей сердце волновал!
Об нем она во мраке ночи,
Пока Морфей не прилетит,
Бывало, девственно грустит,
К луне подъемлет томны очи,
Мечтая с ним когда-нибудь
Свершить смиренный жизни путь!
XXIX
Любви все возрасты покорны;
Но юным, девственным сердцам
Ее порывы благотворны,
Как бури вешние полям:
В дожде страстей они свежеют,
И обновляются, и зреют —
И жизнь могущая дает
И пышный цвет и сладкий плод.
Но в возраст поздний и бесплодный,
На повороте наших лет,
Печален страсти мертвой след:
Так бури осени холодной
В болото обращают луг
И обнажают лес вокруг.
XXX
Сомненья нет: увы! Евгений
В Татьяну как дитя влюблен;
В тоске любовных помышлений
И день и ночь проводит он.
Ума не внемля строгим пеням,
К ее крыльцу, стеклянным сеням
Он подъезжает каждый день;
За ней он гонится как тень;
Он счастлив, если ей накинет
Боа пушистый на плечо,
Или коснется горячо
Ее руки, или раздвинет
Пред нею пестрый полк ливрей,
Или платок подымет ей.
XXXI
Она его не замечает,
Как он ни бейся, хоть умри.
Свободно дома принимает,
В гостях с ним молвит слова три,
Порой одним поклоном встретит,
Порою вовсе не заметит:
Кокетства в ней ни капли нет —
Его не терпит высший свет.
Бледнеть Онегин начинает:
Ей иль не видно, иль не жаль;
Онегин сохнет — и едва ль
Уж не чахоткою страдает.
Все шлют Онегина к врачам,
Те хором шлют его к водам.
XXXII
А он не едет; он заране
Писать ко прадедам готов
О скорой встрече; а Татьяне
И дела нет (их пол таков);
А он упрям, отстать не хочет,
Еще надеется, хлопочет;
Смелей здорового, больной,
Княгине слабою рукой
Он пишет страстное посланье.
Хоть толку мало вообще
Он в письмах видел не вотще;
Но, знать, сердечное страданье
Уже пришло ему невмочь.
Вот вам письмо его точь-в-точь.
Письмо Онегина к Татьяне
Предвижу все: вас оскорбит
Печальной тайны объясненье.
Какое горькое презренье
Ваш гордый взгляд изобразит!
Чего хочу? с какою целью
Открою душу вам свою?
Какому злобному веселью,
Быть может, повод подаю!
Случайно вас когда-то встретя,
В вас искру нежности заметя,
Я ей поверить не посмел:
Привычке милой не дал ходу;
Свою постылую свободу
Я потерять не захотел.
Еще одно нас разлучило…
Несчастной жертвой Ленский пал…
Ото всего, что сердцу мило,
Тогда я сердце оторвал;
Чужой для всех, ничем не связан,
Я думал: вольность и покой
Замена счастью. Боже мой!
Как я ошибся, как наказан.
Нет, поминутно видеть вас,
Повсюду следовать за вами,
Улыбку уст, движенье глаз
Ловить влюбленными глазами,
Внимать вам долго, понимать
Душой все ваше совершенство,
Пред вами в муках замирать,
Бледнеть и гаснуть… вот блаженство!
И я лишен того: для вас
Тащусь повсюду наудачу;
Мне дорог день, мне дорог час:
А я в напрасной скуке трачу
Судьбой отсчитанные дни.
И так уж тягостны они.
Я знаю: век уж мой измерен;
Но чтоб продлилась жизнь моя,
Я утром должен быть уверен,
Что с вами днем увижусь я…
Боюсь: в мольбе моей смиренной
Увидит ваш суровый взор
Затеи хитрости презренной —
И слышу гневный ваш укор.
Когда б вы знали, как ужасно
Томиться жаждою любви,
Пылать — и разумом всечасно
Смирять волнение в крови;
Желать обнять у вас колени
И, зарыдав, у ваших ног
Излить мольбы, признанья, пени,
Все, все, что выразить бы мог,
А между тем притворным хладом
Вооружать и речь и взор,
Вести спокойный разговор,
Глядеть на вас веселым взглядом!..
Но так и быть: я сам себе
Противиться не в силах боле;
Все решено: я в вашей воле
И предаюсь моей судьбе.
XXXIII
Ответа нет. Он вновь посланье:
Второму, третьему письму
Ответа нет. В одно собранье
Он едет; лишь вошел… ему
Она навстречу. Как сурова!
Его не видят, с ним ни слова;
У! как теперь окружена
Крещенским холодом она!
Как удержать негодованье
Уста упрямые хотят!
Вперил Онегин зоркий взгляд:
Где, где смятенье, состраданье?
Где пятна слез?.. Их нет, их нет!
На сем лице лишь гнева след…
XXXIV
Да, может быть, боязни тайной,
Чтоб муж иль свет не угадал
Проказы, слабости случайной…
Всего, что мой Онегин знал…
Надежды нет! Он уезжает,
Свое безумство проклинает —
И, в нем глубоко погружен,
От света вновь отрекся он.
И в молчаливом кабинете
Ему припомнилась пора,
Когда жестокая хандра
За ним гналася в шумном свете,
Поймала, за ворот взяла
И в темный угол заперла.
XXXV
Стал вновь читать он без разбора.
Прочел он Гиббона, Руссо,
Манзони, Гердера, Шамфора,
Madame de Stael, Биша, Тиссо,
Прочел скептического Беля,
Прочел творенья Фонтенеля,
Прочел из наших кой-кого,
Не отвергая ничего:
И альманахи, и журналы,
Где поученья нам твердят,
Где нынче так меня бранят,
А где такие мадригалы
Себе встречал я иногда:
Е sempre bene, господа.
XXXVI
И что ж? Глаза его читали,
Но мысли были далеко;
Мечты, желания, печали
Теснились в душу глубоко.
Он меж печатными строками
Читал духовными глазами
Другие строки. В них-то он
Был совершенно углублен.
То были тайные преданья
Сердечной, темной старины,
Ни с чем не связанные сны,
Угрозы, толки, предсказанья,
Иль длинной сказки вздор живой,
Иль письма девы молодой.
XXXVII
И постепенно в усыпленье
И чувств и дум впадает он,
А перед ним воображенье
Свой пестрый мечет фараон.
То видит он: на талом снеге,
Как будто спящий на ночлеге,
Недвижим юноша лежит,
И слышит голос: что ж? убит.
То видит он врагов забвенных,
Клеветников, и трусов злых,
И рой изменниц молодых,
И круг товарищей презренных,
То сельский дом — и у окна
Сидит она… и все она!..
XXXVIII
Он так привык теряться в этом,
Что чуть с ума не своротил
Или не сделался поэтом.
Признаться: то-то б одолжил!
А точно: силой магнетизма
Стихов российских механизма
Едва в то время не постиг
Мой бестолковый ученик.
Как походил он на поэта,
Когда в углу сидел один,
И перед ним пылал камин,
И он мурлыкал: Веnеdеttа
Иль Idol mio и ронял
В огонь то туфлю, то журнал.
XXXIX
Дни мчались; в воздухе нагретом
Уж разрешалася зима;
И он не сделался поэтом,
Не умер, не сошел с ума.
Весна живит его: впервые
Свои покои запертые,
Где зимовал он, как сурок,
Двойные окны, камелек
Он ясным утром оставляет,
Несется вдоль Невы в санях.
На синих, иссеченных льдах
Играет солнце; грязно тает
На улицах разрытый снег.
Куда по нем свой быстрый бег.
ХL
Стремит Онегин? Вы заране
Уж угадали; точно так:
Примчался к ней, к своей Татьяне
Мой неисправленный чудак.
Идет, на мертвеца похожий.
Нет ни одной души в прихожей.
Он в залу; дальше: никого.
Дверь отворил он. Что ж его
С такою силой поражает?
Княгиня перед ним, одна,
Сидит, не убрана, бледна,
Письмо какое-то читает
И тихо слезы льет рекой,
Опершись на руку щекой.
ХLI
О, кто б немых ее страданий
В сей быстрый миг не прочитал!
Кто прежней Тани, бедной Тани
Теперь в княгине б не узнал!
В тоске безумных сожалений
К ее ногам упал Евгений;
Она вздрогнула и молчит;
И на Онегина глядит
Без удивления, без гнева…
Его больной, угасший взор,
Молящий вид, немой укор,
Ей внятно все. Простая дева,
С мечтами, сердцем прежних дней,
Теперь опять воскресла в ней.
XLII
Она его не подымает
И, не сводя с него очей,
От жадных уст не отымает
Бесчувственной руки своей…
О чем теперь ее мечтанье?
Проходит долгое молчанье,
И тихо наконец она:
«Довольно; встаньте. Я должна
Вам объясниться откровенно.
Онегин, помните ль тот час,
Когда в саду, в аллее нас
Судьба свела, и так смиренно
Урок ваш выслушала я?
Сегодня очередь моя.
XLIII
Онегин, я тогда моложе,
Я лучше, кажется, была,
И я любила вас; и что же?
Что в сердце вашем я нашла?
Какой ответ? одну суровость.
Не правда ль? Вам была не новость
Смиренной девочки любовь?
И нынче — боже! — стынет кровь,
Как только вспомню взгляд холодный
И эту проповедь… Но вас
Я не виню: в тот страшный час
Вы поступили благородно,
Вы были правы предо мной:
Я благодарна всей душой…
XLIV
Тогда — не правда ли? — в пустыне,
Вдали от суетной молвы,
Я вам не нравилась… Что ж ныне
Меня преследуете вы?
Зачем у вас я на примете?
Не потому ль, что в высшем свете
Теперь являться я должна;
Что я богата и знатна,
Что муж в сраженьях изувечен,
Что нас за то ласкает двор?
Не потому ль, что мой позор
Теперь бы всеми был замечен,
И мог бы в обществе принесть
Вам соблазнительную честь?
XLV
Я плачу… если вашей Тани
Вы не забыли до сих пор,
То знайте: колкость вашей брани,
Холодный, строгий разговор,
Когда б в моей лишь было власти,
Я предпочла б обидной страсти
И этим письмам и слезам.
К моим младенческим мечтам
Тогда имели вы хоть жалость,
Хоть уважение к летам…
А нынче! — что к моим ногам
Вас привело? какая малость!
Как с вашим сердцем и умом
Быть чувства мелкого рабом?
XLVI
А мне, Онегин, пышность эта,
Постылой жизни мишура,
Мои успехи в вихре света,
Мой модный дом и вечера,
Что в них? Сейчас отдать я рада
Всю эту ветошь маскарада,
Весь этот блеск, и шум, и чад
За полку книг, за дикий сад,
За наше бедное жилище,
За те места, где в первый раз,
Онегин, видела я вас,
Да за смиренное кладбище,
Где нынче крест и тень ветвей
Над бедной нянею моей…
XLVII
А счастье было так возможно,
Так близко!.. Но судьба моя
Уж решена. Неосторожно,
Быть может, поступила я:
Меня с слезами заклинаний
Молила мать; для бедной Тани
Все были жребии равны…
Я вышла замуж. Вы должны,
Я вас прошу, меня оставить;
Я знаю: в вашем сердце есть
И гордость и прямая честь.
Я вас люблю (к чему лукавить?),
Но я другому отдана;
Я буду век ему верна».
XLVIII
Она ушла. Стоит Евгений,
Как будто громом поражен.
В какую бурю ощущений
Теперь он сердцем погружен!
Но шпор незапный звон раздался,
И муж Татьянин показался,
И здесь героя моего,
В минуту, злую для него,
Читатель, мы теперь оставим,
Надолго… навсегда. За ним
Довольно мы путем одним
Бродили по свету. Поздравим
Друг друга с берегом. Ура!
Давно б (не правда ли?) пора!
XLIX
Кто б ни был ты, о мой читатель,
Друг, недруг, я хочу с тобой
Расстаться нынче как приятель.
Прости. Чего бы ты за мной
Здесь ни искал в строфах небрежных,
Воспоминаний ли мятежных,
Отдохновенья ль от трудов,
Живых картин, иль острых слов,
Иль грамматических ошибок,
Дай бог, чтоб в этой книжке ты
Для развлеченья, для мечты,
Для сердца, для журнальных сшибок
Хотя крупицу мог найти.
За сим расстанемся, прости!
L
Прости ж и ты, мой спутник странный,
И ты, мой верный идеал,
И ты, живой и постоянный,
Хоть малый труд. Я с вами знал
Все, что завидно для поэта:
Забвенье жизни в бурях света,
Беседу сладкую друзей.
Промчалось много, много дней
С тех пор, как юная Татьяна
И с ней Онегин в смутном сне
Явилися впервые мне —
И даль свободного романа
Я сквозь магический кристалл
Еще не ясно различал.
LI
Но те, которым в дружной встрече
Я строфы первые читал…
Иных уж нет, а те далече,
Как Сади некогда сказал.
Без них Онегин дорисован.
А та, с которой образован
Татьяны милый идеал…
О много, много рок отъял!
Блажен, кто праздник жизни рано
Оставил, не допив до дна
Бокала полного вина,
Кто не дочел ее романа
И вдруг умел расстаться с ним,
Как я с Онегиным моим.
Конец
1830 г.
\end{verse}