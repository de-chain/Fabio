\documentclass[12pt,a4paper]{article}
\usepackage[utf8]{inputenc}
\usepackage[russian]{babel}
\usepackage[OT1]{fontenc}
\usepackage{amsmath}
\usepackage{amsfonts}
\usepackage{amssymb}
%\usepackage{makeidx}
\usepackage{lmodern}
\author{Юрий Михайлович Лотман}
\title{Быт и Культура}
\begin{document}
\maketitle % Comando per stampare il titolo, l'autore e la data

\tableofcontents % Comando per generare l'indice
\newpage

\section{Культура как понятие коллективное}

Отдельный человек может быть носителем культуры, может
активно участвовать в ее развитии, тем не менее по своей
природе культура, как и язык, – явление общественное, то
есть социальное.

Она имеет коммуникационную и символическую природу.

Меч также не более чем предмет. Как вещь он может
быть выкован или сломан, его можно поместить в витрину
музея, и им можно убить человека. Это все – употребление
его как предмета, но когда, будучи прикреплен к поясу или
поддерживаемый перевязью помещен на бедре, меч
символизирует свободного человека и является «знаком
свободы», он уже предстает как символ и принадлежит
культуре.

\section{Культура как память}
Культура всегда связана с историей, всегда подразумевает непрерывность
нравственной, интеллектуальной, духовной жизни человека,
общества и человечества.
 
Человек меняется, и, чтобы представить себе логику
поступков литературного героя надо представлять себе, как жили люди тои поры, какой мир их окружал, каковы были их общие
представления и представления нравственные, их
служебные обязанности, обычаи, одежда, почему они поступали так, а не иначе.
 
Быт – это обычное протекание жизни в ее
реально-практических формах; быт – это вещи, которые
окружают нас, наши привычки и каждодневное поведение.

Быт окружает нас как воздух, и, как воздух, он заметен нам только тогда, когда его не хватает или он портится.
\section{Допетровское дворянство}
Дворянство XVIII в представляло собой служилый класс, особенно военное.
За службу оно  получило  землю.
  
Переставая служить, дворянин должен был вернуть пожалованные ему земли в казну.
Иногда земдью предавалась по наследстве.

Дворянин получившии землью за слугу был тесно связан с личной преданностью государю его назвали воинник. 
 
Дворянин получившии землью в наследстве был привязан к земле, к чести его фамилии его назвали вотчинник.
\section{Реформы Пётра 1}
Они представляли собой Идея порядка, «регулярного государства», но они не были внушены Петру I путешествием в Голландию.

XVII век был «бунташным» веком. так называемая "крестьянская война" не отвечала интересам всего крестьянства просто это была безмисленная, беспощадная воина, в которой многочисленные вооруженные банды «гулящих людей»,причинили сельскому населению России неизмеримые страдания.

Опустошенные и разграбленные села, крестьянские избы, забитые трупами, голод, бегство населения – такая картина возникает из документов. Мятежи и бунты вызывали неслыханную вспышку разбойничества. 
 
При этом мы забываем слова Пушкина: «Не приведи Бог видеть русский бунт – бессмысленный и беспощадный».

Реформы отвечали на вопль Россию и дали ей новые работники: офицеры для армии и флота, чиновники и дипломаты, администраторы и инженеры, ученые.
которая еще не Петровская реформа залечила раны «бунташного века» и одновременно не могла себе представить, во что обойдется ей эта «регулярность».
\section{Табель о рангах}
Вершиной Петровскими реформы явилась Табель
о рангах, вырабатывавшаяся в течение ряда лет при
постоянном и активном участии Петра I и опубликованная в январе 1722 года.  
Это новая  петровская государственность по  принципе «регулярности».

Формы петербургской (а в каком-то смысле и всей
русской городской) жизни создал Петр I. Идеалом его было,
как он сам выражался, регулярное – правильное –
государство, где вся жизнь регламентирована, подчинена
правилам, выстроена с соблюдением геометрических
пропорций, сведена к точным, однолинейным отношениям.
Проспекты прямые, дворцы возведены по официально
утвержденным проектам, все выверено и логически
обосновано.
Петербург пробуждался по барабану: по этому
знаку солдаты приступали к учениям, чиновники бежали
в департаменты. Человек XVIII  века жил как бы в двух
измерениях: полдня, полжизни он посвящал
государственной службе, время которой было точно
установлено регламентом, полдня он находился вне ее.

\textbf{Но со времени регулярность перерождалась в реальность
бюрократическую.}

Что представляла собой Табель о рангах?  Люди должны занимать должности по своим способностям и по своему реальному вкладу в государственное дело.
Табель о рангах и устанавливала зависимость
общественного положения человека от его места
в служебной иерархии.
 
Табель о рангах делила все виды службы на:

воинскую
 
статскую

придворную

Первая, в свою очередь, делилась
на сухопутную и морскую (особо была выделена гвардия).

Все чины были разделены на 14 классов,

первые пять составляли генералитет

Классы VI–VIII составляли штаб-офицерские,

IX–XIV – обер-офицерские чины.

Табель о рангах ставила военную службу
в привилегированное положение. Это выражалось,
в частности, в том, что все 14 классов в воинской службе
давали право наследственного дворянства, в статской же
службе такое право давалось лишь начиная с VIII класса.

Это означало, что самый низший обер-офицерский чин
в военной службе уже давал потомственное дворянство.
 
\paragraph{Статская} не считалась «благородной».
Ее называли «подьяческой», в ней всегда было больше
разночинцев, и ею принято было гнушаться.

Правительственная склонность к военному
управлению и та симпатия, которой пользовался мундир
в обществе, – в частности, дамском, – проистекали из разных
источников. Первое обусловлено общим характером власти.

Русские императоры были военными и получали военное
воспитание и образование. Они привыкали с детства
смотреть на армию как на идеал организации; их
эстетические представления складывались под влиянием
парадов, они носили фраки только путешествуя за границей
инкогнито. Нерассуждающий, исполнительный офицер
представлялся им наиболее надежной и психологически
понятной фигурой.

Даже среди статских чиновников
империи трудно назвать лицо, которое хотя бы в молодости,
хоть несколько лет не носило бы офицерского мундира.

Иную основу имел «культ мундира» в дворянском быту.
Конечно, особенно в глазах прекрасного пола, не
последнюю роль играла эстетическая оценка: 
сверкающий золотом или серебром гусарский, сине-красный
уланский, белый (парадный) конногвардейский мундир был
красивее, чем бархатный кафтан щеголя или синий фрак
англомана.
\section{Офицер и чиновник}
Власть государства покоилась на двух фигурах:

офицера

чиновника

Чиновник – человек, самоназвание которого производится от слова «чин». 

Чин» в древнерусском языке означает «порядок». И хотя чин,
вопреки замыслам Петра, очень скоро разошелся
с реальной должностью человека, превратившись в почти
мистическую бюрократическую функцию, функция эта имела
в то же время и совершенно практический смысл.

Чиновник – человек жалованья, его благосостояние
непосредственно зависит от государства. Он привязан
к административной машине и не может без нее
существовать. Чиновник, зависящий от жалованья и чина,
оказался в России наиболее надежным слугой государства.

Но со врем и Запутанность законов и общий дух государственного
произвола, ярчайшим образом проявившийся
в чиновничьей службе, привели (и не могли не привести)
к тому, что русская культура XVIII  – начала XIX века
практически не создала образов беспристрастного судьи,
справедливого администратора – бескорыстного защитника
слабых и угнетенных. 

Чиновник в общественном сознании
ассоциировался с крючкотвором и взяточником.\\
Капнист в комедии  "Ябеда" заставил хор провинциальных чиновников петь
куплет:

Бери, большой тут нет науки;\\
Бери, что можно только взять.\\
На что ж привешены нам руки,\\
Как не на то, чтоб брать?\\
\section{Чинопочетание}
Одновременно с распределением чинов шло
распределение выгод и почестей. Бюрократическое
государство создало огромную лестницу человеческих отношений, нам сейчас совершенно непонятных.\\
Для особ I и II классов таким обращением было
«ваше высокопревосходительство».\\ 
Особы III и IV классов –
«превосходительства» («ваше превосходительство» надо
было писать также и университетскому ректору, независимо
от его чина).\\ V (тот самый выморочный класс бригадиров,
о котором говорилось выше) требовал обращения «ваше
высокородие».\\ К лицам VI–VIII классов обращались «ваше
высокоблагородие».\\ K лицам IX–XIV классов – «ваше
благородие» (впрочем, в быту так обратиться можно было
и к любому дворянину, независимо от его чина).\\
Настоящей наукой являлись правила обращения к царю. На конверте,
например, надо было писать: «Его императорскому
величеству государю императору», а само обращение
начинать словами: «Всемилостивейший монарх» или:
«Августейший монарх».\\ Даже духовную сферу Петр I как бы
регламентировал: митрополит и архиепископ – «ваше
высокопреосвященство» архимандрит и игумен – «ваше высокопреподобие», священник – «ваше преподобие».

\paragraph*{Место чина}
  в служебной иерархии связано было
с получением (или неполучением) многих реальных
привилегий. 
По чинам, к примеру, давали лошадей на
почтовых станциях.

В XVIII веке, при Петре I, в России учреждена была
«регулярная» почта. Она представляла собой сеть станций,
управляемых специальными чиновниками, фигуры которых
сделались позже одними из персонажей «петербургского
мифа» (вспомним «Станционного смотрителя» Пушкина).\\
В распоряжении станционного смотрителя находились
государственные ямщики, кибитки, лошади. Те, кто ездил по
государственной надобности – с подорожной или же по
своей надобности, но на прогонных почтовых лошадях,
приезжая на станцию, оставляли усталых лошадей и брали
свежих. Стоимость езды для фельдъегерей оплачивалась
государством. Едущие «по собственной надобности» платили
за лошадей.
Поэтому провинциальный помещик предпочитал ездить
на собственных лошадях, что замедляло путешествие, но
делало его значительно дешевле.\\
Так, пушкинская Ларина
…тащилась,\\
Боясь прогонов дорогих,\\
Не на почтовых, на своих.\\
Поместье Лариных, видимо, находилось в Псковской
губернии. Путь оттуда в Москву «на почтовых» длился
обычно трое суток, а на курьерских – двое. Экономия
Лариной привела к тому, что\\
…Наша дева насладилась\\
Почтовой скукою вполне:\\
Семь суток ехали оне.\\
При получении лошадей на станциях существовал
строгий порядок: вперед, без очереди, пропускались
фельдъегеря со срочными государственными пакетами,
а остальным давали лошадей по чинам: особы I–III классов
могли брать до двенадцати лошадей, с IV класса – до восьми
и так далее, вплоть до бедных чиновников VI–IX классов,
которым приходилось довольствоваться одной каретой
с двумя лошадьми. Но часто бывало и по-другому:
проезжему генералу отдали всех лошадей – остальные сидят
и ждут… А лихой гусарский поручик, приехавший на станцию
пьяным, мог побить беззащитного станционного смотрителя
и силой забрать лошадей больше, чем ему было положено.

\section{Примеры}

С одной стороны, чин – это некая узаконенная фикция,
слово, обозначающее не реальные свойства человека, а его
место в иерархии. Псевдобытие бюрократии придает
существованию человека-чина призрачность.\\ Это как бы
существование. Герой «Записок сумасшедшего» Гоголя
возмущался: «Что ж из того, что он камер-юнкер. Ведь это
больше ничего кроме достоинство; не какая-нибудь
вещь видимая, которую бы можно взять в руки.\\ Ведь через
то, что камер-юнкер, не прибавится третий глаз на лбу».
Гоголевский Поприщин чувствует фиктивность разделения
людей по чину: «Ведь у него же нос не из золота сделан…
Ведь он им нюхает, а не ест, чихает, а не кашляет.\\
Я несколько раз уже хотел добраться, отчего происходят все
эти разности. Отчего я титулярный советник, и с какой стати
я титулярный советник».\\ Чин – пустая вещь, слово, призрак.
Фикция господствует над жизнью, ею управляет.\\ Эта мысль
становится для Гоголя одной из центральных, и мы не
поймем его произведений, не зная, например, почему так
важно, что чиновник, от которого ушел нос, – майор; почему
Поприщин и Башмачкин – титулярные советники.
...

\end{document}