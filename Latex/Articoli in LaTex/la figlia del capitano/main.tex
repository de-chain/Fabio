% Этот шаблон документа разработан в 2014 году
% Данилом Фёдоровых (danil@fedorovykh.ru) 
% для использования в курсе 
% <<Документы и презентации в \LaTeX>>, записанном НИУ ВШЭ
% для Coursera.org: http://coursera.org/course/latex .
% Вы можете изменять, использовать, распространять
% этот документ любым способом по своему усмотрению. 
% В качестве благодарности автору вы можете сохранить 
% в начале документа данный текст или просто ссылку на
% http://coursera.org/course/latex
% Исходная версия Шаблона --- 
% https://www.writelatex.com/coursera/latex/1.1


\documentclass[a4paper,12pt]{article}

\usepackage{cmap}					% поиск в PDF
\usepackage[T2A]{fontenc}			% кодировка
\usepackage[utf8]{inputenc}			% кодировка исходного текста
\usepackage[english,russian]{babel}	% локализация и переносы
\usepackage{cite}        % Per citare
\usepackage{url}         % Per gestire gli URL correttamente
\usepackage{setspace} %\singlespacing, \onehalfspacing, \doublespacing, modificare l'interlinea solo in una parte del documento, puoi racchiuderla tra parentesi graffe:{\onehalfspacing Questo testo ha un'interlinea una volta e mezza}
\usepackage[colorlinks=true, linkcolor=blue, urlcolor=blue, citecolor=red]{hyperref}

\author{Fabio Rizzi}
\title{А.С. Пушкин Капитанская Дочка}
\date{\today}

\begin{document} % Конец преамбулы, начало текста.

\maketitle
\begin{abstract}
 Главная русская идиллия: история любви и спасения на фоне беспощадного бунта.
O том, что любовь и милость важнее справедливого возмездия, а добрый нрав и чистое сердце спасают даже в самые жестокие времена.
\end{abstract}
\tableofcontents % Crea l'indice qui
\section{История создание романа}
В августе 1832 года Пушкин делает первые наброски новой книги — истории дворянина, перешедшего на сторону Пугачёва. Получив через несколько месяцев доступ в военные архивы, Пушкин начинает работать над «Историей Пугачёва», параллельно меняется и замысел будущего романа: главный герой отдаляется от Пугачёва, для роли изменника вводится отдельный персонаж, возникает романтическая линия, в последней редакции появляется Маша, дочь коменданта крепости. Под рукописью, которую Пушкин передаёт в цензуру, стоит дата «19 октября 1836 года» — его последнее большое произведение закончено в годовщину основания Царскосельского лицея. Через месяц после публикации Гоголь пишет из Парижа, что книга «произвела всеобщий эффект». В восторге даже Белинский, тремя годами раньше писавший в статье «Литературные мечтания»: «Оборвался период Пушкинский, так как кончился и сам Пушкин» (впрочем, главного героя «Капитанской дочки» он объявит впоследствии ничтожным и бесцветным). Для современников это роман о потерянном рае: их завораживает, по выражению Александра Тургенева, «эта эпоха, эти характеры старорусские и эта девичья русская прелесть». Спустя десять лет Гоголь напишет: «Сравнительно с «Капитанской дочкой» все наши романы и повести кажутся приторной размазнёй. <…> …Всё не только самая правда, но ещё как бы лучше её. …На то и призвание поэта, чтобы из нас же взять нас и нас же возвратить нам в очищенном и лучшем виде».

\section{Герои романа}
В первых набросках главным героем был Михаил Шванвич — офицер, попавший в плен к пугачёвцам и перешедший на сторону самозванца. Пушкин несколько раз менял имя героя и его историю. В итоге черты Шванвича перешли к Швабрину, а главный герой получил фамилию подполковника Алексея Гринёва — тот участвовал в подавлении восстания, по ложному оговору был обвинён в сотрудничестве с Пугачёвым, но оправдан судом; этот сюжет присутствует в «Истории Екатерины II» Жана Кастера, которая присутствовала в библиотеке Пушкина. Известно, что баснописец Иван Крылов рассказывал Пушкину о своём отце, капитане Андрее Крылове, защищавшем от восставших Яицкий городок, — на него чем-то похож комендант Миронов. Что касается Пугачёва, на страницах «Истории бунта» и в документах следствия он выглядит куда более жестоким; возможно, на образ Пугачёва в романе повлияли впечатления Пушкина от поездок на Урал, где старые крестьяне по-прежнему называли предводителя восстания «батюшкой Петром Федоровичем» и говорили, что ничего от него не видели, кроме хорошего.
\subsection{Честь}
Кодекс дворянской чести — неписаный свод правил и норм, руководивших жизнью дворянского сословия. Для Петра Гринёва, покидающего отцовский дом без образования и средств к существованию, это основной капитал, его нравственный скелет. Честь — не просто «правила жизни»: это и достоинство, и доблесть, и внутреннее чувство, и оценка извне. Честь для Гринёва важнее, чем жизнь, успех или счастье. Он не раздумывая отдаёт Зурину проигранные деньги, вызывает Швабрина на дуэль, отказывается целовать руку Пугачёву — потому что так диктует честь. Соображение чести то и дело ставят его перед сложнейшей дилеммой: он не может помочь беззащитной девушке, не изменив присяге; обратившись за помощью к Пугачёву, он нарушает требования дворянского кодекса; не желая называть имя Маши Мироновой, он не может опровергнуть обвинения в измене. Его жизнь и достоинство спасают лишь самоотверженный поступок Маши и милость императрицы.

\section{Главные сюжеты романа}

\begin{itemize}
  \item Петр Андреич Гринев
Главным героем исторической повести А. С. Пушкина «Капитанская дочка» является юный дворянин, русский офицер Пётр Гринёв. Это один из наиболее ярких образов в русской литературе, который развивается на протяжении всего повествования. За короткий промежуток времени Гринёв превращается из беззаботного мальчика в отважного, сильного духом воина, способного брать ответственность за свои поступки и жизнь других людей. Характеристика Гринёва, его отношение к любви, долгу позволяет лучше понять авторский замысел: именно честь является главной нравственной ценностью, и воспитывается она с ранней юности.
  \item Марья Ивановна Миронова
Центральным женским образом исторической повести А. С. Пушкина «Капитанская дочка» является юная Марья Ивановна, дочь капитана Миронова. Робкая, пугливая, скромная девушка демонстрирует сильный, твёрдый характер, удивительное благородство и отвагу, когда речь идёт о спасении любимого, его честного имени. Характеристика Маши Мироновой позволяет лучше понять личность главного героя, Петра Гринёва, и его антипода, офицера Швабрина. С помощью этого персонажа автор более полно раскрывает не только любовную линию, но и весь общий замысел произведения.
  \item Алексей Иваныч Швабрин
Одним из важных второстепенных персонажей повести А. С. Пушкина «Капитанская дочка» является молодой офицер Швабрин. Это классический отрицательный герой, в котором автор собрал множество человеческих пороков. Подлый, мстительный и беспринципный, Швабрин выступает в произведении злым гением. Характеристика Швабрина необходима для того, чтобы более полно раскрыть образ главного героя, молодого дворянина Петра Гринёва. Автор намеренно сталкивает героев-антагонистов в сложных жизненных ситуациях, чтобы продемонстрировать разность их характеров и нравственных ценностей.
\item Иван Иванович Зурин 
офицер, с которым Гринев познакомился в трактире в Симбирске.
\item Емильян Пугачев
Одним из главных героев повести А. С. Пушкина «Капитанская дочка» является предводитель народного восстания, донской казак Емельян Пугачёв. Это реально существовавший персонаж, оставивший большой след в российской истории. Пушкин показывает этого героя с разных сторон, демонстрируя всю сложность и противоречивость его натуры. Характеристика Пугачева необходима для демонстрации непростого исторического отрезка, во время которого происходило формирование личности главного героя, молодого дворянина Петра Гринёва.
\item Архип Савельич
Ярким второстепенным персонажем в повести А. С. Пушкина «Капитанская дочка» является Савельич — верный слуга главного героя, Петра Гринёва. Будучи простым крепостным, подневольным человеком, он во многом определяет развитие сюжетной линии. Этому способствует его безграничная верность своему хозяину, порядочность и природное упрямство. Характеристика Савельича, его отношение к своим обязанностям, к Петру Гринёву позволяет сделать вывод о его высоких моральных качествах. Автор наделяет героя лучшими качествами, присущими простому русскому народу: добротой, справедливостью, житейской мудростью, врождённым благородством.
\item Миронов Иван Кузьмич
Одним из второстепенных персонажей исторической повести А. С. Пушкина «Капитанская дочка» является пожилой русский офицер, комендант Белогорской крепости капитан Миронов. Это отважный воин, заботливый отец семейства, глубоко порядочный человек с представлениями о чести и достоинстве. Характеристика Миронова позволяет лучше понять личность героя, его отношение к долгу, любви, семье. Его поведение во время взятия Белогорской крепости, высокие нравственные идеалы становятся примером для юного Петра Гринёва.
\item Миронова Василиса Егоровна 
жена капитана Миронова, мать Маши, женщина властная и отважная.
\item Андрей Петрович Гринев 
отец Петра Андреича, премьер-майор в отставке.
\end{itemize}



\section{Краткое содержание}

\subsection{Глава 1. Сержант гвардии}

Отец главного героя, Андрей Петрович Гринев, вышел в отставку премьер-майором, стал жить в своей Симбирской деревне, женился на дочери тамошнего дворянина. С пяти лет Петрушу отдали на воспитание стремянному Савельичу. Когда главному герою исполнилось 16 лет, отец, вместо того чтобы отправить его в Петербург в Семеновский полк (как ранее планировалось), определил на службу в Оренбург. Вместе с юношей отправили Савельича.
По дороге в Оренбург в трактире в Симбирске Гринев познакомился с ротмистром гусарского полка Зуриным. Тот научил юношу играть в бильярд, предложил сыграть на деньги. Выпив пунша, Гринев разгорячился и проиграл сто рублей. Огорченному Савельичу пришлось отдавать долг.

\subsection{Глава 2. Вожатый}

Они отправились дальше. Погода испортилась, началась сильнейшая метель, дорогу замело. Однако встретившийся им мужик согласился провести их до ближайшего жилья.
По дороге Гринев задремал и увидел сон, в котором увидел нечто пророческое. Петруше приснилось, что он пришел проститься с умирающим отцом, но в постели увидел «мужика с черной бородой». Мать назвала мужика «посаженным отцом» Гринева, сказала поцеловать ему руку, чтобы тот его благословил. Петр отказался. Тогда мужик вскочил, выхватил топор и начал всех убивать. Страшный мужик ласково звал: «Не бойсь, подойди под мое благословение». В эту минуту Гринев проснулся: они приехали к постоялому двору. В благодарность за помощь Гринев отдал вожатому свой заячий тулуп.
В Оренбурге Гринева сразу же направили в Белогорскую крепость, в команду капитана Миронова.

\subsection{Глава 3. Крепость}

«Белогорская крепость находилась в сорока верстах от Оренбурга». В первый же день Гринев познакомился с комендантом и его женой. На следующий день Петр Андреич познакомился с офицером Алексеем Иванычем Швабриным. Его отправили сюда «за смертоубийство» – «заколол поручика» во время поединка. Швабрин постоянно подшучивал над семейством коменданта. Дочь Миронова Марью Ивановну Швабрин описывал как «совершенную дурочку», поэтому сначала Петруша относился к ней с предубеждением. Однако через некоторое время она очень понравилась Петру Андреичу .

\subsection{Глава 4. Поединок}

Со временем Гринев нашел в Марье Ивановне «благоразумную и чувствительную девушку». Петр Андреич начал писать стихи и как-то прочел одно из своих произведений, посвященное Маше, Швабрину. Тот раскритиковал стихи и сказал, что девушка вместо «нежных стишков» предпочла бы «пару серег». Гринев обозвал Швабрина мерзавцем, и тот вызвал Петра Андреича на поединок. Первый раз им не удалось сойтись – их заметили и отвели к коменданту. Вечером Гринев узнал, что Швабрин в прошлом году сватался к Маше и получил отказ.
На следующий день Гринев и Швабрин снова сошлись в поединке. Во время дуэли Петра Андреича окликнул подбежавший Савельич. Гринев оглянулся, и противник нанес ему удар «в грудь пониже правого плеча».

\subsection{Глава 5. Любовь}

Все время, пока Гринев выздоравливал, Маша ухаживала за ним. Петр Андреич предложил девушке стать его женой, она согласилась.
Гринев написал отцу, что собирается жениться. Однако Андрей Петрович ответил, что не даст согласия на женитьбу и даже похлопочет, чтобы сына перевели «куда-нибудь подальше». Узнав об ответе родителей Гринева, Марья Ивановна очень расстроилась, но без их согласия не хотела выходить замуж (в частности потому, что девушка была бесприданницей). С тех пор она начала избегать Петра Андреича.
Глава 6. Пугачевщина
Пришло известие, что из-под караула сбежал «донской казак и раскольник Емельян Пугачев», собрал «злодейскую шайку» и «произвел возмущение в яицких селениях». Вскоре стало известно, что мятежники собираются идти на Белогороскую крепость. Начались приготовления.

\subsection{Глава 7. Приступ}

Гринев не спал всю ночь. К крепости съехалось множество вооруженных людей. Между ними ездил на белом коне сам Пугачев. Мятежники ворвались в крепость, коменданта ранили в голову, Гринева схватили.			
В толпе закричали, «что государь на площади ожидает пленных и принимает присягу». Миронов и поручик Иван Игнатьич отказались присягать и были повешены. Гринева ожидала та же участь, но Савельич в последний момент бросился в ноги Пугачеву и попросил отпустить Петра Андреича. Пугачёв узнал Савельича, вспомнил подаренный Гринёвым тулупчик заячий и помиловал Гринёва. Швабрин же примкнул к мятежникам. Мать Маши была убита.

\subsection{Глава 8. Незваный гость}

Машу спрятала попадья, назвав своей племянницей. Савельич сказал Гриневу, что Пугачев – тот самый мужик, которому Петр Андреич подарил тулуп.
Пугачев вызвал к себе Гринева. Петр Андреич признался, что не сможет служить ему, так как «природный дворянин» и «присягал императрице»: «Голова моя в твоей власти: отпустишь меня – спасибо; казнишь – бог тебе судья; а я сказал тебе правду». Искренность Петра Андреича поразила Пугачева, и он отпустил его «на все четыре стороны».

\subsection{Глава 9. Разлука}

Утром Пугачев сказал Гриневу ехать в Оренбург и передать губернатору и всем генералам, чтобы ждали его через неделю. Швабрина главарь восстания назначил новым командиром в крепости.

\subsection{Глава 10. Осада города}

Через несколько дней пришли известия, что Пугачев движется к Оренбургу. Гриневу передали письмо от Марьи Ивановны. Девушка писала, что Швабрин принуждает ее выйти за него замуж и обходится с ней очень жестоко, поэтому она просит помощи у Гринева.

\subsubsection{Глава 11. Мятежная слобода}

Не получив поддержки у генерала, Гринев отправился в Белогорскую крепость. По дороге их с Савельичем схватили люди Пугачева. Гринев рассказал главарю мятежников, что едет в Белогорскую крепость, так как там Швабрин обижает девушку-сироту – невесту Гринева. Утром Пугачев вместе с Гриневым и своими людьми поехали к крепости.

\subsection{Глава 12. Сирота}

Швабрин сказал, что Марья Ивановна его жена. Но войдя в комнату девушки, Гринев и Пугачев увидели, что она была бледной, худой, а из еды перед ней стоял только «кувшин воды, накрытый ломтем хлеба». Швабрин доложил, что девушка дочь Миронова, однако Пугачев все равно отпустил Гринева с возлюбленной.

\subsection{Глава 13. Арест}

Приближаясь к городку, Гринев с Марьей были остановлены караульными. Петр Андреич пошел к майору и узнал в нем Зурина. Гринев, поговорив с Зуриным, решил отправить Машу к родителям в деревню, а сам остался служить в отряде.
В конце февраля отряд Зурина выступил в поход. Пугачев после того как был разбит, снова собрал шайку и пошел на Москву, наводя смуту. «Шайки разбойников злодействовали повсюду». «Не приведи бог видеть русский бунт, бессмысленный и беспощадный!».
Наконец Пугачева поймали. Гринев собрался к родителям, но пришла бумага о его аресте по делу Пугачева.

\subsection{Глава 14. Суд}

Гринев по приказу прибыл в Казань, его посадили в тюрьму. Во время допроса Петр Андреич, не желая впутывать Марью, промолчал, зачем уезжал из Оренбурга. Обвинитель Гринева – Швабрин – утверждал, что Петр Андреич был шпионом Пугачева.

Марья Ивановна была принята родителями Гринева «с искренним радушием». Весть об аресте Петра Андреича всех поразила: ему грозила пожизненная ссылка в Сибирь. Чтобы спасти возлюбленного, Марья Ивановна поехала в Петербург и остановилась в Царском Селе. Во время утренней прогулки она разговорилась с незнакомой дамой, рассказала ей свою историю и о том, что приехала просить у императрицы помилования для Гринева.
В тот же день за Марьей прислали карету императрицы. Государыня оказалась той самой дамой, с которой девушка разговаривала утром. Императрица помиловала Гринева и пообещала помочь ей с приданым.
По словам уже не Гринева, а автора, в конце 1774 года Петр Андреич был освобожден. «Он присутствовал при казни Пугачева, который узнал его в толпе и кивнул ему головою». Вскоре Гринев женился на Марье Ивановне. «Рукопись Петра Андреевича Гринева доставлена была нам от одного из его внуков».	
\href{https://polka.academy/articles/497}{Polka}	
\url{https://polka.academy/articles/497}
\cite{lambert1994}

\section{знаменитые исторические фразы}
\begin{itemize}
\item «Лучшие и прочнейшие изменения суть те, которые происходят от одного улучшения нравов, без всяких насильственных потрясений» 
\item Будет дождик, будут и грибки; а будут грибки, будет и кузов
\item «Не приведи Бог увидеть русский бунт, бессмысленный и беспощадный»\cite{Polka}
\end{itemize}












\bibliographystyle{apalike}
\bibliography{bibliography}  % Collega il file .bib

\end{document} % Конец текста.

