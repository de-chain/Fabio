\documentclass{article}
\usepackage[T2A]{fontenc}    % Codifica T2A per i caratteri cirillici
\usepackage[utf8]{inputenc}  % Codifica UTF-8
\usepackage[russian, english]{babel} % Supporto per il russo e l'inglese
\tolerance=2000
\emergencystretch=3em

\begin{document}

% Attiva la barra per le righe troppo lunghe
\overfullrule=5pt

\section{Петровский замок}
Дворцовый ансамбль — первая самостоятельная работа Матвея Казакова.\\
Зодчий, впоследствии перестроивший центр Москвы в палладианском стиле,
для дворца на окраине города выбрал псевдоготику, объединив мотивы древнерусской,
готической и восточной архитектуры.\\
 Расположилось величественное здание со стрельчатыми окнами на землях Высоко-Петровского монастыря. За что дворец и стал именоваться Петровским. Второе название — Путевой — объясняется его предназначением: для отдыха по пути из Петербурга в Москву. Дорога неблизкая — 700 километров.
 
Прообразом здания послужил временный город-театр в турецком стиле, построенный Василием Баженовым на Ходынском поле для торжеств по случаю подписания Кучук-Кайнаджирского мирного договора.\\
 Императрице работа понравилась, и она захотела похожий дворец. По задумке архитектора центральный купол символизировал православный храм, возвышаясь над башнями-минаретами, встроенными в боковые флигели. Богатство и величие Российской империи подчеркивали многочисленные лепные украшения, а пейзажный парк с гротами и беседками добавлял резиденции уюта.
Останавливалась Екатерина II в узорчатом замке лишь однажды, в 1787 году.\\
Спустя 10 лет путевую резиденцию посетил перед коронацией Павел I. А впоследствии и Николай II.\\
В 1812 году императорский замок на время стал ставкой Наполеона. Бонапарт покинул охваченный огнем Кремль и наблюдал из окна дворца за тем, как полыхает Москва.

Вот, окружен своей дубравой,\\
Петровский замок. Мрачно он\\ 
Недавнею гордится славой.\\
Напрасно ждал Наполеон,\\
Последним счастьем упоенный,\\
Москвы коленопреклоненной\\
С ключами старого Кремля:\\
Нет, не пошла Москва моя\\
К нему с повинной головою.\\
Не праздник, не приемный дар,\\
Она готовила пожар\\
Нетерпеливому герою.\\
Отселе, в думу погружен,\\
Глядел на грозный пламень он.

Так описывал те события Пушкин в «Евгении Онегине».\\
Но лагерь французской армии размещался здесь недолго, всего четыре дня.\\
Что не помешало солдатам и офицерам разграбить дворец и практически уничтожить парк. Выстояли только стены.\\
После бегства французов в боковых крыльях был устроен временный госпиталь.

\end{document}
