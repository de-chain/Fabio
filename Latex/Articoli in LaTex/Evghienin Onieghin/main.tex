
\documentclass{book}

\usepackage[T2A]{fontenc}    % Codifica T2A per i caratteri cirillici
\usepackage[utf8]{inputenc}  % Codifica UTF-8
\usepackage[russian, english]{babel} % Supporto per il russo e l'inglese
\DeclareUnicodeCharacter{202F}{.}%dichiara il carattere 202F spazio stretto non divisibile, i puntini in O rus capitolo 2
\DeclareUnicodeCharacter{0301}{.}%dichiara il carattere 0301
\usepackage[hidelinks]{hyperref}%rimuove i bordi intorno ai link
\usepackage{marginnote}
\usepackage{microtype}
%\usepackage[stable]{footmisc}  % Usa le opzioni stabili di footmisc
%\renewcommand{\thefootnote}{\roman{footnote}}  % Numerazione romana
%\usepackage{endnotes}
\setlength{\footskip}{3 cm}  % Aumenta lo spazio tra il corpo del testo e le note a piè di pagina
\setlength{\footnotesep}{10pt}  % Aumenta lo spazio tra le note a piè di pagina
\raggedbottom  % Permette alle pagine di avere una spaziatura più flessibile
\usepackage{hyperref}% collegamenti ipertestuali
\newcommand{\ftn}{\footnote}% crea alias
\overfullrule=5pt


%\href{https://www.culture.ru/poems/4481/evgenii-onegin}
\title{Эвгенин Онегин}
\author{А.С. Пушкин}
\date{}

\begin{document}
\maketitle

\begin{verse}
\textbf{Роман в стихах}\\
Не мысля гордый свет забавить,\\
Вниманье дружбы возлюбя,\\
Хотел бы я тебе представить\\
Залог достойнее тебя,\\
Достойнее души прекрасной,\\
Святой исполненной мечты,\\
Поэзии живой и ясной,\\
Высоких дум и простоты;\\
Но так и быть — рукой пристрастной\\
Прими собранье пестрых глав,\\
Полусмешных, полупечальных,\\
Простонародных, идеальных,\\
Небрежный плод моих забав,\\
Бессонниц, легких вдохновений,\\
Незрелых и увядших лет,\\
Ума холодных наблюдений\\
И сердца горестных замет.




Eccoci \ftn{nota}
%Nota a piè di pagina \emph{parola con enfasi.}\footnote{esempio di nota a piè di pagina}

%nota a margine\emph{parola con enfasi}\marginpar{\textit{Damasco}}.

%Questo è un testo con una nota a piè di pagina\footnote{\sffamily Questo è un esempio con un font sans-serif per la nota.}.

%Questo è un testo con una nota a piè di pagina\footnote{\itshape Questo è un esempio con un font itshape per la nota.}.


%\chapter{} % Capitolo di Introduzione
\label{cap1} % Etichetta per riferimenti incrociati

\textit{И жить торопится, и чувствовать спешит.\\
Кн. Вяземский}\\

I\\
«Мой дядя самых честных правил,\\
Когда не в шутку\emph{занемог,}\marginpar{si ammalò}\\
Он уважать себя заставил\\
И лучше выдумать не мог.\\
Его пример другим наука;\\
Но, боже мой, какая скука\\
С больным сидеть и день и ночь,\\
Не отходя ни шагу прочь!\\
Какое низкое коварство\\
Полуживого забавлять,\\
Ему подушки поправлять,\\
Печально подносить лекарство,\\
Вздыхать и думать про себя:\\
Когда же\emph{черт} возьмет тебя!»\footnote{чертикаться мог толбко данди или простой человек}\\
II\\
Так думал молодой повеса,
Летя в пыли на почтовых,
Всевышней волею Зевеса
Наследник всех своих родных.
Друзья Людмилы и Руслана!
С героем моего романа
Без предисловий, сей же час
Позвольте познакомить вас:
Онегин, добрый мой приятель,
Родился на брегах Невы,
Где, может быть, родились вы
Или блистали, мой читатель;
Там некогда гулял и я:
Но вреден север для меня.
III
Служив отлично благородно,
Долгами жил его отец,
Давал три бала ежегодно
И промотался наконец.
Судьба Евгения хранила:
Сперва Madame за ним ходила,
Потом Monsieur ее сменил.
Ребенок был резов, но мил.
Monsieur l’Abbe, француз убогой,
Чтоб не измучилось дитя,
Учил его всему шутя,
Не докучал моралью строгой,
Слегка за шалости бранил
И в Летний сад гулять водил.
IV
Когда же юности мятежной
Пришла Евгению пора,
Пора надежд и грусти нежной,
Monsieur прогнали со двора.
Вот мой Онегин на свободе;
Острижен по последней моде,
Как dandy лондонский одет —
И наконец увидел свет.
Он по-французски совершенно
Мог изъясняться и писал;
Легко мазурку танцевал
И кланялся непринужденно;
Чего ж вам больше? Свет решил,
Что он умен и очень мил.
V
Мы все учились понемногу
Чему-нибудь и как-нибудь,
Так воспитаньем, слава богу,
У нас немудрено блеснуть.
Онегин был по мненью многих
(Судей решительных и строгих)
Ученый малый, но педант:
Имел он счастливый талант
Без принужденья в разговоре
Коснуться до всего слегка,
С ученым видом знатока
Хранить молчанье в важном споре
И возбуждать улыбку дам
Огнем нежданных эпиграмм.
VI
Латынь из моды вышла ныне:
Так, если правду вам сказать,
Он знал довольно по-латыне,
Чтоб эпиграфы разбирать,
Потолковать об Ювенале,
В конце письма поставить vale,
Да помнил, хоть не без греха,
Из Энеиды два стиха.
Он рыться не имел охоты
В хронологической пыли
Бытописания земли:
Но дней минувших анекдоты
От Ромула до наших дней
Хранил он в памяти своей.
VII
Высокой страсти не имея
Для звуков жизни не щадить,
Не мог он ямба от хорея,
Как мы ни бились, отличить.
Бранил Гомера, Феокрита;
Зато читал Адама Смита
И был глубокой эконом,
То есть умел судить о том,
Как государство богатеет,
И чем живет, и почему
Не нужно золота ему,
Когда простой продукт имеет.
Отец понять его не мог
И земли отдавал в залог.
VIII
Всего, что знал еще Евгений,
Пересказать мне недосуг;
Но в чем он истинный был гений,
Что знал он тверже всех наук,
Что было для него измлада
И труд, и мука, и отрада,
Что занимало целый день
Его тоскующую лень, —
Была наука страсти нежной,
Которую воспел Назон,
За что страдальцем кончил он
Свой век блестящий и мятежный
В Молдавии, в глуши степей,
Вдали Италии своей.
IX
. . . . . . . . . . . . . . . . . .
. . . . . . . . . . . . . . . . . .
. . . . . . . . . . . . . . . . . .
X
Как рано мог он лицемерить,
Таить надежду, ревновать,
Разуверять, заставить верить,
Казаться мрачным, изнывать,
Являться гордым и послушным,
Внимательным иль равнодушным!
Как томно был он молчалив,
Как пламенно красноречив,
В сердечных письмах как небрежен!
Одним дыша, одно любя,
Как он умел забыть себя!
Как взор его был быстр и нежен,
Стыдлив и дерзок, а порой
Блистал послушною слезой!
XI
Как он умел казаться новым,
Шутя невинность изумлять,
Пугать отчаяньем готовым,
Приятной лестью забавлять,
Ловить минуту умиленья,
Невинных лет предубежденья
Умом и страстью побеждать,
Невольной ласки ожидать,
Молить и требовать признанья,
Подслушать сердца первый звук,
Преследовать любовь, и вдруг
Добиться тайного свиданья…
И после ей наедине
Давать уроки в тишине!
XII
Как рано мог уж он тревожить
Сердца кокеток записных!
Когда ж хотелось уничтожить
Ему соперников своих,
Как он язвительно злословил!
Какие сети им готовил!
Но вы, блаженные мужья,
С ним оставались вы друзья:
Его ласкал супруг лукавый,
Фобласа давний ученик,
И недоверчивый старик,
И рогоносец величавый,
Всегда довольный сам собой,
Своим обедом и женой.
XIII, XIV
. . . . . . . . . . . . . . . . . .
. . . . . . . . . . . . . . . . . .
. . . . . . . . . . . . . . . . . .
XV
Бывало, он еще в постеле:
К нему записочки несут.
Что? Приглашенья? В самом деле,
Три дома на вечер зовут:
Там будет бал, там детский праздник.
Куда ж поскачет мой проказник?
С кого начнет он? Все равно:
Везде поспеть немудрено.
Покамест в утреннем уборе,
Надев широкий боливар,
Онегин едет на бульвар
И там гуляет на просторе,
Пока недремлющий брегет
Не прозвонит ему обед.
XVI
Уж темно: в санки он садится.
«Пади, пади!» — раздался крик;
Морозной пылью серебрится
Его бобровый воротник.
К Talon помчался: он уверен,
Что там уж ждет его Каверин.
Вошел: и пробка в потолок,
Вина кометы брызнул ток;
Пред ним roast-beef окровавленный,
И трюфли, роскошь юных лет,
Французской кухни лучший цвет,
И Страсбурга пирог нетленный
Меж сыром лимбургским живым
И ананасом золотым.
XVII
Еще бокалов жажда просит
Залить горячий жир котлет,
Но звон брегета им доносит,
Что новый начался балет.
Театра злой законодатель,
Непостоянный обожатель
Очаровательных актрис,
Почетный гражданин кулис,
Онегин полетел к театру,
Где каждый, вольностью дыша,
Готов охлопать entrechat,
Обшикать Федру, Клеопатру,
Моину вызвать (для того,
Чтоб только слышали его).
XVIII
Волшебный край! там в стары годы,
Сатиры смелый властелин,
Блистал Фонвизин, друг свободы,
И переимчивый Княжнин;
Там Озеров невольны дани
Народных слез, рукоплесканий
С младой Семеновой делил;
Там наш Катенин воскресил
Корнеля гений величавый;
Там вывел колкий Шаховской
Своих комедий шумный рой,
Там и Дидло венчался славой,
Там, там под сению кулис
Младые дни мои неслись.
XIX
Мои богини! что вы? где вы?
Внемлите мой печальный глас:
Все те же ль вы? другие ль девы,
Сменив, не заменили вас?
Услышу ль вновь я ваши хоры?
Узрю ли русской Терпсихоры
Душой исполненный полет?
Иль взор унылый не найдет
Знакомых лиц на сцене скучной,
И, устремив на чуждый свет
Разочарованный лорнет,
Веселья зритель равнодушный,
Безмолвно буду я зевать
И о былом воспоминать?
XX
Театр уж полон; ложи блещут;
Партер и кресла — все кипит;
В райке нетерпеливо плещут,
И, взвившись, занавес шумит.
Блистательна, полувоздушна,
Смычку волшебному послушна,
Толпою нимф окружена,
Стоит Истомина; она,
Одной ногой касаясь пола,
Другою медленно кружит,
И вдруг прыжок, и вдруг летит,
Летит, как пух от уст Эола;
То стан совьет, то разовьет
И быстрой ножкой ножку бьет.
XXI
Все хлопает. Онегин входит,
Идет меж кресел по ногам,
Двойной лорнет скосясь наводит
На ложи незнакомых дам;
Все ярусы окинул взором,
Все видел: лицами, убором
Ужасно недоволен он;
С мужчинами со всех сторон
Раскланялся, потом на сцену
В большом рассеянье взглянул,
Отворотился — и зевнул,
И молвил: «Всех пора на смену;
Балеты долго я терпел,
Но и Дидло мне надоел».
XXII
Еще амуры, черти, змеи
На сцене скачут и шумят;
Еще усталые лакеи
На шубах у подъезда спят;
Еще не перестали топать,
Сморкаться, кашлять, шикать, хлопать;
Еще снаружи и внутри
Везде блистают фонари;
Еще, прозябнув, бьются кони,
Наскуча упряжью своей,
И кучера, вокруг огней,
Бранят господ и бьют в ладони —
А уж Онегин вышел вон;
Домой одеться едет он.
XXIII
Изображу ль в картине верной
Уединенный кабинет,
Где мод воспитанник примерный
Одет, раздет и вновь одет?
Все, чем для прихоти обильной
Торгует Лондон щепетильный
И по Балтическим волнам
За лес и сало возит нам,
Все, что в Париже вкус голодный,
Полезный промысел избрав,
Изобретает для забав,
Для роскоши, для неги модной, —
Все украшало кабинет
Философа в осьмнадцать лет.
XXIV
Янтарь на трубках Цареграда,
Фарфор и бронза на столе,
И, чувств изнеженных отрада,
Духи в граненом хрустале;
Гребенки, пилочки стальные,
Прямые ножницы, кривые
И щетки тридцати родов
И для ногтей и для зубов.
Руссо (замечу мимоходом)
Не мог понять, как важный Грим
Смел чистить ногти перед ним,
Красноречивым сумасбродом.
Защитник вольности и прав
В сем случае совсем неправ.
XXV
Быть можно дельным человеком
И думать о красе ногтей:
К чему бесплодно спорить с веком?
Обычай деспот меж людей.
Второй Чадаев, мой Евгений,
Боясь ревнивых осуждений,
В своей одежде был педант
И то, что мы назвали франт.
Он три часа по крайней мере
Пред зеркалами проводил
И из уборной выходил
Подобный ветреной Венере,
Когда, надев мужской наряд,
Богиня едет в маскарад.
XXVI
В последнем вкусе туалетом
Заняв ваш любопытный взгляд,
Я мог бы пред ученым светом
Здесь описать его наряд;
Конечно б это было смело,
Описывать мое же дело:
Но панталоны, фрак, жилет,
Всех этих слов на русском нет;
А вижу я, винюсь пред вами,
Что уж и так мой бедный слог
Пестреть гораздо б меньше мог
Иноплеменными словами,
Хоть и заглядывал я встарь
В Академический словарь.
XXVII
У нас теперь не то в предмете:
Мы лучше поспешим на бал,
Куда стремглав в ямской карете
Уж мой Онегин поскакал.
Перед померкшими домами
Вдоль сонной улицы рядами
Двойные фонари карет
Веселый изливают свет
И радуги на снег наводят;
Усеян плошками кругом,
Блестит великолепный дом;
По цельным окнам тени ходят,
Мелькают профили голов
И дам и модных чудаков.
XXVIII
Вот наш герой подъехал к сеням;
Швейцара мимо он стрелой
Взлетел по мраморным ступеням,
Расправил волоса рукой,
Вошел. Полна народу зала;
Музыка уж греметь устала;
Толпа мазуркой занята;
Кругом и шум и теснота;
Бренчат кавалергарда шпоры;
Летают ножки милых дам;
По их пленительным следам
Летают пламенные взоры,
И ревом скрыпок заглушен
Ревнивый шепот модных жен.
XXIX
Во дни веселий и желаний
Я был от балов без ума:
Верней нет места для признаний
И для вручения письма.
О вы, почтенные супруги!
Вам предложу свои услуги;
Прошу мою заметить речь:
Я вас хочу предостеречь.
Вы также, маменьки, построже
За дочерьми смотрите вслед:
Держите прямо свой лорнет!
Не то… не то, избави боже!
Я это потому пишу,
Что уж давно я не грешу.
XXX
Увы, на разные забавы
Я много жизни погубил!
Но если б не страдали нравы,
Я балы б до сих пор любил.
Люблю я бешеную младость,
И тесноту, и блеск, и радость,
И дам обдуманный наряд;
Люблю их ножки; только вряд
Найдете вы в России целой
Три пары стройных женских ног.
Ах! долго я забыть не мог
Две ножки… Грустный, охладелый,
Я все их помню, и во сне
Они тревожат сердце мне.
XXXI
Когда ж и где, в какой пустыне,
Безумец, их забудешь ты?
Ах, ножки, ножки! где вы ныне?
Где мнете вешние цветы?
Взлелеяны в восточной неге,
На северном, печальном снеге
Вы не оставили следов:
Любили мягких вы ковров
Роскошное прикосновенье.
Давно ль для вас я забывал
И жажду славы и похвал,
И край отцов, и заточенье?
Исчезло счастье юных лет,
Как на лугах ваш легкий след.
XXXII
Дианы грудь, ланиты Флоры
Прелестны, милые друзья!
Однако ножка Терпсихоры
Прелестней чем-то для меня.
Она, пророчествуя взгляду
Неоцененную награду,
Влечет условною красой
Желаний своевольный рой.
Люблю ее, мой друг Эльвина,
Под длинной скатертью столов,
Весной на мураве лугов,
Зимой на чугуне камина,
На зеркальном паркете зал,
У моря на граните скал.
XXXIII
Я помню море пред грозою:
Как я завидовал волнам,
Бегущим бурной чередою
С любовью лечь к ее ногам!
Как я желал тогда с волнами
Коснуться милых ног устами!
Нет, никогда средь пылких дней
Кипящей младости моей
Я не желал с таким мученьем
Лобзать уста младых Армид,
Иль розы пламенных ланит,
Иль перси, полные томленьем;
Нет, никогда порыв страстей
Так не терзал души моей!
XXXIV
Мне памятно другое время!
В заветных иногда мечтах
Держу я счастливое стремя…
И ножку чувствую в руках;
Опять кипит воображенье,
Опять ее прикосновенье
Зажгло в увядшем сердце кровь,
Опять тоска, опять любовь!..
Но полно прославлять надменных
Болтливой лирою своей;
Они не стоят ни страстей,
Ни песен, ими вдохновенных:
Слова и взор волшебниц сих
Обманчивы… как ножки их.
XXXV
Что ж мой Онегин? Полусонный
В постелю с бала едет он:
А Петербург неугомонный
Уж барабаном пробужден.
Встает купец, идет разносчик,
На биржу тянется извозчик,
С кувшином охтенка спешит,
Под ней снег утренний хрустит.
Проснулся утра шум приятный.
Открыты ставни; трубный дым
Столбом восходит голубым,
И хлебник, немец аккуратный,
В бумажном колпаке, не раз
Уж отворял свой васисдас.
XXXVI
Но, шумом бала утомленный
И утро в полночь обратя,
Спокойно спит в тени блаженной
Забав и роскоши дитя.
Проснется за полдень, и снова
До утра жизнь его готова,
Однообразна и пестра.
И завтра то же, что вчера.
Но был ли счастлив мой Евгений,
Свободный, в цвете лучших лет,
Среди блистательных побед,
Среди вседневных наслаждений?
Вотще ли был он средь пиров
Неосторожен и здоров?
XXXVII
Нет: рано чувства в нем остыли;
Ему наскучил света шум;
Красавицы не долго были
Предмет его привычных дум;
Измены утомить успели;
Друзья и дружба надоели,
Затем, что не всегда же мог
Beef-stеаks и страсбургский пирог
Шампанской обливать бутылкой
И сыпать острые слова,
Когда болела голова;
И хоть он был повеса пылкой,
Но разлюбил он наконец
И брань, и саблю, и свинец.
XXXVIII
Недуг, которого причину
Давно бы отыскать пора,
Подобный английскому сплину,
Короче: русская хандра
Им овладела понемногу;
Он застрелиться, слава богу,
Попробовать не захотел,
Но к жизни вовсе охладел.
Как Child-Harold, угрюмый, томный
В гостиных появлялся он;
Ни сплетни света, ни бостон,
Ни милый взгляд, ни вздох нескромный,
Ничто не трогало его,
Не замечал он ничего.
XXXIX, ХL, ХLI
. . . . . . . . . . . . . . . . . .
. . . . . . . . . . . . . . . . . .
. . . . . . . . . . . . . . . . . .
ХLII
Причудницы большого света!
Всех прежде вас оставил он;
И правда то, что в наши лета
Довольно скучен высший тон;
Хоть, может быть, иная дама
Толкует Сея и Бентама,
Но вообще их разговор
Несносный, хоть невинный вздор;
К тому ж они так непорочны,
Так величавы, так умны,
Так благочестия полны,
Так осмотрительны, так точны,
Так неприступны для мужчин,
Что вид их уж рождает сплин.
XLIII
И вы, красотки молодые,
Которых позднею порой
Уносят дрожки удалые
По петербургской мостовой,
И вас покинул мой Евгений.
Отступник бурных наслаждений,
Онегин дома заперся,
Зевая, за перо взялся,
Хотел писать — но труд упорный
Ему был тошен; ничего
Не вышло из пера его,
И не попал он в цех задорный
Людей, о коих не сужу,
Затем, что к ним принадлежу.
ХLIV
И снова, преданный безделью,
Томясь душевной пустотой,
Уселся он — с похвальной целью
Себе присвоить ум чужой;
Отрядом книг уставил полку,
Читал, читал, а все без толку:
Там скука, там обман иль бред;
В том совести, в том смысла нет;
На всех различные вериги;
И устарела старина,
И старым бредит новизна.
Как женщин, он оставил книги,
И полку, с пыльной их семьей,
Задернул траурной тафтой.
ХLV
Условий света свергнув бремя,
Как он, отстав от суеты,
С ним подружился я в то время.
Мне нравились его черты,
Мечтам невольная преданность,
Неподражательная странность
И резкий, охлажденный ум.
Я был озлоблен, он угрюм;
Страстей игру мы знали оба;
Томила жизнь обоих нас;
В обоих сердца жар угас;
Обоих ожидала злоба
Слепой Фортуны и людей
На самом утре наших дней.
XLVI
Кто жил и мыслил, тот не может
В душе не презирать людей;
Кто чувствовал, того тревожит
Призрак невозвратимых дней:
Тому уж нет очарований,
Того змия воспоминаний,
Того раскаянье грызет.
Все это часто придает
Большую прелесть разговору.
Сперва Онегина язык
Меня смущал; но я привык
К его язвительному спору,
И к шутке, с желчью пополам,
И злости мрачных эпиграмм.
XLVII
Как часто летнею порою,
Когда прозрачно и светло
Ночное небо над Невою
И вод веселое стекло
Не отражает лик Дианы,
Воспомня прежних лет романы,
Воспомня прежнюю любовь,
Чувствительны, беспечны вновь,
Дыханьем ночи благосклонной
Безмолвно упивались мы!
Как в лес зеленый из тюрьмы
Перенесен колодник сонный,
Так уносились мы мечтой
К началу жизни молодой.
XLVIII
С душою, полной сожалений,
И опершися на гранит,
Стоял задумчиво Евгений,
Как описал себя пиит.
Все было тихо; лишь ночные
Перекликались часовые,
Да дрожек отдаленный стук
С Мильонной раздавался вдруг;
Лишь лодка, веслами махая,
Плыла по дремлющей реке:
И нас пленяли вдалеке
Рожок и песня удалая…
Но слаще, средь ночных забав,
Напев Торкватовых октав!
XLIX
Адриатические волны,
О Брента! нет, увижу вас
И, вдохновенья снова полный,
Услышу ваш волшебный глас!
Он свят для внуков Аполлона;
По гордой лире Альбиона
Он мне знаком, он мне родной.
Ночей Италии златой
Я негой наслажусь на воле,
С венецианкою младой,
То говорливой, то немой,
Плывя в таинственной гондоле;
С ней обретут уста мои
Язык Петрарки и любви.
L
Придет ли час моей свободы?
Пора, пора! — взываю к ней;
Брожу над морем, жду погоды,
Маню ветрила кораблей.
Под ризой бурь, с волнами споря,
По вольному распутью моря
Когда ж начну я вольный бег?
Пора покинуть скучный брег
Мне неприязненной стихии
И средь полуденных зыбей,
Под небом Африки моей,
Вздыхать о сумрачной России,
Где я страдал, где я любил,
Где сердце я похоронил.
LI
Онегин был готов со мною
Увидеть чуждые страны;
Но скоро были мы судьбою
На долгой срок разведены.
Отец его тогда скончался.
Перед Онегиным собрался
Заимодавцев жадный полк.
У каждого свой ум и толк:
Евгений, тяжбы ненавидя,
Довольный жребием своим,
Наследство предоставил им,
Большой потери в том не видя
Иль предузнав издалека
Кончину дяди старика.
LII
Вдруг получил он в самом деле
От управителя доклад,
Что дядя при смерти в постеле
И с ним проститься был бы рад.
Прочтя печальное посланье,
Евгений тотчас на свиданье
Стремглав по почте поскакал
И уж заранее зевал,
Приготовляясь, денег ради,
На вздохи, скуку и обман
(И тем я начал мой роман);
Но, прилетев в деревню дяди,
Его нашел уж на столе,
Как дань готовую земле.
LIII
Нашел он полон двор услуги;
К покойнику со всех сторон
Съезжались недруги и други,
Охотники до похорон.
Покойника похоронили.
Попы и гости ели, пили
И после важно разошлись,
Как будто делом занялись.
Вот наш Онегин — сельский житель,
Заводов, вод, лесов, земель
Хозяин полный, а досель
Порядка враг и расточитель,
И очень рад, что прежний путь
Переменил на что-нибудь.
LIV
Два дня ему казались новы
Уединенные поля,
Прохлада сумрачной дубровы,
Журчанье тихого ручья;
На третий роща, холм и поле
Его не занимали боле;
Потом уж наводили сон;
Потом увидел ясно он,
Что и в деревне скука та же,
Хоть нет ни улиц, ни дворцов,
Ни карт, ни балов, ни стихов.
Хандра ждала его на страже,
И бегала за ним она,
Как тень иль верная жена.
LV
Я был рожден для жизни мирной,
Для деревенской тишины;
В глуши звучнее голос лирный,
Живее творческие сны.
Досугам посвятясь невинным,
Брожу над озером пустынным,
И far nientе мой закон.
Я каждым утром пробужден
Для сладкой неги и свободы:
Читаю мало, долго сплю,
Летучей славы не ловлю.
Не так ли я в былые годы
Провел в бездействии, в тени
Мои счастливейшие дни?
LVI
Цветы, любовь, деревня, праздность,
Поля! я предан вам душой.
Всегда я рад заметить разность
Между Онегиным и мной,
Чтобы насмешливый читатель
Или какой-нибудь издатель
Замысловатой клеветы,
Сличая здесь мои черты,
Не повторял потом безбожно,
Что намарал я свой портрет,
Как Байрон, гордости поэт,
Как будто нам уж невозможно
Писать поэмы о другом,
Как только о себе самом.
LVII
Замечу кстати: все поэты —
Любви мечтательной друзья.
Бывало, милые предметы
Мне снились, и душа моя
Их образ тайный сохранила;
Их после муза оживила:
Так я, беспечен, воспевал
И деву гор, мой идеал,
И пленниц берегов Салгира.
Теперь от вас, мои друзья,
Вопрос нередко слышу я:
«О ком твоя вздыхает лира?
Кому, в толпе ревнивых дев,
Ты посвятил ее напев?
LVIII
Чей взор, волнуя вдохновенье,
Умильной лаской наградил
Твое задумчивое пенье?
Кого твой стих боготворил?»
И, други, никого, ей-богу!
Любви безумную тревогу
Я безотрадно испытал.
Блажен, кто с нею сочетал
Горячку рифм: он тем удвоил
Поэзии священный бред,
Петрарке шествуя вослед,
А муки сердца успокоил,
Поймал и славу между тем;
Но я, любя, был глуп и нем.
LIX
Прошла любовь, явилась муза,
И прояснился темный ум.
Свободен, вновь ищу союза
Волшебных звуков, чувств и дум;
Пишу, и сердце не тоскует,
Перо, забывшись, не рисует,
Близ неоконченных стихов,
Ни женских ножек, ни голов;
Погасший пепел уж не вспыхнет,
Я все грущу; но слез уж нет,
И скоро, скоро бури след
В душе моей совсем утихнет:
Тогда-то я начну писать
Поэму песен в двадцать пять.
LX
Я думал уж о форме плана
И как героя назову;
Покамест моего романа
Я кончил первую главу;
Пересмотрел все это строго:
Противоречий очень много,
Но их исправить не хочу.
Цензуре долг свой заплачу
И журналистам на съеденье
Плоды трудов моих отдам:
Иди же к невским берегам,
Новорожденное творенье,
И заслужи мне славы дань:
Кривые толки, шум и брань!
%\chapter{} % Capitolo di Metodologia
\label{cap2} % Etichetta per riferimenti incrociati

\textit{O rus!..\\
Hor.\\ 
(О Русь!)}\\

I

Деревня, где скучал Евгений,\\
Была прелестный уголок;\\
Там друг невинных наслаждений\\
Благословить бы небо мог.\\
Господский дом уединенный,\\
Горой от ветров огражденный,\\
Стоял над речкою. Вдали\\
Пред ним пестрели и цвели\\
Луга и нивы золотые,\\
Мелькали селы; здесь и там\\
Стада бродили по лугам,\\
И сени расширял густые\\
Огромный, запущенный сад,\\
Приют задумчивых \emph{дриад.}\footnote{в греческой мифологии нимфы, покровительницы деревьев}

II

Почтенный замок был построен,\\
Как замки строиться должны:\\
\emph{Отменно}\footnote{Оценочная характеристика чего-либо как очень хорошего, превосходного.} прочен и спокоен\\
Во вкусе умной старины.\\
Везде высокие покои,\\
В гостиной \emph{штофные}\marginpar{\textit{Damasco}} обои,\\
Царей портреты на стенах,\\
И печи в пестрых изразцах.\\
Все это ныне \emph{обветшало,}\marginpar{fatiscente}\\
Не знаю, право, почему;\\
Да, впрочем, другу моему\\
В том нужды было очень мало,\\
Затем, что он равно зевал\\
Средь модных и старинных зал.\\

III

Он в том покое поселился,\\
Где деревенский старожил\\
Лет сорок с ключницей бранился,\\
В окно смотрел и мух давил.\\
Все было просто: пол дубовый,\\
Два шкафа, стол, диван пуховый,\\
Нигде ни пятнышка чернил.\\
Онегин шкафы отворил;\\
В одном нашел тетрадь расхода,\\
В другом наливок целый строй,\\
Кувшины с яблочной водой\\
И календарь осьмого года:\\
Старик, имея много дел,\\
В иные книги не глядел.\\

IV

Один среди своих владений,\\
\textit{Чтоб только время проводить,\\
Сперва задумал наш Евгений\\
Порядок новый учредить.\\
В своей глуши мудрец пустынный,\\
Ярем он барщины старинной\\
Оброком легким заменил;\\
И раб судьбу благословил.}\footnote{При барщине крестьяне бесплатно работали на земле своего хозяина-крепостника, в лучшем случае им давали месячину - продовольственный паёк. Производительность труда, сами понимаете, крайне невелика, невелики и доходы. Онегин разделил свои владения между крестьянами, которые теперь сами обрабатывали землю, крутились как могли, и платили своему хозяину оброк. На эту же тему см. тургеневский "Хорь и Калиныч" из "Записок охотника". Оброк, кстати, тоже большого дохода не приносил - в среднем по 5 рублей с крестьянского двора в год. 
Сроки барщины были неопределёнными. Да, император Павел Первый в 1797 г. попытался запретить крестьянам работать в праздники и ограничил барщину 3 днями в неделю, но фактически указ не выполнялся. По закону крестьянин должен был обрабатывать столько же помещичьей земли, сколько ему выделялось на личное хозяйство. Точнее, помещик выделял землю крестьянской общине, а делили её между собой уже на сходе. }\\
Зато в углу своем надулся,\\
Увидя в этом страшный вред,\\\
Его расчетливый сосед;\\
Другой лукаво улыбнулся,\\
И в голос все решили так,\\
Что он опаснейший чудак.\\

V

Сначала все к нему езжали;\\
Но так как с заднего крыльца\\
Обыкновенно подавали\\
Ему донского жеребца,\\
Лишь только вдоль большой дороги\\
Заслышат их домашни дроги, —\\
Поступком оскорбясь таким,\\
Все дружбу прекратили с ним.\footnote{Вначале, сразу после того, как Онегин поселился в поместье умершего дяди, все соседи стали ездить к нему в гости с визитами - познакомиться или поддержать знакомство. Так у них было принято. 
Но Онегин ни с кем из сельских соседей дружить не хотел. И даже просто поддерживать какие-либо отношения тоже не хотел. Поэтому, когда с дороги, ведущей к его дому, слышался звук приближающегося экипажа (домашних дрог, дрожек, то есть, коляски, сделанной местным крепостным умельцем), Онегин требовал верхового коня и уезжал из дома на прогулку. Причем уезжал с заднего крыльца, чтобы попасть на другую дорогу, на которой он не повстречался бы с непрошеными гостями.
Соседи поняли все правильно, обиделись и прекратили дружбу с ним.}\\
«Сосед наш неуч; сумасбродит;\\
Он фармазон; он пьет одно\\
Стаканом красное вино;\\
Он дамам к ручке не подходит;\\
Все да да нет; не скажет да-с\\
Иль нет-с». Таков был общий глас.\\

VI\footnote{характер Ленского. Изъять из Быта}

В свою деревню в ту же пору\\
Помещик новый прискакал\\
И столь же строгому разбору\\
В соседстве повод подавал:\\
По имени \emph{Владимир Ленской,}\\
С душою прямо геттингенской,\\
Красавец, в полном цвете лет,\\
Поклонник Канта и поэт.\\
Он из Германии туманной\\
Привез учености плоды:\\
Вольнолюбивые мечты,\\
Дух пылкий\marginpar{Сильно чувствующий} и довольно странный,\\
Всегда восторженную\marginpar{легко приходит в состояние восторга...} речь\\
И кудри черные до плеч.\\

VII

От хладного разврата света\\
Еще увянуть не успев,\\
Его душа была согрета\\
Приветом друга, лаской дев;\\
Он сердцем милый был невежда,\\
Его лелеяла надежда,\\
И мира новый блеск и шум\\
Еще пленяли юный ум.\\
Он забавлял мечтою сладкой\\
Сомненья сердца своего;\\
Цель жизни нашей для него\\
Была заманчивой загадкой,\\
Над ней он голову ломал\\
И чудеса подозревал.\\

VIII

Он верил, что душа родная\\
Соединиться с ним должна,\\
Что, безотрадно изнывая,\\
Его вседневно ждет она;\\
Он верил, что друзья готовы\\
За честь его приять оковы\\
И что не дрогнет их рука\\
Разбить сосуд клеветника;\\
Что есть избранные судьбами,\\
Людей священные друзья;\\
Что их бессмертная семья\\
Неотразимыми лучами\\
Когда-нибудь нас озарит\\
И мир блаженством одарит.\\

IX

Негодованье, сожаленье,\\
Ко благу чистая любовь\\
И славы сладкое мученье\\
В нем рано волновали кровь.\\
Он с лирой странствовал на свете;\\
Под небом Шиллера и Гете\\
Их поэтическим огнем\\
Душа воспламенилась в нем;\\
И муз возвышенных искусства,\\
Счастливец, он не постыдил:\\
Он в песнях гордо сохранил\\
Всегда возвышенные чувства,\\
Порывы девственной мечты\\
И прелесть важной простоты.\\

X

Он пел любовь, любви послушный,\\
И песнь его была ясна,\\
Как мысли девы простодушной,\\
Как сон младенца, как луна\\
В пустынях неба безмятежных,\\
Богиня тайн и вздохов нежных.\\
Он пел разлуку и печаль,\\
И нечто, и туманну даль,\\
И романтические розы;\\
Он пел те дальные страны,\\
Где долго в лоно тишины\\
Лились его живые слезы;\\
Он пел поблеклый жизни цвет\\
Без малого в осьмнадцать лет.\\

XI

В пустыне, где один Евгений\\
Мог оценить его дары,\\
Господ соседственных селений\\
Ему не нравились пиры;\\
Бежал он их беседы шумной.\\
Их разговор благоразумный\\
О сенокосе, о вине,\\
О псарне, о своей родне,\\
Конечно, не блистал ни чувством,\\
Ни поэтическим огнем,\\
Ни остротою, ни умом,\\
Ни общежития искусством;\\
Но разговор их милых жен\\
Гораздо меньше был умен.\\

XII

Богат, хорош собою, Ленский\\
Везде был принят как жених;\\
Таков обычай деревенский;\\
Все дочек прочили своих\\
За полурусского соседа;\\
Взойдет ли он, тотчас беседа\\
Заводит слово стороной\\
О скуке жизни холостой;\\
Зовут соседа к самовару,\\
А Дуня разливает чай;\\
Ей шепчут: «Дуня, примечай!»\marginpar{за кем-чем. Наблюдать, следить}\\
Потом приносят и гитару:\\
И запищит она (бог мой!):\\
Приди в чертог ко мне златой!..\\

XIII

Но Ленский, не имев, конечно,\\
Охоты узы\marginpar{vincoli} брака несть,\\
С Онегиным желал сердечно\\
Знакомство покороче свесть.\\
Они сошлись. Волна и камень,\\
Стихи и проза, лед и пламень\\
Не столь различны меж собой.\\
Сперва взаимной разнотой\\
Они друг другу были скучны;\\
Потом понравились; потом\\
Съезжались каждый день верхом\\
И скоро стали неразлучны.\\
Так люди (первый каюсь я)\\
От делать нечего друзья.\\

XIV

Но дружбы нет и той меж нами.\\
Все предрассудки истребя,\\
Мы почитаем всех нулями,\\
А единицами — себя.\\
Мы все глядим в Наполеоны;\\
Двуногих тварей миллионы\\
Для нас орудие одно;\\
Нам чувство дико и смешно.\\
Сноснее многих был Евгений;\\
Хоть он людей, конечно, знал\\
И вообще их презирал, —\\
Но (правил нет без исключений)\\
Иных он очень отличал\\
И вчуже чувство уважал.\\

XV

\emph{Он слушал Ленского с улыбкой.\\
Поэта пылкий разговор,\\
И ум, еще в сужденьях зыбкой,\\
И вечно вдохновенный взор, —\\
Онегину все было ново;\\
Он охладительное слово\\
В устах старался удержать\\
И думал: глупо мне мешать\\
Его минутному блаженству;\\
И без меня пора придет;\\
Пускай покамест он живет\\
Да верит мира совершенству;\\
Простим горячке юных лет\\
И юный жар и юный бред.}\\

XVI

Меж ими все рождало споры\\
И к размышлению влекло:\\
Племен минувших договоры,\\
Плоды наук, добро и зло,\\
И предрассудки вековые,\\
И гроба тайны роковые,\\
Судьба и жизнь в свою чреду,\\
Все подвергалось их суду.\\
Поэт в жару своих суждений\\
Читал, забывшись, между тем\\
Отрывки северных поэм,\\
И снисходительный Евгений,\\
Хоть их не много понимал,\\
Прилежно юноше внимал.\\

XVII

Но чаще занимали страсти\\
Умы пустынников моих.\\
Ушед от их мятежной власти,\\
Онегин говорил об них\\
С невольным вздохом сожаленья:\\
Блажен, кто ведал их волненья\\
И наконец от них отстал;\\
Блаженней тот, кто их не знал,\\
Кто охлаждал любовь — разлукой,\\
Вражду — злословием; порой\\
Зевал с друзьями и с женой,\\
Ревнивой не тревожась мукой,\\
И дедов верный капитал\\
Коварной двойке не вверял.\\

XVIII

Когда прибегнем мы под знамя\\
Благоразумной тишины,\\
Когда страстей угаснет пламя,\\
И нам становятся смешны\\
Их своевольство иль порывы\\
И запоздалые отзывы, —\\
Смиренные не без труда,\\
Мы любим слушать иногда\\
Страстей чужих язык мятежный,\\
И нам он сердце шевелит.\\
Так точно старый инвалид\\
Охотно клонит слух прилежный\\
Рассказам юных усачей,\\
Забытый в хижине своей.\\

XIX

Зато и пламенная младость\\
Не может ничего скрывать.\\
Вражду, любовь, печаль и радость\\
Она готова разболтать.\\
В любви считаясь инвалидом,\\
Онегин слушал с важным видом,\\
Как, сердца исповедь любя,\\
Поэт высказывал себя;\\
Свою доверчивую совесть\\
Он простодушно обнажал.\\
Евгений без труда узнал\\
Его любви младую повесть,\\
Обильный чувствами рассказ,\\
Давно не новыми для нас.\\

XX

Ах, он любил, как в наши лета\\
Уже не любят; как одна\\
Безумная душа поэта\\
Еще любить осуждена:\\
Всегда, везде одно мечтанье,\\
Одно привычное желанье,\\
Одна привычная печаль.\\
Ни охлаждающая даль,\\
Ни долгие лета разлуки,\\
Ни музам данные часы,\\
Ни чужеземные красы,\\
Ни шум веселий, ни науки\\
Души не изменили в нем,\\
Согретой девственным огнем.\\

XXI

Чуть отрок, Ольгою плененный,\\
Сердечных мук еще не знав,\\
Он был свидетель умиленный\\
Ее младенческих забав;\\
В тени хранительной дубравы\\
Он разделял ее забавы,\\
И детям прочили венцы\\
Друзья-соседы, их отцы.\\
В глуши, под сению смиренной,\\
Невинной прелести полна,\\
В глазах родителей, она\\
Цвела, как ландыш потаенный,\\
Незнаемый в траве глухой\\
Ни мотыльками, ни пчелой.\\

XXII

Она поэту подарила\\
Младых восторгов первый сон,\\
И мысль об ней одушевила\\
Его цевницы первый стон.\\
Простите, игры золотые!\\
Он рощи полюбил густые,\\
Уединенье, тишину,\\
И ночь, и звезды, и луну,\\
Луну, небесную лампаду,\\
Которой посвящали мы\\
Прогулки средь вечерней тьмы,\\
И слезы, тайных мук отраду…\\
Но нынче видим только в ней\\
Замену тусклых фонарей.\\

XXIII

Всегда скромна, всегда послушна,\\
Всегда как утро весела,\\
Как жизнь поэта простодушна,\\
Как поцелуй любви мила;\\
Глаза, как небо, голубые,\\
Улыбка, локоны льняные,\\
Движенья, голос, легкий стан,\\
Все в Ольге… но любой роман\\
Возьмите и найдете верно\\
Ее портрет: он очень мил,\\
Я прежде сам его любил,\\
Но надоел он мне безмерно.\\
Позвольте мне, читатель мой,\\
Заняться старшею сестрой.\\

XXIV

Ее сестра звалась Татьяна…\\
Впервые именем таким\\
Страницы нежные романа\\
Мы своевольно освятим.\\
И что ж? оно приятно, звучно;\\
Но с ним, я знаю, неразлучно\\
Воспоминанье старины\\
Иль девичьей! Мы все должны\\
Признаться: вкусу очень мало\\
У нас и в наших именах\\
(Не говорим уж о стихах);\\
Нам просвещенье не пристало,\\
И нам досталось от него\\
Жеманство, — больше ничего.\\

XXV

Итак, она звалась Татьяной.\\
Ни красотой сестры своей,\\
Ни свежестью ее румяной\\
Не привлекла б она очей.\\
Дика, печальна, молчалива,\\
\emph{Как лань лесная боязлива,\\
Она в семье своей родной}\\
Казалась девочкой чужой.\\
Она ласкаться не умела\\
К отцу, ни к матери своей;\\
Дитя сама, в толпе детей\\
Играть и прыгать не хотела\\
И часто целый день одна\\
Сидела молча у окна.\\

XXVI\footnote{[задумчивый мичтательный характер Татьяны}

Задумчивость, ее подруга\\
От самых колыбельных дней,\\
Теченье сельского досуга\\
Мечтами украшала ей.\\
Ее изнеженные пальцы\\
Не знали игл; склонясь на пяльцы,\\
Узором шелковым она\
Не оживляла полотна.\\
Охоты властвовать примета,\\
С послушной куклою дитя\\
Приготовляется шутя\\
К приличию — закону света,\\
И важно повторяет ей\\
Уроки маменьки своей.\\

XXVII

Но куклы даже в эти годы\\
Татьяна в руки не брала;\\
Про вести города, про моды\\
Беседы с нею не вела.\\
И были детские проказы\\
Ей чужды: страшные рассказы\\
Зимою в темноте ночей\\
Пленяли больше сердце ей.\\
Когда же няня собирала\\
Для Ольги на широкий луг\\
Всех маленьких ее подруг,\\
Она в горелки не играла,\\
Ей скучен был и звонкий смех,\\
И шум их ветреных утех.\\

XXVIII

Она любила на балконе\\\
Предупреждать зари восход,\\
Когда на бледном небосклоне\
Звезд исчезает хоровод,\\
И тихо край земли светлеет,\\
И, вестник утра, ветер веет,\\
И всходит постепенно день.\\
Зимой, когда ночная тень\\
Полмиром доле обладает,\\
И доле в праздной тишине,\\
При отуманенной луне,\\
Восток ленивый почивает,\\
В привычный час пробуждена\\
Вставала при свечах она.\\

XXIX

Ей рано нравились романы;\\
Они ей заменяли все;\\
Она влюблялася в обманы\\
И Ричардсона и Руссо.\\
Отец ее был добрый малый,\\
В прошедшем веке запоздалый;\\
Но в книгах не видал вреда;\\
\emph{Он, не читая никогда,\\
Их почитал пустой игрушкой}\\
И не заботился о том,\\
Какой у дочки тайный том\\
Дремал до утра под подушкой.\\
Жена ж его была сама\\
От Ричардсона без ума.\\

XXX

Она любила Ричардсона\\
Не потому, чтобы прочла,\\
Не потому, чтоб Грандисона\\
Она Ловласу предпочла;\\
Но в старину княжна Алина,\\
Ее московская кузина,\\
Твердила часто ей об них.\\
В то время был еще жених\\
Ее супруг, но по неволе;\\
Она вздыхала по другом,\\
Который сердцем и умом\\
Ей нравился гораздо боле:\\
Сей Грандисон был славный франт,\\
Игрок и гвардии сержант.\\

XXXI

Как он, она была одета\\
Всегда по моде и к лицу;\\
Но, не спросясь ее совета,\\
Девицу повезли к венцу.\\
И, чтоб ее рассеять горе,\\
Разумный муж уехал вскоре\\
В свою деревню, где она,\\
Бог знает кем окружена,\\
Рвалась и плакала сначала,\\
С супругом чуть не развелась;\\
Потом хозяйством занялась,\\
Привыкла и довольна стала.\\
\textbf{Привычка свыше нам дана:\\
Замена счастию она.}\\

XXXII

Привычка усладила горе,\\
Не отразимое ничем;\\
Открытие большое вскоре\\
Ее утешило совсем:\\
Она меж делом и досугом\\
Открыла тайну, как супругом\\
Самодержавно управлять,\\
И все тогда пошло на стать.\\
Она езжала по работам,\\
Солила на зиму грибы,\\
Вела расходы, брила лбы,\\
Ходила в баню по субботам,\\
Служанок била осердясь —\\
Все это мужа не спросясь.\\

XXXIII

Бывало, писывала кровью\\
Она в альбомы нежных дев,\\
Звала Полиною Прасковью\\
И говорила нараспев,\\
Корсет носила очень узкий,\\
И русский Н как N французский\\
Произносить умела в нос;\\
Но скоро все перевелось:\\
Корсет, альбом, княжну Алину,\\
Стишков чувствительных тетрадь\\
Она забыла: стала звать\\
Акулькой прежнюю Селину\\
И обновила наконец\\
На вате шлафор и чепец.\\

XXXIV

Но муж любил ее сердечно,\\
В ее затеи не входил,\\
Во всем ей веровал беспечно,\\
А сам в халате ел и пил;\\
Покойно жизнь его катилась;\\
Под вечер иногда сходилась\\
Соседей добрая семья,\\
Нецеремонные друзья,\\
И потужить, и позлословить,\\
И посмеяться кой о чем.\\
Проходит время; между тем\\
Прикажут Ольге чай готовить,\\
Там ужин, там и спать пора,\\
И гости едут со двора.\\

XXXV

Они хранили в жизни мирной\\
Привычки милой старины;\\
У них на масленице жирной\\
Водились русские блины;\\
Два раза в год они говели;\\
Любили круглые качели,\\
Подблюдны песни, хоровод;\\
В день Троицын, когда народ,\\
Зевая, слушает молебен,\\
Умильно на пучок зари\\
Они роняли слезки три;\\
Им квас как воздух был потребен,\\
И за столом у них гостям\\
Носили блюды по чинам.\\

XXXVI

И так они старели оба.\\
И отворились наконец\\
Перед супругом двери гроба,\
И новый он приял венец.\\
Он умер в час перед обедом,\\
Оплаканный своим соседом,\\
Детьми и верною женой\\
Чистосердечней, чем иной.\\
Он был простой и добрый барин,\\
И там, где прах его лежит,\\
Надгробный памятник гласит:\\
Смиренный грешник, Дмитрий Ларин,\\
Господний раб и бригадир,\\
Под камнем сим вкушает мир.\\

XXXVII

Своим пенатам возвращенный,\\
Владимир Ленский посетил\\
Соседа памятник смиренный,\\
И вздох он пеплу посвятил;\\
И долго сердцу грустно было.\\
«Рооr Yorick! — молвил он уныло. —\\
Он на руках меня держал.\\
Как часто в детстве я играл\\
Его Очаковской медалью!\\
Он Ольгу прочил за меня,\\
Он говорил: дождусь ли дня?..»\\
И, полный искренней печалью,\\
Владимир тут же начертал\\
Ему надгробный мадригал.\\

XXXVIII

И там же надписью печальной\\
Отца и матери, в слезах,\\
Почтил он прах патриархальный…\\
Увы! на жизненных браздах\\
Мгновенной жатвой поколенья,\\
По тайной воле провиденья,\\
Восходят, зреют и падут;\\
Другие им вослед идут…\\
Так наше ветреное племя\\
Растет, волнуется, кипит\\
И к гробу прадедов теснит.\\
Придет, придет и наше время,\\
И наши внуки в добрый час\\
Из мира вытеснят и нас!\\

XXXIX

Покамест упивайтесь ею,\\
Сей легкой жизнию, друзья!\\
Ее ничтожность разумею\\
И мало к ней привязан я;\\
Для призраков закрыл я вежды;\\
Но отдаленные надежды\\
Тревожат сердце иногда:\\
Без неприметного следа\\
Мне было б грустно мир оставить.\\
Живу, пишу не для похвал;\\
Но я бы, кажется, желал\\
Печальный жребий свой прославить,\\
Чтоб обо мне, как верный друг,\\
Напомнил хоть единый звук.\\

XL

И чье-нибудь он сердце тронет;\\
И, сохраненная судьбой,\\
Быть может, в Лете не потонет\\
Строфа, слагаемая мной;\\
Быть может (лестная надежда!),\\
Укажет будущий невежда\\
На мой прославленный портрет\\
И молвит: то-то был поэт!\\
Прими ж мои благодаренья,\\
Поклонник мирных аонид,\\
О ты, чья память сохранит\\
Мои летучие творенья,\\
Чья благосклонная рука\\
Потреплет лавры старика!\\

%\chapter{} % Capitolo 3
\label{cap3} % Etichetta per riferimenti incrociati

\textit{Elle était fille, elle était amoureuse.\\
Malfilâtre}

I

«Куда? \textbf{Уж эти мне поэты!}»\\
— Прощай, Онегин, мне пора.	\\
«\textbf{Я не держу тебя}; но где ты\\
Свои проводишь вечера?»\\
— У Лариных. — «Вот это чудно.\\
Помилуй! \textbf{и тебе не трудно}\\
Там каждый вечер убивать?»\\
— Нимало. — «Не могу понять.\\
\emph{Отселе}\footnote{откуда} вижу, что такое:\\
Во-первых (слушай, прав ли я?),\\
Простая, русская семья,\\
К гостям усердие большое,\\
Варенье, вечный разговор\\
Про дождь, про лен, про скотный двор…»\\

II

— Я тут еще беды не вижу.\\
«Да скука, вот беда, мой друг».\\
— Я модный свет ваш ненавижу;\\
Милее мне домашний круг,\\
Где я могу… — «Опять\emph{эклога!}\footnote{Стихотворение в повествовательной или диалогической форме на тему о пастушеской жизни, сходное с идиллией.}\\
Да полно, милый, ради бога.\\
Ну что ж? ты едешь: очень жаль.\\
Ах, слушай, Ленский; да нельзя ль\\
Увидеть мне Филлиду эту,\\
Предмет и мыслей, и пера,\\
И слез, и рифм et cetera?..\\
Представь меня». — Ты шутишь. — «Нету».\\
— Я рад. — «Когда же?» — Хоть сейчас.\\
Они с охотой примут нас.\\

III

Поедем. —\\
Поскакали други,\\
Явились; им расточены\\
\emph{Порой}\\footnote{Иногда, иной раз} тяжелые услуги\\
Гостеприимной старины.\\
Обряд известный угощенья:\\
Несут на блюдечках варенья,\\
На столик ставят \emph{вощаной}\footnote{Натёртый или пропитанный воском; вощёный.}\\
Кувшин с \emph{брусничною водой}.\footnote{БРУСНИ́КА, Мелкие красные кисловатые ягоды этого кустарничка.}\\
. . . . . . . . . . . . . .\\
. . . . . . . . . . . . . .\\
. . . . . . . . . . . . . .\\
. . . . . . . . . . . . . .\\
. . . . . . . . . . . . . .\\
. . . . . . . . . . . . . .\\

IV

Они дорогой самой краткой\\
Домой летят во весь опор.\\
Теперь подслушаем \emph{украдкой}`\footnote{Скрытно, незаметно от других.}\\
Героев наших разговор:\\
— Ну что ж, Онегин? ты зеваешь. —\\
«Привычка, Ленский». — Но скучаешь\\
Ты как-то больше. — «Нет, равно.\\
Однако в поле уж темно;\\
Скорей! пошел, пошел, Андрюшка!\\
Какие глупые места!\\
А кстати: Ларина проста,\\
Но очень милая старушка;\\
Боюсь: брусничная вода\\
Мне не наделала б вреда.\\% fa mal di pancia ?

V

Скажи: которая Татьяна?»\\
— Да та, которая, грустна\\
И молчалива, как Светлана,\\% chi è Svetlana ?
Вошла и села у окна. —\\
«Неужто ты влюблен в меньшую?»\\
— А что? — «Я выбрал бы другую,\\
Когда б я был, как ты, поэт.\\
В чертах у Ольги жизни нет.\\
Точь-в-точь в Вандиковой Мадоне:\\ %cioè ?
Кругла, красна лицом она,\\
Как эта глупая луна\\
На этом глупом небосклоне».\\
Владимир сухо отвечал\\
И после во весь путь молчал.\\

VI

Меж тем Онегина явленье\\
У Лариных произвело\\
На всех большое впечатленье\\
И всех соседей развлекло.\\
Пошла догадка за догадкой.\\
Все стали толковать украдкой,\\
Шутить, судить не без греха,\\% interessante frase
Татьяне прочить жениха;\\
Иные даже утверждали,\\
Что свадьба слажена совсем,\\
Но остановлена затем,\\
Что модных колец не достали.\\
О свадьбе Ленского давно\\
У них уж было решено.\\

VII% come scoppia l'amore per Oneghin

Татьяна слушала с \emph{досадой}\footnote{Раздражение, неудовольствие}\\
Такие сплетни; но тайком\\
С \emph'{неизъяснимою отрадой}\footnote{невыразимый радость}\\ 
Невольно думала о том;\\
И в сердце дума заронилась;\\
Пора пришла, она влюбилась.\\
Так в землю падшее зерно\\
Весны огнем оживлено.\\
Давно ее воображенье,\\
Сгорая негой и тоской,\\
Алкало пищи роковой;\\
Давно сердечное томленье\\
Теснило ей младую грудь;\\
Душа ждала… кого-нибудь,\\

VIII

И дождалась… Открылись очи;\\
Она сказала: это он!\\
Увы! теперь и дни и ночи,\\
И жаркий одинокий сон,\\
Все полно им; все деве милой\\
Без умолку волшебной силой\\
Твердит о нем. Докучны ей\\
И звуки ласковых речей,\\
И взор заботливой прислуги.\\
В уныние погружена,\\
Гостей не слушает она\\
И проклинает их \emph{досуги,}\footnote{свободное время}\\
Их неожиданный приезд\\
И продолжительный присест.\\

IX

Теперь с каким она вниманьем\\
Читает сладостный роман,\\
С каким живым очарованьем\\
Пьет \emph{обольстительный}\footnote{заманчивый, привлекательный} обман!\\
Счастливой силою мечтанья\\
Одушевленные созданья,\\
Любовник Юлии Вольмар,\\
Малек-Адель и де Линар,\\
И Вертер, мученик мятежный,\\
И бесподобный Грандисон,\\%??
Который нам наводит сон,\\
Все для мечтательницы нежной\\
В единый образ облеклись,\\
\textbf{В одном Онегине слились.}\\

X

Воображаясь героиной?\\
Своих возлюбленных творцов,\\
Кларисой, Юлией, Дельфиной,\\
Татьяна в тишине лесов\\
\textbf{Одна с опасной книгой бродит,}\\
Она в ней ищет и находит\\
Свой тайный жар, свои мечты,\\
Плоды сердечной полноты,\\
Вздыхает и, себе присвоя\\
Чужой восторг, чужую грусть,\\
В забвенье шепчет наизусть\\
Письмо для милого героя…\\
Но наш герой, кто б ни был он,\\
Уж верно был не Грандисон.\\

XI

Свой слог на важный лад настроя,\\
Бывало, пламенный творец\\
Являл нам своего героя\\
Как совершенства образец.\\
Он одарял предмет любимый,\\
Всегда неправедно гонимый,\\
Душой чувствительной, умом\\
И привлекательным лицом.\\
Питая жар чистейшей страсти,\\
Всегда восторженный герой\\
Готов был жертвовать собой,\\
И при конце последней части\\
Всегда наказан был \emph{порок,}\footnote{vizio}\\
Добру достойный был венок.\\

XII

А нынче все умы в тумане,\\
Мораль на нас наводит сон,\\
Порок любезен — и в романе,\\
И там уж торжествует он.\\
Британской музы \emph{небылицы}\footnote{delle favole}\\
Тревожат сон \emph{отроковицы,}\footnote{Девочка-подросток в возрасте между ребенком и девушкой}\\
И стал теперь ее кумир\\
Или задумчивый Вампир,\\
Или Мельмот, бродяга мрачный,\\
Иль Вечный жид, или Корсар,\\
Или таинственный Сбогар.\\
Лорд Байрон прихотью удачной\\
Облек в унылый романтизм\\
И безнадежный эгоизм.\\

XIII\\footnote{простая русская семья, простая литература намичает путь к эволуций прозе}

Друзья мои, что ж толку в этом?\\
Быть может, волею небес,\\
Я перестану быть поэтом,\\
В меня вселится новый бес,\\
И, Фебовы\\footnote{Аполлон-Представитель высокого исскуства в классицизме} презрев угрозы,\\
Унижусь до смиренной прозы;\\
Тогда роман на старый лад\\
Займет веселый мой закат.\\
Не муки тайные злодейства\\
Я грозно в нем изображу,\\
Но просто вам перескажу\\
Преданья русского семейства,\\
Любви \emph{пленительные}\footnote{привлекательный} сны\\
Да нравы нашей старины.\\

XIV

Перескажу простые речи\\
Отца иль дяди-старика,\\
Детей условленные встречи\\
У старых лип, у ручейка;\\
Несчастной ревности мученья,\\
Разлуку, слезы примиренья,\\
Поссорю вновь, и наконец\\
Я поведу их под венец…\\
Я вспомню речи неги страстной,\\
Слова тоскующей любви,\\
Которые в минувши дни\\
У ног любовницы прекрасной\\
Мне приходили на язык,\\
От коих я теперь отвык.\\

XV\footnote{П. описывает Татьяну с ееромантическим языком}

Татьяна, милая Татьяна!\\
С тобой теперь я слезы лью;\\
Ты в руки модного тирана\\
Уж отдала судьбу свою.\\
Погибнешь, милая; но прежде\\
Ты в ослепительной надежде\\
Блаженство темное зовешь,\\
Ты негу жизни узнаешь,\\
Ты пьешь волшебный яд желаний,\\
Тебя преследуют мечты:\\
Везде воображаешь ты\\
Приюты счастливых свиданий;\\
Везде, везде перед тобой\\
Твой \emph{искуситель роковой.}`\footnote{соблазнитель неизбежный.}\\

XVI

Тоска любви Татьяну гонит,\\
И в сад идет она грустить,\\
И вдруг недвижны очи клонит,\\
И лень ей далее ступить.\\
Приподнялася грудь, ланиты\\
Мгновенным пламенем покрыты,\\
Дыханье замерло в устах,\\
И в слухе шум, и блеск в очах…\\
Настанет ночь; луна обходит\\
Дозором дальный свод небес,\\
И соловей во мгле древес\\
Напевы звучные заводит.\\
Татьяна в темноте не спит\\
И тихо с няней говорит:\\

XVII

«Не спится, няня: здесь так душно!\\
Открой окно да сядь ко мне».\\
— Что, Таня, что с тобой? — «Мне скучно,\\
Поговорим о старине».\\
— О чем же, Таня? Я, бывало,\\
Хранила в памяти не мало\\
Старинных былей, небылиц\\
Про злых духов и про девиц;\\
А нынче все мне темно, Таня:\\
Что знала, то забыла. Да,\\
Пришла худая череда!\\
Зашибло… — «Расскажи мне, няня,\\
Про ваши старые года:\\
Была ты влюблена тогда?»\\

XVIII\footnote{Т. нуждается в совет и обращается к няне}

— И, полно, Таня! В эти лета\\
Мы не слыхали про любовь;\\
А то бы согнала со света\\
Меня покойница свекровь. —\\
«Да как же ты венчалась, няня?»\\
— Так, видно, бог велел. Мой Ваня\\
Моложе был меня, мой свет,\\
А было мне тринадцать лет.\\
Недели две ходила сваха\\
К моей родне, и наконец\\
Благословил меня отец.\\
Я горько плакала со страха,\\
Мне с плачем \emph{косу расплели}\footnote{одна коса это девичья, две нже замужная женшина}\\
Да с пеньем в церковь повели.\\

XIX

И вот ввели в семью чужую…\\
Да ты не слушаешь меня… —\\
«Ах, няня, няня, я тоскую,\\
Мне тошно, милая моя:\\
Я плакать, я рыдать готова!..»\\
— Дитя мое, ты нездорова;\\
Господь помилуй и спаси!\\
Чего ты хочешь, попроси…\\
Дай окроплю святой водою,\\
Ты вся горишь… — «Я не больна:\\
Я… знаешь, няня… влюблена».\\
— Дитя мое, господь с тобою! —\\
И няня девушку с мольбой\marginnote{молитваЪ\\
Крестила дряхлою рукой.\\

XX

«Я влюблена», — шептала снова\\
Старушке с горестью она.\\
— Сердечный друг, ты нездорова.\\
«Оставь меня: я влюблена».\\
И между тем луна сияла\\
И томным светом озаряла\\
Татьяны бледные красы,\\
И распущенные власы,\\
И капли слез, и на скамейке\\
Пред героиней молодой,\\
С платком на голове седой,\\
Старушку в длинной телогрейке;\\
И все дремало в тишине\\
При вдохновительной луне.\\

XXI

И сердцем далеко носилась\\
Татьяна, смотря на луну…\\
Вдруг мысль в уме ее родилась…\\
«Поди, оставь меня одну.\\
Дай, няня, мне перо, бумагу,\\
Да стол подвинь; я скоро лягу;\\
Прости». И вот она одна.\\
Все тихо. Светит ей луна.\\
Облокотясь\marginnote'{appoggiare i gomiti sul tavolo}, Татьяна пишет\\,
И все Евгений на уме,\\
И в необдуманном письме\\
Любовь невинной девы дышит.\\
Письмо готово, сложено…\\
Татьяна! для кого ж оно?\\

XXII

Я знал красавиц недоступных,\\
Холодных, чистых, как зима,\\
Неумолимых, неподкупных,\\
Непостижимых для ума;\\
Дивился я их спеси модной,\\
Их добродетели природной,\\
И, признаюсь, от них бежал,\\
И, мнится, с ужасом читал\\
Над их бровями надпись ада:\\
Оставь надежду навсегда.\\
Внушать любовь для них беда,\\
Пугать людей для них отрада.\\
Быть может, на брегах Невы\\
Подобных дам видали вы.\\

XXIII

Среди поклонников послушных\\
Других причудниц я видал,\\
Самолюбиво равнодушных\\
Для вздохов страстных и похвал.\\
И что ж нашел я с изумленьем?\\
Они, суровым повеленьем\\
Пугая робкую любовь,\\
Ее привлечь умели вновь\\
По крайней мере сожаленьем,\\
По крайней мере звук речей\\
Казался иногда нежней,\\
И с легковерным ослепленьем\\
Опять любовник молодой\\
Бежал за милой суетой.\\

XXIV

За что ж виновнее Татьяна?\\
За то ль, что в милой простоте\\
Она не ведает обмана\\
И верит избранной мечте?\\
За то ль, что любит без искусства,\\
Послушная влеченью чувства,\\
Что так доверчива она,\\
Что от небес одарена\\
Воображением мятежным,\\
Умом и волею живой,\\
И своенравной головой,\\
И сердцем пламенным и нежным?\\
Ужели не простите ей\\
Вы легкомыслия страстей?\\

XXV

Кокетка судит хладнокровно,\\
Татьяна любит не шутя\\
И предается безусловно\\
Любви, как милое дитя.\\
Не говорит она: отложим —\\
Любви мы цену тем умножим,\\
Вернее в сети заведем;\\
Сперва тщеславие кольнем\\\
Надеждой, там недоуменьем\\
Измучим сердце, а потом\\
Ревнивым оживим огнем;\\
А то, скучая наслажденьем\\,
Невольник хитрый из оков\\
Всечасно вырваться готов.\\

XXVI

Еще предвижу затрудненья:\\
Родной земли спасая честь,\\
Я должен буду, без сомненья,\\
Письмо Татьяны перевесть.\\
Она по-русски плохо знала,\\
Журналов наших не читала\\
И выражалася с трудом\\
На языке своем родном,\\
Итак, писала по-французски…\\
Что делать! повторяю вновь:\\
Доныне \marginnote{ло сих пор}дамская любовь\\
Не изьяснялася по-русски,\\
Доныне гордый наш язык\\
К почтовой прозе не привык.\\

XXVII

Я знаю: дам хотят заставить\\
Читать по-русски. Право, страх!\\
Могу ли их себе представить\\
С «Благонамеренным» в руках!\\
Я шлюсь на вас, мои поэты;\\
Не правда ль: милые предметы,\\
Которым, за свои грехи,\\
Писали втайне вы стихи,\\
Которым сердце посвящали,\\
Не все ли, русским языком\\
Владея слабо и с трудом,\\
Его так мило `emph{искажали,}\marginnote{портить, коверкать}\\
И в их устах язык чужой\\
Не обратился ли в родной?\\

XXVIII

Не дай мне бог сойтись на бале\\
Иль при разъезде на крыльце\\
С семинаристом в желтой шале\\
Иль с академиком в чепце!\\
Как уст румяных без улыбки,\\
Без грамматической ошибки\\
Я русской речи не люблю.\\	
Быть может, на беду мою,\\
Красавиц новых поколенье,\\
Журналов вняв молящий глас,\\
К грамматике приучит нас;\\
Стихи введут в употребленье;\\
Но я… какое дело мне?\\
Я верен буду старине.\\

XXIX

Неправильный, небрежный лепет,\\
Неточный выговор речей\\
По-прежнему сердечный трепет\\
Произведут в груди моей;\\
Раскаяться во мне нет силы,\\
Мне галлицизмы будут милы,\\
Как прошлой юности грехи,\\
Как Богдановича стихи.\\
Но полно. Мне пора заняться\\
Письмом красавицы моей;\\
Я слово дал, и что ж? ей-ей\\
Теперь готов уж отказаться.\\
Я знаю: нежного Парни\\
Перо не в моде в наши дни.\\

XXX

Певец Пиров и грусти томной,\\
Когда б еще ты был со мной,\\
Я стал бы просьбою нескромной\\
Тебя тревожить, милый мой:\\
Чтоб на волшебные напевы\\
Переложил ты страстной девы\\
Иноплеменные слова.\\
Где ты? приди: свои права\\
Передаю тебе с поклоном…\\
Но посреди печальных скал,\\
Отвыкнув сердцем от похвал,\\
Один, под финским небосклоном,\\
Он бродит, и душа его\\
Не слышит горя моего.\\

XXXI

Письмо Татьяны предо мною;\\
Его я свято берегу,\\
Читаю с тайною тоскою\\
И начитаться не могу.\\
Кто ей внушал и эту нежность,\\
И слов любезную небрежность\marginnote{negligenza}?\\
Кто ей внушал умильный вздор,\\
Безумный сердца разговор,\\
И увлекательный и вредный?\\
Я не могу понять. Но вот\\
Неполный, слабый перевод,\\
С живой картины список бледный\\
Или разыгранный Фрейшиц\\
Перстами робких учениц:\\
Письмо Татьяны к Онегину\\
Я к вам пишу — чего же боле?\\
Что я могу еще сказать?\\
Теперь, я знаю, в вашей воле\\
Меня презреньем наказать.\\
Но вы, к моей несчастной доле\\
Хоть каплю жалости храня,\\
Вы не оставите меня.\\
Сначала я молчать хотела;\\
Поверьте: моего стыда\\
Вы не узнали б никогда,\\
Когда б надежду я имела\\
Хоть редко, хоть в неделю раз\\
В деревне нашей видеть вас,\\
Чтоб только слышать ваши речи,\\
Вам слово молвить, и потом\\
Все думать, думать об одном\\
И день и ночь до новой встречи.\\
Но, говорят, вы нелюдим;\\
В глуши, в деревне все вам скучно,\\
А мы… ничем мы не блестим,\\
Хоть вам и рады простодушно.\\
Зачем вы посетили нас?\\
В глуши забытого селенья\\
Я никогда не знала б вас,\\
Не знала б горького мученья.\\
Души неопытной волненья\\
Смирив со временем (как знать?),\\
По сердцу я нашла бы друга,\\
Была бы верная супруга\\
И добродетельная мать.\\
Другой!.. Нет, никому на свете\\
Не отдала бы сердца я!\\
То в вышнем суждено совете…\\
То воля неба: я твоя;\\
Вся жизнь моя была залогом\\
Свиданья верного с тобой;\\
Я знаю, ты мне послан богом,\\
До гроба ты хранитель мой…\\
Ты в сновиденьях мне являлся\\
Незримый, ты мне был уж мил,\\
Твой чудный взгляд меня томил,\\
В душе твой голос раздавался\\
Давно… нет, это был не сон!\\
Ты чуть вошел, я вмиг узнала,\\
Вся обомлела, запылала\\
И в мыслях молвила: вот он!\\
Не правда ль? я тебя слыхала:\\
Ты говорил со мной в тиши,\\
Когда я бедным помогала\\
Или молитвой услаждала\\
Тоску волнуемой души?\\
И в это самое мгновенье\\
Не ты ли, милое виденье,\\
В прозрачной темноте мелькнул,\\
Приникнул тихо к изголовью?\\
Не ты ль, с отрадой и любовью,\\
Слова надежды мне шепнул?\\
Кто ты, мой ангел ли хранитель,\\
Или коварный искуситель:\\
Мои сомненья разреши.\\
Быть может, это все пустое,\\
Обман неопытной души!\\
И суждено совсем иное…\\
Но так и быть! Судьбу мою\\
Отныне я тебе вручаю,\\
Перед тобою слезы лью,\\
Твоей защиты умоляю…\\
Вообрази: я здесь одна,\\
Никто меня не понимает,\\
Рассудок мой изнемогает,\\
И молча гибнуть я должна.\\
Я жду тебя: единым взором\\
Надежды сердца оживи\\
Иль сон тяжелый перерви,\\
Увы, заслуженным укором!\\
Кончаю! Страшно перечесть…\\
Стыдом и страхом замираю…\\
Но мне порукой ваша честь,\\
И смело ей себя вверяю…\\

XXXII

Татьяна то вздохнет, то охнет\marginnote{Воскликнуть ох, выражая какое-л. ощущение, чувство: боль, досаду, испуг и т. п.}\\
Письмо дрожит в ее руке;\\
Облатка розовая сохнет\\
На воспаленном\marginnote{infiammata} языке.\\
К плечу головушкой склонилась,\\
Соро'чка\\marginnote{intimo} легкая спустилась\\
С ее прелестного плеча…\\
Но вот уж лунного луча\\
Сиянье гаснет. Там долина\marginnote{Ровное пространство вдоль речного русла}\\
Сквозь пар яснеет. Там поток\\
Засеребрился; там рожок\\
Пастуший будит селянина.\marginnote{Крестьянин}\\
Вот утро: встали все давно,\\
Моей Татьяне все равно.\\

XXXIII

Она зари не замечает,\\
Сидит с поникшею главой\\
И на письмо не напирает\\
Своей печати вырезной.\\
Но, дверь тихонько отпирая,\\
Уж ей Филипьевна седая\\
Приносит на подносе чай.\\
«Пора, дитя мое, вставай:\\
Да ты, красавица, готова!\\
О пташка ранняя моя!\\
Вечор уж как боялась я!\\
Да, слава богу, ты здорова!\\
Тоски ночной и следу нет,\\
Лицо твое как маков цвет».\\

XXXIV

— Ах! няня, сделай одолженье. —\\
«Изволь, родная, прикажи».\\
— Не думай… право… подозренье…\\
Но видишь… ах! не откажи. —\\
«Мой друг, вот бог тебе порука».\\
— Итак, пошли тихонько внука\\
С запиской этой к О… к тому…\\
К соседу… да велеть ему,\\
Чтоб он не говорил ни слова,\\
Чтоб он не называл меня… —\\
«Кому же, милая моя?\\
Я нынче стала бестолкова.\\
Кругом соседей много есть;\\
Куда мне их и перечесть».\\

XXXV

— Как недогадлива ты, няня! —\\
«Сердечный друг, уж я стара,\\
Стара; тупеет разум, Таня;\\
А то, бывало, я востра,\\
Бывало, слово барской воли…»\\
— Ах, няня, няня! до того ли?\\
Что нужды мне в твоем уме?\\
Ты видишь, дело о письме\\
К Онегину. — «Ну, дело, дело.\\
Не гневайся, душа моя,\\
Ты знаешь, непонятна я…\\
Да что ж ты снова побледнела?»\\
— Так, няня, право ничего.\\
Пошли же внука своего.\\

XXXVI

Но день протек, и нет ответа.\\
Другой настал: все нет как нет.\\
Бледна, как тень, с утра одета,\\
Татьяна ждет: когда ж ответ?\\
Приехал Ольгин обожатель.\\
«Скажите: где же ваш приятель? —\\
Ему вопрос хозяйки был. —\\
Он что-то нас совсем забыл».\\
Татьяна, вспыхнув, задрожала.\\
— Сегодня быть он обещал, —\\
Старушке Ленский отвечал, —\\
Да, видно, почта задержала. —\\
Татьяна потупила взор,\\
Как будто слыша злой укор.\\

XXXVII

Смеркалось; на столе, блистая,\\
Шипел вечерний самовар,\\
Китайский чайник нагревая;\\
Под ним клубился легкий пар.\\
Разлитый Ольгиной рукою,\\
По чашкам темною струею\\
Уже душистый чай бежал,\\
И сливки мальчик подавал;\\
Татьяна пред окном стояла,\\
На стекла хладные дыша,\\
Задумавшись, моя душа,\\
Прелестным пальчиком писала\\
На отуманенном стекле\\
Заветный вензель\marginnote{Особенно ценимый Начальные буквы имени и фамилии} О да Е.\\

XXXVIII

И между тем душа в ней ныла,
И слез был полон томный взор.
Вдруг топот!.. кровь ее застыла.
Вот ближе! скачут… и на двор
Евгений! «Ах!» — и легче тени
Татьяна прыг в другие сени,
С крыльца на двор, и прямо в сад,
Летит, летит; взглянуть назад
Не смеет; мигом обежала
Куртины, мостики, лужок,
Аллею к озеру, лесок,
Кусты сирен переломала,
По цветникам летя к ручью.
И, задыхаясь, на скамью
XXXIX
Упала…
«Здесь он! здесь Евгений!
О боже! что подумал он!»
В ней сердце, полное мучений,
Хранит надежды темный сон;
Она дрожит и жаром пышет,
И ждет: нейдет ли? Но не слышит.
В саду служанки, на грядах,
Сбирали ягоду в кустах
И хором по наказу пели
(Наказ, основанный на том,
Чтоб барской ягоды тайком
Уста лукавые не ели
И пеньем были заняты:
Затея сельской остроты!)
Песня девушек
Девицы, красавицы,
Душеньки, подруженьки,
Разыграйтесь девицы,
Разгуляйтесь, милые!
Затяните песенку,
Песенку заветную,
Заманите молодца
К хороводу нашему,
Как заманим молодца,
Как завидим издали,
Разбежимтесь, милые,
Закидаем вишеньем,
Вишеньем, малиною,
Красною смородиной.
Не ходи подслушивать
Песенки заветные,
Не ходи подсматривать
Игры наши девичьи.
ХL
Они поют, и, с небреженьем
Внимая звонкий голос их,
Ждала Татьяна с нетерпеньем,
Чтоб трепет сердца в ней затих,
Чтобы прошло ланит пыланье.
Но в персях то же трепетанье,
И не проходит жар ланит,
Но ярче, ярче лишь горит…
Так бедный мотылек и блещет
И бьется радужным крылом,
Плененный школьным шалуном;
Так зайчик в озими трепещет,
Увидя вдруг издалека
В кусты припадшего стрелка.
ХLI
Но наконец она вздохнула
И встала со скамьи своей;
Пошла, но только повернула
В аллею, прямо перед ней,
Блистая взорами, Евгений
Стоит подобно грозной тени,
И, как огнем обожжена,
Остановилася она.
Но следствия нежданной встречи
Сегодня, милые друзья,
Пересказать не в силах я;
Мне должно после долгой речи
И погулять и отдохнуть:
Докончу после как-нибудь.

\section{Analisi cap.3}
Строфа I - II   П. вводит в поэтическое произведение буднучную прозаичесукую беседу
%\chapter{} % Capitolo di Metodologia
\label{cap2} % Etichetta per riferimenti incrociati

\textit{O rus!..\\
Hor.\\ 
(О Русь!)}\\

I

Деревня, где скучал Евгений,\\
Была прелестный уголок;\\
Там друг невинных наслаждений\\
Благословить бы небо мог.\\
Господский дом уединенный,\\
Горой от ветров огражденный,\\
Стоял над речкою. Вдали\\
Пред ним пестрели и цвели\\
Луга и нивы золотые,\\
Мелькали селы; здесь и там\\
Стада бродили по лугам,\\
И сени расширял густые\\
Огромный, запущенный сад,\\
Приют задумчивых \emph{дриад.}\footnote{в греческой мифологии нимфы, покровительницы деревьев}

II

Почтенный замок был построен,\\
Как замки строиться должны:\\
\emph{Отменно}\footnote{Оценочная характеристика чего-либо как очень хорошего, превосходного.} прочен и спокоен\\
Во вкусе умной старины.\\
Везде высокие покои,\\
В гостиной \emph{штофные}\marginpar{\textit{Damasco}} обои,\\
Царей портреты на стенах,\\
И печи в пестрых изразцах.\\
Все это ныне \emph{обветшало,}\marginpar{fatiscente}\\
Не знаю, право, почему;\\
Да, впрочем, другу моему\\
В том нужды было очень мало,\\
Затем, что он равно зевал\\
Средь модных и старинных зал.\\

III

Он в том покое поселился,\\
Где деревенский старожил\\
Лет сорок с ключницей бранился,\\
В окно смотрел и мух давил.\\
Все было просто: пол дубовый,\\
Два шкафа, стол, диван пуховый,\\
Нигде ни пятнышка чернил.\\
Онегин шкафы отворил;\\
В одном нашел тетрадь расхода,\\
В другом наливок целый строй,\\
Кувшины с яблочной водой\\
И календарь осьмого года:\\
Старик, имея много дел,\\
В иные книги не глядел.\\

IV

Один среди своих владений,\\
\textit{Чтоб только время проводить,\\
Сперва задумал наш Евгений\\
Порядок новый учредить.\\
В своей глуши мудрец пустынный,\\
Ярем он барщины старинной\\
Оброком легким заменил;\\
И раб судьбу благословил.}\footnote{При барщине крестьяне бесплатно работали на земле своего хозяина-крепостника, в лучшем случае им давали месячину - продовольственный паёк. Производительность труда, сами понимаете, крайне невелика, невелики и доходы. Онегин разделил свои владения между крестьянами, которые теперь сами обрабатывали землю, крутились как могли, и платили своему хозяину оброк. На эту же тему см. тургеневский "Хорь и Калиныч" из "Записок охотника". Оброк, кстати, тоже большого дохода не приносил - в среднем по 5 рублей с крестьянского двора в год. 
Сроки барщины были неопределёнными. Да, император Павел Первый в 1797 г. попытался запретить крестьянам работать в праздники и ограничил барщину 3 днями в неделю, но фактически указ не выполнялся. По закону крестьянин должен был обрабатывать столько же помещичьей земли, сколько ему выделялось на личное хозяйство. Точнее, помещик выделял землю крестьянской общине, а делили её между собой уже на сходе. }\\
Зато в углу своем надулся,\\
Увидя в этом страшный вред,\\\
Его расчетливый сосед;\\
Другой лукаво улыбнулся,\\
И в голос все решили так,\\
Что он опаснейший чудак.\\

V

Сначала все к нему езжали;\\
Но так как с заднего крыльца\\
Обыкновенно подавали\\
Ему донского жеребца,\\
Лишь только вдоль большой дороги\\
Заслышат их домашни дроги, —\\
Поступком оскорбясь таким,\\
Все дружбу прекратили с ним.\footnote{Вначале, сразу после того, как Онегин поселился в поместье умершего дяди, все соседи стали ездить к нему в гости с визитами - познакомиться или поддержать знакомство. Так у них было принято. 
Но Онегин ни с кем из сельских соседей дружить не хотел. И даже просто поддерживать какие-либо отношения тоже не хотел. Поэтому, когда с дороги, ведущей к его дому, слышался звук приближающегося экипажа (домашних дрог, дрожек, то есть, коляски, сделанной местным крепостным умельцем), Онегин требовал верхового коня и уезжал из дома на прогулку. Причем уезжал с заднего крыльца, чтобы попасть на другую дорогу, на которой он не повстречался бы с непрошеными гостями.
Соседи поняли все правильно, обиделись и прекратили дружбу с ним.}\\
«Сосед наш неуч; сумасбродит;\\
Он фармазон; он пьет одно\\
Стаканом красное вино;\\
Он дамам к ручке не подходит;\\
Все да да нет; не скажет да-с\\
Иль нет-с». Таков был общий глас.\\

VI\footnote{характер Ленского. Изъять из Быта}

В свою деревню в ту же пору\\
Помещик новый прискакал\\
И столь же строгому разбору\\
В соседстве повод подавал:\\
По имени \emph{Владимир Ленской,}\\
С душою прямо геттингенской,\\
Красавец, в полном цвете лет,\\
Поклонник Канта и поэт.\\
Он из Германии туманной\\
Привез учености плоды:\\
Вольнолюбивые мечты,\\
Дух пылкий\marginpar{Сильно чувствующий} и довольно странный,\\
Всегда восторженную\marginpar{легко приходит в состояние восторга...} речь\\
И кудри черные до плеч.\\

VII

От хладного разврата света\\
Еще увянуть не успев,\\
Его душа была согрета\\
Приветом друга, лаской дев;\\
Он сердцем милый был невежда,\\
Его лелеяла надежда,\\
И мира новый блеск и шум\\
Еще пленяли юный ум.\\
Он забавлял мечтою сладкой\\
Сомненья сердца своего;\\
Цель жизни нашей для него\\
Была заманчивой загадкой,\\
Над ней он голову ломал\\
И чудеса подозревал.\\

VIII

Он верил, что душа родная\\
Соединиться с ним должна,\\
Что, безотрадно изнывая,\\
Его вседневно ждет она;\\
Он верил, что друзья готовы\\
За честь его приять оковы\\
И что не дрогнет их рука\\
Разбить сосуд клеветника;\\
Что есть избранные судьбами,\\
Людей священные друзья;\\
Что их бессмертная семья\\
Неотразимыми лучами\\
Когда-нибудь нас озарит\\
И мир блаженством одарит.\\

IX

Негодованье, сожаленье,\\
Ко благу чистая любовь\\
И славы сладкое мученье\\
В нем рано волновали кровь.\\
Он с лирой странствовал на свете;\\
Под небом Шиллера и Гете\\
Их поэтическим огнем\\
Душа воспламенилась в нем;\\
И муз возвышенных искусства,\\
Счастливец, он не постыдил:\\
Он в песнях гордо сохранил\\
Всегда возвышенные чувства,\\
Порывы девственной мечты\\
И прелесть важной простоты.\\

X

Он пел любовь, любви послушный,\\
И песнь его была ясна,\\
Как мысли девы простодушной,\\
Как сон младенца, как луна\\
В пустынях неба безмятежных,\\
Богиня тайн и вздохов нежных.\\
Он пел разлуку и печаль,\\
И нечто, и туманну даль,\\
И романтические розы;\\
Он пел те дальные страны,\\
Где долго в лоно тишины\\
Лились его живые слезы;\\
Он пел поблеклый жизни цвет\\
Без малого в осьмнадцать лет.\\

XI

В пустыне, где один Евгений\\
Мог оценить его дары,\\
Господ соседственных селений\\
Ему не нравились пиры;\\
Бежал он их беседы шумной.\\
Их разговор благоразумный\\
О сенокосе, о вине,\\
О псарне, о своей родне,\\
Конечно, не блистал ни чувством,\\
Ни поэтическим огнем,\\
Ни остротою, ни умом,\\
Ни общежития искусством;\\
Но разговор их милых жен\\
Гораздо меньше был умен.\\

XII

Богат, хорош собою, Ленский\\
Везде был принят как жених;\\
Таков обычай деревенский;\\
Все дочек прочили своих\\
За полурусского соседа;\\
Взойдет ли он, тотчас беседа\\
Заводит слово стороной\\
О скуке жизни холостой;\\
Зовут соседа к самовару,\\
А Дуня разливает чай;\\
Ей шепчут: «Дуня, примечай!»\marginpar{за кем-чем. Наблюдать, следить}\\
Потом приносят и гитару:\\
И запищит она (бог мой!):\\
Приди в чертог ко мне златой!..\\

XIII

Но Ленский, не имев, конечно,\\
Охоты узы\marginpar{vincoli} брака несть,\\
С Онегиным желал сердечно\\
Знакомство покороче свесть.\\
Они сошлись. Волна и камень,\\
Стихи и проза, лед и пламень\\
Не столь различны меж собой.\\
Сперва взаимной разнотой\\
Они друг другу были скучны;\\
Потом понравились; потом\\
Съезжались каждый день верхом\\
И скоро стали неразлучны.\\
Так люди (первый каюсь я)\\
От делать нечего друзья.\\

XIV

Но дружбы нет и той меж нами.\\
Все предрассудки истребя,\\
Мы почитаем всех нулями,\\
А единицами — себя.\\
Мы все глядим в Наполеоны;\\
Двуногих тварей миллионы\\
Для нас орудие одно;\\
Нам чувство дико и смешно.\\
Сноснее многих был Евгений;\\
Хоть он людей, конечно, знал\\
И вообще их презирал, —\\
Но (правил нет без исключений)\\
Иных он очень отличал\\
И вчуже чувство уважал.\\

XV

\emph{Он слушал Ленского с улыбкой.\\
Поэта пылкий разговор,\\
И ум, еще в сужденьях зыбкой,\\
И вечно вдохновенный взор, —\\
Онегину все было ново;\\
Он охладительное слово\\
В устах старался удержать\\
И думал: глупо мне мешать\\
Его минутному блаженству;\\
И без меня пора придет;\\
Пускай покамест он живет\\
Да верит мира совершенству;\\
Простим горячке юных лет\\
И юный жар и юный бред.}\\

XVI

Меж ими все рождало споры\\
И к размышлению влекло:\\
Племен минувших договоры,\\
Плоды наук, добро и зло,\\
И предрассудки вековые,\\
И гроба тайны роковые,\\
Судьба и жизнь в свою чреду,\\
Все подвергалось их суду.\\
Поэт в жару своих суждений\\
Читал, забывшись, между тем\\
Отрывки северных поэм,\\
И снисходительный Евгений,\\
Хоть их не много понимал,\\
Прилежно юноше внимал.\\

XVII

Но чаще занимали страсти\\
Умы пустынников моих.\\
Ушед от их мятежной власти,\\
Онегин говорил об них\\
С невольным вздохом сожаленья:\\
Блажен, кто ведал их волненья\\
И наконец от них отстал;\\
Блаженней тот, кто их не знал,\\
Кто охлаждал любовь — разлукой,\\
Вражду — злословием; порой\\
Зевал с друзьями и с женой,\\
Ревнивой не тревожась мукой,\\
И дедов верный капитал\\
Коварной двойке не вверял.\\

XVIII

Когда прибегнем мы под знамя\\
Благоразумной тишины,\\
Когда страстей угаснет пламя,\\
И нам становятся смешны\\
Их своевольство иль порывы\\
И запоздалые отзывы, —\\
Смиренные не без труда,\\
Мы любим слушать иногда\\
Страстей чужих язык мятежный,\\
И нам он сердце шевелит.\\
Так точно старый инвалид\\
Охотно клонит слух прилежный\\
Рассказам юных усачей,\\
Забытый в хижине своей.\\

XIX

Зато и пламенная младость\\
Не может ничего скрывать.\\
Вражду, любовь, печаль и радость\\
Она готова разболтать.\\
В любви считаясь инвалидом,\\
Онегин слушал с важным видом,\\
Как, сердца исповедь любя,\\
Поэт высказывал себя;\\
Свою доверчивую совесть\\
Он простодушно обнажал.\\
Евгений без труда узнал\\
Его любви младую повесть,\\
Обильный чувствами рассказ,\\
Давно не новыми для нас.\\

XX

Ах, он любил, как в наши лета\\
Уже не любят; как одна\\
Безумная душа поэта\\
Еще любить осуждена:\\
Всегда, везде одно мечтанье,\\
Одно привычное желанье,\\
Одна привычная печаль.\\
Ни охлаждающая даль,\\
Ни долгие лета разлуки,\\
Ни музам данные часы,\\
Ни чужеземные красы,\\
Ни шум веселий, ни науки\\
Души не изменили в нем,\\
Согретой девственным огнем.\\

XXI

Чуть отрок, Ольгою плененный,\\
Сердечных мук еще не знав,\\
Он был свидетель умиленный\\
Ее младенческих забав;\\
В тени хранительной дубравы\\
Он разделял ее забавы,\\
И детям прочили венцы\\
Друзья-соседы, их отцы.\\
В глуши, под сению смиренной,\\
Невинной прелести полна,\\
В глазах родителей, она\\
Цвела, как ландыш потаенный,\\
Незнаемый в траве глухой\\
Ни мотыльками, ни пчелой.\\

XXII

Она поэту подарила\\
Младых восторгов первый сон,\\
И мысль об ней одушевила\\
Его цевницы первый стон.\\
Простите, игры золотые!\\
Он рощи полюбил густые,\\
Уединенье, тишину,\\
И ночь, и звезды, и луну,\\
Луну, небесную лампаду,\\
Которой посвящали мы\\
Прогулки средь вечерней тьмы,\\
И слезы, тайных мук отраду…\\
Но нынче видим только в ней\\
Замену тусклых фонарей.\\

XXIII

Всегда скромна, всегда послушна,\\
Всегда как утро весела,\\
Как жизнь поэта простодушна,\\
Как поцелуй любви мила;\\
Глаза, как небо, голубые,\\
Улыбка, локоны льняные,\\
Движенья, голос, легкий стан,\\
Все в Ольге… но любой роман\\
Возьмите и найдете верно\\
Ее портрет: он очень мил,\\
Я прежде сам его любил,\\
Но надоел он мне безмерно.\\
Позвольте мне, читатель мой,\\
Заняться старшею сестрой.\\

XXIV

Ее сестра звалась Татьяна…\\
Впервые именем таким\\
Страницы нежные романа\\
Мы своевольно освятим.\\
И что ж? оно приятно, звучно;\\
Но с ним, я знаю, неразлучно\\
Воспоминанье старины\\
Иль девичьей! Мы все должны\\
Признаться: вкусу очень мало\\
У нас и в наших именах\\
(Не говорим уж о стихах);\\
Нам просвещенье не пристало,\\
И нам досталось от него\\
Жеманство, — больше ничего.\\

XXV

Итак, она звалась Татьяной.\\
Ни красотой сестры своей,\\
Ни свежестью ее румяной\\
Не привлекла б она очей.\\
Дика, печальна, молчалива,\\
\emph{Как лань лесная боязлива,\\
Она в семье своей родной}\\
Казалась девочкой чужой.\\
Она ласкаться не умела\\
К отцу, ни к матери своей;\\
Дитя сама, в толпе детей\\
Играть и прыгать не хотела\\
И часто целый день одна\\
Сидела молча у окна.\\

XXVI\footnote{[задумчивый мичтательный характер Татьяны}

Задумчивость, ее подруга\\
От самых колыбельных дней,\\
Теченье сельского досуга\\
Мечтами украшала ей.\\
Ее изнеженные пальцы\\
Не знали игл; склонясь на пяльцы,\\
Узором шелковым она\
Не оживляла полотна.\\
Охоты властвовать примета,\\
С послушной куклою дитя\\
Приготовляется шутя\\
К приличию — закону света,\\
И важно повторяет ей\\
Уроки маменьки своей.\\

XXVII

Но куклы даже в эти годы\\
Татьяна в руки не брала;\\
Про вести города, про моды\\
Беседы с нею не вела.\\
И были детские проказы\\
Ей чужды: страшные рассказы\\
Зимою в темноте ночей\\
Пленяли больше сердце ей.\\
Когда же няня собирала\\
Для Ольги на широкий луг\\
Всех маленьких ее подруг,\\
Она в горелки не играла,\\
Ей скучен был и звонкий смех,\\
И шум их ветреных утех.\\

XXVIII

Она любила на балконе\\\
Предупреждать зари восход,\\
Когда на бледном небосклоне\
Звезд исчезает хоровод,\\
И тихо край земли светлеет,\\
И, вестник утра, ветер веет,\\
И всходит постепенно день.\\
Зимой, когда ночная тень\\
Полмиром доле обладает,\\
И доле в праздной тишине,\\
При отуманенной луне,\\
Восток ленивый почивает,\\
В привычный час пробуждена\\
Вставала при свечах она.\\

XXIX

Ей рано нравились романы;\\
Они ей заменяли все;\\
Она влюблялася в обманы\\
И Ричардсона и Руссо.\\
Отец ее был добрый малый,\\
В прошедшем веке запоздалый;\\
Но в книгах не видал вреда;\\
\emph{Он, не читая никогда,\\
Их почитал пустой игрушкой}\\
И не заботился о том,\\
Какой у дочки тайный том\\
Дремал до утра под подушкой.\\
Жена ж его была сама\\
От Ричардсона без ума.\\

XXX

Она любила Ричардсона\\
Не потому, чтобы прочла,\\
Не потому, чтоб Грандисона\\
Она Ловласу предпочла;\\
Но в старину княжна Алина,\\
Ее московская кузина,\\
Твердила часто ей об них.\\
В то время был еще жених\\
Ее супруг, но по неволе;\\
Она вздыхала по другом,\\
Который сердцем и умом\\
Ей нравился гораздо боле:\\
Сей Грандисон был славный франт,\\
Игрок и гвардии сержант.\\

XXXI

Как он, она была одета\\
Всегда по моде и к лицу;\\
Но, не спросясь ее совета,\\
Девицу повезли к венцу.\\
И, чтоб ее рассеять горе,\\
Разумный муж уехал вскоре\\
В свою деревню, где она,\\
Бог знает кем окружена,\\
Рвалась и плакала сначала,\\
С супругом чуть не развелась;\\
Потом хозяйством занялась,\\
Привыкла и довольна стала.\\
\textbf{Привычка свыше нам дана:\\
Замена счастию она.}\\

XXXII

Привычка усладила горе,\\
Не отразимое ничем;\\
Открытие большое вскоре\\
Ее утешило совсем:\\
Она меж делом и досугом\\
Открыла тайну, как супругом\\
Самодержавно управлять,\\
И все тогда пошло на стать.\\
Она езжала по работам,\\
Солила на зиму грибы,\\
Вела расходы, брила лбы,\\
Ходила в баню по субботам,\\
Служанок била осердясь —\\
Все это мужа не спросясь.\\

XXXIII

Бывало, писывала кровью\\
Она в альбомы нежных дев,\\
Звала Полиною Прасковью\\
И говорила нараспев,\\
Корсет носила очень узкий,\\
И русский Н как N французский\\
Произносить умела в нос;\\
Но скоро все перевелось:\\
Корсет, альбом, княжну Алину,\\
Стишков чувствительных тетрадь\\
Она забыла: стала звать\\
Акулькой прежнюю Селину\\
И обновила наконец\\
На вате шлафор и чепец.\\

XXXIV

Но муж любил ее сердечно,\\
В ее затеи не входил,\\
Во всем ей веровал беспечно,\\
А сам в халате ел и пил;\\
Покойно жизнь его катилась;\\
Под вечер иногда сходилась\\
Соседей добрая семья,\\
Нецеремонные друзья,\\
И потужить, и позлословить,\\
И посмеяться кой о чем.\\
Проходит время; между тем\\
Прикажут Ольге чай готовить,\\
Там ужин, там и спать пора,\\
И гости едут со двора.\\

XXXV

Они хранили в жизни мирной\\
Привычки милой старины;\\
У них на масленице жирной\\
Водились русские блины;\\
Два раза в год они говели;\\
Любили круглые качели,\\
Подблюдны песни, хоровод;\\
В день Троицын, когда народ,\\
Зевая, слушает молебен,\\
Умильно на пучок зари\\
Они роняли слезки три;\\
Им квас как воздух был потребен,\\
И за столом у них гостям\\
Носили блюды по чинам.\\

XXXVI

И так они старели оба.\\
И отворились наконец\\
Перед супругом двери гроба,\
И новый он приял венец.\\
Он умер в час перед обедом,\\
Оплаканный своим соседом,\\
Детьми и верною женой\\
Чистосердечней, чем иной.\\
Он был простой и добрый барин,\\
И там, где прах его лежит,\\
Надгробный памятник гласит:\\
Смиренный грешник, Дмитрий Ларин,\\
Господний раб и бригадир,\\
Под камнем сим вкушает мир.\\

XXXVII

Своим пенатам возвращенный,\\
Владимир Ленский посетил\\
Соседа памятник смиренный,\\
И вздох он пеплу посвятил;\\
И долго сердцу грустно было.\\
«Рооr Yorick! — молвил он уныло. —\\
Он на руках меня держал.\\
Как часто в детстве я играл\\
Его Очаковской медалью!\\
Он Ольгу прочил за меня,\\
Он говорил: дождусь ли дня?..»\\
И, полный искренней печалью,\\
Владимир тут же начертал\\
Ему надгробный мадригал.\\

XXXVIII

И там же надписью печальной\\
Отца и матери, в слезах,\\
Почтил он прах патриархальный…\\
Увы! на жизненных браздах\\
Мгновенной жатвой поколенья,\\
По тайной воле провиденья,\\
Восходят, зреют и падут;\\
Другие им вослед идут…\\
Так наше ветреное племя\\
Растет, волнуется, кипит\\
И к гробу прадедов теснит.\\
Придет, придет и наше время,\\
И наши внуки в добрый час\\
Из мира вытеснят и нас!\\

XXXIX

Покамест упивайтесь ею,\\
Сей легкой жизнию, друзья!\\
Ее ничтожность разумею\\
И мало к ней привязан я;\\
Для призраков закрыл я вежды;\\
Но отдаленные надежды\\
Тревожат сердце иногда:\\
Без неприметного следа\\
Мне было б грустно мир оставить.\\
Живу, пишу не для похвал;\\
Но я бы, кажется, желал\\
Печальный жребий свой прославить,\\
Чтоб обо мне, как верный друг,\\
Напомнил хоть единый звук.\\

XL

И чье-нибудь он сердце тронет;\\
И, сохраненная судьбой,\\
Быть может, в Лете не потонет\\
Строфа, слагаемая мной;\\
Быть может (лестная надежда!),\\
Укажет будущий невежда\\
На мой прославленный портрет\\
И молвит: то-то был поэт!\\
Прими ж мои благодаренья,\\
Поклонник мирных аонид,\\
О ты, чья память сохранит\\
Мои летучие творенья,\\
Чья благосклонная рука\\
Потреплет лавры старика!\\

%\chapter{Capitolo 5} % Capitolo di Metodologia
\label{cap5} % Etichetta per riferimenti incrociati

\textit{О, не знай сих страшных снов\\
Ты, моя Светлана!\\
Жуковский
}\\

I
В тот год осенняя погода\\
Стояла долго на дворе,\\
Зимы ждала, ждала природа.\\
Снег выпал только в январе\\
На третье в ночь. Проснувшись рано,\\
В окно увидела Татьяна\\
Поутру побелевший двор,\\
Куртины, кровли и забор,\\
На стеклах легкие узоры,\\
Деревья в зимнем серебре,\\
Сорок веселых на дворе\\
И мягко устланные горы\\
Зимы блистательным ковром.\\
Все ярко, все бело кругом.\\

II

Зима!.. Крестья'нин, торжествуя,\\
На дровнях обновляет путь;\\
Его лошадка, снег почуя,\\
Плетется рысью как-нибудь;\\
Бразды пушистые взрывая,\\
Летит кибитка удалая;\\
Ямщи'к сидит на облучке\\
В тулупе, в красном кушаке.\\
Вот бегает дворовый мальчик,\\
В салазки жучку посадив,\\
Себя в коня преобразив;\\
Шалун уж заморозил пальчик:\\
Ему и больно и смешно,\\
А мать грозит ему в окно…\\

III

Но, может быть, такого рода\\
Картины вас не привлекут:\\
Все это низкая природа;\\
Изящного не много тут.\\
Согретый вдохновенья богом,\\
Другой поэт роскошным слогом\\
Живописал нам первый снег\\
И все оттенки зимних нег;\\
Он вас пленит, я в том уверен,\\
Рисуя в пламенных стихах\\
Прогулки тайные в санях;\\
Но я бороться не намерен\\
Ни с ним покамест, ни с тобой,\\
Певец финляндки молодой!\\

IV

Татьяна (русская душою,\\
Сама не зная почему)\\
С ее холодною красою\\
Любила русскую зиму,\\
На солнце иний в день морозный,\\
И сани, и зарею поздной\\
Сиянье розовых снегов,\\
И мглу крещенских вечеров.\\
По старине торжествовали\\
В их доме эти вечера:\\
Служанки со всего двора\\
Про барышень своих гадали\\
И им сулили каждый год\\
Мужьев военных и поход.\\

V

Татьяна верила преданьям\marginnote{leggende}\\
Простонародной старины,\\
И снам, и карточным гаданьям,\\
И предсказаниям луны.\\
Ее тревожили приметы;\\
Таинственно ей все предметы\\
Провозглашали что-нибудь,\\
Предчувствия теснили грудь.\\
Жеманный кот, на печке сидя,\\
Мурлыча, лапкой рыльце мыл:\\
То несомненный знак ей был,\\
Что едут гости. Вдруг увидя\\
Младой двурогий лик луны\\
На небе с левой стороны,\\

VI

Она дрожала и бледнела.\\
Когда ж падучая звезда\\
По небу темному летела\\
И рассыпалася, — тогда\\
В смятенье Таня торопилась,\\
Пока звезда еще катилась,\\
Желанье сердца ей шепнуть.\\
Когда случалось где-нибудь\\
Ей встретить черного монаха\\
Иль быстрый заяц меж полей\\
Перебегал дорогу ей,\\
Не зная, что начать со страха,\\
Предчувствий горестных полна,\\
Ждала несчастья уж она.\\

VII

Что ж? Тайну прелесть находила\\
И в самом ужасе она:\\
Так нас природа сотворила,\\
К противуречию склонна.\\
Настали святки. То-то радость!\\
Гадает ветреная младость,\\
Которой ничего не жаль,\\
Перед которой жизни даль\\
Лежит светла, необозрима;\narginnote{infinito}\\
Гадает старость сквозь очки\\
У гробовой своей доски,\\
Все потеряв невозвратимо;\\
И все равно: надежда им\\
Лжет детским лепетом своим.\\

VIII

Татьяна любопытным взором\\
На воск потопленный глядит:\\
Он чудно вылитым узором\\
Ей что-то чудное гласит;\\
Из блюда, полного водою,\\
Выходят кольцы чередою;\\
И вынулось колечко ей\\
Под песенку старинных дней:\\
«Там мужички-то все богаты,\\
Гребут лопатой серебро;\\
Кому поем, тому добро\\
И слава!» Но сулит утраты\\
Сей песни жалостный напев;\\
Милей кошурка сердцу дев.\\

IX

Морозна ночь, все небо ясно;\\
Светил небесных дивный хор\\
Течет так тихо, так согласно…\\
Татьяна на широкой двор\\
В открытом платьице выходит,\\
На месяц зеркало наводит;\\
Но в темном зеркале одна\\
Дрожит печальная луна…\\
Чу… снег хрустит… прохожий; дева\\
К нему на цыпочках летит,\\
И голосок ее звучит\\
Нежней свирельного напева:\marginnote'{Народная музыка del flauto}\\
Как ваше имя? Смотрит он\\
И отвечает: Агафон.\\

X

Татьяна, по совету няни\\
Сбираясь ночью ворожить,\\
Тихонько приказала в бане\\
На два прибора стол накрыть;\\
Но стало страшно вдруг Татьяне…\\
И я — при мысли о Светлане\\
Мне стало страшно — так и быть…\\
С Татьяной нам не ворожить.\\
Татьяна поясок шелковый\\
Сняла, разделась и в постель\\
Легла. Над нею вьется Лель,\\
А под подушкою пуховой\\
Девичье зеркало лежит.\\
Утихло все. Татьяна спит.\\

XI

И снится чудный сон Татьяне.\\
Ей снится, будто бы она\\
Идет по снеговой поляне,\\
Печальной мглой окружена;\\
В сугробах снежных перед нею\\
Шумит, клубит волной своею\\
Кипучий, темный и седой\\
Поток, не скованный зимой;\\
Две жердочки, склеены льдиной,\\
Дрожащий, гибельный мосток,\\
Положены через поток;\\
И пред шумящею пучиной,\\
Недоумения полна,\\
Остановилася она.\\

XII

Как на досадную\marginnote{досадное недоразумение — spiacevole malinteso} разлуку,\\
Татьяна ропщет\\marginnote{выражать недовольство) mormorare роптать на начальство} на ручей;\\
Не видит никого, кто руку\\
С той стороны подал бы ей;\\
Но вдруг сугроб зашевелился.\\
И кто ж из-под него явился?\\
Большой, взъерошенный медведь;\\
Татьяна ах! а он реветь,\\
И лапу с острыми когтями\\
Ей протянул; она скрепясь\\
Дрожащей ручкой оперлась\\
И боязливыми шагами\\
Перебралась через ручей;\\
Пошла — и что ж? медведь за ней!\\

XIII

Она, взглянуть назад не смея,\\
Поспешный ускоряет шаг;\\
Но от косматого лакея\\
Не может убежать никак;\\
Кряхтя, валит медведь несносный;\\
Пред ними лес; недвижны сосны\\
В своей нахмуренной красе;\\
Отягчены их ветви все\\
Клоками снега; сквозь вершины\\
Осин, берез и лип нагих\\
Сияет луч светил ночных;\\
Дороги нет; кусты, стремнины\\
Метелью все занесены,\\
Глубоко в снег погружены.\\

XIV

Татьяна в лес; медведь за нею;\\
Снег рыхлый по колено ей;\\
То длинный сук ее за шею\\
Зацепит вдруг, то из ушей\\
Златые серьги вырвет силой;\\
То в хрупком снеге с ножки милой\\
Увязнет мокрый башмачок;\\
То выронит она платок;\\
Поднять ей некогда; боится,\\
Медведя слышит за собой,\\
И даже трепетной рукой\\
Одежды край поднять стыдится;\\
Она бежит, он все вослед,\\
И сил уже бежать ей нет.\\

XV

Упала в снег; медведь проворно\\
Ее хватает и несет;\\
Она бесчувственно-покорна,\\
Не шевельнется, не дохнет;\\
Он мчит ее лесной дорогой;\\
Вдруг меж дерев шалаш убогой;\\
Кругом все глушь; отвсюду он\\
Пустынным снегом занесен,\\
И ярко светится окошко,\\
И в шалаше и крик и шум;\\
Медведь промолвил: «Здесь мой кум:\\
Погрейся у него немножко!»\\
И в сени прямо он идет\\
И на порог ее кладет.\\

XVI

Опомнилась, глядит Татьяна:\\
Медведя нет; она в сенях;\\
\emph{За дверью крик и звон стакана,\\
Как на больших похоронах;}\\
Не видя тут ни капли толку,\\
Глядит она тихонько в щелку,\\
И что же видит?.. за столом\\
Сидят чудовища кругом:\\
Один в рогах с собачьей мордой,\\
Другой с петушьей головой,\\
Здесь ведьма с козьей бородой,\\
Тут остов\marginnote{scheletro} чо'порный и гордый,\\
Там карла с хвостиком, а вот\\
Полужуравль и полукот.\\

XVII

Еще страшней, еще чуднее:\\
Вот рак верхом на пауке,\
Вот череп на гусиной шее\\
Вертится в красном колпаке,\\
Вот мельница вприсядку пляшет\\
И крыльями трещит и машет;\\
Лай, хо'хот, пенье, свист и хлоп,\\
Людская молвь и конской топ!\\
Но что подумала Татьяна,\\
Когда узнала меж гостей\\
Того, кто мил и страшен ей,\\
Героя нашего романа!\\
Онегин за столом сидит\\
И в дверь украдкою глядит.\\

XVIII

Он знак подаст — и все хлопочут;\\
Он пьет — все пьют и все кричат;\\
Он засмеется — все хохочут;\\
Нахмурит брови — все молчат;\\
Он там хозяин, это ясно:\\
И Тане уж не так ужасно,\\
И, любопытная, теперь\\
Немного растворила дверь…\\
Вдруг ветер дунул, загашая\\
Огонь светильников ночных;\\
Смутилась шайка домовых;\\
Онегин, взорами сверкая,\\
Из-за стола, гремя, встает;\\
Все встали: он к дверям идет.\\

XIX

И страшно ей; и торопливо\\
Татьяна силится бежать:\\
Нельзя никак; нетерпеливо\\
Метаясь, хочет закричать:\\
Не может; дверь толкнул Евгений:\\
И взорам адских привидений\\
Явилась дева; ярый смех\\
Раздался дико; очи всех,\\
Копыты, хоботы кривые,\\
Хвосты хохлатые, клыки,\\
Усы, кровавы языки,\\
Рога и пальцы костяные,\\
Все указует на нее,\\
И все кричат: мое! мое!\\

XX

Мое! — сказал Евгений грозно,
И шайка вся сокрылась вдруг;
Осталася во тьме морозной
Младая дева с ним сам-друг;
Онегин тихо увлекает
Татьяну в угол и слагает
Ее на шаткую скамью
И клонит голову свою
К ней на плечо; вдруг Ольга входит,
За нею Ленский; свет блеснул;
Онегин руку замахнул,
И дико он очами бродит,
И незваных гостей бранит;
Татьяна чуть жива лежит.

XXI

Спор громче, громче; вдруг Евгений
Хватает длинный нож, и вмиг
Повержен Ленский; страшно тени
Сгустились; нестерпимый крик
Раздался… хижина шатнулась…
И Таня в ужасе проснулась…
Глядит, уж в комнате светло;
В окне сквозь мерзлое стекло
Зари багряный луч играет;
Дверь отворилась. Ольга к ней,
Авроры северной алей
И легче ласточки, влетает;
«Ну, говорит, скажи ж ты мне,
Кого ты видела во сне?»
XXII
Но та, сестры не замечая,
В постеле с книгою лежит,
За листом лист перебирая,
И ничего не говорит.
Хоть не являла книга эта
Ни сладких вымыслов поэта,
Ни мудрых истин, ни картин,
Но ни Виргилий, ни Расин,
Ни Скотт, ни Байрон, ни Сенека,
Ни даже Дамских Мод Журнал
Так никого не занимал:
То был, друзья, Мартын Задека,
Глава халдейских мудрецов,
Гадатель, толкователь снов.
XXIII
Сие глубокое творенье
Завез кочующий купец
Однажды к ним в уединенье
И для Татьяны наконец
Его с разрозненной «Мальвиной»
Он уступил за три с полтиной,
В придачу взяв еще за них
Собранье басен площадных,
Грамматику, две Петриады
Да Мармонтеля третий том.
Мартын Задека стал потом
Любимец Тани… Он отрады
Во всех печалях ей дарит
И безотлучно с нею спит.
XXIV
Ее тревожит сновиденье.
Не зная, как его понять,
Мечтанья страшного значенье
Татьяна хочет отыскать.
Татьяна в оглавленье кратком
Находит азбучным порядком
Слова: бор, буря, ведьма, ель,
Еж, мрак, мосток, медведь, метель
И прочая. Ее сомнений
Мартын Задека не решит;
Но сон зловещий ей сулит
Печальных много приключений.
Дней несколько она потом
Все беспокоилась о том.
XXV
Но вот багряною рукою
Заря от утренних долин
Выводит с солнцем за собою
Веселый праздник именин.
С утра дом Лариных гостями
Весь полон; целыми семьями
Соседи съехались в возках,
В кибитках, в бричках и в санях.
В передней толкотня, тревога;
В гостиной встреча новых лиц,
Лай мосек, чмоканье девиц,
Шум, хохот, давка у порога,
Поклоны, шарканье гостей,
Кормилиц крик и плач детей.
XXVI
С своей супругою дородной
Приехал толстый Пустяков;
Гвоздин, хозяин превосходный,
Владелец нищих мужиков;
Скотинины, чета седая,
С детьми всех возрастов, считая
От тридцати до двух годов;
Уездный франтик Петушков,
Мой брат двоюродный, Буянов,
В пуху, в картузе с козырьком
(Как вам, конечно, он знаком),
И отставной советник Флянов,
Тяжелый сплетник, старый плут,
Обжора, взяточник и шут.
XXVII
С семьей Панфила Харликова
Приехал и мосье Трике,
Остряк, недавно из Тамбова,
В очках и в рыжем парике.
Как истинный француз, в кармане
Трике привез куплет Татьяне
На голос, знаемый детьми:
Reveillez vous, belle endormie.
Меж ветхих песен альманаха
Был напечатан сей куплет;
Трике, догадливый поэт,
Его на свет явил из праха,
И смело вместо belle Nina
Поставил belle Tatiana.
XXVIII
И вот из ближнего посада
Созревших барышень кумир,
Уездных матушек отрада,
Приехал ротный командир;
Вошел… Ах, новость, да какая!
Музыка будет полковая!
Полковник сам ее послал.
Какая радость: будет бал!
Девчонки прыгают заране;
Но кушать подали. Четой
Идут за стол рука с рукой.
Теснятся барышни к Татьяне;
Мужчины против; и, крестясь,
Толпа жужжит, за стол садясь.
XXIX
На миг умолкли разговоры;
Уста жуют. Со всех сторон
Гремят тарелки и приборы
Да рюмок раздается звон.
Но вскоре гости понемногу
Подъемлют общую тревогу.
Никто не слушает, кричат,
Смеются, спорят и пищат.
Вдруг двери настежь. Ленский входит,
И с ним Онегин. «Ах, творец! —
Кричит хозяйка: — наконец!»
Теснятся гости, всяк отводит
Приборы, стулья поскорей;
Зовут, сажают двух друзей.
XXX
Сажают прямо против Тани,
И, утренней луны бледней
И трепетней гонимой лани,
Она темнеющих очей
Не подымает: пышет бурно
В ней страстный жар; ей душно, дурно;
Она приветствий двух друзей
Не слышит, слезы из очей
Хотят уж капать; уж готова
Бедняжка в обморок упасть;
Но воля и рассудка власть
Превозмогли. Она два слова
Сквозь зубы молвила тишком
И усидела за столом.
XXXI
Траги-нервических явлений,
Девичьих обмороков, слез
Давно терпеть не мог Евгений:
Довольно их он перенес.
Чудак, попав на пир огромный,
Уж был сердит. Но девы томной
Заметя трепетный порыв,
С досады взоры опустив,
Надулся он и, негодуя,
Поклялся Ленского взбесить
И уж порядком отомстить.
Теперь, заране торжествуя,
Он стал чертить в душе своей
Карикатуры всех гостей.
XXXII
Конечно, не один Евгений
Смятенье Тани видеть мог;
Но целью взоров и суждений
В то время жирный был пирог
(К несчастию, пересоленный);
Да вот в бутылке засмоленной,
Между жарким и блан-манже,
Цимлянское несут уже;
За ним строй рюмок узких, длинных,
Подобно талии твоей,
Зизи, кристалл души моей,
Предмет стихов моих невинных,
Любви приманчивый фиал,
Ты, от кого я пьян бывал!
XXXIII
Освободясь от пробки влажной,
Бутылка хлопнула; вино
Шипит; и вот с осанкой важной,
Куплетом мучимый давно,
Трике встает; пред ним собранье
Хранит глубокое молчанье.
Татьяна чуть жива; Трике,
К ней обратясь с листком в руке,
Запел, фальшивя. Плески, клики
Его приветствуют. Она
Певцу присесть принуждена;
Поэт же скромный, хоть великий,
Ее здоровье первый пьет
И ей куплет передает.
XXXIV
Пошли приветы, поздравленья;
Татьяна всех благодарит.
Когда же дело до Евгенья
Дошло, то девы томный вид,
Ее смущение, усталость
В его душе родили жалость:
Он молча поклонился ей,
Но как-то взор его очей
Был чудно нежен. Оттого ли,
Что он и вправду тронут был,
Иль он, кокетствуя, шалил,
Невольно ль, иль из доброй воли,
Но взор сей нежность изъявил:
Он сердце Тани оживил.
XXXV
Гремят отдвинутые стулья;
Толпа в гостиную валит:
Так пчел из лакомого улья
На ниву шумный рой летит.
Довольный праздничным обедом,
Сосед сопит перед соседом;
Подсели дамы к камельку;
Девицы шепчут в уголку;
Столы зеленые раскрыты:
Зовут задорных игроков
Бостон и ломбер стариков,
И вист, доныне знаменитый,
Однообразная семья,
Все жадной скуки сыновья.
XXXVI
Уж восемь робертов сыграли
Герои виста; восемь раз
Они места переменяли;
И чай несут. Люблю я час
Определять обедом, чаем
И ужином. Мы время знаем
В деревне без больших сует:
Желудок — верный наш брегет;
И кстати я замечу в скобках,
Что речь веду в моих строфах
Я столь же часто о пирах,
О разных кушаньях и пробках,
Как ты, божественный Омир,
Ты, тридцати веков кумир!
XXXVII. XXXVIII. XXXIX
Но чай несут; девицы чинно
Едва за блюдички взялись,
Вдруг из-за двери в зале длинной
Фагот и флейта раздались.
Обрадован музыки громом,
Оставя чашку чаю с ромом,
Парис окружных городков,
Подходит к Ольге Петушков,
К Татьяне Ленский; Харликову,
Невесту переспелых лет,
Берет тамбовский мой поэт,
Умчал Буянов Пустякову,
И в залу высыпали все.
И бал блестит во всей красе.
ХL
В начале моего романа
(Смотрите первую тетрадь)
Хотелось вроде мне Альбана
Бал петербургский описать;
Но, развлечен пустым мечтаньем,
Я занялся воспоминаньем
О ножках мне знакомых дам.
По вашим узеньким следам,
О ножки, полно заблуждаться!
С изменой юности моей
Пора мне сделаться умней,
В делах и в слоге поправляться,
И эту пятую тетрадь
От отступлений очищать.
ХLI
Однообразный и безумный,
Как вихорь жизни молодой,
Кружится вальса вихорь шумный;
Чета мелькает за четой.
К минуте мщенья приближаясь,
Онегин, втайне усмехаясь,
Подходит к Ольге. Быстро с ней
Вертится около гостей,
Потом на стул ее сажает,
Заводит речь о том о сем;
Спустя минуты две потом
Вновь с нею вальс он продолжает;
Все в изумленье. Ленский сам
Не верит собственным глазам.
ХLII
Мазурка раздалась. Бывало,
Когда гремел мазурки гром,
В огромной зале все дрожало,
Паркет трещал под каблуком,
Тряслися, дребезжали рамы;
Теперь не то: и мы, как дамы,
Скользим по лаковым доскам.
Но в городах, по деревням
Еще мазурка сохранила
Первоначальные красы:
Припрыжки, каблуки, усы
Все те же: их не изменила
Лихая мода, наш тиран,
Недуг новейших россиян.
XLIII. XLIV
Буянов, братец мой задорный,
К герою нашему подвел
Татьяну с Ольгою; проворно
Онегин с Ольгою пошел;
Ведет ее, скользя небрежно,
И, наклонясь, ей шепчет нежно
Какой-то пошлый мадригал,
И руку жмет — и запылал
В ее лице самолюбивом
Румянец ярче. Ленский мой
Все видел: вспыхнул, сам не свой;
В негодовании ревнивом
Поэт конца мазурки ждет
И в котильон ее зовет.
ХLV
Но ей нельзя. Нельзя? Но что же?
Да Ольга слово уж дала
Онегину. О боже, боже!
Что слышит он? Она могла…
Возможно ль? Чуть лишь из пеленок,
Кокетка, ветреный ребенок!
Уж хитрость ведает она,
Уж изменять научена!
Не в силах Ленский снесть удара;
Проказы женские кляня,
Выходит, требует коня
И скачет. Пистолетов пара,
Две пули — больше ничего —
Вдруг разрешат судьбу его.

%\include{capitolo6}
\chapter{Capitolo 7}\label{section:cap7}% Etichetta per riferimenti incrociati

\begin{verse}

\textit{Москва, России дочь любима,\\
Где равную тебе сыскать?\\
Дмитриев.\\
Как не любить родной Москвы?\\
Баратынский.\\
Гоненье на Москву! что значит видеть свет!\\
Где ж лучше?\\
Где нас нет.\\
Грибоедов.}

I

Гонимы вешними лучами,\\
С окрестных гор уже снега\\
Сбежали мутными ручьями\\
На потопленные луга.\\
Улыбкой ясною природа\\
Сквозь сон встречает утро года;\\
Синея блещут небеса.\\
Еще прозрачные, леса\\
Как будто пухом зеленеют.\\
Пчела за данью полевой\\
Летит из кельи восковой.\\
Долины сохнут и пестреют;\\
Стада шумят, и соловей\\
Уж пел в безмолвии ночей.

II

Как грустно мне твое явленье,\\
Весна, весна! пора любви!\\
Какое томное волненье\\
В моей душе, в моей крови!\\
С каким тяжелым умиленьем\\
Я наслаждаюсь дуновеньем\\
В лицо мне веющей весны\\
На лоне сельской тишины!\\
Или мне чуждо наслажденье,\\
И все, что радует, живит,\\
Все, что ликует и блестит\\
Наводит скуку и томленье\\
На душу мертвую давно\\
И все ей кажется темно?

III

Или, не радуясь возврату\\
Погибших осенью листов,\\
Мы помним горькую утрату,\\
Внимая новый шум лесов;\\
Или с природой оживленной\\
Сближаем думою смущенной\\
Мы увяданье наших лет,\\
Которым возрожденья нет?\\
Быть может, в мысли нам приходит\\
Средь поэтического сна\\
Иная, старая весна\\
И в трепет сердце нам приводит\\
Мечтой о дальной стороне,\\
О чудной ночи, о луне…

IV

Вот время: добрые ленивцы,\\
Эпикурейцы-мудрецы,\\
Вы, равнодушные счастливцы,\\
Вы, школы Левшина птенцы,\\
Вы, деревенские Приамы,\\
И вы, чувствительные дамы,\\
Весна в деревню вас зовет,\\
Пора тепла, цветов, работ,\\
Пора гуляний вдохновенных\\
И соблазнительных ночей.\\
В поля, друзья! скорей, скорей,\\
В каретах, тяжко нагруженных,\\
На долгих иль на почтовых\\
Тянитесь из застав градских.

V

И вы, читатель благосклонный,\\
В своей коляске выписной\\
Оставьте град неугомонный,\\
Где веселились вы зимой;\\
С моею музой своенравной\\
Пойдемте слушать шум дубравный\\
Над безыменною рекой\\
В деревне, где Евгений мой,\\
Отшельник праздный и унылый,\\
Еще недавно жил зимой\\
В соседстве Тани молодой,\\
Моей мечтательницы милой,\\
Но где его теперь уж нет…\\
Где грустный он оставил след.

VI

Меж гор, лежащих полукругом,\\
Пойдем туда, где ручеек,\\
Виясь, бежит зеленым лугом\\
К реке сквозь липовый лесок.\\
Там соловей, весны любовник,\\
Всю ночь поет; цветет шиповник,\\
И слышен говор ключевой,\\
Там виден камень гробовой\\
В тени двух сосен устарелых.\\
Пришельцу надпись говорит:\\
«Владимир Ленский здесь лежит,\\
Погибший рано смертью смелых,\\
В такой-то год, таких-то лет.\\
Покойся, юноша-поэт!»

VII

На ветви сосны преклоненной,\\
Бывало, ранний ветерок\\
Над этой урною смиренной\\
Качал таинственный венок.\\
Бывало, в поздние досуги\\
Сюда ходили две подруги,\\
И на могиле при луне,\\
Обнявшись, плакали оне.\\
Но ныне… памятник унылый\\
Забыт. К нему привычный след\\
Заглох. Венка на ветви нет;\\
Один, под ним, седой и хилый\\
Пастух по-прежнему поет\\
И обувь бедную плетет.

VIII. IX. X

Мой бедный Ленский! изнывая,\\
Не долго плакала она.\\
Увы! невеста молодая\\
Своей печали неверна.\\
Другой увлек ее вниманье,\\
Другой успел ее страданье\\
Любовной лестью усыпить,\\
Улан умел ее пленить,\\
Улан любим ее душою…\\
И вот уж с ним пред алтарем\\
Она стыдливо под венцом\\
Стоит с поникшей головою,\\
С огнем в потупленных очах,\\
С улыбкой легкой на устах.

XI

Мой бедный Ленский! за могилой\\
В пределах вечности глухой\\
Смутился ли, певец унылый,\\
Измены вестью роковой,\\
Или над Летой усыпленный\\
Поэт, бесчувствием блаженный,\\
Уж не смущается ничем,\\
И мир ему закрыт и нем?..\\
Так! равнодушное забвенье\\
За гробом ожидает нас.\\
Врагов, друзей, любовниц глас\\
Вдруг молкнет. Про одно именье\\
Наследников сердитый хор\\
Заводит непристойный спор.

XII

И скоро звонкий голос Оли\\
В семействе Лариных умолк.\\
Улан, своей невольник доли,\\
Был должен ехать с нею в полк.\\
Слезами горько обливаясь,\\
Старушка, с дочерью прощаясь,\\
Казалось, чуть жива была,\\
Но Таня плакать не могла;\\
Лишь смертной бледностью покрылось\\
Ее печальное лицо.\\
Когда все вышли на крыльцо,\\
И все, прощаясь, суетилось\\
Вокруг кареты молодых,\\
Татьяна проводила их.

XIII

И долго, будто сквозь тумана,\\
Она глядела им вослед…\\
И вот одна, одна Татьяна!\\
Увы! подруга стольких лет,\\
Ее голубка молодая,\\
Ее наперсница родная,\\
Судьбою вдаль занесена,\\
С ней навсегда разлучена.\\
Как тень она без цели бродит,\\
То смотрит в опустелый сад…\\
Нигде, ни в чем ей нет отрад,\\
И облегченья не находит\\
Она подавленным слезам,\\
И сердце рвется пополам.

XIV

И в одиночестве жестоком\\
Сильнее страсть ее горит,\\
И об Онегине далеком\\
Ей сердце громче говорит.\\
Она его не будет видеть;\\
Она должна в нем ненавидеть\\
Убийцу брата своего;\\
Поэт погиб… но уж его\\
Никто не помнит, уж другому\\
Его невеста отдалась.\\
Поэта память пронеслась\\
Как дым по небу голубому,\\
О нем два сердца, может быть,\\
Еще грустят… На что грустить?..

XV

Был вечер. Небо меркло. Воды\\
Струились тихо. Жук жужжал.\\
Уж расходились хороводы;\\
Уж за рекой, дымясь, пылал\\
Огонь рыбачий. В поле чистом,\\
Луны при свете серебристом,\\
В свои мечты погружена,\\
Татьяна долго шла одна.\\
Шла, шла. И вдруг перед собою\\
С холма господский видит дом,\\
Селенье, рощу под холмом\\
И сад над светлою рекою.\\
Она глядит — и сердце в ней\\
Забилось чаще и сильней.

XVI

Ее сомнения смущают:\\
«Пойду ль вперед, пойду ль назад?..\\
Его здесь нет. Меня не знают…\\
Взгляну на дом, на этот сад».\\
И вот с холма Татьяна сходит,\\
Едва дыша; кругом обводит\\
Недоуменья полный взор…\\
И входит на пустынный двор.\\
К ней, лая, кинулись собаки.\\
На крик испуганный ея\\
Ребят дворовая семья\\
Сбежалась шумно. Не без драки\\
Мальчишки разогнали псов,\\
Взяв барышню под свой покров.

XVII

«Увидеть барской дом нельзя ли?»\\
Спросила Таня. Поскорей\\
К Анисье дети побежали\\
У ней ключи взять от сеней;\\
Анисья тотчас к ней явилась,\\
И дверь пред ними отворилась,\\
И Таня входит в дом пустой,\\
Где жил недавно наш герой.\\
Она глядит: забытый в зале\\
Кий на бильярде отдыхал,\\
На смятом канапе лежал\\
Манежный хлыстик. Таня дале;\\
Старушка ей: «А вот камин;\\
Здесь барин сиживал один.

XVIII

Здесь с ним обедывал зимою\\
Покойный Ленский, наш сосед.\\
Сюда пожалуйте, за мною.\\
Вот это барский кабинет;\\
Здесь почивал он, кофей кушал,\\
Приказчика доклады слушал\\
И книжку поутру читал…\\
И старый барин здесь живал;\\
Со мной, бывало, в воскресенье,\\
Здесь под окном, надев очки,\\
Играть изволил в дурачки.\\
Дай бог душе его спасенье,\\
А косточкам его покой\\
В могиле, в мать-земле сырой!»

XIX

Татьяна взором умиленным\\
Вокруг себя на все глядит,\\
И все ей кажется бесценным,\\
Все душу томную живит\\
Полумучительной отрадой:\\
И стол с померкшею лампадой,\\
И груда книг, и под окном\\
Кровать, покрытая ковром,\\
И вид в окно сквозь сумрак лунный,\\
И этот бледный полусвет,\\
И лорда Байрона портрет,\\
И столбик с куклою чугунной\\
Под шляпой с пасмурным челом,\\
С руками, сжатыми крестом.

XX

Татьяна долго в келье модной\\
Как очарована стоит.\\
Но поздно. Ветер встал холодный.\\
Темно в долине. Роща спит\\
Над отуманенной рекою;\\
Луна сокрылась за горою,\\
И пилигримке молодой\\
Пора, давно пора домой.\\
И Таня, скрыв свое волненье,\\
Не без того, чтоб не вздохнуть,\\
Пускается в обратный путь.\\
Но прежде просит позволенья\\
Пустынный замок навещать,\\
Чтоб книжки здесь одной читать.

XXI

Татьяна с ключницей простилась\\
За воротами. Через день\\
Уж утром рано вновь явилась\\
Она в оставленную сень.\\
И в молчаливом кабинете,\\
Забыв на время все на свете,\\
Осталась наконец одна,\\
И долго плакала она.\\
Потом за книги принялася.\\
Сперва ей было не до них,\\
Но показался выбор их\\
Ей странен. Чтенью предалася\\
Татьяна жадною душой;\\
И ей открылся мир иной.

XXII

Хотя мы знаем, что Евгений\\
Издавна чтенье разлюбил,\\
Однако ж несколько творений\\
Он из опалы исключил:\\
Певца Гяура и Жуана\\
Да с ним еще два-три романа,\\
В которых отразился век\\
И современный человек\\
Изображен довольно верно\\
С его безнравственной душой,\\
Себялюбивой и сухой,\\
Мечтанью преданной безмерно,\\
С его озлобленным умом,\\
Кипящим в действии пустом.

XXIII

Хранили многие страницы\\
Отметку резкую ногтей;\\
Глаза внимательной девицы\\
Устремлены на них живей.\\
Татьяна видит с трепетаньем,\\
Какою мыслью, замечаньем\\
Бывал Онегин поражен,\\
В чем молча соглашался он.\\
На их полях она встречает\\
Черты его карандаша.\\
Везде Онегина душа\\
Себя невольно выражает\\
То кратким словом, то крестом,\\
То вопросительным крючком.

XXIV

И начинает понемногу\\
Моя Татьяна понимать\\
Теперь яснее — слава богу —\\
Того, по ком она вздыхать\\
Осуждена судьбою властной:\\
Чудак печальный и опасный,\\
Созданье ада иль небес,\\
Сей ангел, сей надменный бес,\\
Что ж он? Ужели подражанье,\\
Ничтожный призрак, иль еще\\
Москвич в Гарольдовом плаще,\\
Чужих причуд истолкованье,\\
Слов модных полный лексикон?..\\
Уж не пародия ли он?

XXV

Ужель загадку разрешила?\\
Ужели слово найдено?\\
Часы бегут; она забыла,\\
Что дома ждут ее давно,\\
Где собралися два соседа\\
И где об ней идет беседа.\\
— Как быть? Татьяна не дитя,\\
Старушка молвила кряхтя.\\
Ведь Оленька ее моложе.\\
Пристроить девушку, ей-ей,\\
Пора; а что мне делать с ней?\\
Всем наотрез одно и то же:\\
Нейду. И все грустит она,\\
Да бродит по лесам одна.

XXVI

«Не влюблена ль она?» — В кого же?\\
Буянов сватался: отказ.\\
Ивану Петушкову — тоже.\\
Гусар Пыхтин гостил у нас;\\
Уж как он Танею прельщался,\\
Как мелким бесом рассыпался!\\
Я думала: пойдет авось;\\
Куда! и снова дело врозь.\\
«Что ж, матушка? за чем же стало?\\
В Москву, на ярманку невест!\\
Там, слышно, много праздных мест».\\
— Ох, мой отец! доходу мало.\\
«Довольно для одной зимы,\\
Не то уж дам хоть я взаймы».

XXVII

Старушка очень полюбила\\
Совет разумный и благой;\\
Сочлась — и тут же положила\\
В Москву отправиться зимой.\\
И Таня слышит новость эту.\\
На суд взыскательному свету\\
Представить ясные черты\\
Провинциальной простоты,\\
И запоздалые наряды,\\
И запоздалый склад речей;\\
Московских франтов и цирцей\\
Привлечь насмешливые взгляды!..\\
О страх! нет, лучше и верней\\
В глуши лесов остаться ей.

XXVIII

Вставая с первыми лучами,\\
Теперь она в поля спешит\\
И, умиленными очами\\
Их озирая, говорит:\\
«Простите, мирные долины,\\
И вы, знакомых гор вершины,\\
И вы, знакомые леса;\\
Прости, небесная краса,\\
Прости, веселая природа;\\
Меняю милый, тихий свет\\
На шум блистательных сует…\\
Прости ж и ты, моя свобода!\\
Куда, зачем стремлюся я?\\
Что мне сулит судьба моя?»

XXIX

Ее прогулки длятся доле.\\
Теперь то холмик, то ручей\\
Остановляют поневоле\\
Татьяну прелестью своей.\\
Она, как с давними друзьями,\\
С своими рощами, лугами\\
Еще беседовать спешит.\\
Но лето быстрое летит.\\
Настала осень золотая.\\
Природа трепетна, бледна,\\
Как жертва, пышно убрана…\\
Вот север, тучи нагоняя,\\
Дохнул, завыл — и вот сама\\
Идет волшебница зима.

XXX

Пришла, рассыпалась; клоками\\
Повисла на суках дубов;\\
Легла волнистыми коврами\\
Среди полей, вокруг холмов;\\
Брега с недвижною рекою\\
Сравняла пухлой пеленою;\\
Блеснул мороз. И рады мы\\
Проказам матушки зимы.\\
Не радо ей лишь сердце Тани.\\
Нейдет она зиму встречать,\\
Морозной пылью подышать\\
И первым снегом с кровли бани\\
Умыть лицо, плеча и грудь:\\
Татьяне страшен зимний путь.

XXXI

Отъезда день давно просрочен,\\
Проходит и последний срок.\\
Осмотрен, вновь обит, упрочен\\
Забвенью брошенный возок.\\
Обоз обычный, три кибитки\\
Везут домашние пожитки,\\
Кастрюльки, стулья, сундуки,\\
Варенье в банках, тюфяки,\\
Перины, клетки с петухами,\\
Горшки, тазы et cetera,\\
Ну, много всякого добра.\\
И вот в избе между слугами\\
Поднялся шум, прощальный плач:\\
Ведут на двор осьмнадцать кляч,

XXXII

В возок боярский их впрягают,\\
Готовят завтрак повара,\\
Горой кибитки нагружают,\\
Бранятся бабы, кучера.\\
На кляче тощей и косматой\\
Сидит форейтор бородатый,\\
Сбежалась челядь у ворот\\
Прощаться с барами. И вот\\
Уселись, и возок почтенный,\\
Скользя, ползет за ворота.\\
«Простите, мирные места!\\
Прости, приют уединенный!\\
Увижу ль вас?..» И слез ручей\\
У Тани льется из очей.

XXXIII

Когда благому просвещенью\\
Отдвинем более границ,\\
Современем (по расчисленью\\
Философических таблиц,\\
Лет чрез пятьсот) дороги, верно,\\
У нас изменятся безмерно:\\
Шоссе Россию здесь и тут,\\
Соединив, пересекут.\\
Мосты чугунные чрез воды\\
Шагнут широкою дугой,\\
Раздвинем горы, под водой\\
Пророем дерзостные своды,\\
И заведет крещеный мир\\
На каждой станции трактир.

XXXIV

Теперь у нас дороги плохи,\\
Мосты забытые гниют,\\
На станциях клопы да блохи\\
Заснуть минуты не дают;\\
Трактиров нет. В избе холодной\\
Высокопарный, но голодный\\
Для виду прейскурант висит\\
И тщетный дразнит аппетит,\\
Меж тем как сельские циклопы\\
Перед медлительным огнем\\
Российским лечат молотком\\
Изделье легкое Европы,\\
Благословляя колеи\\
И рвы отеческой земли.

XXXV

Зато зимы порой холодной\\
Езда приятна и легка.\\
Как стих без мысли в песне модной,\\
Дорога зимняя гладка.\\
Автомедоны наши бойки,\\
Неутомимы наши тройки,\\
И версты, теша праздный взор,\\
В глазах мелькают, как забор.\\
К несчастью, Ларина тащилась,\\
Боясь прогонов дорогих,\\
Не на почтовых, на своих,\\
И наша дева насладилась\\
Дорожной скукою вполне:\\
Семь суток ехали оне.

XXXVI

Но вот уж близко. Перед ними\\
Уж белокаменной Москвы\\
Как жар, крестами золотыми\\
Горят старинные главы.\\
Ах, братцы! как я был доволен,\\
Когда церквей и колоколен,\\
Садов, чертогов полукруг\\
Открылся предо мною вдруг!\\
Как часто в горестной разлуке,\\
В моей блуждающей судьбе,\\
Москва, я думал о тебе!\\
Москва… как много в этом звуке\\
Для сердца русского слилось!\\
Как много в нем отозвалось!

XXXVII

Вот, окружен своей дубравой,\\
\hyperref[ch:дд]{Петровский Замок}. Мрачно он\\
Недавнею гордится славой.\\
Напрасно ждал Наполеон,\\
Последним счастьем упоенный,\\
Москвы коленопреклоненной\\
С ключами старого Кремля:\\
Нет, не пошла Москва моя\\
К нему с повинной головою.\\
Не праздник, не приемный дар,\\
Она готовила пожар\\
Нетерпеливому герою.\\
Отселе, в думу погружен,\\
Глядел на грозный пламень он.

XXXVIII\ftn{П. показывает две Москве Героическая и боярская}

Прощай, свидетель падшей славы,\\
Петровский замок. Ну! не стой,\\
Пошел! Уже столпы заставы\ftn{граница Москвы}\\
Белеют: вот уж по Тверской\\
Возок несется чрез ухабы.\\
Мелькают мимо будки, бабы,\\
Мальчишки\ftn{разносчики}, лавки, фонари,\\
Дворцы, сады, монастыри,\\
Бухарцы,\ftn{из города Бухара, эти продавцы восточных женских шали} сани, огороды,\ftn{много дворянины имели свою усадбу в Москве с огородом, конюшнои...}\\
Купцы, лачужки, мужики,\\
Бульвары, башни, казаки,\\
Аптеки, магазины моды,\\
Балконы, львы на воротах\ftn{герп}\\
И стаи галок\ftn{птичка} на крестах.

XXXIX. ХL

В сей утомительной прогулке\\
Проходит час-другой, и вот\\
У Харитонья в переулке\\
Возок пред домом у ворот\\
Остановился. К старой тетке,\\
Четвертый год больной в чахотке,\\
Они приехали теперь.\\
Им настежь отворяет дверь,\\
В очках, в изорванном кафтане,\\
С чулком в руке, седой калмык.\\
Встречает их в гостиной крик\\
Княжны, простертой на диване.\\
Старушки с плачем обнялись,\\
И восклицанья полились.

ХLI

— Княжна, mon аngе! —\\
«Раchеttе!» — Алина! —\\
«Кто б мог подумать? Как давно!\\
Надолго ль? Милая! Кузина!\\
Садись — как это мудрено!\\
Ей-богу, сцена из романа…»\\
— А это дочь моя, Татьяна. —\\
«Ах, Таня! подойди ко мне —\\
Как будто брежу я во сне…\\
Кузина, помнишь Грандисона?»\\
— Как, Грандисон?.. а, Грандисон!\\
Да, помню, помню. Где же он? —\\
«В Москве, живет у Симеона;\\
Меня в сочельник навестил;\\
Недавно сына он женил.

ХLII

А тот… но после все расскажем,\\
Не правда ль? Всей ее родне\\
Мы Таню завтра же покажем.\\
Жаль, разъезжать нет мочи мне;\\
Едва, едва таскаю ноги.\\
Но вы замучены с дороги;\\
Пойдемте вместе отдохнуть…\\
Ох, силы нет… устала грудь…\\
Мне тяжела теперь и радость,\\
Не только грусть… душа моя,\\
Уж никуда не годна я…\\
Под старость жизнь такая гадость…»\\
И тут, совсем утомлена,\\
В слезах раскашлялась она.

XLIII

Больной и ласки и веселье\\
Татьяну трогают; но ей\\
Нехорошо на новоселье,\\
Привыкшей к горнице своей.\\
Под занавескою шелковой\\
Не спится ей в постеле новой,\\
И ранний звон колоколов,\\
Предтеча утренних трудов,\\
Ее с постели подымает.\\
Садится Таня у окна.\\
Редеет сумрак; но она\\
Своих полей не различает:\\
Пред нею незнакомый двор,\\
Конюшня, кухня и забор.

XLIV

И вот: по родственным обедам\\
Развозят Таню каждый день\\
Представить бабушкам и дедам\\
Ее рассеянную лень.\\
Родне, прибывшей издалеча,\\
Повсюду ласковая встреча,\\
И восклицанья, и хлеб-соль.\\
«Как Таня выросла! Давно ль\\
Я, кажется, тебя крестила?\\
А я так на руки брала!\\
А я так за уши драла!\\
А я так пряником кормила!»\\
И хором бабушки твердят:\\
«Как наши годы-то летят!»

XLV

Но в них не видно перемены;\\
Все в них на старый образец:\\
У тетушки княжны Елены\\
Все тот же тюлевый чепец;\\
Все белится Лукерья Львовна,\\
Все то же лжет Любовь Петровна,\\
Иван Петрович так же глуп,\\
Семен Петрович так же скуп,\\
У Пелагеи Николавны\\
Все тот же друг мосье Финмуш,\\
И тот же шпиц, и тот же муж;\\
А он, все клуба член исправный,\\
Все так же смирен, так же глух\\
И так же ест и пьет за двух.

XLVI

Их дочки Таню обнимают.\\
Младые грации Москвы\\
Сначала молча озирают\\
Татьяну с ног до головы;\\
Ее находят что-то странной,\\
Провинциальной и жеманной,\\
И что-то бледной и худой,\\
А впрочем очень недурной;\\
Потом, покорствуя природе,\\
Дружатся с ней, к себе ведут,\\
Целуют, нежно руки жмут,\\
Взбивают кудри ей по моде\\
И поверяют нараспев\\
Сердечны тайны, тайны дев,

XLVII

Чужие и свои победы,\\
Надежды, шалости, мечты.\\
Текут невинные беседы\\
С прикрасой легкой клеветы.\\
Потом, в отплату лепетанья,\\
Ее сердечного признанья\\
Умильно требуют оне.\\
Но Таня, точно как во сне,\\
Их речи слышит без участья,\\
Не понимает ничего,\\
И тайну сердца своего,\\
Заветный клад и слез и счастья,\\
Хранит безмолвно между тем\\
И им не делится ни с кем.

XLVIII

Татьяна вслушаться желает\\
В беседы, в общий разговор;\\
Но всех в гостиной занимает\\
Такой бессвязный, пошлый вздор;\\
Все в них так бледно, равнодушно;\\
Они клевещут даже скучно;\\
В бесплодной сухости речей,\\
Расспросов, сплетен и вестей\\
Не вспыхнет мысли в целы сутки,\\
Хоть невзначай, хоть наобум;\\
Не улыбнется томный ум,\\
Не дрогнет сердце, хоть для шутки.\\
И даже глупости смешной\\
В тебе не встретишь, свет пустой.

XLIX

Архивны юноши толпою\\
На Таню чопорно глядят\\
И про нее между собою\\
Неблагосклонно говорят.\\
Один какой-то шут печальный\\
Ее находит идеальной\\
И, прислонившись у дверей,\\
Элегию готовит ей.\\
У скучной тетки Таню встретя,\\
К ней как-то Вяземский подсел\\
И душу ей занять успел.\\
И, близ него ее заметя,\\
Об ней, поправя свой парик,\\
Осведомляется старик.

L

Но там, где Мельпомены бурной\\
Протяжный раздается вой,\\
Где машет мантией мишурной\\
Она пред хладною толпой,\\
Где Талия тихонько дремлет\\
И плескам дружеским не внемлет,\\
Где Терпсихоре лишь одной\\
Дивится зритель молодой\\
(Что было также в прежни леты,\\
Во время ваше и мое),\\
Не обратились на нее\\
Ни дам ревнивые лорнеты,\\
Ни трубки модных знатоков\\
Из лож и кресельных рядов.

LI

Ее привозят и в Собранье.\\
Там теснота, волненье, жар,\\
Музыки грохот, свеч блистанье,\\
Мельканье, вихорь быстрых пар,\\
Красавиц легкие уборы,\\
Людьми пестреющие хоры,\\
Невест обширный полукруг,\\
Все чувства поражает вдруг.\\
Здесь кажут франты записные\\
Свое нахальство, свой жилет\\
И невнимательный лорнет.\\
Сюда гусары отпускные\\
Спешат явиться, прогреметь,\\
Блеснуть, пленить и улететь.

LII

У ночи много звезд прелестных,\\
Красавиц много на Москве.\\
Но ярче всех подруг небесных\\
Луна в воздушной синеве.\\
Но та, которую не смею\\
Тревожить лирою моею,\\
Как величавая луна,\\
Средь жен и дев блестит одна.\\
С какою гордостью небесной\\
Земли касается она!\\
Как негой грудь ее полна!\\
Как томен взор ее чудесный!..\\
Но полно, полно; перестань:\\
Ты заплатил безумству дань.

LIII

Шум, хохот, беготня, поклоны,\\
Галоп, мазурка, вальс… Меж тем,\\
Между двух теток у колонны,\\
Не замечаема никем,\\
Татьяна смотрит и не видит,\\
Волненье света ненавидит;\\
Ей душно здесь… она мечтой\\
Стремится к жизни полевой,\\
В деревню, к бедным поселянам,\\
В уединенный уголок,\\
Где льется светлый ручеек,\\
К своим цветам, к своим романам\\
И в сумрак липовых аллей,\\
Туда, где он являлся ей.

LIV

Так мысль ее далече\\
Забыт и свет и шумный бал,\\
А глаз меж тем с нее не сводит\\
Какой-то важный генерал.\\
Друг другу тетушки мигнули\\
И локтем Таню враз толкнули,\\
И каждая шепнула ей:\\
— Взгляни налево поскорей. —\\
«Налево? где? что там такое?»\\
— Ну, что бы ни было, гляди…\\
В той кучке, видишь? впереди,\\
Там, где еще в мундирах двое…\\
Вот отошел… вот боком стал… —\\
«Кто? толстый этот генерал?»

LV

Но здесь с победою поздравим\\
Татьяну милую мою\\
И в сторону свой путь направим,\\
Чтоб не забыть, о ком пою…\\
Да кстати, здесь о том два слова:\\
Пою приятеля младого\\
И множество его причуд.\\
Благослови мой долгий труд,\\
О ты, эпическая муза!\\
И, верный посох мне вручив,\\
Не дай блуждать мне вкось и вкрив.\\
Довольно. С плеч долой обуза!\\
Я классицизму отдал честь:\\
Хоть поздно, а вступленье есть.
\end{verse}
\chapter{Седмая глава}  
Время работы П над серединой седьмой главы (апрель 1828 г., подробнее см. в разделе «Хро­нология работы Пушкина над романом») совпадало со сложными процессами в творчестве П. В сознании поэта боролись две —в этот период противоположные и не находившие синтеза —тенденции. Первая из них — стремление к историзму, которое толкало П к принятию объективного хода исторических событий в том виде, в каком они даны в реальной действительности. 
С этих позиций требования, предъявляемые отдельной лич­
ностью к истории, третировались как «романтизм» и
«эгоизм». Не лишенные оттенка «примирения с дейст­
вительностью», такие настроения давали, однако, мощ­
ный толчок реалистическому и историческому сознанию и
определили целый ряд антиромантических выступлений
в творчестве П этих лет (от «Полтавы» и «Стансов»
(1826) до заметки о драмах Байрона). Однако пока еще
подспудно, в черновиках и глубинах сознания зрела
мысль о непреходящей ценности человеческой личности
и о необходимости мерить исторический прогресс
счастьем и правами отдельного человека. «Герой, будь
прежде человек» (1826) (VI, 411). «И нас они (домаш­
ние божества. —Ю. Л.У науке первой учат — Чтить
самого себя» (1829) (III, 1, 193), «Оставь герою сердце!
Что же Он будет без него? Тиран...» (1830) (III, 1, 253) —
такова цепь высказываний, которая закономерно при­
ведет к «Медному всаднику» и «Капитанской дочке».
\chapter{Петровский Замок}\label{ch:дд}
Дворцовый ансамбль — первая самостоятельная работа Матвея Казакова.\\
Зодчий, впоследствии перестроивший центр Москвы в палладианском стиле,
для дворца на окраине города выбрал псевдоготику, объединив мотивы древнерусской,
готической и восточной архитектуры.\\
 Расположилось величественное здание со стрельчатыми окнами на землях Высоко-Петровского монастыря. За что дворец и стал именоваться Петровским. Второе название — Путевой — объясняется его предназначением: для отдыха по пути из Петербурга в Москву. Дорога неблизкая — 700 километров.
 
Прообразом здания послужил временный город-театр в турецком стиле, построенный Василием Баженовым на Ходынском поле для торжеств по случаю подписания Кучук-Кайнаджирского мирного договора.\\
 Императрице работа понравилась, и она захотела похожий дворец. По задумке архитектора центральный купол символизировал православный храм, возвышаясь над башнями-минаретами, встроенными в боковые флигели. Богатство и величие Российской империи подчеркивали многочисленные лепные украшения, а пейзажный парк с гротами и беседками добавлял резиденции уюта.
Останавливалась Екатерина II в узорчатом замке лишь однажды, в 1787 году.\\
Спустя 10 лет путевую резиденцию посетил перед коронацией Павел I. А впоследствии и Николай II.\\
В 1812 году императорский замок на время стал ставкой Наполеона. Бонапарт покинул охваченный огнем Кремль и наблюдал из окна дворца за тем, как полыхает Москва.

Вот, окружен своей дубравой,\\
Петровский замок. Мрачно он\\ 
Недавнею гордится славой.\\
Напрасно ждал Наполеон,\\
Последним счастьем упоенный,\\
Москвы коленопреклоненной\\
С ключами старого Кремля:\\
Нет, не пошла Москва моя\\
К нему с повинной головою.\\
Не праздник, не приемный дар,\\
Она готовила пожар\\
Нетерпеливому герою.\\
Отселе, в думу погружен,\\
Глядел на грозный пламень он.

Так описывал те события Пушкин в «Евгении Онегине».\\
Но лагерь французской армии размещался здесь недолго, всего четыре дня.\\
Что не помешало солдатам и офицерам разграбить дворец и практически уничтожить парк. Выстояли только стены.\\
После бегства французов в боковых крыльях был устроен временный госпиталь.





%\label{Note} % Etichetta per riferimenti incrociati


\documentclass{article}
\usepackage[italian]{babel}

\begin{document}

% Attiva la barra per le righe troppo lunghe
\overfullrule=10pt

% Esempio di riga che causa un Overfull \hbox
Questa è una frase molto lunga che non si adatta correttamente nella larghezza del testo perché contiene parole lunghe e spazi fissi.

% Una frase normale che non causa problemi
Questa frase dovrebbe adattarsi senza problemi.

\end{document}






\end{verse}








\end{document}
