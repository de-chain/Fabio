\documentclass{contratto}

\begin{document}

\begin{center}
    \textbf{ LOCAZIONE ABITATIVA DI NATURA TRANSITORIA}\\
    \vspace{0.5cm}
   (Legge 9 dicembre 1998, n. 431, articolo 5, comma 1)
\end{center}

% Rest of your contract body as shown in Option 1
Il sig. Rizzi Fabio nato a Padova (PD) il 11 ottobre 1970 e residente a Mestrino (PD)  in Via Fratelli Bandiera 11/B, codice fiscale RZZFBA70R11G224F proprietà 50\% e la Sig.ra Parchina Larissa nata in Russia (EE) il 08/09/1970 residente a Mestrino (PD) in Via Fratelli Bandiera 11/B, codice fiscale PRCLLSS70P48Z154S proprietà 50\%di seguito denominati locatore concede in locazione al sig. Aghayev Tural nato a Baku (EE) il 12 febbraio 2002 e residente a Baku, Azerbaijan (EE) codice fiscale GHYTRL02B12Z253F di seguito denominato conduttore,  che accetta, per sé  la locazione parziale dell'unità immobiliare posta in MESTRINO (PD) via Fratelli Bandiera 11/B, censita al catasto urbano: foglio 13 particella 1488, subalterno 9, categoria catastale A/2 Classe 2 superficie calpestabile mq 111,68, rendita catastale € 542,28 e il garage censito al catasto urbano: foglio 13, particella 1488, subalterno 7 categoria catastale C/6 Classe 3, superficie calpestabile mq 23, rendita catastale € 44,62.
Visto l’articolo 1, comma 7, del Decreto-legge 23 dicembre 2013, n. 145, il conduttore dà atto e dichiara di aver ricevuto le informazioni e la documentazione, comprensiva dell'attestato, in ordine alla prestazione energetica degli edifici. Classe energetica: E.\\
La porzione dell’unità immobiliare locata è di mq 11, situata al 1 piano rispetto all'ingresso e composta da: letto singolo, scrivania con sedia, armadio, riscaldamento a termosifone, prese corrente elettrica, presa lan, wifi.\\
Il conduttore avrà inoltre l’uso condiviso di servizi e spazi comuni composti da: al 1 piano un bagno, al piano terra  zona cucina, zona giorno, zona ingresso e bagno.\\
Il locatore riserva per se dove risiede a dimora abitualmente la stanza situata al 1 piano rispetto all'ingresso, di mq 13,5 adiacente alla stanza locata e composta da: letto matrimoniale, armadio, riscaldamento a termosifone, prese corrente elettrica, wifi.\\
Superficie locativa totale: mq. 125,48\\
Elementi caratterizzanti: 1, 8, 12, 17, 19, 21, 25, 26, 29, 30, 31, 33, 34, 38, 44, 48, 49, 51, 52, si veda l'Allegato B. \\
Elementi arredo: 1, 2, 3, 4, 5, 6, 7, 8, 9, 10, 11, 12, 14, 15, 16, 17, 18, 19, 22, 24, 25, 26, 27, 28, 29, 30, 31, 32, 33, 34, si veda l'Allegato B.\\
\begin{center}
\textbf{\Large Articolo 1}\\
\textit{(Durata)}
\end{center}
Il contratto è stipulato per la durata di 10 mesi, dal 01 ottobre 2024 al 31 luglio 2024 allorché, fatto salvo quanto previsto dall’articolo 2 cessa senza bisogno di alcuna disdetta.

\newpage
\begin{center}
\textbf{\Large Articolo 2}\\
\textit{(Esigenza del locatore/conduttore)}
\end{center}
 Il conduttore, nel rispetto di quanto previsto dal decreto del Ministro delle infrastrutture e dei trasporti di concerto con il Ministro dell’economia e delle finanze, emanato ai sensi dell'articolo 4, comma 2, della legge n. 431/98 - di cui il presente tipo di contratto costituisce l’Allegato D - e dall'Accordo territoriale tra le organizzazioni sindacali dei locatori ASPPI, CONFEDILIZIA, UPPI e dei conduttori SICET, SUNIA, UNIAT depositato il 11/07/2016 al prot. n° 8226 presso il Comune di MESTRINO dichiara la seguente esigenza che giustifica la transitorietà del contratto per motivi di studio non rientranti nell’ipotesi di cui all’art. 5, comma 2 e 3 della L. 431/98, e che documenta, in caso di durata superiore a 30 giorni, allegando copia dell’autocertificazione attestante l’iscrizione all'Università di Padova.


\begin{center}
\textbf{\Large Articolo 3}\\
\textit{(Inadempimento delle modalità di stipula)}
\end{center}
Il presente contratto è ricondotto alla durata prevista dall’art. 2 comma 1 della legge 9 dicembre 1998, n.
431, in caso di inadempimento delle modalità di stipula previste dall’art. 2, commi 1, 2, 3, 4, 5 e 6 del decre -
to dei Ministri delle infrastrutture e dell’economia e delle finanze ex art. 4 comma 2 della legge 431/98.
In ogni caso, ove il locatore abbia riacquistato la disponibilità dell'alloggio alla scadenza dichiarando di volerlo adibire ad un uso determinato e non lo adibisca, senza giustificato motivo, nel termine di sei mesi dalla
data in cui ha riacquistato la detta disponibilità, a tale uso, il conduttore ha diritto al ripristino del rapporto di
locazione alle condizioni di cui all'articolo 2, comma 1, della legge n. 431/98 o, in alternativa, ad un risarcimento in misura pari a trentasei mensilità dell'ultimo canone di locazione corrisposto.

\begin{center}
\textbf{\Large Articolo 4}\\
\textit{(Canone)}
\end{center}
Nei Comuni con un numero di abitanti superiore a diecimila, come risultanti dai dati ufficiali dell’ultimo censimento, il canone di locazione, secondo quanto stabilito dall'Accordo territoriale tra le organizzazioni sindacali dei locatori ASPPI, CONFEDILIZIA, UPPI e dei conduttori SICET, SUNIA, UNIAT depositato il 11/07/2016 al prot. n° 8226 presso il Comune di MESTRINO è convenuto in euro 2000 (duemila/00), importo che il conduttore si obbliga a corrispondere nel domicilio del locatore ovvero a mezzo di bonifico bancario, in n. 10 rate eguali anticipate di euro 200 (duecento/00) ciascuna, alle seguenti date: il 27 di ogni mese.

\begin{center}
\textbf{\Large Articolo 5}\\
\textit{(Deposito cauzionale e altre forme di garanzia)}
\end{center}
A garanzia delle obbligazioni assunte col presente contratto, il conduttore versa al locatore (che con la firma del contratto ne rilascia, in caso, quietanza) una somma di euro 400 (quattrocento/00) pari a numero 02 mensilità del canone, non imputabile in conto canoni e produttiva di interessi legali, riconosciuti al conduttore al termine della locazione. Il deposito cauzionale così costituito viene reso al termine della locazione previa verifica dello stato dell'unità immobiliare e dell'osservanza di ogni obbligazione contrattuale. 

\begin{center}
\textbf{\Large Articolo 6}\\
\textit{(Oneri accessori)}
\end{center}
Per gli oneri accessori le parti fanno applicazione alla “Ripartizione delle spese condominiali”, il pagamento degli oneri anzidetti, per la quota parte di quelli condominiali/comuni a carico del conduttore, deve avvenire entro sessanta giorni dalla richiesta. Prima di effettuare il pagamento, il conduttore ha diritto di ottenere l'indicazione specifica delle spese anzidette e dei criteri di ripartizione. Ha inoltre diritto di prendere visione - anche tramite organizzazioni sindacali - presso il locatore (o il suo amministratore o l'amministratore condominiale, ove esistente) dei documenti giustificativi delle spese effettuate. Insieme con il pagamento della prima rata del canone annuale, il conduttore versa  una quota di acconto non superiore a quella di sua spettanza risultante dal rendiconto dell'anno precedente. Sono interamente a carico del conduttore le spese relative ad ogni utenza (energia elettrica, acqua, gas, telefono e altro ……..). Per le spese di cui al presente articolo, il conduttore versa una quota di euro 50 (cinquanta/00) mensili, salvo conguaglio. 

\begin{center}
\textbf{\Large Articolo 7}\\
\textit{(Spese di bollo e registrazione)}
\end{center}
Le spese di bollo per il presente contratto e per le ricevute conseguenti sono a carico del conduttore. Il locatore provvede alla registrazione del contratto, ove dovuta, dandone comunicazione al conduttore - che corrisponde la quota di sua spettanza, pari alla metà - e all’Amministratore del Condominio ai sensi dell’art. 13 della legge 431/98. Le parti possono delegare alla registrazione del contratto una delle organizzazioni sindacali che abbia prestato assistenza ai fini della stipula del contratto medesimo.

\begin{center}
\textbf{\Large Articolo 8}\\
\textit{(Pagamento)}
\end{center}
 Il pagamento del canone o di quant'altro dovuto anche per oneri accessori non può venire sospeso o ritardato da pretese o eccezioni del conduttore, qualunque
ne sia il titolo. Il mancato puntuale pagamento, per qualunque causa, anche di una sola rata del canone (nonché di quant'altro dovuto, ove di importo pari almeno ad una mensilità del canone), costituisce in mora il conduttore, fatto salvo quanto previsto dall’articolo 55 della legge n. 392/78.

\begin{center}
\textbf{\Large Articolo 9}\\
\textit{(Uso)}
\end{center}
 L'immobile deve essere destinato esclusivamente a civile abitazione del conduttore. Salvo patto scritto contrario, è fatto divieto di sublocare o dare in comodato, né in tutto né in parte, l’unità immobiliare, pena la risoluzione di diritto del contratto. Per la successione nel contratto, si applica l'articolo 6 della legge n. 392/78, nel testo vigente a seguito della sentenza della Corte costituzionale n. 404 del 1988.

\newpage
\begin{center}
\textbf{\Large Articolo 10}\\
\textit{(Recesso del conduttore)}
\end{center}
Il conduttore ha facoltà di recedere per gravi motivi dal contratto previo avviso da recapitarsi mediante lettera raccomandata almeno1 mese prima.

\begin{center}
\textbf{\Large Articolo 11}\\
\textit{(Consegna)}
\end{center}
 Il conduttore dichiara di aver visitato l'unità immobiliare locatagli, di averla trovata adatta all'uso convenuto e, pertanto, di prenderla in consegna ad ogni effetto col
ritiro delle chiavi, costituendosi da quel momento custode della stessa. Il conduttore si impegna a riconsegnare l'unità immobiliare nello stato in cui l'ha ricevuta, salvo il
deperimento d'uso, pena il risarcimento del danno; si impegna, altresì, a rispettare le norme del regolamento dello stabile ove esistente, accusando in tal caso ricevuta dello stesso con la firma del presente contratto, così come si impegna ad osservare le deliberazioni dell'assemblea dei condomini. È in ogni caso vietato al conduttore compiere atti e tenere comportamenti che possano recare molestia agli altri abitanti dello stabile. Le parti danno atto, in relazione allo stato dell’unità immobiliare, ai sensi dell'articolo 1590 del Codice civile, di quanto segue: l'immobile in oggetto del presente contratto viene consegnato in buono stato di conservazione, con normali segni di usura dovuti al tempo e all'uso ordinario. Qualsiasi eventuale difetto che ecceda l'usura naturale è stato comunicato e accettato dalle parti prima della sottoscrizione del contratto.

\begin{center}
\textbf{\Large Articolo 12}\\
\textit{(Modifica e danni)}
\end{center}
Il conduttore non può apportare alcuna modifica, innovazione, miglioria o addizione ai locali locati ed alla loro destinazione, o agli impianti esistenti, senza il preventivo consenso scritto del locatore. 

\begin{center}
\textbf{\Large Articolo 13}\\
\textit{(Assemblee)}
\end{center}
Il conduttore esonera espressamente il locatore da ogni
responsabilità per danni diretti o indiretti che possano derivargli da fatti dei dipendenti del
locatore medesimo nonché per interruzioni incolpevoli dei servizi. Il conduttore ha diritto di
voto, in luogo del proprietario dell'unità immobiliare locatagli, nelle deliberazioni
dell'assemblea condominiale relative alle spese ed alle modalità di gestione dei servizi di
riscaldamento e di condizionamento d'aria. Ha inoltre diritto di intervenire, senza voto, sulle
deliberazioni relative alla modificazione degli altri servizi comuni. Quanto stabilito in materia
di riscaldamento e di condizionamento d'aria si applica anche ove si tratti di edificio non in
condominio. In tale caso (e con l'osservanza, in quanto applicabili, delle disposizioni del
codice civile sull'assemblea dei condomini) i conduttori si riuniscono in apposita assemblea,
convocata dalla proprietà o da almeno tre conduttori. 

\newpage
\begin{center}
\textbf{\Large Articolo 14}\\
\textit{(Impianti)}
\end{center}
 Il conduttore - in caso d'installazione sullo stabile di antenna
televisiva centralizzata - si obbliga a servirsi unicamente dell'impianto relativo, restando sin
d'ora il locatore, in caso di inosservanza, autorizzato a far rimuovere e demolire ogni antenna
individuale a spese del conduttore, il quale nulla può pretendere a qualsiasi titolo, fatte salve
le eccezioni di legge. Per quanto attiene all'impianto termico autonomo, ove presente, ai sensi
della normativa del d.lgs. n.192/05, con particolare riferimento all’art. 7 comma 1, il
conduttore subentra per la durata della detenzione alla figura del proprietario nell’onere di
adempiere alle operazioni di controllo e di manutenzione.

\begin{center}
\textbf{\Large Articolo 15}\\
\textit{(Accesso)}
\end{center}
 Il conduttore deve consentire l'accesso all'unità immobiliare al
locatore, al suo amministratore nonché ai loro incaricati ove gli stessi ne abbiano -
motivandola - ragione. Nel caso in cui il locatore intenda vendere o locare l'unità immobiliare,
in caso di recesso anticipato del conduttore, questi deve consentirne la visita una volta la
settimana, per almeno due ore, con esclusione dei giorni festivi oppure con le seguenti modalità: previo accordi telefonici.

\begin{center}
\textbf{\Large Articolo 16}\\
\textit{(Commissione di negoziazione paritetica e conciliazione stragiudiziale)}
\end{center}
Commissione di cui all’articolo 6 del decreto del Ministro delle infrastrutture e dei trasporti di
concerto con il Ministro dell’economia e delle finanze, emanato ai sensi dell’articolo 4, comma 2,
della legge 431 del 1998, è composta da due membri scelti fra appartenenti alle rispettive
organizzazioni firmatarie dell'Accordo territoriale sulla base delle designazioni, rispettivamente, del
locatore e del conduttore ed un terzo – che svolge funzioni di presidente – sulla base della scelta
operata dai due componenti come sopra designati qualora gli stessi ritengano di nominarlo. La
richiesta di intervento della Commissione non determina la sospensione delle obbligazioni
contrattuali.
\begin{center}
\textbf{\Large Articolo 17}\\
\textit{(Varie)}
\end{center}
A tutti gli effetti del presente contratto, comprese la notifica degli atti
esecutivi, e ai fini della competenza a giudicare, il conduttore elegge domicilio nei locali a lui
locati e, ove egli più non li occupi o comunque detenga, presso l'ufficio di segreteria del
Comune ove è situato l'immobile locato. Qualunque modifica al presente contratto non può
aver luogo, e non può essere provata, se non con atto scritto. Il locatore ed il conduttore si
autorizzano reciprocamente a comunicare a terzi i propri dati personali in relazione ad
adempimenti connessi col rapporto di locazione (d.lgs. n. 196/03). Per quanto non previsto
dal presente contratto le parti rinviano a quanto in materia disposto dal Codice civile, dalle
leggi n. 392/78 e n. 431/98 o comunque dalle norme vigenti e dagli usi locali nonché alla
normativa ministeriale emanata in applicazione della legge n. 431/98 ed all'Accordo
territoriale. \\

\begin{center}
\textbf{\Large Articolo 18}\\
\textit{(Cedolare secca)}
\end{center}
La parte locatrice dichiara di esercitare l'opzione per il sistema denominato "cedolare secca"
introdotto dall’art. 3 del D.Lgs. 23/2011 restando pertanto esonerata dall'obbligo di inviare al
conduttore la prevista comunicazione mediante lettera raccomandata e dall'obbligo di
corrispondere l'imposta di registro e di bollo. L’opzione in oggetto comporta, per il periodo
corrispondente alla sua durata, il venir meno dell’obbligo di versamento delle imposte di registro e
di bollo relative al presente contratto, comprese quelle dovute sulla risoluzione e sulle proroghe del
contratto stesso. Non potrà inoltre essere applicata, negli anni di decorrenza del contratto, alcuna
maggiorazione del canone, inclusa la variazione accertata dall’ISTAT. La parte locatrice si riserva
tuttavia la facoltà di revocare tale opzione in ciascuna annualità contrattuale successiva a quella in
corso, da comunicarsi alla parte conduttrice a mezzo lettera raccomandata con ricevuta di ritorno
prima della scadenza dell’annualità contrattuale. Se in futuro la parte locatrice decidesse di
revocare l’applicazione di tale regime fiscale, sull’importo annuo del canone dovrà essere pagata
l’imposta di registro (di cui metà a carico della parte locatrice e metà a carico del conduttore). In tal
caso il canone sarà aggiornato nella misura del 75% della variazione in aumento dell’indice dei
prezzi al consumo accertato dall’ISTAT per le famiglie degli operai e degli impiegati seguendo le
modalità previste dalla normativa della cedolare secca.\\

Letto, approvato e sottoscritto\\\\
Padova, \\\\
Il locatore ………………………………………..\\\\
Il conduttore ……………………………………..\\

A mente degli articoli 1341 e 1342, del Codice civile, le parti specificamente approvano i patti
di cui agli articoli 2 (Esigenza del locatore/conduttore), 3 (Cessazione delle condizioni di
transitorietà), 4 (Canone), 5 (Deposito cauzionale e altre forme di garanzia), 6 (Oneri
accessori), 8 (Pagamento, risoluzione), 9 (Uso), 10 (Recesso del conduttore), 11 (Consegna),
12 (Modifiche e danni), 14 (Impianti), 15 (Accesso), 16 (Commissione di negoziazione
paritetica e conciliazione stragiudiziale) e 17 (Varie) del presente contratto. \\\\
Il locatore ………………………………………...\\\\
Il conduttore ……………………………………… 

\end{document}
